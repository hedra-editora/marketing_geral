% AUTOR_LIVRO_AUTOR.tex
% quem é "autor"
% > "VIDA & OBRA"

% Preencher com o nome das cor ou composição RGB (ex: [r=0.862, g=0.118, b=0.118]) 
\usecolors[crayola] 			   % Paleta de cores pré-definida: wiki.contextgarden.net/Color#Pre-defined_colors

% Cores definidas pelo designer:
% MyGreen		r=0.251, g=0.678, b=0.290 % 40ad4a
% MyCyan		r=0.188, g=0.749, b=0.741 % 30bfbd
% MyRed			r=0.820, g=0.141, b=0.161 % d12429
% MyPink		r=0.980, g=0.780, b=0.761 % fac7c2
% MyGray		r=0.812, g=0.788, b=0.780 % cfc9c7
% MyOrange		r=0.980, g=0.671, b=0.290 % faab4a

% Configuração de cores
\definecolor[MyColor]	  [x=fac7c2]      % ou ex: [r=0.862, g=0.118, b=0.118] % corresponde a RGB(220, 30, 30)
\definecolor[MyColorText] [r=0.655, g=0.663, b=0.675]      % ou ex: [r=0.862, g=0.118, b=0.118] % corresponde a RGB(167, 169, 172)

% Classe para diagramação dos posts
\environment{marketing.env}		   

% Cabeço e rodapé: Informações (caso queira trocar alguma coisa)
 		\def\MensagemSaibaMais  {SAIBA MAIS:}
 		\def\MensagemSite		{HEDRA.COM.BR}
 		\def\MensagemLink       {LINK NA BIO}


% - Largura: 1200 pixels * 0.26458 mm ≈ 317 mm
% - Altura: 400 pixels * 0.26458 mm ≈ 106 mm

\def\MyPicture#1{\vfill
                \starttikzpicture
                    \node[yshift=-5mm, inner sep=0] (image) at (0,0) {\externalfigure[#1][width=\paperwidth]};
                    \fill[MyColorText,opacity=0.3] (image.south west) rectangle (image.north east);
                \stoptikzpicture} 

\definepapersize    [post]        [width=1200px, height=400px]
\definepapersize    [14x21]       [width=140mm, height=210mm]
\setuppapersize     [post]        %[14x21]

\definelayer[cover][
  x=0mm,
  y=0mm,
  width=\paperwidth,
  height=\paperheight,
]


\def\figura{   
			}

\setlayer[cover][
  hoffset=0mm,
  voffset=0mm,
]{%
  \framed[
    frame=off
    width=\paperwidth,
    height=\paperheight,
  ]{%
  {
                \starttikzpicture
                    \node[yshift=-5mm, inner sep=0] (image) at (0,0) {\externalfigure[banner2][width=\textwidth]};
                    \fill[MyColorText,opacity=0.3] (image.south west) rectangle (image.north east);
                \stoptikzpicture}  
    %\externalfigure[banner2][width=\paperwidth]
  }
}

\starttext %--------------------------------------------------------|

\Hashtag{Oioi}

\setupbackgrounds[page][background=cover]


%\MyPicture{banner2}

Mussum Ipsum, cacilds vidis litro abertis.  Suco de cevadiss, é um leite
divinis, qui tem lupuliz, matis, aguis e fermentis. 

\stoptext %---------------------------------------------------------|
			