\usecolors[crayola] 			   % wiki.contextgarden.net/Color#Pre-defined_colors
\definecolor[MyColor][Almond]      % ou ex: [r=0.862, g=0.118, b=0.118] % corresponde a RGB(220, 30, 30)
\definecolor[MyColorText][Salmon]  % ou ex: [r=0.862, g=0.118, b=0.118] % corresponde a RGB(167, 169, 172)
\environment{marketing.env}		   % onde está todo o código


% Comandos %%%%%%%%%%%%%%%%%%%%%%%%%%%%%%%%%%%%%%%%%%%%%%%%%%%%%%%%%%%%%%%%%%%%%%%%%%|

% Cores:

% Pesos para os títulos:
%		\startMyCampaign...		 \stopMyCampaign
%		\stopMyCampaignSection...   \stopMyCampaignSection

% Aplicação de imagens: 
% 		\MyCover{capa.pdf}
%		\MyPicture{Imagem.png}  % Imagem se possíve deitada. Comando aplica filtro
%		\MyPhoto{}

% Cabeço e rodabé: Opções
% 		\Mensagem{AGORA É QUE SÃO ELAS}
% 		\Hashtag{campanha de natal}
% 		\Mensagem{campanha de natal}

% Cabeço e rodabé: Informações (caso queira trocar alguma coisa)
% 		\def\MensagemSaibaMais{SAIBA MAIS:}
% 		\def\MensagemSite{HEDRA.COM.BR}
% 		\def\MensagemLink{LINK NA BIO}

% Atalhos: 						
% 		\Seta  % Seta para baixo

% Logos e selos: 				
% \Hedra
% \HedraAyllon	% Não está pronto
% \HedraAcorde	% Não está pronto
% \Ayllon		% Não está pronto
% \Acorde		% Não está pronto

%%%%%%%%%%%%%%%%%%%%%%%%%%%%%%%%%%%%%%%%%%%%%%%%%%%%%%%%%%%%%%%%%%%%%%%%%%%%%%%%%%%%%|

\starttext
%\showframe  %Para mostrar somente as linhas.


\startMyCampaign
{\bf NARRATIVAS

DA ESCRAVIDÃO}

\Seta LANÇAMENTO
\stopMyCampaign


\page %---------------------------------------------------------|



\Hedra

\stoptext