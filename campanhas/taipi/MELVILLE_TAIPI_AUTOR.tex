% MELVILLE_TAIPI_AUTOR.tex
\usecolors[crayola]

% Configuração de cores TickleMePink
\definecolor[MyColor][x=414b47]      % ou ex: [r=0.862, g=0.118, b=0.118] % corresponde a RGB(220, 30, 30)
\definecolor[MyColorText][white]  % ou ex: [r=0.862, g=0.118, b=0.118] % corresponde a RGB(167, 169, 172)
% Classe para diagramação dos posts
\environment{marketing.env}        

% Cabeço e rodapé: Informações (caso queira trocar alguma coisa)
        \def\MensagemSaibaMais  {SAIBA MAIS:}
        \def\MensagemSite       {HEDRA.COM.BR}
        \def\MensagemLink       {LINK NA BIO}
      

\def\MyBackground#1{
\defineoverlay
  [backgroundimage]
  [{\externalfigure[#1][height=\overlayheight]}]
}
\environment{extra.env}

\starttext  %---------------------------------------------------------|

\def\MyBackgroundMessage{ESCRITOR A BORDO}
\MyBackground{MELVILLE_TAIPI__1.jpg}

\startMyCampaign
\hyphenpenalty=10000
\exhyphenpenalty=10000
\position(0,7.8){\scale[factor=4]{\Seta\,HERMAN MELVILLE (1819--1891)}}
\stopMyCampaign

\page 

\Mensagem{ESCRITOR A BORDO}
\setupbackgrounds[page][background=color,backgroundcolor=MyColor]

Escritor, poeta e ensaísta norte-americano, {\bf MELVILLE} é autor de um dos livros mais emblemáticos de literatura de aventura, o célebre {\bf MOBY DICK} (1851), considerado um dos romances mais importantes da literatura ocidental. 

\page %----------------------------------------------------------|

A escrita de Melville baseia-se sobretudo em suas vivências de {\bf MARINHEIRO}, as quais lhe proporcionaram um distanciamento que certamente contribuiu para a minuciosa análise que realiza das {\bf CONTRADIÇÕES DA SOCIEDADE NORTE-AMERICANA}. 

\page

Apesar de hoje {\bf MOBY DICK} ser sua obra mais conhecida, quando vivo Melville era conhecido como o autor de {\bf TAIPI}, seu romance de estreia e responsável por consagrá-lo como um dos mais conhecidos autores dos {\cap EUA}. 

% Sua morte, bem como o centenário de seu nascimento, comemorado em 1919, foram de extrema importância para renovar o interesse pela figura de Melville e reavivar os estudos acadêmicos voltados para sua obra, a qual tinha finalmente ascendido à categoria dos {\bf CLÁSSICOS}.

\page

\MyCover{MELVILLE_TAIPI_THUMB}

\page %----------------------------------------------------------|

\Hedra

\stoptext %---------------------------------------------------------|
