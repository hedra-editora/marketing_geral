% MELVILLE_TAIPI_CURIOSIDADES_3.tex
% Preencher com o nome das cor ou composição RGB (ex: [r=0.862, g=0.118, b=0.118]) 
\usecolors[crayola] 			   % Paleta de cores pré-definida: wiki.contextgarden.net/Color#Pre-defined_colors

% Cores definidas pelo designer:
% MyGreen		r=0.251, g=0.678, b=0.290 % 40ad4a
% MyCyan		r=0.188, g=0.749, b=0.741 % 30bfbd
% MyRed			r=0.820, g=0.141, b=0.161 % d12429
% MyPink		r=0.980, g=0.780, b=0.761 % fac7c2
% MyGray		r=0.812, g=0.788, b=0.780 % cfc9c7
% MyOrange		r=0.980, g=0.671, b=0.290 % faab4a

% Configuração de cores

\definecolor[MyColor][x=414b47]      % ou ex: [r=0.862, g=0.118, b=0.118] % corresponde a RGB(220, 30, 30)
\definecolor[MyColorText][white]  % ou ex: [r=0.862, g=0.118, b=0.118] % corresponde a RGB(167, 169, 172)

% Classe para diagramação dos posts
\environment{marketing.env}		   

\starttext %---------------------------------------------------------|

\hyphenpenalty=10000
\exhyphenpenalty=10000

\Mensagem{EM CONTEXTO} %Sempre usar esse header

\startMyCampaign

\hyphenpenalty=10000
\exhyphenpenalty=10000

QUAL É A {\bf HISTÓRIA DA TATUAGEM}?
%Aqui a manchete pode ser mais longa

\stopMyCampaign

\page %---------------------------------------------------------| 

\hyphenpenalty=10000
\exhyphenpenalty=10000

A tatuagem é um elemento central em {\bf TAIPI}. O protagonista do romance sente profundo horror perante às tatuagens que os indígenas exibem.

\page


\page %---------------------------------------------------------|

\MyCover{MELVILLE_TAIPI_THUMB}

\page %---------------------------------------------------------|

\Hedra

\stoptext %---------------------------------------------------------|

% Existem muitas provas arqueológicas que afirmam que tatuagens foram feitas no Egito entre 4000 a.C. e 2000 a.C e também por nativos da Polinésia, Filipinas, Indonésia e Nova Zelândia (maori),tatuavam-se em rituais ligados a religião.[1] Os Ainu, um povo indígena do norte do Japão, tradicionalmente tinham tatuagens faciais, assim como os austro-asiáticos. Hoje, pode-se encontrar em diversas etnias espalhadas pelo mundo o costume de se utilizar tatuagens faciais, entre estes povos tem se os berberes do Norte da África, os iorubás, fulas e hauçás da Nigéria e os maoris da Nova Zelândia.[2]

% Múmias tatuadas foram recuperadas de pelo menos 49 sítios arqueológicos, incluindo locais na Groenlândia, no Alasca, na Sibéria, na Mongólia, no oeste da China, no Egito, no Sudão, nas Filipinas e nos Andes.[3] Estes incluem Amunet, Sacerdotisa da Deusa Hathor do antigo Egito (c. 2134–1991 a.C.), múltiplas múmias da Sibéria, incluindo a cultura Pazyryk da Rússia e de várias culturas em toda a América do Sul pré-colombiana.[4] Em 2015, a reavaliação científica da idade das duas mais antigas múmias tatuadas conhecidas, identificou Ötzi como o exemplo mais antigo atualmente conhecido. Este corpo, com 61 tatuagens, foi encontrado embutido em gelo glacial nos Alpes, e datado de 3250 a.C.[4][4]
% Tatuagens na Europa Antiga e Medieval

% Os registros escritos em grego sobre tatuagens datam de pelo menos o século V a.C. Os antigos gregos e romanos usavam tatuagens para penalizar escravos, criminosos e prisioneiros de guerra. Embora conhecida por estes, a tatuagem decorativa era desprezada e a tatuagem religiosa continuou sendo utilizada quase que exclusivamente no Egipto e na Síria após a anexação romana.[5] Porém mais tarde os romanos da antiguidade tardia também passaram a ter o costume de tatuar soldados e fabricantes de armas, uma prática que continuou no século IX.

% As tribos germânicas, celtas e outras tribos da Europa central e setentrional pré-cristã possuíam o costume de utilizar tatuagens, de acordo com registros sobreviventes, mas também pode ter sido tinta normal. Os pictos da Escócia podem ter sido tatuados com desenhos elaborados, inspirados na guerra, em preto ou azul escuro (ou, possivelmente, cobre para tons azuis). Júlio César descreveu essas tatuagens no Livro V de sua obra De Bello Gallico (c. 50 a.C.). No entanto, estas podem ter sido marcas pintadas em vez de tatuagens.[6]

% Amade ibne Fadalane escreveu sobre o seu encontro com uma tribo escandinava da Rússia no início do século X, descrevendo-os como tatuados de "unhas a pescoço" com "padrões de árvore azul-escuro" e outras "figuras".[7] No entanto, isso também pode ter sido pintado, uma vez que a palavra usada pode significar tatuagem e pintura.

% Durante o processo gradual de cristianização na Europa, as tatuagens eram muitas vezes consideradas elementos remanescentes do paganismo e geralmente proibidas legalmente. A Igreja na Idade Média baniu a tatuagem da Europa (em 787, ela foi proibida pelo Papa), sendo considerada como uma prática demoníaca, comumente caracterizando-a como prática de vandalismo no próprio corpo, afirmando em sua doutrina como maneira de vilipendiar o templo do Espírito Santo, o corpo, levando seus fiéis a uma forma verdadeiramente reta de louvor a Deus. Esta posição da Igreja nesta época veio a partir de uma interpretação do livro de Levítico, um livro do Antigo Testamento. De acordo com Robert Graves em seu livro "The Greek Myths", a tatuagem era comum entre certos grupos religiosos no antigo mundo mediterrâneo, o que pode ter contribuído para a proibição da tatuagem entre os judeus, como se pode ver no terceiro livro da Torá, o Levítico.
% Origem

% O termo tatuagem, pelo francês tatouage e, por sua vez, do inglês tattoo, tem sua origem em línguas polinésias (taitiano) na palavra tatau[8] e supõe-se que todos os povos circunvizinhos ao Oceano Pacífico possuíam a tradição da tatuagem além das dos Mares do Sul.

% O pai da palavra "tattoo" que conhecemos atualmente foi o capitão James Cook (também descobridor do surf), que escreveu em seu diário a palavra "tattow", também conhecida como "tatau" (era o som feito durante a execução da tatuagem, em que se utilizavam ossos finos como agulhas e uma espécie de martelinho para introduzir a tinta na pele). Com a circulação dos marinheiros ingleses a tatuagem e a palavra Tattoo entraram em contato com diversas outras civilizações pelo mundo novamente. Porém o Governo da Inglaterra adotou a tatuagem como uma forma de identificação de criminosos em 1879, a partir daí a tatuagem ganhou uma conotação fora-da-lei no Ocidente.

% O primeiro tatuador profissional documentado na Grã-Bretanha foi estabelecido no porto de Liverpool na década de 1870. Na Grã-Bretanha, a tatuagem ainda estava amplamente associada aos marinheiros e aos criminosos, mas a partir da década de 1870, se tornara moda entre alguns membros das classes superiores, incluindo a realeza.[9] Porém uma clara divisão de opiniões sobre a aceitabilidade da prática continuou por algum tempo na Grã-Bretanha.

% Inúmeras personalidades históricas, tais como: o Czar Nicolau II da Rússia; o Kaiser Guilherme II da Alemanha; a Rainha Vitória do Reino Unido; os Presidentes americanos James K. Polk, Theodore Roosevelt e Andrew Jackson; o Primeiro-ministro Britânico Winston Churchill; os escritores George Orwell e Dorothy Parker; o inventor Thomas Edison; o Rei Haroldo II de Inglaterra; o Rei Eduardo VII do Reino Unido; etc.; também tiveram tatuagem.[10][11][12] 