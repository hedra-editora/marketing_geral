% MELVILLE_TAIPI_CURIOSIDADES_2.tex
% Preencher com o nome das cor ou composição RGB (ex: [r=0.862, g=0.118, b=0.118]) 
\usecolors[crayola] 			   % Paleta de cores pré-definida: wiki.contextgarden.net/Color#Pre-defined_colors

% Cores definidas pelo designer:
% MyGreen		r=0.251, g=0.678, b=0.290 % 40ad4a
% MyCyan		r=0.188, g=0.749, b=0.741 % 30bfbd
% MyRed			r=0.820, g=0.141, b=0.161 % d12429
% MyPink		r=0.980, g=0.780, b=0.761 % fac7c2
% MyGray		r=0.812, g=0.788, b=0.780 % cfc9c7
% MyOrange		r=0.980, g=0.671, b=0.290 % faab4a

% Configuração de cores
\definecolor[MyColor][x=414b47]      % ou ex: [r=0.862, g=0.118, b=0.118] % corresponde a RGB(220, 30, 30)
\definecolor[MyColorText][white]  % ou ex: [r=0.862, g=0.118, b=0.118] % corresponde a RGB(167, 169, 172)

% Classe para diagramação dos posts
\environment{marketing.env}		   

\starttext %---------------------------------------------------------|

\hyphenpenalty=10000
\exhyphenpenalty=10000

\Mensagem{O MOVIMENTO PENDULAR DE TAIPI} %Sempre usar esse header

\startMyCampaign

\hyphenpenalty=10000
\exhyphenpenalty=10000

{\bf O OLHAR DE COMPAIXÃO E O MEDO DE ASSIMILAÇÃO}

\stopMyCampaign

\page %---------------------------------------------------------| 

\hyphenpenalty=10000
\exhyphenpenalty=10000

% COLOCAR TRECHOS QUE DEMONSTREM AS DUAS TENDENCIAS QUE PENDEM O OLHAR DE MELVILLE: DESPREZO/CHOQUE (TATUAGENS, CANIBALISMO, MEDO DE ASSIMILAÇAO) VS EMPATIA, BELEZA EXOTICA  (QUEM SAO OS VERDADEIROS SELVAGENS)

Em seu romance autobiográfico, {\bf TAIPI}, Melville expõe os {\bf SENTIMENTOS CONTRADITÓRIOS} que conviveram durante seu tempo como prisioneiro de um povo nativo das Ilhas Marquesas.

\page

O escritor demonstra compaixão, empatia e compreensão frente aos nativos, mas certas práticas
lhe inspiram profundo {\bf TERROR}, que atinge seu ápice na {\bf TATUAGEM E CANIBALISMO}, pelo qual os taipi são famosos entre os estrangeiros.

\page %---------------------------------------------------------|

O horror que lhe causa a {\bf TATUAGEM}, contrasta com a beleza, bastante exotizante, que ele identifica nos taipi. Ao mesmo tempo que descreve uma indígena como uma «a própria perfeição da graça e beleza femininas», ele diz que ela não estava livre da «horrível mácula da tatuagem».

\page

A tatuagem e o canibalismo funcionam como {\bf ENGRENAGENS DA NARRATIVA}, que geram suspense e angústia no protagonista que teme tanto ser assimilado quanto devorado. Numa ocasião em que os taipi mostram-se inclinados a tatuar sua pele, profere estar «horrorizado com a simples ideia de quedar medonho por toda a vida».

\page

Ao mesmo tempo que não se desfaz completamente do medo de ser sacrificado pelos supostos canibais, Melville reconhece que eles estão longe da brutalidade do mundo dos brancos. Ele questiona se rituais canibais seriam piores do que os crimes e guerras perpetrados pelo {\bf MUNDO “CIVILIZADO”}.

\page

«Ora, eles são {\bf CANIBAIS!} — exclamou Toby certa feita, quando elogiei a tribo.\\
— É verdade — respondi —, mas não creio que não exista grupo mais humano, cavalheiresco e amigável de epicuristas neste Pacífico inteiro.»
\page


\MyCover{MELVILLE_TAIPI_THUMB}

\page %---------------------------------------------------------|

\Hedra

\stoptext %---------------------------------------------------------|



% O limite entre a integração e a dissolução se constrói a partir de
% duas instituições, o canibalismo e a tatuagem. Ambas ganham
% do narrador tratamentos etnográficos heterodoxos em razão do
% terror que lhe inspiram e, portanto, devem ser pensadas e tra-
% tadas, antes de tudo, como engrenagens do suspense que cerca,
% mesmo em seus momentos cômicos ou bucólicos, a narrativa do
% cativeiro de Tommo até os rompantes de seu desenlace. (pag 20 pra mais)


% Havia uma leve imperfeição, contudo, em sua
% aparência: uma mancha larga de tatuagem se estendia por todo
% o seu rosto, com uma linha passando por seus olhos, fazendo
% com que parecesse usar um enorme par de óculos; e a realeza em
% óculos sugere umas ideias risíveis.

% Se me perguntassem se as belas formas de Fayaway estavam
% de todo livres da horrível mácula da tatuagem, eu seria obri-
% gado a responder que não.

% À medida que avançávamos mais ao longo da construção,
% surpreendeu-nos o aspecto de quatro ou cinco velhos medonhos,
% em cujas formas decrépitas o tempo e a tatuagem pareciam ter des-
% truído quaisquer vestígios de humanidade.


% Horrorizado com a simples ideia de quedar medonho por
% toda a vida, caso o infeliz executasse seu propósito sobre mim,
% lutei para me afastar dele, enquanto Kori-Kori, tornando-se um
% traidor, permaneceu parado e me implorou para consentir com o
% pedido ultrajante. Diante de minhas reiteradas recusas, o entusi-
% asmado artista ficou um tanto fora de si, e transido de tristeza por
% perder tão nobre oportunidade de se fazer notar em sua profissão.
% A ideia de fixar sua tatuagem em minha pele branca o en-
% cheu de todo o entusiasmo de um pintor; ele observou meu sem-
% blante fixamente repetidas vezes, e a cada novo vislumbre sentia
% que aumentava a veemência de sua ambição. (PAG 301 -- TATUAGEM NELE)


% Dentre todas, no entanto, devo destacar a bela ninfa Fayaway,
% minha favorita. Sua figura livre e flexível era a própria perfeição
% da graça e beleza femininas. Sua pele era de um vivo e uniforme
% oliva; e quando lhe mirava o brilho nas maçãs do rosto, era capaz
% de jurar que, sob o meio transparente, espreitavam os rubores
% de um leve cinabre. (136 -- Beleza de fayaway)

% Embora o mais dedicado e bem-
% -disposto criado do mundo, Kori-Kori era, ai!, uma criatura me-
% donha de se ver. Tinha seus vinte e cinco anos de idade e cerca de
% seis pés de altura, dotado de compleição robusta e bem proporcio-
% nada e do aspecto mais extraordinário. Tinha a cabeça meticulo-
% samente raspada, com exceção de dois pontos circulares, do tama-
% nho de uma moeda de dólar, próximos ao topo do crânio, onde o
% cabelo, ao qual era permitido crescer a um incrível comprimento,
% % se entrelaçava em dois proeminentes nós, que conferiam a ele aaparência de ser decorado com um par de chifres. À sua barba, ar-
% rancada pela raiz de todas as demais partes de seu rosto, cabia pen-
% der em pingentes peludos, dois dos quais lhe enfeitavam o lábio
% inferior, mesmo número que lhe caía da extremidade do queixo.
% Kori-Kori, com o objetivo de aperfeiçoar a obra da natureza,
% e talvez movido por um desejo de tornar ainda mais atraente a
% expressão de seu semblante, achou adequado embelezar o rosto
% com três largas listas longitudinais de tatuagem, que, como aque-
% las estradas rurais que seguem em linha reta desafiando quais-
% quer obstáculos, cruzavam seu órgão nasal, desciam-lhe pelas
% órbitas dos olhos e até contornavam os limites de sua boca. Cada
% uma delas se estendia por todo o rosto; uma em linha com os
% olhos, a outra cruzando as proximidades do nariz e a terceira var-
% rendo os lábios de orelha a orelha. Seu semblante, desse modo
% três vezes cingido de tatuagens, por assim dizer, sempre me lem-
% brava daqueles infelizes coitados que por vezes observei em nos-
% tálgica contemplação detrás das grades de uma janela de prisão;
% enquanto todo o corpo do meu criado selvagem, coberto por
% toda parte de representações de pássaros e peixes e uma varie-
% dade de criaturas de aparência inexplicável, sugeria-me a ideia
% de um museu pictórico da história natural ou uma cópia ilus-
% trada de A natureza animada de Goldsmith. (APARENCIA DE KORI KORI)


% O estranho não podia ter mais de vinte e cinco anos de idade
% e tinha altura ligeiramente acima da média; se fosse um único
% fio de cabelo mais alto, a simetria incomparável de suas formas
% teria sido destruída. Seus membros nus eram belamente forma-
% dos; enquanto as linhas elegantes de sua figura, juntamente com
% suas faces sem barba, o poderiam ter qualificado à distinção de
% representar a estátua de um Apolo polinésio; (BELEZA DE MARNU)

