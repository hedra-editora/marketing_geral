% MELVILLE_TAIPI_CURIOSIDADES_2.tex
% Preencher com o nome das cor ou composição RGB (ex: [r=0.862, g=0.118, b=0.118]) 
\usecolors[crayola] 			   % Paleta de cores pré-definida: wiki.contextgarden.net/Color#Pre-defined_colors

% Cores definidas pelo designer:
% MyGreen		r=0.251, g=0.678, b=0.290 % 40ad4a
% MyCyan		r=0.188, g=0.749, b=0.741 % 30bfbd
% MyRed			r=0.820, g=0.141, b=0.161 % d12429
% MyPink		r=0.980, g=0.780, b=0.761 % fac7c2
% MyGray		r=0.812, g=0.788, b=0.780 % cfc9c7
% MyOrange		r=0.980, g=0.671, b=0.290 % faab4a

% Configuração de cores
\definecolor[MyColor][x=414b47]      % ou ex: [r=0.862, g=0.118, b=0.118] % corresponde a RGB(220, 30, 30)
\definecolor[MyColorText][white]  % ou ex: [r=0.862, g=0.118, b=0.118] % corresponde a RGB(167, 169, 172)

% Classe para diagramação dos posts
\environment{marketing.env}		   

\starttext %---------------------------------------------------------|

\hyphenpenalty=10000
\exhyphenpenalty=10000

\Mensagem{O MOVIMENTO PENDULAR DE TAIPI} %Sempre usar esse header

\startMyCampaign

\hyphenpenalty=10000
\exhyphenpenalty=10000

{\bf O OLHAR DE COMPAIXÃO E O MEDO DE ASSIMILAÇÃO}

\stopMyCampaign

\page %---------------------------------------------------------| 

\hyphenpenalty=10000
\exhyphenpenalty=10000

% COLOCAR TRECHOS QUE DEMONSTREM AS DUAS TENDENCIAS QUE PENDEM O OLHAR DE MELVILLE: DESPREZO/CHOQUE (TATUAGENS, CANIBALISMO, MEDO DE ASSIMILAÇAO) VS EMPATIA, BELEZA EXOTICA  (QUEM SAO OS VERDADEIROS SELVAGENS)


Pulvinar ante, a ultricies magna {\bf TRECHO EM DESTAQUE, MAS PODE HAVER MAIS DE UM}, sempre em negrito e caixa alta. Lorem ipsum dolor sit amet,
consectetur ou {\bf ALGUM DESTAQUE} adipiscing elit. Praesent sit amet
pulvinar ante, a ultricies magna. Etiam placerat quis tellus sed ultrices.
Duis aliquet sed quam non tincidunt. Donec sit amet tempor urna. Quisque
auctor justo enim. Curabitur vel est consectetur.

\page %---------------------------------------------------------|

\MyCover{THUMB_LIVRO.pdf}

\page %---------------------------------------------------------|

\Hedra

\stoptext %---------------------------------------------------------|



% O limite entre a integração e a dissolução se constrói a partir de
% duas instituições, o canibalismo e a tatuagem. Ambas ganham
% do narrador tratamentos etnográficos heterodoxos em razão do
% terror que lhe inspiram e, portanto, devem ser pensadas e tra-
% tadas, antes de tudo, como engrenagens do suspense que cerca,
% mesmo em seus momentos cômicos ou bucólicos, a narrativa do
% cativeiro de Tommo até os rompantes de seu desenlace. (pag 20 pra mais)


% Havia uma leve imperfeição, contudo, em sua
% aparência: uma mancha larga de tatuagem se estendia por todo
% o seu rosto, com uma linha passando por seus olhos, fazendo
% com que parecesse usar um enorme par de óculos; e a realeza em
% óculos sugere umas ideias risíveis.

% Se me perguntassem se as belas formas de Fayaway estavam
% de todo livres da horrível mácula da tatuagem, eu seria obri-
% gado a responder que não.

% À medida que avançávamos mais ao longo da construção,
% surpreendeu-nos o aspecto de quatro ou cinco velhos medonhos,
% em cujas formas decrépitas o tempo e a tatuagem pareciam ter des-
% truído quaisquer vestígios de humanidade.


% Horrorizado com a simples ideia de quedar medonho por
% toda a vida, caso o infeliz executasse seu propósito sobre mim,
% lutei para me afastar dele, enquanto Kori-Kori, tornando-se um
% traidor, permaneceu parado e me implorou para consentir com o
% pedido ultrajante. Diante de minhas reiteradas recusas, o entusi-
% asmado artista ficou um tanto fora de si, e transido de tristeza por
% perder tão nobre oportunidade de se fazer notar em sua profissão.
% A ideia de fixar sua tatuagem em minha pele branca o en-
% cheu de todo o entusiasmo de um pintor; ele observou meu sem-
% blante fixamente repetidas vezes, e a cada novo vislumbre sentia
% que aumentava a veemência de sua ambição. (PAG 301 -- TATUAGEM NELE)


% Dentre todas, no entanto, devo destacar a bela ninfa Fayaway,
% minha favorita. Sua figura livre e flexível era a própria perfeição
% da graça e beleza femininas. Sua pele era de um vivo e uniforme
% oliva; e quando lhe mirava o brilho nas maçãs do rosto, era capaz
% de jurar que, sob o meio transparente, espreitavam os rubores
% de um leve cinabre. (136 -- Beleza de fayaway)