% MELVILLE_TAIPI_EDICAO.tex
% Preencher com o nome das cor ou composição RGB (ex: [r=0.862, g=0.118, b=0.118]) 
\usecolors[crayola] 			   % Paleta de cores pré-definida: wiki.contextgarden.net/Color#Pre-defined_colors

% Cores definidas pelo designer:
% MyGreen		r=0.251, g=0.678, b=0.290 % 40ad4a
% MyCyan		r=0.188, g=0.749, b=0.741 % 30bfbd
% MyRed			r=0.820, g=0.141, b=0.161 % d12429
% MyPink		r=0.980, g=0.780, b=0.761 % fac7c2
% MyGray		r=0.812, g=0.788, b=0.780 % cfc9c7
% MyOrange		r=0.980, g=0.671, b=0.290 % faab4a

% Configuração de cores
\definecolor[MyColor][x=414b47]      % ou ex: [r=0.862, g=0.118, b=0.118] % corresponde a RGB(220, 30, 30)
\definecolor[MyColorText][white]  % ou ex: [r=0.862, g=0.118, b=0.118] % corresponde a RGB(167, 169, 172)

% Classe para diagramação dos posts
\environment{marketing.env}		   

\starttext %---------------------------------------------------------|

\Mensagem{ENTRE FATO E FICÇÃO}

\startMyCampaign

\hyphenpenalty=10000
\exhyphenpenalty=10000

{\bf COMO MELVILLE ESCREVEU TAIPI?}

\stopMyCampaign

%\vfill\scale[lines=1.5]{\MyStar[MyColorText][none]}

\page %---------------------------------------------------------| 

\MyCover{MELVILLE_TAIPI_THUMB}

\page %---------------------------------------------------------| 

\hyphenpenalty=10000
\exhyphenpenalty=10000

O conteúdo de {\bf TAIPI} é uma mistura. Combinando sua experiência biográfica com relatos de viagens de seus contemporâneos, Melville concebe uma narrativa que caminha no limite entre {\bf ROMANCE E ETNOGRAFIA}.

\page

A vivência pessoal que Melville adapta literariamente nesse romance é seu período como {\bf CATIVO} nas Ilhas Marquesas. Como seu protagonista, o escritor desertou um baleeiro e se embrenhou nos vales desconhecidos de Nuku Hiva, ilha habitada por povos temidos pelos colonizadores que lá aportavam.

\page

 Melville, apesar das precauções, acaba como prisioneiro do povo mais mal-afamado da região --- os taipi, conhecidos como {\bf GUERREIROS CANIBAIS IMPIEDOSOS}. Será que é isso mesmo?

\page

\Hedra

\stoptext %---------------------------------------------------------|

%FALAR DA EXPERIENCIA DE MELVILLE; MATIZAR A IDEIA DO CANIBALISMO DOS TAIPIS; DIZER QUE DAI QUE VEIO O CONTEUDO DO LIVRO TAIPI


