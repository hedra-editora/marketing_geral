% MELVILLE_TAIPI_EDICAO.tex
% Preencher com o nome das cor ou composição RGB (ex: [r=0.862, g=0.118, b=0.118]) 
\usecolors[crayola] 			   % Paleta de cores pré-definida: wiki.contextgarden.net/Color#Pre-defined_colors

% Cores definidas pelo designer:
% MyGreen		r=0.251, g=0.678, b=0.290 % 40ad4a
% MyCyan		r=0.188, g=0.749, b=0.741 % 30bfbd
% MyRed			r=0.820, g=0.141, b=0.161 % d12429
% MyPink		r=0.980, g=0.780, b=0.761 % fac7c2
% MyGray		r=0.812, g=0.788, b=0.780 % cfc9c7
% MyOrange		r=0.980, g=0.671, b=0.290 % faab4a

% Configuração de cores
\definecolor[MyColor][x=414b47]      % ou ex: [r=0.862, g=0.118, b=0.118] % corresponde a RGB(220, 30, 30)
\definecolor[MyColorText][white]  % ou ex: [r=0.862, g=0.118, b=0.118] % corresponde a RGB(167, 169, 172)

% Classe para diagramação dos posts
\environment{marketing.env}		   

\starttext %---------------------------------------------------------|

\Mensagem{MANCHETE CATIVANTE}

\startMyCampaign

\hyphenpenalty=10000
\exhyphenpenalty=10000

{\bf MELVILLE FOI PRISIONEIRO DE UM POVO CANIBAL?}

\stopMyCampaign

%\vfill\scale[lines=1.5]{\MyStar[MyColorText][none]}

\page %---------------------------------------------------------| 

\MyCover{MELVILLE_TAIPI_THUMB}

\page %---------------------------------------------------------| 

\hyphenpenalty=10000
\exhyphenpenalty=10000

O romance {\bf TAIPI} tem como base a experiência biográfica de Melville enquanto prisioneiro de um povo nativo das Ilhas Marquesas.

\page

Após {\bf DESERTAREM} o baleeiro cuja tripulação integravam, cientes da má fama dos locais, Melville
e seu companheiro Toby seguem cautelosamente. 

\page

Apesar de evitarem ao máximo aproximar-se do vale que servia de residência aos famigerados taipis, conhecidos pelos colonizadores como {\bf GUERREIROS CANIBAIS IMPIEDOSOS}, não só eles cruzam com os taipi, como estes os mantêm como {\bf PRISIONEIROS} por cerca de um mês.

\page

Contudo, tudo transcorre muito diferentemente do que Melville esperava: os nativos cuidam de seus ferimentos, alimentam-nos e até mesmo lhes concedem um servo. Eram estes os {\bf VIOLENTOS CANIBAIS}?

\page

Sem desfazer-se completamente do medo, o tratamento que lhe foi dado faz Melville {\bf QUESTIONAR} a imagem que os colonizadores cultivavam desse povo.
Muito a frente de seu tempo, Melville se pergunta {\bf QUEM SERIAM OS VERDADEIROS SELVAGENS}: os colonizadores ou os nativos? 

\page

«A habilidade demoníaca que ostentamos na invenção de toda
a sorte de {\bf MECANISMOS DE MORTE}, a vingança com que conduzimos
nossas guerras e a infelicidade e a desolação que se seguem em sua
esteira, são o bastante para identificar o homem branco civilizado
como o animal mais feroz da face da terra.»
\vfill
\scale[factor=10]{\Seta\,Herman Melville}


\page %---------------------------------------------------------|

\hyphenpenalty=10000
\exhyphenpenalty=10000

% «Pulvinar ante, a ultricies magna {\bf TRECHO EM DESTAQUE, MAS PODE HAVER MAIS DE UM}, sempre em negrito e caixa alta. Aqui entra um trecho cativante do texto.»

% {\vfill\scale[factor=5]{{\bf Nome de quem escreveu a análise}, qualificação de}\setupinterlinespace[line=1.5ex]\scale[factor=5]{XPTO professora na Universidade de Nova York. Lembre}\setupinterlinespace[line=1.5ex]\scale[factor=5]{de quebrar as linhas nos códigos.}}

\page %---------------------------------------------------------|

\Hedra

\stoptext %---------------------------------------------------------|

%FALAR DA EXPERIENCIA DE MELVILLE; MATIZAR A IDEIA DO CANIBALISMO DOS TAIPIS; DIZER QUE DAI QUE VEIO O CONTEUDO DO LIVRO TAIPI