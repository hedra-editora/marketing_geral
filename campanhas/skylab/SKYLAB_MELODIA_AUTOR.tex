% AUTOR_LIVRO_AUTOR.tex
% Preencher com o nome das cor ou composição RGB (ex: [r=0.862, g=0.118, b=0.118]) 
\usecolors[crayola]                            % Paleta de cores pré-definida: wiki.contextgarden.net/Color#Pre-defined_colors

% Cores definidas pelo designer:
% MyGreen                r=0.251, g=0.678, b=0.290 % 40ad4a
% MyCyan                r=0.188, g=0.749, b=0.741 % 30bfbd
% MyRed                        r=0.820, g=0.141, b=0.161 % d12429
% MyPink                r=0.980, g=0.780, b=0.761 % fac7c2
% MyGray                r=0.812, g=0.788, b=0.780 % cfc9c7
% MyOrange                r=0.980, g=0.671, b=0.290 % faab4a

% Configuração de cores
\definecolor[MyColor][BrickRed]      % ou ex: [r=0.862, g=0.118, b=0.118] % corresponde a RGB(220, 30, 30)
\definecolor[MyColorText][white]  % ou ex: [r=0.862, g=0.118, b=0.118] % corresponde a RGB(167, 169, 172)

% Classe para diagramação dos posts
\environment{marketing.env}                   

% Cabeço e rodapé: Informações (caso queira trocar alguma coisa)
                 \def\MensagemSaibaMais  {SAIBA MAIS:}
                 \def\MensagemSite                {HEDRA.COM.BR}
                 \def\MensagemLink       {LINK NA BIO}

\starttext %--------------------------------------------------------|

\Mensagem{MANCHETE CATIVANTE}

\hyphenpenalty=10000
\exhyphenpenalty=10000

%\startMyCampaign

\MyPicture{SKYLAB_FOTO}

%\stopMyCampaign

\vfill\scale[factor=6]{\Seta\,ROGÉRIO SKYLAB}

\page %----------------------------------------------------------|

\hyphenpenalty=10000
\exhyphenpenalty=10000

Compositor consagrado pela internet, crítico da canção popular, entrevistador de {\cap TV} --- {\bf ROGÉRIO SKYLAB} é múltiplo. 

\page %----------------------------------------------------------|

Carioca da gema, Skylab tem 67 anos. Cursou letras e filosofia na {\cap UFRJ}. Trabalhou por quase 30 anos no Banco do Brasil, ocupação que lhe garantiu o sustento e, depois de aposentado, a {\bf DEDICAÇÃO EXCLUSIVA À CANÇÃO}. 

\page

Desde 1992, Skylab já lançou mais de 20 álbuns, transitando pelos mais diferentes gêneros, sempre com {\bf LETRAS PROVOCADORAS}.

\page

Nos ensaios de {\bf A MELODIA TRÁGICA}, Skylab analisa as obras de Torquato Neto, José Agrippino de Paula, Fausto Fawcett e Arrigo Barnabé, entre outros, e se consagra também como crítico da canção popular e da literatura. 

\page

\MyCover{SKYLAB_MELODIA_THUMB}

\page %----------------------------------------------------------|

\Hedra

\stoptext %---------------------------------------------------------|
