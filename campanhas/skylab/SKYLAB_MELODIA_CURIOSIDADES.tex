% SKYLAB_MELODIA_CURIOSIDADES.tex
% Preencher com o nome das cor ou composição RGB (ex: [r=0.862, g=0.118, b=0.118]) 
\usecolors[crayola] 			   % Paleta de cores pré-definida: wiki.contextgarden.net/Color#Pre-defined_colors

% Cores definidas pelo designer:
% MyGreen		r=0.251, g=0.678, b=0.290 % 40ad4a
% MyCyan		r=0.188, g=0.749, b=0.741 % 30bfbd
% MyRed			r=0.820, g=0.141, b=0.161 % d12429
% MyPink		r=0.980, g=0.780, b=0.761 % fac7c2
% MyGray		r=0.812, g=0.788, b=0.780 % cfc9c7
% MyOrange		r=0.980, g=0.671, b=0.290 % faab4a

% Configuração de cores
\definecolor[MyColor][BrickRed]      % ou ex: [r=0.862, g=0.118, b=0.118] % corresponde a RGB(220, 30, 30)
\definecolor[MyColorText][white]  % ou ex: [r=0.862, g=0.118, b=0.118] % corresponde a RGB(167, 169, 172)

% Classe para diagramação dos posts
\environment{marketing.env}		   

\starttext %---------------------------------------------------------|

\hyphenpenalty=10000
\exhyphenpenalty=10000

\Mensagem{EM CONTEXTO} %Sempre usar esse header

\startMyCampaign

\hyphenpenalty=10000
\exhyphenpenalty=10000

{\bf TORQUATO NETO 
SEGUNDO 
ROGÉRIO SKYLAB} %Aqui a manchete pode ser mais longa

\stopMyCampaign

\page %---------------------------------------------------------| 

\MyPicture{SKYLAB_MELODIA_TORQUATO}
\page

\hyphenpenalty=10000
\exhyphenpenalty=10000

«Em busca de um Torquato \\Tropicalista» é o primeiro e mais longo ensaio de {\bf A MELODIA TRÁGICA}, de Rogério Skylab
\page %---------------------------------------------------------|

A obra do poeta piauiense Torquato Neto foi consagrada pela \\{\bf TROPICÁLIA}: são dele os versos de «Geleia Geral», parceria com Gilberto Gil, e «A coisa mais linda que existe», também com Gil, interpretada por Gal Costa. Também as letras de «Let’s play that», de Jards Macalé, e  «Go back», dos Titãs, são de autoria Torquato.

\page

Segundo Rogério Skylab, o poeta é um «{\bf MITO CULT}, representando tudo que é marginal, desde os anos 60».

\page

 Torquato se distanciava dos nacionalistas populistas de esquerda e direita, repudiava a sociedade de consumo e ligava-se à Antropofagia e, evidentemente, ao Tropicalismo. Mas, para Skylab, a poesia de Torquato vai além do próprio Tropicalismo por {\bf UM PROCESSO CONSTANTE DE FUGA}.    

\page

\MyCover{SKYLAB_MELODIA_THUMB}

\page %---------------------------------------------------------|

\Hedra

\stoptext %---------------------------------------------------------|


 

        
  

