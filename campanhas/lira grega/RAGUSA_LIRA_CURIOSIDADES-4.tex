% AUTOR_LIVRO_CURIOSIDADES.tex
% Preencher com o nome das cor ou composição RGB (ex: [r=0.862, g=0.118, b=0.118]) 
\usecolors[crayola] 			   % Paleta de cores pré-definida: wiki.contextgarden.net/Color#Pre-defined_colors

% Cores definidas pelo designer:
% MyGreen		r=0.251, g=0.678, b=0.290 % 40ad4a
% MyCyan		r=0.188, g=0.749, b=0.741 % 30bfbd
% MyRed			r=0.820, g=0.141, b=0.161 % d12429
% MyPink		r=0.980, g=0.780, b=0.761 % fac7c2
% MyGray		r=0.812, g=0.788, b=0.780 % cfc9c7
% MyOrange		r=0.980, g=0.671, b=0.290 % faab4a

% Configuração de cores
\definecolor[MyColor][x=c0e016]      % ou ex: [r=0.862, g=0.118, b=0.118] % corresponde a RGB(220, 30, 30)
\definecolor[MyColorText][black]     % ou ex: [r=0.862, g=0.118, b=0.118] % corresponde a RGB(167, 169, 172)

% Classe para diagramação dos posts
\environment{marketing.env}		   

\starttext %---------------------------------------------------------|

\hyphenpenalty=10000
\exhyphenpenalty=10000

\Mensagem{EM CONTEXTO} %Sempre usar esse header

\startMyCampaign

\hyphenpenalty=10000
\exhyphenpenalty=10000

COMO A MÉLICA ARCAICA CHEGOU ATÉ NÓS?

\stopMyCampaign

\page %---------------------------------------------------------| 

\hyphenpenalty=10000
\exhyphenpenalty=10000

A circulação da mélica na era arcaica e clássica se deu sobretudo oralmente, em repetições de performances por amadores ou profissionais. Deu-se também por inscrições celebrativas aos poetas,
muitos deles cultuados nas póleis, Grécia afora, ou a eventos
com os quais se relacionavam as canções; e ainda por cópias.

\page %---------------------------------------------------------|


Tudo isso contribuiu para a sobrevivência dos textos até que
na famosa Biblioteca de Alexandria fossem copiados, compilados, editados, pela primeira vez.

\page

A mélica, então, passa de poesia musical, social e performática para texto, privada do que lhe fora essencial nas eras arcaica e clássica, isto é, daquilo que definia sua natureza: a performance. Com isso, perdeu-se a relação direta que a comunicação oral estabelece entre destinatário (audiência) e remetente (poeta e/ ou performer), que está na base das distinções entre a poesia moderna e a antiga, pois se nesta “o eu privado” está “inserido numa moldura social”, naquela é
“narcisista” (Most, 1982, p. 97); nem há na poesia antiga, à diferença do que se passa na moderna, o embate entre a voz poética, a sociedade e o mundo ao seu redor, pois, em larga medida, aquela poesia, em sua composição genérica — pautada prevalentemente pelas práticas tradicionais associadas às espécies de poesia —, adere “ao ‘paladar social’, às regras estabelecidas e às expectativas suscitadas por elas no público” (ibid.)


\page


\MyCover{THUMB_LIVRO.pdf}

\page %---------------------------------------------------------|

\Hedra

\stoptext %---------------------------------------------------------|

 Considerando que a mélica
arcaica e tardo-arcaica teve seu apogeu entre os anos aproxima-
dos de 620 — em que Álcman esteve ativo — e 446 a. C. — em que
se situa a última ode datável de Píndaro —, e que Aristófanes
de Bizâncio, editor dos mélicos, morreu em c. 180 a. C., concluí-
mos que a distância entre aquela poesia e sua edição não é nada
desprezível e atinge a dimensão cultural, na medida em que, na
era helenística, tinham mudado profundamente “as condições
fundamentais de produção poética, assim como a relação entre
o poeta e sua audiência” (Clay, 1998, p. 28). O Egito dos Ptolo-
meus, acentuadamente filo-heleno, tinha o grego como espécie
de língua franca, e produziu escribas, os copistas profissionais,
que, junto a eruditos, trabalharam para guardar em Alexandria
a poesia dos antigos gregos, então “lida como literatura pura e
simples” (Gentili, 1990, p. 37), embora ainda recitada.

