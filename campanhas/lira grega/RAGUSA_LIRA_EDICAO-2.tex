% AUTOR_LIVRO_EDICAO.tex
% Preencher com o nome das cor ou composição RGB (ex: [r=0.862, g=0.118, b=0.118]) 
\usecolors[crayola] 			   % Paleta de cores pré-definida: wiki.contextgarden.net/Color#Pre-defined_colors

% Cores definidas pelo designer:
% MyGreen		r=0.251, g=0.678, b=0.290 % 40ad4a
% MyCyan		r=0.188, g=0.749, b=0.741 % 30bfbd
% MyRed			r=0.820, g=0.141, b=0.161 % d12429
% MyPink		r=0.980, g=0.780, b=0.761 % fac7c2
% MyGray		r=0.812, g=0.788, b=0.780 % cfc9c7
% MyOrange		r=0.980, g=0.671, b=0.290 % faab4a

% Configuração de cores
\definecolor[MyColor][x=2a3e92]      % ou ex: [r=0.862, g=0.118, b=0.118] % corresponde a RGB(220, 30, 30)
\definecolor[MyColorText][white]     

% Classe para diagramação dos posts
\environment{marketing.env}		   

\starttext %---------------------------------------------------------|

\Mensagem{POR DENTRO DA EDIÇÃO}

\startMyCampaign

\hyphenpenalty=10000
\exhyphenpenalty=10000

{\bf 
A TRADUÇÃO DOS FRAGMENTOS DE LIRA GREGA}

\stopMyCampaign

%\vfill\scale[lines=1.5]{\MyStar[MyColorText][none]}

\page %---------------------------------------------------------| 

\MyCover{RAGUSA_LIRA_THUMB}

\page %---------------------------------------------------------| 

\hyphenpenalty=10000
\exhyphenpenalty=10000

«Apresentar, contextualizar, chamar a atenção para conceitos importantes, para dados estilísticos e sonoros, explicar referências culturais sobre mitos, cultos e outros temas» 
\vfill\hfill →
\page

«oferecer elementos que auxiliem e {\bf ENRIQUEÇAM AS LEITURAS DOS FRAGMENTOS} remanescentes de canções que cruzaram séculos até nós — eis a tarefa empreendida em parceria com a Hedra desde a tradução de Safo e da primeira edição de Lira grega, e agora renovada»
\vfill\hfill →

\page %---------------------------------------------------------|

\hyphenpenalty=10000
\exhyphenpenalty=10000
«O molde que para ambos os trabalhos concebi configura-se como {\bf INÉDITO EM NOSSO PAÍS}, tanto pelas introduções às obras, quanto pelos paratextos aos fragmentos e visão em conjunto de um relevante gênero poético da Grécia arcaica, a {\bf MÉLICA}.» 

{\vfill\scale[factor=5]{{\bf Giuliana Ragusa}, na «Nota à segunda edição»}\setupinterlinespace[line=1.5ex]\scale[factor=5]{de {\it Lira grega}.}}

\page %---------------------------------------------------------|

\Hedra

\stoptext %---------------------------------------------------------|
