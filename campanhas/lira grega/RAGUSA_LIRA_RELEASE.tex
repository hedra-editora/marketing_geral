\setuppapersize[A4]
\usecolors[crayola]
\setupbackgrounds[paper][background=color,backgroundcolor=Almond]
\mainlanguage[pt]	

	\definefontfeature
		[default]
		[default]
		[expansion=quality,protrusion=quality,onum=yes]
	\setupalign[fullhz,hanging]
	\definefontfamily [mainface] [sf] [Formular]
	\setupbodyfont[mainface,11pt]

% Indenting [4.4 cont-enp.p.65]
			\setupindenting[yes, 3ex]  % none small medium big next first dimension
			\indenting[next]           % never not no yes always first next
			
			% [cont-ent.p.76]
			\setupspacing[broad]  %broad packed
			% O tamanho do espaço entre o ponto final e o começo de uma sentença. 


\startsetups[grid][mypenalties]
    \setdefaultpenalties
    \setpenalties\widowpenalties{2}{10000}
    \setpenalties\clubpenalties {2}{10000}
\stopsetups

\setuppagenumbering
  [location={}]            % Estilo dos números de páginat

\setuphead[subject]
[style=bfb]		

\setuplayout[
          location=middle,
          %
          leftedge=0mm,
          leftedgedistance=0mm,
          leftmargin=20mm,
          leftmargindistance=0mm,
          width=100mm,
          rightmargindistance=0mm,
          rightmargin=20mm,
          rightedgedistance=0mm,
          rightedge=0mm,
          backspace=20mm,
          %
          top=21mm,
          topdistance=0mm,
          header=0mm,
          headerdistance=0mm,
          height=250mm,
          footerdistance=0mm,
          footer=0mm,
          bottomdistance=0mm,
          bottom=21mm,
          topspace=21mm,
        setups=mypenalties,
]

\setupalign[right]

\starttext
{\bfc Lira grega}\\
{\tfb\it  Antologia de poesia arcaica}

\blank[0.3cm]

\noindent{\tfb Giuliana Ragusa (org.)}

\blank[big]

\hyphenpenalty=10000
\exhyphenpenalty=10000


\noindent {\it Com tradução direta do grego antigo,}  Lira grega {\it verte para o português fragmentos selecionados dos maiores poetas da mélica grega.} 



\blank[1cm]

\inoutermargin[width=60mm,hoffset=1cm,style=tfx,,voffset=2cm]{
\externalfigure[RAGUSA_LIRA_THUMB][width=50mm]
}


\inoutermargin[width=70mm,hoffset=1.1cm,voffset=2.8cm,style=tfx]
{\noindent{\bf Título} {\it Lira grega: antologia de poesia arcaica}\\
% {\bf Autor} Júlia Lopes de Almeida\\
{\bf Organização e tradução} Giuliana Ragusa\\
% {\bf Aparatos} Alfredo Sousa, Norma Telles e Rafael Balseiro Zin\\
{\bf Editora} Hedra\\
{\bf ISBN} 978-85-7715-987-1\\
{\bf Pág.}  440\\
{\bf Preço} 99,90 
}


\inoutermargin[width=70mm,hoffset=1.1cm,voffset=7.5cm,style=tfx]
{{\bf Sobre a organizadora}  {\it Giuliana Ragusa} é\\ professora livre-docente de Língua e\\ Literatura Grega na Faculdade de Filosofia, Letras e Ciências Humanas da Universidade de São Paulo ({\cap USP}).  Possui mestrado e\\ doutorado em Letras Clássicas, além de ter realizado um pós-doutorado na University of Wisconsin, Madison, com bolsa da {\cap FAPESP}. Desde 2004, ela atua no ensino e pesquisa na área de Estudos Clássicos, sendo uma\\ referência no estudo da lírica grega arcaica.
Entre suas obras estão {\it Fragmentos de uma deusa: a representação de Afrodite na\\ lírica de Safo}, que recebeu o segundo lugar no Prêmio Jabuti de 2006, e {\it Lira, mito e\\ erotismo: Afrodite na poesia mélica grega arcaica}. Além de livros, Ragusa publica\\ artigos em periódicos especializados e\\ desenvolve projetos de pesquisa sobre a\\ mélica grega arcaica, integrando o programa de pós-graduação em Letras Clássicas da {\cap usp}.}


\inoutermargin[width=70mm,hoffset=-10cm,voffset=19.7cm,style=tfx]
{\definefontfamily [Times] [rm] [Times New Roman]
                   [tf=file:TimesLTStd-Roman.otf]

\setcharacterkerning[reset] \switchtobodyfont[Times,50pt] hedra \hfill \mbox{}
}


\noindent{\tfb Uma viagem à origem da poesia ocidental}

\blank[.5cm]

\hyphenpenalty=10000
\exhyphenpenalty=10000


Quase uma década depois da sua primeira publicação, a nova edição bilíngue de {\it Lira grega} revisa e expande a obra.
A antologia reúne fragmentos dos nove poetas mais influentes da mélica grega arcaica --- como Safo, Alceu, Píndaro e Baquílides ---, traduzidos diretamente do grego antigo por Giuliana Ragusa.


Com introdução crítica, notas explicativas e vasta bibliografia de apoio, a edição contribui para que a mélica, gênero poético performático associado à música e à dança, ganhe uma interpretação cuidadosa e acessível, conectando o leitor contemporâneo às vivências sociais e rituais da Grécia antiga.

A articulação dos poemas a referências culturais e traços estilísticos enriquece a leitura. E a abordagem inovadora da antologia, que inclui uma visão abrangente da mélica, reflete o compromisso de sua organizadora em difundir a poesia grega antiga de forma consistente e criteriosa.  

\setuplayout[
          location=middle,
          %
          leftedge=0mm,
          leftedgedistance=0mm,
          leftmargin=20mm,
          leftmargindistance=0mm,
          width=160mm,
          rightmargindistance=0mm,
          rightmargin=20mm,
          rightedgedistance=0mm,
          rightedge=0mm,
          backspace=20mm,
          %
          top=21mm,
          topdistance=0mm,
          header=0mm,
          headerdistance=0mm,
          height=250mm,
          footerdistance=0mm,
          footer=0mm,
          bottomdistance=0mm,
          bottom=21mm,
          topspace=21mm,
        setups=mypenalties,
]

% \subject{Sobre a autora}

%  {\it Júlia Lopes de Almeida}, nascida no Rio de Janeiro em 1862, destacou-se como um fenômeno literário, escrevendo romances, contos, peças teatrais e crônicas que capturaram a Belle Époque carioca. Participou ativamente do meio literário e foi uma das idealizadoras da Academia Brasileira de Letras, da qual foi excluída por ser mulher. Defensora da emancipação feminina, criticou a educação restrita às mulheres e incentivou a independência financeira, deixando um legado que foi injustamente esquecido ao longo do tempo.


\stoptext
