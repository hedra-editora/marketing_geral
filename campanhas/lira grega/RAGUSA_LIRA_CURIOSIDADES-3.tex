% AUTOR_LIVRO_CURIOSIDADES.tex
% Preencher com o nome das cor ou composição RGB (ex: [r=0.862, g=0.118, b=0.118]) 
\usecolors[crayola] 			   % Paleta de cores pré-definida: wiki.contextgarden.net/Color#Pre-defined_colors

% Cores definidas pelo designer:
% MyGreen		r=0.251, g=0.678, b=0.290 % 40ad4a
% MyCyan		r=0.188, g=0.749, b=0.741 % 30bfbd
% MyRed			r=0.820, g=0.141, b=0.161 % d12429
% MyPink		r=0.980, g=0.780, b=0.761 % fac7c2
% MyGray		r=0.812, g=0.788, b=0.780 % cfc9c7
% MyOrange		r=0.980, g=0.671, b=0.290 % faab4a

% Configuração de cores
\definecolor[MyColor][x=2a3e92]      % ou ex: [r=0.862, g=0.118, b=0.118] % corresponde a RGB(220, 30, 30)
\definecolor[MyColorText][white]     % ou ex: [r=0.862, g=0.118, b=0.118] % corresponde a RGB(167, 169, 172)

% Classe para diagramação dos posts
\environment{marketing.env}		   

\starttext %---------------------------------------------------------|

\hyphenpenalty=10000
\exhyphenpenalty=10000

\Mensagem{EM CONTEXTO} %Sempre usar esse header

\startMyCampaign

\hyphenpenalty=10000
\exhyphenpenalty=10000

QUEM SÃO OS POETAS DA {\bf LIRA GREGA}?
\stopMyCampaign

\page %---------------------------------------------------------| 

\hyphenpenalty=10000
\exhyphenpenalty=10000

{\bfc ÁLCMAN} é o primeiro poeta da mélica arcaica do qual há um corpus consistente. Há dúvidas quanto à origem espartana do poeta, mas o fato é que suas canções foram compostas no dialeto lacônio da região, e lá ele exerceu seu canto.
Sua mélica insere-se plenamente no universo histórico, mítico e cultual de Esparta

\page %---------------------------------------------------------|

{\bfc ALCEU} foi um guerreiro e poeta mélico nascido na aristocracia da proeminente cidade lésbia Mitilene. Sua produção em dialeto lésbio-eólico que é o da região foi volumosa, e em meio aos fragmentos acha-se grande variedade temática e formal.

\page

{\bfc ESTESÍCORO}, como os demais poetas arcaicos, tem uma biografia bastante obscura. 
A sua poesia é narrativa, de conteúdo e dicção épicos, mas feita em métrica mélica, o que lhe confere singularidade, se comparada à poesia dos demais mélicos conhecidos.

\page

{\bfc ÍBICO} é encarado tanto como monodista quanto poeta coral. Sua proximidade com seu contemporâneo Anacreonte é com frequência assinalada, pois conviveram na Samos de Polícrates e partilharam suas canções da temática
erótica.

\page

{\bfc PÍNDARO} foi contemporâneo de Baquílides e de Simônides, com os quais compõe a trinca final
de grandes poetas mélicos. Entre a mélica arcaica e tardo-arcaica, a poesia jâmbica e a elegíaca, os epinícios de Píndaro compõem o maior corpus preservado. A poesia de Píndaro mostra a
proeminência da ética aristocrática, de cuja tradição ele se
afirma herdeiro.

\page

\MyCover{RAGUSA_LIRA_THUMB}

\page %---------------------------------------------------------|

\Hedra

\stoptext %---------------------------------------------------------|

