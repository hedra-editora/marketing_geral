% AUTOR_LIVRO_CURIOSIDADES.tex
% Preencher com o nome das cor ou composição RGB (ex: [r=0.862, g=0.118, b=0.118]) 
\usecolors[crayola] 			   % Paleta de cores pré-definida: wiki.contextgarden.net/Color#Pre-defined_colors

% Cores definidas pelo designer:
% MyGreen		r=0.251, g=0.678, b=0.290 % 40ad4a
% MyCyan		r=0.188, g=0.749, b=0.741 % 30bfbd
% MyRed			r=0.820, g=0.141, b=0.161 % d12429
% MyPink		r=0.980, g=0.780, b=0.761 % fac7c2
% MyGray		r=0.812, g=0.788, b=0.780 % cfc9c7
% MyOrange		r=0.980, g=0.671, b=0.290 % faab4a

% Configuração de cores
\definecolor[MyColor][x=2a3e92]      % ou ex: [r=0.862, g=0.118, b=0.118] % corresponde a RGB(220, 30, 30)
\definecolor[MyColorText][white]     % ou ex: [r=0.862, g=0.118, b=0.118] % corresponde a RGB(167, 169, 172)

% Classe para diagramação dos posts
\environment{marketing.env}		   

\starttext %---------------------------------------------------------|

\hyphenpenalty=10000
\exhyphenpenalty=10000

\Mensagem{EM CONTEXTO} %Sempre usar esse header

\startMyCampaign

\hyphenpenalty=10000
\exhyphenpenalty=10000

O QUE É A {\bf POESIA MÉLICA?}
\stopMyCampaign

\page %---------------------------------------------------------| 

\hyphenpenalty=10000
\exhyphenpenalty=10000

«Mélica» é apenas uma das formas usada para designar o gênero poético que teve seu auge na {\bf GRÉCIA ARCAICA}, entre 800 e 480 a.C. 

\page
Essa forma de arte, a partir da era helenística, passou a ser chamada de «lírica», evocando a lira, instrumento essencial para a expressão deste gênero em que a {\bf MÚSICA E POESIA SÃO INDISSOCIÁVEIS}.

\page %---------------------------------------------------------|

 A poesia mélica foi criada para performances, tanto em coro quanto solo. Essas apresentações contavam não apenas com música, mas também com {\bf DANÇA} e ocorriam em simpósios, eventos sociais nas casas de aristocratas, ou em festivais públicos organizados pelas {\it pólis}, momentos essenciais para a {\bf VIDA COMUNITÁRIA}, celebrando rituais e conquistas importantes.

\page

A mélica, portanto, se baseia na performance. Combinando música e poesia, é uma espécie de canção que {\bf CUMPRE UMA FUNÇÃO CENTRAL NA VIDA PRÁTICA DAS COMUNIDADES}.

\page

\MyCover{RAGUSA_LIRA_THUMB}

\page %---------------------------------------------------------|

\Hedra

\stoptext %---------------------------------------------------------|

