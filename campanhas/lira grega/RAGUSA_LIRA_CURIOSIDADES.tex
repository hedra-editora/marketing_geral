% AUTOR_LIVRO_CURIOSIDADES.tex
% Preencher com o nome das cor ou composição RGB (ex: [r=0.862, g=0.118, b=0.118]) 
\usecolors[crayola] 			   % Paleta de cores pré-definida: wiki.contextgarden.net/Color#Pre-defined_colors

% Cores definidas pelo designer:
% MyGreen		r=0.251, g=0.678, b=0.290 % 40ad4a
% MyCyan		r=0.188, g=0.749, b=0.741 % 30bfbd
% MyRed			r=0.820, g=0.141, b=0.161 % d12429
% MyPink		r=0.980, g=0.780, b=0.761 % fac7c2
% MyGray		r=0.812, g=0.788, b=0.780 % cfc9c7
% MyOrange		r=0.980, g=0.671, b=0.290 % faab4a

% Configuração de cores
\definecolor[MyColor][x=2a3e92]      % ou ex: [r=0.862, g=0.118, b=0.118] % corresponde a RGB(220, 30, 30)
\definecolor[MyColorText][white]     % ou ex: [r=0.862, g=0.118, b=0.118] % corresponde a RGB(167, 169, 172)

% Classe para diagramação dos posts
\environment{marketing.env}		   

\starttext %---------------------------------------------------------|

\hyphenpenalty=10000
\exhyphenpenalty=10000

\Mensagem{EM CONTEXTO} %Sempre usar esse header

\startMyCampaign

\hyphenpenalty=10000
\exhyphenpenalty=10000

O ESSENCIAL SOBRE 
{\bf SAFO DE LESBOS}
\stopMyCampaign

\page %---------------------------------------------------------| 

\hyphenpenalty=10000
\exhyphenpenalty=10000

{\bf SAFO}, poeta lésbia de Mitilene, contemporânea de Alceu e, como ele, de família aristocrata, é a única poeta da Grécia arcaica da qual temos um corpus preservado.

\page %---------------------------------------------------------|

«As composições de Safo foram {\bf OBJETO DE PERFORMANCES} para audiências em Mitilene, como é próprio da poesia na “cultura da canção”.» 

\vfill\hfill →

\page

«São os seus fragmentos permeados de nomes femininos, formando um grupo que inclui a persona, e que está no centro das mais diversas leituras, das mais distintas {\bf HIPÓTESES INCONSISTENTES}.»

\vfill\hfill →

\page

«Safo seduz sua audiência com uma {\bf ILUSÃO DE INTIMIDADE E CUMPLICIDADE} retoricamente construídas. E também a {\bf DIMENSÃO ERÓTICA}, própria às relações internas ao coro que Safo liderava e acompanhava nas performances, como cabia ao poeta coral.»

\vfill{\scale[factor=6]{\Seta\,{\bf Lira grega}, Giuliana Ragusa.}}

\page

\MyCover{RAGUSA_LIRA_THUMB}

\page %---------------------------------------------------------|

\Hedra

\stoptext %---------------------------------------------------------|
