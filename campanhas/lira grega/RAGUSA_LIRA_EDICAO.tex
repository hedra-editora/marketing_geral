% AUTOR_LIVRO_EDICAO.tex
% Preencher com o nome das cor ou composição RGB (ex: [r=0.862, g=0.118, b=0.118]) 
\usecolors[crayola] 			   % Paleta de cores pré-definida: wiki.contextgarden.net/Color#Pre-defined_colors

% Cores definidas pelo designer:
% MyGreen		r=0.251, g=0.678, b=0.290 % 40ad4a
% MyCyan		r=0.188, g=0.749, b=0.741 % 30bfbd
% MyRed			r=0.820, g=0.141, b=0.161 % d12429
% MyPink		r=0.980, g=0.780, b=0.761 % fac7c2
% MyGray		r=0.812, g=0.788, b=0.780 % cfc9c7
% MyOrange		r=0.980, g=0.671, b=0.290 % faab4a

% Configuração de cores
\definecolor[MyColor][x=2a3e92]      % ou ex: [r=0.862, g=0.118, b=0.118] % corresponde a RGB(220, 30, 30)
\definecolor[MyColorText][white]     

% Classe para diagramação dos posts
\environment{marketing.env}		   

\starttext %---------------------------------------------------------|

\Mensagem{POR DENTRO DA EDIÇÃO}

\startMyCampaign

\hyphenpenalty=10000
\exhyphenpenalty=10000

{\bf A POESIA MÉLICA COMO CANÇÃO}

\stopMyCampaign

%\vfill\scale[lines=1.5]{\MyStar[MyColorText][none]}

\page %---------------------------------------------------------| 

\MyCover{RAGUSA_LIRA_THUMB}

\page %---------------------------------------------------------| 

\hyphenpenalty=10000
\exhyphenpenalty=10000

«Evidencia-se que a poesia aqui traduzida, ao contrário do que ocorre hoje, não se destinava à leitura — muito menos a solitária e silenciosa; ela não existia na forma do texto e não era aquilo que o nome “poesia” identifica em nosso mundo, mas {\bf ALGO MAIS PRÓXIMO À “CANÇÃO”}.»

\vfill\hfill →

\page %---------------------------------------------------------|

\hyphenpenalty=10000
\exhyphenpenalty=10000

«Ainda {\bf NÃO SE TINHAM\\ DIVORCIADO VERSO E MÚSICA}, e tampouco tinha a música autonomia em relação às palavras. Na era arcaica e mesmo no início da clássica, em que estamos em plena “cultura da canção”, a poesia era, em todos os seus gêneros, “performance ao vivo, diante de seres humanos vivos, sob o sol”»

{\vfill\scale[factor=5]{{\bf Giuliana Ragusa}, na introdução de «Lira Grega».}}

\page %---------------------------------------------------------|

\Hedra

\stoptext %---------------------------------------------------------|
