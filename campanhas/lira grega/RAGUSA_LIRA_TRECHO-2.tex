% AUTOR_LIVRO_TRECHO.tex
% Preencher com o nome das cor ou composição RGB (ex: [r=0.862, g=0.118, b=0.118]) 
\usecolors[crayola] 			   % Paleta de cores pré-definida: wiki.contextgarden.net/Color#Pre-defined_colors

% Cores definidas pelo designer:
% MyGreen		r=0.251, g=0.678, b=0.290 % 40ad4a
% MyCyan		r=0.188, g=0.749, b=0.741 % 30bfbd
% MyRed			r=0.820, g=0.141, b=0.161 % d12429
% MyPink		r=0.980, g=0.780, b=0.761 % fac7c2
% MyGray		r=0.812, g=0.788, b=0.780 % cfc9c7
% MyOrange		r=0.980, g=0.671, b=0.290 % faab4a

% Configuração de cores
\definecolor[MyColor][x=2a3e92]      % ou ex: [r=0.862, g=0.118, b=0.118] % corresponde a RGB(220, 30, 30)
\definecolor[MyColorText][white]     

% Classe para diagramação dos posts
\environment{marketing.env}		   

\starttext %---------------------------------------------------------|

\Mensagem{DESTAQUE}

\startMyCampaign

\hyphenpenalty=10000
\exhyphenpenalty=10000
«Bebamos!
Toma as grandes taças adornadas, meu caro, pois o filho de Sêmele e Zeus deu aos homens vinho
que traz olvido»

\stopMyCampaign

{\vfill\scale[factor=6]{\Seta\,Trecho do {\bf Fragmento 346}, de Ateneu,}\setupinterlinespace[line=1.5ex]\scale[factor=6]{presente no livro «Lira grega».}}
\page %---------------------------------------------------------| 
\hyphenpenalty=10000
\exhyphenpenalty=10000

Esse famoso fragmento de Ateneu, de caráter filosófico e linguagem exortativa, elabora uma {\bf RESPOSTA HEDONISTA À EFEMERIDADE DA VIDA}, tema presente na poesia grega antiga desde o canto {\cap VI} da Ilíada. 
\page
\hyphenpenalty=10000
\exhyphenpenalty=10000

Alceu, porém, parece ter sido o primeiro a fazer a associação entre ambos os temas, tão próprios ao que a tradição posterior aos poetas latinos — notadamente a Horácio — conhecerá como tema do {\bf CARPE DIEM}.

\page

\MyCover{RAGUSA_LIRA_THUMB}

\page %---------------------------------------------------------|

\Hedra

\stoptext %---------------------------------------------------------|
