% Preencher com o nome das cor ou composição RGB (ex: [r=0.862, g=0.118, b=0.118]) 
\usecolors[crayola] 			   % Paleta de cores pré-definida: wiki.contextgarden.net/Color#Pre-defined_colors

% Cores definidas pelo designer:
% MyGreen		r=0.251, g=0.678, b=0.290 % 40ad4a
% MyCyan		r=0.188, g=0.749, b=0.741 % 30bfbd
% MyRed			r=0.820, g=0.141, b=0.161 % d12429
% MyPink		r=0.980, g=0.780, b=0.761 % fac7c2
% MyGray		r=0.812, g=0.788, b=0.780 % cfc9c7
% MyOrange		r=0.980, g=0.671, b=0.290 % faab4a

% Configuração de cores
\definecolor[MyColor][x=79716e]      % ou ex: [r=0.862, g=0.118, b=0.118] % corresponde a RGB(220, 30, 30)
\definecolor[MyColorText][white]     % ou ex: [r=0.862, g=0.118, b=0.118] % corresponde a RGB(167, 169, 172)

% Classe para diagramação dos posts
\environment{marketing.env}		   

\starttext %---------------------------------------------------------|

\Mensagem{DESTAQUE}

\startMyCampaign

\hyphenpenalty=10000
\exhyphenpenalty=10000
«“Vitor me disse algo surpreendente: “Eu não me sinto nem músico, nem escritor, eu não me sinto nada. Eu sou um cara interessado em muitas coisas”. Essa afirmação veio quando estava falando a respeito da sua relação com o ambiente da música, sobre algumas necessidades próprias ao ofício, como a de viajar para encontrar músicos, produtores, arranjadores, o que for. Viver dentro das ambiências do campo cultural, vamos dizer assim, como protagonista entre os atores sociais que compõem este campo”.»

\stopMyCampaign

{\vfill\scale[factor=fit]{\Seta\,Trecho do livro {\bf Vitor Ramil, o astronauta lírico}, }\setupinterlinespace[line=1.5ex]\scale[factor=6]{de Marcos Lacerda}}

\page %---------------------------------------------------------| 

\MyPhoto{RAMIL_CAPAINTEIRA.pdf}

\page %---------------------------------------------------------|

\Hedra

\stoptext %---------------------------------------------------------|