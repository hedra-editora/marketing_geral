\setuppapersize[A4]
\usecolors[crayola]
\setupbackgrounds[paper][background=color,backgroundcolor=Almond]
	
	\definefontfeature
		[default]
		[default]
		[expansion=quality,protrusion=quality,onum=yes]
	\setupalign[fullhz,hanging]
	\definefontfamily [mainface] [sf] [Formular]
	\setupbodyfont[mainface,11pt]

% Indenting [4.4 cont-enp.p.65]
			\setupindenting[yes, 3ex]  % none small medium big next first dimension
			\indenting[next]           % never not no yes always first next
			
			% [cont-ent.p.76]
			\setupspacing[broad]  %broad packed
			% O tamanho do espaço entre o ponto final e o começo de uma sentença. 


\startsetups[grid][mypenalties]
    \setdefaultpenalties
    \setpenalties\widowpenalties{2}{10000}
    \setpenalties\clubpenalties {2}{10000}
\stopsetups

\setuppagenumbering
  [location={}]            % Estilo dos números de páginat

\setuphead[subject]
[style=bfb]		

\setuplayout[
          location=middle,
          %
          leftedge=0mm,
          leftedgedistance=0mm,
          leftmargin=20mm,
          leftmargindistance=0mm,
          width=100mm,
          rightmargindistance=0mm,
          rightmargin=20mm,
          rightedgedistance=0mm,
          rightedge=0mm,
          backspace=20mm,
          %
          top=21mm,
          topdistance=0mm,
          header=0mm,
          headerdistance=0mm,
          height=250mm,
          footerdistance=0mm,
          footer=0mm,
          bottomdistance=0mm,
          bottom=21mm,
          topspace=21mm,
        setups=mypenalties,
]

\setupalign[right]

\starttext
{\bfb Biografia inédita preenche lacuna nos estudos críticos sobre a música popular brasileira}

\blank[big]

\noindent{\it Ensaio biográfico sobre o cantor e compositor gaúcho Vitor Ramil desvela uma das trajetórias artísticas mais instigantes da cena musical do Brasil contemporâneo}

\blank[1cm]

\noindent{\em Vitor Ramil, o astronauta lírico} é um ensaio biográfico sobre o
cantor e compositor gaúcho Vitor Ramil, escrito pelo crítico Marcos
Lacerda. O livro se concentra na trajetória artística do biografado, mas
sem deixar de lado as informações relevantes de momentos de sua vida
pessoal. As análises da obra de Vitor Ramil se condensam assim, na
proporção exata, com as inquietudes do homem crescido no extremo sul do
país, nas derradeiras décadas do século XX. Trata-se de um livro
essencial sobre vida e a poética musical única do cantautor tanto para o
leitor leigo quanto para o já iniciado na “estética do frio”.

\inoutermargin[width=60mm,hoffset=1.5cm,style=tfx]{
\externalfigure[RAMIL_CAPA.pdf][width=60mm]
}


\inoutermargin[width=60mm,hoffset=1.5cm,voffset=1cm,style=tfx]{
\noindent{\bf Título} {\it Vitor Ramil, o astronauta lírico}\\
{\bf Autor} Marcos Lacerda\\
{\bf Editora} Acorde Editorial e Hedra\\
{\bf ISBN} 978-65-84716-19-3\\
{\bf Pág.} 340\\
{\bf Pré-venda} 05/04/2024\\
{\bf Lançamento} Maio
}

Construído a partir de diálogos do autor com o biografado ao longo de
quatro anos e de minuciosa pesquisa em periódicos e materiais inéditos,
o livro de Marcos Lacerda trata da vida e da obra de Vitor Ramil de
maneira singular e propõe uma leitura crítica e informativa. Por sua
formação como sociólogo e sua atuação como crítico musical, o autor não
desperdiça linha ou vírgula neste livro.

{\em Vitor Ramil, o astronauta lírico} vem cumprir uma lacuna de jogar
luz à obra de um dos mais inventivos artistas contemporâneos
brasileiros.

\blank[big]

\subject{Sobre o autor}

Marcos Lacerda é sociólogo e ensaísta. Foi diretor de música da Funarte,
responsável por políticas de âmbito nacional. É autor de {\em Hotel
Universo: a poética de Ronaldo Bastos} (2019) e organizador de
{\em Música: ensaios brasileiros contemporâneos} (2016) e {\em A canção
como música de invenção} (2018). É um dos curadores da coleção Cadernos
Ultramares e da coleção Certas Canções (Hedra/Acorde).

\subject{Trechos do livro}

\startblockquote
“O palco está já preparado. Acendem as primeiras luzes. O teatro lotado
aguarda, com certa ansiedade. Começam a entrar os primeiros músicos.
Gutcha Ramil, Ian Ramil, Thiago Ramil, João Ramil, Kleiton e Kledir
Ramil e, por fim, Vitor Ramil. Estão reunidos os músicos da Casa Ramil,
projeto de shows, gravações e encontros entre os artistas da família. O
nome é sugestivo. A importância da casa como lugar real e metáfora na
obra de Vitor é enorme. A casa das canções em “Autorretrato” (1984); a
descrição de quartos, salas, portas, pátios em “Espaço” (2000); no mesmo
ano de 2000 a “Ilusão da casa”; a importância da casa dos pais como
cenário fundamental na novela Pequod, com quartos secretos, poltronas,
relógios, um casarão de um amigo próximo do pai (1995); os pátios
pequenos da concisão onde se revelam o universo e o sentido das coisas
em “Milonga de sete cidades” (1997); a alegoria da casa da família em
ruínas no romance Satolep (2008); a descrição de corredores, mesas da
“casa nova” na canção “Satolep” (1984); o tapete deslocado da sala para
os pais dançarem tango no texto do encarte do disco Ramilonga -- A
estética do frio (1997); a descrição de cama, livro, televisão, abajur
em “Livro aberto” (2007) e, claro, a volta para a casa onde nasceu na
cidade de Pelotas, fundamental para uma reorientação na sua carreira, no
ano de 1992”.
\stopblockquote

\blank[big]

\startblockquote
“Vitor me disse algo surpreendente: “Eu não me sinto nem músico, nem
escritor, eu não me sinto nada. Eu sou um cara interessado em muitas
coisas”. Essa afirmação veio quando estava falando a respeito da sua
relação com o ambiente da música, sobre algumas necessidades próprias ao
ofício, como a de viajar para encontrar músicos, produtores,
arranjadores, o que for. Viver dentro das ambiências do campo cultural,
vamos dizer assim, como protagonista entre os atores sociais que compõem
este campo”.
\stopblockquote

\stoptext