% AUTOR_LIVRO_CURIOSIDADE.tex
% Vamos falar sobre isso "curiosidades"
% > "EM CONTEXTO"

% Preencher com o nome das cor ou composição RGB (ex: [r=0.862, g=0.118, b=0.118]) 
\usecolors[crayola] 			   % Paleta de cores pré-definida: wiki.contextgarden.net/Color#Pre-defined_colors

% Cores definidas pelo designer:
% MyGreen		r=0.251, g=0.678, b=0.290 % 40ad4a
% MyCyan		r=0.188, g=0.749, b=0.741 % 30bfbd
% MyRed			r=0.820, g=0.141, b=0.161 % d12429
% MyPink		r=0.980, g=0.780, b=0.761 % fac7c2
% MyGray		r=0.812, g=0.788, b=0.780 % cfc9c7
% MyOrange		r=0.980, g=0.671, b=0.290 % faab4a

% Configuração de cores
\definecolor[MyColor][x=79716e]      % ou ex: [r=0.862, g=0.118, b=0.118] % corresponde a RGB(220, 30, 30)
\definecolor[MyColorText][white]     % ou ex: [r=0.862, g=0.118, b=0.118] % corresponde a RGB(167, 169, 172)

% Classe para diagramação dos posts
\environment{marketing.env}		   

\starttext %---------------------------------------------------------|

\hyphenpenalty=10000
\exhyphenpenalty=10000

\Mensagem{EM CONTEXTO} %Sempre usar esse header

\starttikzpicture[remember picture,overlay]
\node at (8,-4)
{\externalfigure
              [./RAMIL_CURIOSIDADES_FOTO1.jpg]
              [width=.6\textwidth]};
\stoptikzpicture

\startMyCampaign
\hyphenpenalty=10000
\exhyphenpenalty=10000

ENTENDA A {\bf ESTÉTICA\\ DO FRIO}
EM QUATRO\\ PONTOS

\stopMyCampaign

\page %---------------------------------------------------------| 

\hyphenpenalty=10000
\exhyphenpenalty=10000

\starttikzpicture[remember picture,overlay]
\node at (-1.5,-4)
{\externalfigure
              [./RAMIL_CURIOSIDADES_FOTO1.jpg]
              [width=.6\textwidth]};
\stoptikzpicture

\startnarrower[3*left]
Conceito que o músico {\bf VITOR RAMIL} começou a esboçar em 1992, a {\bf ESTÉTICA DO FRIO} parte da consciência de uma {\bf ESTÉTICA TROPICAL} que uniria o país, ao lado da tomada de consciência da necessidade de criar uma estética própria ao Rio Grande do Sul.
\stopnarrower

{\vfill\scale[factor=fit]{\Seta\,O escritor argentino Jorge Luis Borges, que influenciou Vitor Ramil desde a juventude com suas milongas e poemas}}


\page %---------------------------------------------------------| 

\hyphenpenalty=10000
\exhyphenpenalty=10000

Ela é caracterizada por:

\startitemize[n]
\item Um espaço geográfico, os pampas, com sua linearidade, clareza, leveza e melancolia;

\item Uma ambiência cultural por onde confluem Brasil, Argentina e Uruguai; 

\item A milonga, ritmo que descende da {\it habanera} cubana e nasce dos negros dos países platinos;

\page

\item E o personagem mítico do gaúcho, filho de indígena com europeu.


% \starttikzpicture[remember picture,overlay]
% \node at (3,-1)
% {\externalfigure
%               [./RAMIL_CURIOSIDADES_FOTO2.png]
%               [width=.6\textwidth]};
% \stoptikzpicture
\MyPhoto{RAMIL_CURIOSIDADES_FOTO2}

{\vfill\scale[factor=fit]{\Seta\,Foto de divulgação do show {\it É proibido o uso de salto alto?}}}

\page %---------------------------------------------------------|

\MyPhoto{RAMIL_CAPAINTEIRA.pdf}

\page %---------------------------------------------------------|

\Hedra

\stoptext %---------------------------------------------------------|

%Conceito que o músico {\bf VITOR RAMIL} começou a esboçar em 1992, no livro {\it Nós, os gaúchos}, e depois desenvolveu no disco {\it Ramilonga – a estética do frio} (1997), a {\bf ESTÉTICA DO FRIO} relaciona-se a uma forma de pensar a arte no contexto gaúcho.
% \page %---------------------------------------------------------| 
%Ela parte da consciência de uma {\bf ESTÉTICA TROPICAL} que uniria o país, ao lado da tomada de consciência da necessidade de criar uma estética própria ao Rio Grande do Sul.
% \page %---------------------------------------------------------| 
% Segundo o autor, o fato de existir uma identificação da maior parte dos brasileiros que vivem em regiões tropicais com o ambiente festeiro da rua, da agregação efusiva de corpos, sendo os habitantes do Sul pouco afeitos a estes tipos de experimentação da vida social e da cultura em geral, faz com que se justifique a impressão de os rio-grandenses se sentirem «mais diferentes em um país feito de diferenças».
% \page %---------------------------------------------------------| 
% Um elemento fundamental dessa diferença são as fronteiras: «Muitos de nós, rio-grandenses, consideravam-se mais uruguaios que brasileiros; outros tinham em Buenos Aires, Argentina, um referencial de grande polo irradiador de informação e cultura mais presente que São Paulo ou Rio de Janeiro», escreve Ramil.
% \page %---------------------------------------------------------| 
% Enquanto gênero, a milonga compõe bem o quadro do que seria a estética do frio: «Em sua inteireza e essencialidade, a milonga, assim como a imagem, opunha-se ao excesso, à redundância. Intensas e extensas, ambas tendiam ao monocromatismo, à horizontalidade».
% \page %---------------------------------------------------------| 
% Em suma, a estética do frio é caracterizada por: um espaço geográfico, os pampas, com sua linearidade, clareza, leveza e melancolia; uma ambiência cultural por onde confluem Brasil, Argentina e Uruguai; o personagem mítico do gaúcho, filho de índio com europeu; e a milonga, ritmo que descende da {\it habanera} cubana e nasce dos negros dos países platinos, sendo designado por uma palavra de origem africana.
%\page %---------------------------------------------------------| 
%«Unidade. A própria ideia do frio como metáfora amplamente definidora apontava para este caminho: o frio nos tocava a todos em nossa heterogeneidade.»
%{\vfill\scale[factor=fit]{\Seta\,Vitor Ramil na biografia {\bf Vitor Ramil, o astronauta lírico}, de Marcos Lacerda}}