% AUTOR_LIVRO_EFEMERIDE.tex
% Preencher com o nome das cor ou composição RGB (ex: [r=0.862, g=0.118, b=0.118]) 
\usecolors[crayola]                            % Paleta de cores pré-definida: wiki.contextgarden.net/Color#Pre-defined_colors

% Cores definidas pelo designer:
% MyGreen                r=0.251, g=0.678, b=0.290 % 40ad4a
% MyCyan                r=0.188, g=0.749, b=0.741 % 30bfbd
% MyRed                        r=0.820, g=0.141, b=0.161 % d12429
% MyPink                r=0.980, g=0.780, b=0.761 % fac7c2
% MyGray                r=0.812, g=0.788, b=0.780 % cfc9c7
% MyOrange                r=0.980, g=0.671, b=0.290 % faab4a

% Configuração de cores
\definecolor[MyColor][x=79716e]      % ou ex: [r=0.862, g=0.118, b=0.118] % corresponde a RGB(220, 30, 30)
\definecolor[MyColorText][white]     % ou ex: [r=0.862, g=0.118, b=0.118] % corresponde a RGB(167, 169, 172)

% Classe para diagramação dos posts
\environment{marketing.env}                   

\starttext %---------------------------------------------------------|

\hyphenpenalty=10000
\exhyphenpenalty=10000

\Mensagem{VITOR RAMIL 6.1} %Sempre usar esse header

\MyPicture{THUMB_AUTOR2.jpg}

%\vfill\scale[factor=6]{\Seta\,QUEM É {\bf VITOR RAMIL}}

%	\page %---------------------------------------------------------| 

\hyphenpenalty=10000
\exhyphenpenalty=10000

Nascido em 7 de abril de 1962 em Pelotas (RS), {\bf VITOR RAMIL} é um dos mais inventivos músicos contemporâneos brasileiros.

\page %---------------------------------------------------------|
\hyphenpenalty=10000
\exhyphenpenalty=10000

Lançou seu primeiro álbum aos 18 anos, \emph{Estrela, estrela} (1981), ao qual se seguiram mais 11 discos, sendo o mais recente \emph{Avenida Angélica}, de 2022, com poemas de sua conterrânea Angélica Freitas.
No ensaio biográfico {\it Vitor Ramil, o astronauta lírico}, o crítico {\bf MARCOS LACERDA} explora sua carreira e formação desde a adolescência e juventude até suas incursões na literatura e suas últimas produções artísticas.

\hyphenpenalty=10000
\exhyphenpenalty=10000

\page

«Vitor, além de exímio compositor de canções, letrista versátil e
cultíssimo, é também um artista da canção que transforma poemas em
música com tal nível de excelência que fica a impressão de que o poema
musicado foi feito para ser mesmo letra de canção popular.»

\vfill

{\scale[factor=fit]{{\bf Marcos Lacerda}, autor de {\it Vitor Ramil, o astronauta lírico}}

\page

\MyPhoto{RAMIL_CAPAINTEIRA.pdf}

\vfill

\scale[factor=5]{\Seta\,De 89,00 {\bf por 60,00}, 336 páginas, 1ª edição.}

\page %---------------------------------------------------------|

\Hedra

\stoptext %---------------------------------------------------------|