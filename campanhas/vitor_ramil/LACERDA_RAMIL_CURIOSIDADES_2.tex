% AUTOR_LIVRO_CURIOSIDADES.tex
% Preencher com o nome das cor ou composição RGB (ex: [r=0.862, g=0.118, b=0.118]) 
\usecolors[crayola] 			   % Paleta de cores pré-definida: wiki.contextgarden.net/Color#Pre-defined_colors

% Cores definidas pelo designer:
% MyGreen		r=0.251, g=0.678, b=0.290 % 40ad4a
% MyCyan		r=0.188, g=0.749, b=0.741 % 30bfbd
% MyRed			r=0.820, g=0.141, b=0.161 % d12429
% MyPink		r=0.980, g=0.780, b=0.761 % fac7c2
% MyGray		r=0.812, g=0.788, b=0.780 % cfc9c7
% MyOrange		r=0.980, g=0.671, b=0.290 % faab4a

% Configuração de cores
\definecolor[MyColor][x=f8982a]      % ou ex: [r=0.862, g=0.118, b=0.118] % corresponde a RGB(220, 30, 30)
\definecolor[MyColorText][black]     % ou ex: [r=0.862, g=0.118, b=0.118] % corresponde a RGB(167, 169, 172)

% Classe para diagramação dos posts
\environment{marketing.env}		   

\starttext %---------------------------------------------------------|

\hyphenpenalty=10000
\exhyphenpenalty=10000

\Mensagem{EM CONTEXTO} %Sempre usar esse header

\startMyCampaign

\hyphenpenalty=10000
\exhyphenpenalty=10000

 5 FATOS SOBRE
{\bf VITOR RAMIL}

\stopMyCampaign

\page %---------------------------------------------------------| 

\hyphenpenalty=10000
\exhyphenpenalty=10000

{\tfa \bf 1.} A canção «Estrela, Estrela» de Vitor Ramil, uma das suas mais conhecidas, foi gravada por artistas como {\bf GAL COSTA E MILTON NASCIMENTO}.

\page %---------------------------------------------------------|

{\tfa \bf 2.} O seu terceiro álbum, {\bf TANGO} (1987), teve como capa um
retrato seu feito pelo renomado pintor Carlos Scliar, autor de obras importantes da arte moderna
brasileira. 

\page
% \MyPhoto{LACERDA_RAMIL_1}
\page

{\tfa \bf 3.} Vitor Ramil participou do importante {\bf PROJETO PIXINGUINHA}, da Funarte. O projeto foi idealizado pelo pesquisador e também compositor Hermínio Bello de Carvalho
(1935) e promovia a circulação de artistas consagrados com
artistas iniciantes pelo Brasil.

\page
\MyPhoto{LACERDA_RAMIL_2}

\scale[factor=fit]{Recorte do jornal {\bf O Liberal} publicado em 29 de abril de 1986.}
\page

{\tfa \bf 4.} A obra de Vitor Ramil está recheada de {\bf CANÇÕES FEITAS A PARTIR
DE POEMAS} de diversos poetas. Entre eles: Fernando Pessoa, Mayakovski,
Allen Ginsberg, e.e. cummings e Emily Dickinson, 
 Paulo Leminski, Augusto de Campos e Haroldo de Campos, Augusto
dos Anjos e Angélica Freitas.

\page %---------------------------------------------------------|

{\tfa \bf 5.} Vitor Ramil fez shows pelo Brasil em parceria com {\bf CHICO CÉSAR}. Chico também participou de seu  álbum Tambong (2000) e fez uma composição com Vitor no álbum «Campos Neutrais» (2017): «Olho d'água, água d'olho». 

\page
\MyPhoto{LACERDA_RAMIL_4}

\page
\MyCover{RAMIL_CAPA.png}

\page %---------------------------------------------------------|

\Hedra

\stoptext %---------------------------------------------------------|