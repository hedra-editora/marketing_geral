% LACERDA_RAMIL_EDICAO.tex
% Preencher com o nome das cor ou composição RGB (ex: [r=0.862, g=0.118, b=0.118]) 
\usecolors[crayola] 			   % Paleta de cores pré-definida: wiki.contextgarden.net/Color#Pre-defined_colors

% Cores definidas pelo designer:
% MyGreen		r=0.251, g=0.678, b=0.290 % 40ad4a
% MyCyan		r=0.188, g=0.749, b=0.741 % 30bfbd
% MyRed			r=0.820, g=0.141, b=0.161 % d12429
% MyPink		r=0.980, g=0.780, b=0.761 % fac7c2
% MyGray		r=0.812, g=0.788, b=0.780 % cfc9c7
% MyOrange		r=0.980, g=0.671, b=0.290 % faab4a

% Configuração de cores
\definecolor[MyColor][x=f8982a]      % ou ex: [r=0.862, g=0.118, b=0.118] % corresponde a RGB(220, 30, 30)
\definecolor[MyColorText][black]     % ou ex: [r=0.862, g=0.118, b=0.118] % corresponde a RGB(167, 169, 172)

% Classe para diagramação dos posts
\environment{marketing.env}		   

\starttext %---------------------------------------------------------|

\Mensagem{POR DENTRO DA EDIÇÃO}

\startMyCampaign

\hyphenpenalty=10000
\exhyphenpenalty=10000

{\bf 
AS IMAGENS DE VITOR RAMIL}

\stopMyCampaign

%\vfill\scale[lines=1.5]{\MyStar[MyColorText][none]}

\page %---------------------------------------------------------| 
\hyphenpenalty=10000
\exhyphenpenalty=10000

\MyPicture{Imagem1b}
\page %---------------------------------------------------------|

\MyPhoto{Imagem34}

\scale[factor=4]{{Foto do encarte de {\it Avenida Angélica}, 2022.}}
\page %----------------------------------------------------------|

\MyPhoto{Imagem6}
\scale[factor=4]{{Texto-Manifesto. Jornal {\cap sDCE UFRGS}, 1985.}}
\page

\MyPhoto{Imagem16}
\scale[factor=4]{{Foto de divulgação do show {\it É proibido o uso de salto alto?}}}
\page

\MyPhoto{Imagem32}
\scale[factor=4]{Foto do encarte de {\it Campos Neutrais}. Marcelo Soares.}

\page
\MyCover{RAMIL_CAPA}

\page %----------------------------------------------------------|

\Hedra

\stoptext %---------------------------------------------------------|