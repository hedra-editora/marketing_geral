% AUTOR_LIVRO_CURIOSIDADES.tex
% Preencher com o nome das cor ou composição RGB (ex: [r=0.862, g=0.118, b=0.118]) 
\usecolors[crayola] 			   % Paleta de cores pré-definida: wiki.contextgarden.net/Color#Pre-defined_colors

% Cores definidas pelo designer:
% MyGreen		r=0.251, g=0.678, b=0.290 % 40ad4a
% MyCyan		r=0.188, g=0.749, b=0.741 % 30bfbd
% MyRed			r=0.820, g=0.141, b=0.161 % d12429
% MyPink		r=0.980, g=0.780, b=0.761 % fac7c2
% MyGray		r=0.812, g=0.788, b=0.780 % cfc9c7
% MyOrange		r=0.980, g=0.671, b=0.290 % faab4a

% Configuração de cores
\definecolor[MyColor][x=f8982a]      % ou ex: [r=0.862, g=0.118, b=0.118] % corresponde a RGB(220, 30, 30)
\definecolor[MyColorText][black]     % ou ex: [r=0.862, g=0.118, b=0.118] % corresponde a RGB(167, 169, 172)

% Classe para diagramação dos posts
\environment{marketing.env}		   

\starttext %---------------------------------------------------------|

\hyphenpenalty=10000
\exhyphenpenalty=10000

\Mensagem{EM CONTEXTO} %Sempre usar esse header

\startMyCampaign

\hyphenpenalty=10000
\exhyphenpenalty=10000

{\bf VITOR RAMIL E JORGE LUIS BORGES}
DA POESIA À CANÇÃO
\stopMyCampaign

\page %---------------------------------------------------------| 

\hyphenpenalty=10000
\exhyphenpenalty=10000

{\bf VITOR RAMIL} tem uma relação profunda e multifacetada com a literatura. Entre colaborações com poetas e musicalizações de poemas, a poesia revela-se intrinsecamente ligada à obra do compositor pelotense. Dentre suas influências literárias, destaca-se {\bf JORGE LUIS BORGES}, um dos escritores mais influentes do século XX. 

\page

\MyPicture{borges2}

\page

A presença de Borges na obra de Vitor Ramil se explicita especialmente no álbum {\bf DÉLIBÁB} (2010), onde foram musicalizados alguns dos seus poemas, como «Milonga de Manuel Flores», «Milonga de dos hermanos» e «Milonga de los morenos», com participações de {\bf CAETANO VELOSO} e {\bf CARLOS MOSCARDINI}.

\page

O músico admira a precisão e a profundidade do trabalho de Borges, que ele considera uma referência na construção da sua  {\bf ESTÉTICA DO FRIO}. Para Ramil, os temas e a linguagem de Borges se alinham com sua visão artística, de modo que o escritor argentino seria de enorme importância em sua formação artística.

\page %---------------------------------------------------------|

\MyCover{RAMIL_CAPA}

\page %---------------------------------------------------------|

\Hedra

\stoptext %---------------------------------------------------------|s
