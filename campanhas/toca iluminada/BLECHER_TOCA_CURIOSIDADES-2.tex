% AUTOR_LIVRO_CURIOSIDADES.tex
% Preencher com o nome das cor ou composição RGB (ex: [r=0.862, g=0.118, b=0.118]) 
\usecolors[crayola] 			   % Paleta de cores pré-definida: wiki.contextgarden.net/Color#Pre-defined_colors

% Cores definidas pelo designer:
% MyGreen		r=0.251, g=0.678, b=0.290 % 40ad4a
% MyCyan		r=0.188, g=0.749, b=0.741 % 30bfbd
% MyRed			r=0.820, g=0.141, b=0.161 % d12429
% MyPink		r=0.980, g=0.780, b=0.761 % fac7c2
% MyGray		r=0.812, g=0.788, b=0.780 % cfc9c7
% MyOrange		r=0.980, g=0.671, b=0.290 % faab4a

% Configuração de cores
\definecolor[MyColor][x=98df88]      % ou ex: [r=0.862, g=0.118, b=0.118] % corresponde a RGB(220, 30, 30)
\definecolor[MyColorText][black]     % ou ex: [r=0.862, g=0.118, b=0.118] % corresponde a RGB(167, 169, 172)

% Classe para diagramação dos posts
\environment{marketing.env}		   

\starttext %---------------------------------------------------------|

\hyphenpenalty=10000
\exhyphenpenalty=10000

\Mensagem{EM CONTEXTO} %Sempre usar esse header

\startMyCampaign

\hyphenpenalty=10000
\exhyphenpenalty=10000

QUAL É O PAPEL DO ESCRITOR
PARA {\bf MAX BLECHER}?
\stopMyCampaign

\page %---------------------------------------------------------| 

\hyphenpenalty=10000
\exhyphenpenalty=10000

«O escritor deveria descer da “torre de marfim” e se juntar ao “fórum”. O ponto de vista dos intelectuais deveria ser facilmente compreendido por qualquer pessoa interessada em seus trabalhos, já que seu propósito é {\bf LANÇAR LUZ SOBRE QUESTÕES IMPORTANTES} e guiar as massas que estão interessadas nas opiniões dos intelectuais de hoje.»

\page

«No momento, estou me sentindo bastante cético em relação à “importância geral” da literatura. {\bf SERÁ QUE UM ESCRITOR TEM ALGUMA INFLUÊNCIA REAL?} Duvido, e não consigo pensar em muitos exemplos na história em que a sociedade foi mudada por romances ou pelas opiniões de escritores.»

\page

«Mas talvez as coisas sejam diferentes agora, talvez as pessoas estejam lendo mais e fazendo uso do conhecimento que adquiriram. Se este for o caso, então, é claro, um escritor que esteja ciente do que está acontecendo no mundo deve tentar provocar mudanças sociais, fazendo uma contribuição ideológica.» 

\page

«No entanto, essa contribuição deve estar em consonância com sua integridade como escritor, e suas habilidades devem empregadas para promover justiça e liberdade espiritual.» 

\page

«É bastante triste que, hoje em dia, alguns escritores só desçam de sua “torre de marfim” para trazer veneno e paixões políticas cegas à mesa, suas vozes apenas adicionando confusão, ideias obscuras e intolerância ao debate geral. Sim, é bastante triste que esses “intelectuais” alimentem a virulência e a violência em um momento em que isso já é tão abundante.»

{\vfill\scale[factor=6]{\Seta\,Trecho de entrevista concedida por Max}\setupinterlinespace[line=1.5ex]\scale[factor=6]{Blecher para revista {\bf Rampa}, em 1937.}}

\page %---------------------------------------------------------|

\MyCover{BLECHER_TOCA_THUMB}

\page %---------------------------------------------------------|

\Hedra

\stoptext %---------------------------------------------------------|

