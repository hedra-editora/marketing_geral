\setuppapersize[A4]
\usecolors[crayola]
\setupbackgrounds[paper][background=color,backgroundcolor=Almond]
	
	\definefontfeature
		[default]
		[default]
		[expansion=quality,protrusion=quality,onum=yes]
	\setupalign[fullhz,hanging]
	\definefontfamily [mainface] [sf] [Formular]
	\setupbodyfont[mainface,11pt]

% Indenting [4.4 cont-enp.p.65]
			\setupindenting[yes, 3ex]  % none small medium big next first dimension
			\indenting[next]           % never not no yes always first next
			
			% [cont-ent.p.76]
			\setupspacing[broad]  %broad packed
			% O tamanho do espaço entre o ponto final e o começo de uma sentença. 


\startsetups[grid][mypenalties]
    \setdefaultpenalties
    \setpenalties\widowpenalties{2}{10000}
    \setpenalties\clubpenalties {2}{10000}
\stopsetups

\setuppagenumbering
  [location={}]            % Estilo dos números de páginat

\setuphead[subject]
[style=bfb]		

\setuplayout[
          location=middle,
          %
          leftedge=0mm,
          leftedgedistance=0mm,
          leftmargin=20mm,
          leftmargindistance=0mm,
          width=100mm,
          rightmargindistance=0mm,
          rightmargin=20mm,
          rightedgedistance=0mm,
          rightedge=0mm,
          backspace=20mm,
          %
          top=21mm,
          topdistance=0mm,
          header=0mm,
          headerdistance=0mm,
          height=250mm,
          footerdistance=0mm,
          footer=0mm,
          bottomdistance=0mm,
          bottom=21mm,
          topspace=21mm,
        setups=mypenalties,
]

\setupalign[right]

\starttext
{\bfb CHAMADA CATIVANTE}

\blank[big]

\noindent A toca iluminada {\it apresenta as experiências de Blecher em sanatórios durante a década de 1930, onde a vivência na espécie de um mundo-bolha pode se sobrepor à realidade, por vezes estranha e cheio de rotinas. E é nesse lugar que sua vida interior passa a assumir um papel cada vez mais ampliado}

\blank[1cm]

\inoutermargin[width=60mm,hoffset=1cm,style=tfx,,voffset=3.5cm]{
\externalfigure[BLECHER_TOCA_THUMB][width=45mm]
}


\inoutermargin[width=70mm,hoffset=1cm,voffset=4.5cm,style=tfx]
{\noindent{\bf Título} {\em A toca iluminada: Diário de sanatório}\\
{\bf Autor} Max Blecher\\
{\bf Tradução} Fernando Klabin\\
{\bf Editora} Hedra\\
{\bf ISBN} 978-85-7715-835-5\\
{\bf Páginas} 138\\
%{\bf Pré-venda} XXXX\\
%{\bf Preço} XXXXX
}

\noindent Publicado postumamente em 1971, {\em A toca iluminada: Diário de sanatório} é estruturado a partir de eventos biográficos do autor. Em meio ao período que esteve internado em um sanatório, Blecher confronta-se com os limites da memória, na medida que busca capturar momentos de sua vida enquanto esvaem-se como “cinzas que passam por uma peneira”. Focando em “cada instante narrado, como quem coloca uma espécie de lupa imaterial sobre a própria passagem do tempo”, Blecher descreve o período na fronteira entre a realidade e o sonho. À medida que sua condição se agrava, devendo permanecer permanentemente acamado, a vida do narrador migra para os limites de sua consciência: uma {\em toca iluminada}, onde a realidade se confunde com a fantasia, o surreal com o mundano, captando, o mais plenamente possível, o mundo que aos poucos lhe escapa. É nesse movimento, de completa interiorização das experiências, que Blecher mostra-se capaz de extrair “dos abismos, das trevas e do nada toda uma constelação iluminada: aquela de uma vida interior que fulgura na escuridão”.

O autor {\bf Max Blecher} (1909--1938) nasceu em Botoșani, Romênia, filho de bem-sucedidos comerciantes judeus do ramo da porcelana. Cursou o liceu em Roman, e em 1928 matriculou-se no curso de medicina da Universidade de Rouen, na França, mas foi obrigado a abandoná-lo pouco tempo depois por conta de sua saúde. Volta então para Roman, onde faleceria em 1938, dez anos após uma sequência de internações hospitalares. A década de internações lhe rendeu muitos escritos e correspondências, como por exemplo as cartas trocadas com André Breton, líder do movimento surrealista francês, e os livros {\em Corpo transparente}, {\em Corações cicatrizados} e {\em Acontecimentos na irrealidade imediata}, além de {\em A toca iluminada}, uma publicação póstuma.
    
\page
\subject{Trechos do livro}

\startitemize
\item Tudo aquilo que escrevo foi, um dia, vida de verdade. Mas, sempre que penso isoladamente em cada instante que passou e tento revê-lo, reconstituí-lo, ou seja, restabelecer sua luz específica, sua tristeza ou sua alegria específica, a impressão que ressurge, antes de qualquer coisa, é a da efemeridade da vida que se escoa e, em seguida, a da completa ausência de valor com que esses instantes se integram naquilo a que chamamos, em poucas palavras, de existência de uma pessoa. Seria possível dizer que as lembranças da memória desbotam do mesmo modo como as que conservamos numa gaveta.

\item Não raro me ocorre ver, e ver de olhos bem abertos, coisas estranhas que só têm como acontecer em sonho e, noutras ocasiões, sonhar de olhos fechados durante o sono ou em simples devaneio coisas que, quando tento recordar, não consigo mais discernir em que mundo, em que realidade haviam se sucedido.

\item É o deserto dos acontecimentos do mundo que rodeia cada vida, e cada vida permanece solitária e isolada nesse deserto absoluto de fatos que não param de ocorrer, sempre.

\item Pois então, observei que justamente isso forma o núcleo do sofrimento, e a conclusão foi simples: para escapar da dor, não devemos procurar “escapar” dela, mas, pelo contrário, devemos “cuidar” dela com atenção máxima. Atenção máxima, e proximidade máxima. Até o ponto de a perceber em suas mínimas fibras.
\stopitemize

\page

\subject {Trecho do posfácio}

\startitemize
\item Aliás, se é que é possível classificar de alguma maneira a em tudo extraordinária narrativa de Blecher, então no gênero do testemunho — esta antítese do triunfante Bildungsroman ou romance de formação oitocentista, devotado ao protagonismo do indivíduo: seus livros, ao contrário, trata da precariedade e da fragilidade da vida, da vitimização do homem pelo destino, do triunfo inexorável da morte mas, ao mesmo tempo, do potencial de eternidade que habita cada instante. Como no diálogo entre Alice e o coelho, em Alice no país das maravilhas, de Lewis Carroll, quando Alice pergunta: "Quanto tempo dura a eternidade?" e recebe como resposta: "Às vezes, um instante."
\stopitemize

\stoptext