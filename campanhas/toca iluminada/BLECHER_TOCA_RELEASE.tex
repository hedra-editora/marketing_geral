\setuppapersize[A4]
\usecolors[crayola]
\setupbackgrounds[paper][background=color,backgroundcolor=Almond]
	\mainlanguage[pt]  

	\definefontfeature
		[default]
		[default]
		[expansion=quality,protrusion=quality,onum=yes]
	\setupalign[fullhz,hanging]
	\definefontfamily [mainface] [sf] [Formular]
	\setupbodyfont[mainface,11pt]

% Indenting [4.4 cont-enp.p.65]
			\setupindenting[yes, 3ex]  % none small medium big next first dimension
			\indenting[next]           % never not no yes always first next
			
			% [cont-ent.p.76]
			\setupspacing[broad]  %broad packed
			% O tamanho do espaço entre o ponto final e o começo de uma sentença. 


\startsetups[grid][mypenalties]
    \setdefaultpenalties
    \setpenalties\widowpenalties{2}{10000}
    \setpenalties\clubpenalties {2}{10000}
\stopsetups

\setuppagenumbering
  [location={}]            % Estilo dos números de páginat

\setuphead[subject]
[style=bfb]		

\setuplayout[
          location=middle,
          %
          leftedge=0mm,
          leftedgedistance=0mm,
          leftmargin=20mm,
          leftmargindistance=0mm,
          width=100mm,
          rightmargindistance=0mm,
          rightmargin=20mm,
          rightedgedistance=0mm,
          rightedge=0mm,
          backspace=20mm,
          %
          top=21mm,
          topdistance=0mm,
          header=0mm,
          headerdistance=0mm,
          height=250mm,
          footerdistance=0mm,
          footer=0mm,
          bottomdistance=0mm,
          bottom=21mm,
          topspace=21mm,
        setups=mypenalties,
]

\setupalign[right]

\starttext
{\bfc A toca iluminada: diário\\ de sanatório\\} 
{\tfb Max Blecher}

\blank[big]

\hyphenpenalty=10000
\exhyphenpenalty=10000

\noindent {\it A partir da sua experiência em sanatórios, Blecher produziu} A toca iluminada, {\it visionária em suas incursões metafísicas e insuportavelmente concreta em seu retrato da dor física e da degradação.}


% A toca iluminada {\it apresenta as experiências de Blecher em sanatórios durante a década de 1930, onde a vivência na espécie de um mundo-bolha pode se sobrepor à realidade, por vezes estranha e cheio de rotinas. E é nesse lugar que sua vida interior passa a assumir um papel cada vez mais ampliado}

\blank[1cm]

\inoutermargin[width=60mm,hoffset=1cm,style=tfx,,voffset=2cm]{
\externalfigure[BLECHER_TOCA_THUMB][width=50mm]
}


\inoutermargin[width=70mm,hoffset=1.1cm,voffset=2.7cm,style=tfx]
{\noindent{\bf Título} {\em A toca iluminada: diário de sanatório}\\
{\bf Autor} Max Blecher\\
{\bf Posfácio} Luis S.\,Krausz\\
{\bf Tradução} Fernando Klabin\\
{\bf Editora} Hedra\\
{\bf ISBN} 978-85-7715-835-5\\
{\bf Páginas} 138\\
{\bf Preço} 49,00
}

\inoutermargin[width=70mm,hoffset=1.1cm,voffset=7.8cm,style=tfx]
{{\bf Sobre o autor} Max Blecher (1909--1938), saudado por Eugène Ionescu como o “Kafka romeno” e frequentemente comparado pela crítica a Bruno Schulz e Robert Walser, nasceu em Botoșani, Romênia, filho de bem-sucedidos comerciantes judeus. Diagnosticado com tuberculose espinhal aos 19 anos, falece em 1938, dez anos após uma sequência de internações hospitalares. A década em sanatórios lhe rendeu muitos escritos, como as correspondências com André Breton, líder do movimento surrealista francês, e com o filósofo alemão Martin Heidegger, além dos livros {\em Corpo transparente}, {\em Corações cicatrizados}, {\em Acontecimentos na irrealidade imediata} e {\em A toca iluminada}, uma publicação póstuma.
    }

\inoutermargin[width=70mm,hoffset=1.1cm,voffset=17cm,style=tfx]
{{\bf Sobre o tradutor} Fernando Klabin nasceu em São Paulo e formou-se em Ciência Política pela Universidade de Bucareste, onde foi agraciado com a Ordem do Mérito Cultural da Romênia no grau de Oficial, em 2016.}

% \inoutermargin[width=70mm,hoffset=1.1cm,voffset=16.6cm,style=tfx]
% {{\bf Sobre o apresentador} Leonardo Francisco Soares é professor do Instituto de Letras e Linguística da Universidade Federal de Uberlândia ({\cap ILEEL/UFU}) e do programa de pós-graduação em Estudos Literários do {\cap ILEEL/UFU}. Publicou, dentre outros, um texto na coletânea {\it Guerra e literatura: ensaios em emergência} (Alameda, 2022)}


\inoutermargin[width=70mm,hoffset=-10cm,voffset=18.5cm,style=tfx]
{\definefontfamily [Times] [rm] [Times New Roman]
                   [tf=file:TimesLTStd-Roman.otf]

\setcharacterkerning[reset] \switchtobodyfont[Times,50pt] hedra \hfill \mbox{}
}


\noindent{\tfb \it Um dos maiores escritores romenos e seu íntimo caso com a morte}

\blank[.5cm]

\noindent Publicado postumamente em 1971, {\em A toca iluminada: diário de sanatório} é estruturado a partir de eventos biográficos de Max Blecher, o “Kafka romeno”. Em meio ao período que esteve hospitalizado com tuberculose espinhal, o escritor confronta-se com os limites da memória e busca capturar momentos de sua vida enquanto se esvaem como “cinzas que passam por uma peneira”. 

Focando em cada instante narrado, o romance se situa na fronteira entre a realidade e o sonho. As histórias delirantes que o protagonista elabora funcionam como recurso escapista, capaz de distanciá-lo momentaneamente da realidade assombrosa que o cerca e invade.

À medida que sua condição se agrava, devendo permanecer permanentemente acamado, a vida do narrador migra para os limites da  consciência: uma {\em toca iluminada}, onde a realidade se confunde com a fantasia, o surreal com o mundano, captando, o mais plenamente possível, o mundo que aos poucos lhe escapa. Nesse movimento de completa interiorização das experiências, Blecher mostra-se capaz de extrair “dos abismos, das trevas e do nada toda uma constelação iluminada: aquela de uma vida interior que fulgura na escuridão”.


    % \textbf{A toca iluminada} \textls[-10]{(1971) é o último romance de Max Blecher, publicado postumamente. A matéria fundamental desta obra autobiográfica é a experiência do autor em internações durante os anos 1930. À debilidade do corpo, à falta de mobilidade, ao desconforto e à dor corresponde uma vida interior agitada, nos vários sentidos da palavra, que Blecher registra febrilmente em primeira pessoa. Mesmo fora das instituições hospitalares, o narrador vive nas fronteiras do isolamento, de cujas bordas tenta aproximar-se. Para ele, \textit{mundo real} e sanatório são sobreposições de lugares estranhos, nos quais a rotina e as tentativas de levar uma vida normal não fazem sentido em meio à atmosfera impregnada de morte --- dado que muitos dos pacientes, incluindo ele próprio, são terminais.}

%  «Blecher trata da precariedade e da fragilidade da vida, da vitimização do homem pelo destino, do triunfo inexorável da morte mas, ao mesmo tempo, do potencial de eternidade que habita cada instante. Como no diálogo entre Alice e o coelho, em Alice no país das maravilhas, de Lewis Carroll, quando Alice pergunta: “Quanto tempo dura a eternidade?” e recebe como resposta: “Às vezes, um instante”.»




% \page

% \subject{Sobre o autor}


% \subject{Sobre o tradutor}


% \subject{Sobre o posfaciador}


% \subject{Trechos do livro}

% \startitemize
% \item Tudo aquilo que escrevo foi, um dia, vida de verdade. Mas, sempre que penso isoladamente em cada instante que passou e tento revê-lo, reconstituí-lo, ou seja, restabelecer sua luz específica, sua tristeza ou sua alegria específica, a impressão que ressurge, antes de qualquer coisa, é a da efemeridade da vida que se escoa e, em seguida, a da completa ausência de valor com que esses instantes se integram naquilo a que chamamos, em poucas palavras, de existência de uma pessoa. Seria possível dizer que as lembranças da memória desbotam do mesmo modo como as que conservamos numa gaveta.

% \item Não raro me ocorre ver, e ver de olhos bem abertos, coisas estranhas que só têm como acontecer em sonho e, noutras ocasiões, sonhar de olhos fechados durante o sono ou em simples devaneio coisas que, quando tento recordar, não consigo mais discernir em que mundo, em que realidade haviam se sucedido.

% \item É o deserto dos acontecimentos do mundo que rodeia cada vida, e cada vida permanece solitária e isolada nesse deserto absoluto de fatos que não param de ocorrer, sempre.

% \item Pois então, observei que justamente isso forma o núcleo do sofrimento, e a conclusão foi simples: para escapar da dor, não devemos procurar “escapar” dela, mas, pelo contrário, devemos “cuidar” dela com atenção máxima. Atenção máxima, e proximidade máxima. Até o ponto de a perceber em suas mínimas fibras.
% \stopitemize

\stoptext


% «No que exatamente consiste o valor de um instante? Em que se podem
% reconhecer sua profundidade e irreversibilidade definitivas? Em que se diferencia o instante em que uma pessoa morre de outros instantes, em
% que só acontecem fatos simples e banais?»



% «É o deserto dos acontecimentos do mundo que rodeia cada vida, e cada vida permanece solitária e isolada nesse deserto absoluto de fatos que não param de ocorrer, sempre.»