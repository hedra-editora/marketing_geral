% AUTOR_LIVRO_EDICAO.tex
% Preencher com o nome das cor ou composição RGB (ex: [r=0.862, g=0.118, b=0.118]) 
\usecolors[crayola] 			   % Paleta de cores pré-definida: wiki.contextgarden.net/Color#Pre-defined_colors

% Cores definidas pelo designer:
% MyGreen		r=0.251, g=0.678, b=0.290 % 40ad4a
% MyCyan		r=0.188, g=0.749, b=0.741 % 30bfbd
% MyRed			r=0.820, g=0.141, b=0.161 % d12429
% MyPink		r=0.980, g=0.780, b=0.761 % fac7c2
% MyGray		r=0.812, g=0.788, b=0.780 % cfc9c7
% MyOrange		r=0.980, g=0.671, b=0.290 % faab4a

% Configuração de cores
\definecolor[MyColor][x=98df88]      % ou ex: [r=0.862, g=0.118, b=0.118] % corresponde a RGB(220, 30, 30)
\definecolor[MyColorText][black]     % ou ex: [r=0.862, g=0.118, b=0.118] % corresponde a RGB(167, 169, 172)

% Classe para diagramação dos posts
\environment{marketing.env}		   

\starttext %---------------------------------------------------------|

\Mensagem{ROMÊNIA EM FOCO}

\startMyCampaign

\hyphenpenalty=10000
\exhyphenpenalty=10000

{\bf DENTRO E FORA DOS MUROS DO SANATÓRIO}

\stopMyCampaign

%\vfill\scale[lines=1.5]{\MyStar[MyColorText][none]}

\page %---------------------------------------------------------| 

\MyCover{BLECHER_TOCA_THUMB}

\page %---------------------------------------------------------| 

\hyphenpenalty=10000
\exhyphenpenalty=10000

Publicado postumamente em 1971, {\bf A TOCA ILUMINADA: DIÁRIO DE SANATÓRIO} apresenta as experiências de Blecher em sanatórios durante a década de 1930, quando esteve doente com tuberculose.

\page

Blecher confronta-se com os limites da memória, na medida que busca capturar momentos de sua vida enquanto esvaem-se como «cinzas que passam por uma peneira». Focando em «cada instante narrado, como quem coloca uma espécie de lupa imaterial sobre a própria passagem do tempo», Blecher descreve o período na fronteira entre a realidade e o sonho. À medida que sua condição se agrava, devendo permanecer permanentemente acamado, a vida do narrador migra para os limites de sua consciência: uma {\em toca iluminada}, onde a realidade se confunde com a fantasia, o surreal com o mundano, captando, o mais plenamente possível, o mundo que aos poucos lhe escapa. É nesse movimento, de completa interiorização das experiências, que Blecher mostra-se capaz de extrair «dos abismos, das trevas e do nada toda uma constelação iluminada: aquela de uma vida interior que fulgura na escuridão».
\page %---------------------------------------------------------|

\hyphenpenalty=10000
\exhyphenpenalty=10000

«Pulvinar ante, a ultricies magna {\bf TRECHO EM DESTAQUE, MAS PODE HAVER MAIS DE UM}, sempre em negrito e caixa alta. Aqui entra um trecho cativante do texto.»

{\vfill\scale[factor=5]{{\bf Nome de quem escreveu a análise}, qualificação de}\setupinterlinespace[line=1.5ex]\scale[factor=5]{XPTO professora na Universidade de Nova York. Lembre}\setupinterlinespace[line=1.5ex]\scale[factor=5]{de quebrar as linhas nos códigos.}}

\page %---------------------------------------------------------|

\Hedra

\stoptext %---------------------------------------------------------|




    \item Aliás, se é que é possível classificar de alguma maneira a em tudo extraordinária narrativa de Blecher, então no gênero do testemunho — esta antítese do triunfante Bildungsroman ou romance de formação oitocentista, devotado ao protagonismo do indivíduo: seus livros, ao contrário, trata da precariedade e da fragilidade da vida, da vitimização do homem pelo destino, do triunfo inexorável da morte mas, ao mesmo tempo, do potencial de eternidade que habita cada instante. Como no diálogo entre Alice e o coelho, em Alice no país das maravilhas, de Lewis Carroll, quando Alice pergunta: "Quanto tempo dura a eternidade?" e recebe como resposta: "Às vezes, um instante."
