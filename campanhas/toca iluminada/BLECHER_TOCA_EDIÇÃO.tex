% AUTOR_LIVRO_EDICAO.tex
% Preencher com o nome das cor ou composição RGB (ex: [r=0.862, g=0.118, b=0.118]) 
\usecolors[crayola] 			   % Paleta de cores pré-definida: wiki.contextgarden.net/Color#Pre-defined_colors

% Cores definidas pelo designer:
% MyGreen		r=0.251, g=0.678, b=0.290 % 40ad4a
% MyCyan		r=0.188, g=0.749, b=0.741 % 30bfbd
% MyRed			r=0.820, g=0.141, b=0.161 % d12429
% MyPink		r=0.980, g=0.780, b=0.761 % fac7c2
% MyGray		r=0.812, g=0.788, b=0.780 % cfc9c7
% MyOrange		r=0.980, g=0.671, b=0.290 % faab4a

% Configuração de cores
\definecolor[MyColor][x=98df88]      % ou ex: [r=0.862, g=0.118, b=0.118] % corresponde a RGB(220, 30, 30)
\definecolor[MyColorText][black]     % ou ex: [r=0.862, g=0.118, b=0.118] % corresponde a RGB(167, 169, 172)

% Classe para diagramação dos posts
\environment{marketing.env}		   

\starttext %---------------------------------------------------------|

\Mensagem{LITERATURA ROMENA EM FOCO}

\startMyCampaign

\hyphenpenalty=10000
\exhyphenpenalty=10000
{\bf A TUBERCULOSE NA LITERATURA}
QUANDO A DOENÇA
SE TORNA INSPIRAÇÃO

\stopMyCampaign

%\vfill\scale[lines=1.5]{\MyStar[MyColorText][none]}

\page %---------------------------------------------------------| 

\MyCover{BLECHER_TOCA_THUMB}

\page %---------------------------------------------------------| 

\hyphenpenalty=10000
\exhyphenpenalty=10000

Publicado postumamente em 1971, {\bf A TOCA ILUMINADA: DIÁRIO DE SANATÓRIO} apresenta as experiências de Blecher em sanatórios durante a década de 1930, quando esteve doente com tuberculose.

\page

\starttikzpicture[remember picture,overlay]
    \node at (4.45,-3.8) {\externalfigure[blecher4][width=9cm]};
\stoptikzpicture

\page

Blecher confronta-se com os limites da memória e busca capturar momentos de sua vida enquanto esvaem-se como «cinzas que passam por uma peneira», descrevendo acontecimentos que se passam na fronteira entre {\bf A REALIDADE E O SONHO}. 

\page

À medida que sua condição se agrava, devendo permanecer permanentemente acamado, a vida do narrador migra para os limites de sua consciência: uma {\em toca iluminada}, onde a {\bf REALIDADE SE CONFUNDE COM A FANTASIA}, o surreal com o mundano, captando, o mais plenamente possível, o mundo que aos poucos lhe escapa. 

\page
É nesse movimento, de completa interiorização das experiências, que Blecher mostra-se capaz de extrair «dos abismos, das trevas e do nada toda uma constelação iluminada: aquela de uma {\bf VIDA INTERIOR} que fulgura na escuridão».

\page %---------------------------------------------------------|


 «Blecher trata da {\bf PRECARIEDADE E DA FRAGILIDADE DA VIDA}, da vitimização do homem pelo destino, do triunfo inexorável da morte mas, ao mesmo tempo, do {\bf POTENCIAL DE ETERNIDADE QUE HABITA CADA INSTANTE}. Como no diálogo entre Alice e o coelho, em Alice no país das maravilhas, de Lewis Carroll, quando Alice pergunta: “Quanto tempo dura a eternidade?” e recebe como resposta: “Às vezes, um instante”.»
{\vfill\scale[factor=6]{\Seta\,Trecho do posfácio de {\bf A toca iluminada}, de}\setupinterlinespace[line=1.5ex]\scale[factor=6]{Luis S.\,Krausz.}}

\page

\Hedra

\stoptext %---------------------------------------------------------|




    