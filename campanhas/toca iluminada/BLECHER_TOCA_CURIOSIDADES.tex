% AUTOR_LIVRO_CURIOSIDADES.tex
% Preencher com o nome das cor ou composição RGB (ex: [r=0.862, g=0.118, b=0.118]) 
\usecolors[crayola] 			   % Paleta de cores pré-definida: wiki.contextgarden.net/Color#Pre-defined_colors

% Cores definidas pelo designer:
% MyGreen		r=0.251, g=0.678, b=0.290 % 40ad4a
% MyCyan		r=0.188, g=0.749, b=0.741 % 30bfbd
% MyRed			r=0.820, g=0.141, b=0.161 % d12429
% MyPink		r=0.980, g=0.780, b=0.761 % fac7c2
% MyGray		r=0.812, g=0.788, b=0.780 % cfc9c7
% MyOrange		r=0.980, g=0.671, b=0.290 % faab4a

% Configuração de cores
\definecolor[MyColor][x=98df88]      % ou ex: [r=0.862, g=0.118, b=0.118] % corresponde a RGB(220, 30, 30)
\definecolor[MyColorText][black]     % ou ex: [r=0.862, g=0.118, b=0.118] % corresponde a RGB(167, 169, 172)

% Classe para diagramação dos posts
\environment{marketing.env}		   

\starttext %---------------------------------------------------------|

\hyphenpenalty=10000
\exhyphenpenalty=10000

\Mensagem{ROMÊNIA EM FOCO} %Sempre usar esse header

\startMyCampaign

\hyphenpenalty=10000
\exhyphenpenalty=10000

{\bf MAX BLECHER}

O {\bf KAFKA} ROMENO
\stopMyCampaign

\page %---------------------------------------------------------| 

\hyphenpenalty=10000
\exhyphenpenalty=10000

Em 1936, o autor Mihail Sebastian visitou um amigo doente de cama, em uma calma cidade do nordeste romeno. Ele voltou à Bucareste «saturado, exausto, sentindo que eu não conseguiria voltar à vida», como escreveu em seu diário.

\page
«Tudo parecia sem sentido e absurdo.» Seu amigo de vinte e seis anos tem vivido «em íntima companhia com a morte», o que Sebastian considerava ambos humilhante e aterrorizante.

\page

Este amigo de Sebastian era Max Blecher, quem Eugène Ionesco saudou como o {\bf KAFKA ROMENO} após sua estreia literária nos anos 1930. 

\MyPhoto{blecher}

\page

Enquanto estudava medicina em Paris, Blecher foi diagnosticado com tuberculose espinhal, o que fez com que passasse a última década de sua vida registrando seu lento caso com a morte, produzindo um corpo de trabalho assombroso, ao mesmo tempo visionário em suas incursões metafísicas e insuportavelmente concreto em seu retrato da dor física e da degradação.

\page %---------------------------------------------------------|

\MyCover{BLECHER_TOCA_THUMB}

\page %---------------------------------------------------------|

\Hedra

\stoptext %---------------------------------------------------------|