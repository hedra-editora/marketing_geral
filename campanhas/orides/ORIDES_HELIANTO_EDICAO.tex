ORIDES_HELIANTO_EDICAO.tex
% Preencher com o nome das cor ou composição RGB (ex: [r=0.862, g=0.118, b=0.118]) 
\usecolors[crayola] 			   % Paleta de cores pré-definida: wiki.contextgarden.net/Color#Pre-defined_colors

% Cores definidas pelo designer:
% MyGreen		r=0.251, g=0.678, b=0.290 % 40ad4a
% MyCyan		r=0.188, g=0.749, b=0.741 % 30bfbd
% MyRed			r=0.820, g=0.141, b=0.161 % d12429
% MyPink		r=0.980, g=0.780, b=0.761 % fac7c2
% MyGray		r=0.812, g=0.788, b=0.780 % cfc9c7
% MyOrange		r=0.980, g=0.671, b=0.290 % faab4a

% Configuração de cores
\definecolor[MyColor][x=7ac766]      % ou ex: [r=0.862, g=0.118, b=0.118] % corresponde a RGB(220, 30, 30)
\definecolor[MyColorText][black]     % ou ex: [r=0.862, g=0.118, b=0.118] % corresponde a RGB(167, 169, 172)


% Classe para diagramação dos posts
\environment{marketing.env}		   
\def\MyOldCover#1{\starttikzpicture[overlay, remember picture]
              \node
                   [draw=none, 
                    blur shadow={shadow xshift=5.5pt,shadow yshift=-1.5pt, shadow scale=0.93},
                    shadow opacity=50, 
                    shadow blur extra rounding] 
                    at (0.50  \textwidth,-.4\textheight) {\externalfigure[#1][width=4cm]};
                \stoptikzpicture}

\def\MyCover#1{\starttikzpicture[overlay, remember picture]
              \node
                   [draw=none, 
                    blur shadow={shadow xshift=5.5pt,shadow yshift=-1.5pt, shadow scale=0.93},
                    shadow opacity=50, 
                    shadow blur extra rounding] 
                    at (0.25  \textwidth,-.3\textheight) {\externalfigure[#1][width=3cm]};
                \stoptikzpicture}


\def\MySecondCover#1{\starttikzpicture[overlay, remember picture]
              \node
                   [draw=none, 
                    blur shadow={shadow xshift=5.5pt,shadow yshift=-1.5pt, shadow scale=0.93},
                    shadow opacity=50, 
                    shadow blur extra rounding] 
                    at (0.70  \textwidth,-.3\textheight) {\externalfigure[#1][width=3cm]};
                \stoptikzpicture}


\starttext %---------------------------------------------------------|

\Mensagem{NOVO FORMATO}

\startMyCampaign

\hyphenpenalty=10000
\exhyphenpenalty=10000

{\bf 
A ORIDES ESTÁ MUDANDO DE
CARA}

\stopMyCampaign

%\vfill\scale[lines=1.5]{\MyStar[MyColorText][none]}

\page %---------------------------------------------------------| 

{\bf VOCÊ JÁ VIU ESSA CAPA POR AÍ?}

\MyOldCover{ORIDES_ANTIGA_THUMB}

\page %---------------------------------------------------------| 

\hyphenpenalty=10000
\exhyphenpenalty=10000

Agora, a poesia completa de Orides Fontela está de volta em um {\bf NOVO VISUAL.}

\MyCover{ORIDES_TRANSPOSICAO_THUMB}
\MySecondCover{ORIDES_HELIANTO_THUMB}
\page %---------------------------------------------------------|

\hyphenpenalty=10000
\exhyphenpenalty=10000

Os livros «Transposição», «Helianto», «Alba», «Rosácea» e «Teia» serão {\bf PUBLICADOS INDIVIDUALMENTE}, como foram concebidos originalmente pela poeta.

\page

O novo formato da coleção pretende, portanto, ser mais fiel ao {\bf PROJETO E ORGANIZAÇÃO QUE ORIDES DEU A SUA PRÓPRIA OBRA.}

\page %---------------------------------------------------------|

O segundo livro da coleção, {\bf HELIANTO}, saíra em breve da pré-venda.
Adquira já em nosso site.

\vfill\scale[factor=5]{\Seta\,De R\$\,49 {\bf por R\$\,34}, 80 páginas, 2ª edição.}

\page
\Hedra

\stoptext %---------------------------------------------------------|