% ORIDES_HELIANTO_CURIOSIDADES.tex
% Preencher com o nome das cor ou composição RGB (ex: [r=0.862, g=0.118, b=0.118]) 
\usecolors[crayola] 			   % Paleta de cores pré-definida: wiki.contextgarden.net/Color#Pre-defined_colors

% Cores definidas pelo designer:
% MyGreen		r=0.251, g=0.678, b=0.290 % 40ad4a
% MyCyan		r=0.188, g=0.749, b=0.741 % 30bfbd
% MyRed			r=0.820, g=0.141, b=0.161 % d12429
% MyPink		r=0.980, g=0.780, b=0.761 % fac7c2
% MyGray		r=0.812, g=0.788, b=0.780 % cfc9c7
% MyOrange		r=0.980, g=0.671, b=0.290 % faab4a

% Configuração de cores
\definecolor[MyColor][x=7ac766]      % ou ex: [r=0.862, g=0.118, b=0.118] % corresponde a RGB(220, 30, 30)
\definecolor[MyColorText][black]     % ou ex: [r=0.862, g=0.118, b=0.118] % corresponde a RGB(167, 169, 172)

% Classe para diagramação dos posts
\environment{marketing.env}		   

\starttext %---------------------------------------------------------|

\hyphenpenalty=10000
\exhyphenpenalty=10000

\Mensagem{EM CONTEXTO} %Sempre usar esse header

\startMyCampaign

\hyphenpenalty=10000
\exhyphenpenalty=10000


{\bf ORIDES FONTELA\\}
 UMA VIDA TODA PARA A POESIA

\stopMyCampaign

\page %---------------------------------------------------------| 

\MyPhoto{004.jpeg}
\hyphenpenalty=10000
\exhyphenpenalty=10000
Orides Fontela levou uma {\bf “VIDA DE POETA”}. Isto é, uma vida voltada para a poesia, para o pensamento e para a arte, e avessa às vicissitudes e exigências da existência cotidiana.

\page %---------------------------------------------------------|

Jamais se casou, não teve filhos, não tinha trabalho fixo nem paciência para as convenções e conveniências sociais e materiais.

\page
\MyPhoto{009.jpeg}

Foi uma vida dedicada à atividade poética e que culminou em uma {\bf OBRA FUNDAMENTAL PARA A POESIA CONTEMPORÂNEA BRASILEIRA.}

\page

Apesar da importância de sua obra, Orides morreu na {\bf POBREZA}, tendo de recorrer a amigos para pagar as contas. Quando estes insistiam para que saísse do centro de São Paulo para morar na periferia, respondia: «Não sou árvore para ser removida de São Paulo!».

\page

\MyCover{ORIDES_HELIANTO_THUMB}

\page %---------------------------------------------------------|

\Hedra

\stoptext %---------------------------------------------------------|