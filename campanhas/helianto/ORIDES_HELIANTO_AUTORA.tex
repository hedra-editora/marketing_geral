% Preencher com o nome das cor ou composição RGB (ex: [r=0.862, g=0.118, b=0.118]) 
\usecolors[crayola]                        % Paleta de cores pré-definida: wiki.contextgarden.net/Color#Pre-defined_colors

% Cores definidas pelo designer:
% MyGreen               r=0.251, g=0.678, b=0.290 % 40ad4a
% MyCyan                r=0.188, g=0.749, b=0.741 % 30bfbd
% MyRed                 r=0.820, g=0.141, b=0.161 % d12429
% MyPink                r=0.980, g=0.780, b=0.761 % fac7c2
% MyGray                r=0.812, g=0.788, b=0.780 % cfc9c7
% MyOrange              r=0.980, g=0.671, b=0.290 % faab4a

% Configuração de cores
\definecolor[MyColor][x=7ac766]      % ou ex: [r=0.862, g=0.118, b=0.118] % corresponde a RGB(220, 30, 30)
\definecolor[MyColorText][black]     % ou ex: [r=0.862, g=0.118, b=0.118] % corresponde a RGB(167, 169, 172)

% Classe para diagramação dos posts
\environment{marketing.env}                

\starttext %---------------------------------------------------------|

\hyphenpenalty=10000
\exhyphenpenalty=10000

\Mensagem{A POETA IMPOSSÍVEL} %Sempre usar esse header

\MyPicture{./008.jpeg}

\vfill\scale[factor=6]{\Seta\,{\bf ORIDES FONTELA (1940-1998)}}

\page %---------------------------------------------------------| 

\hyphenpenalty=10000
\exhyphenpenalty=10000

{\bf ORIDES FONTELA} (1940-1998) nasceu em São João da Boa Vista, onde concluiu o curso normal e tornou-se professora. Na juventude, os poemas que publicava no jornal da cidade foram lidos por um antigo colega de escola, o jovem crítico {\bf DAVI ARRIGUCCI JÚNIOR}.  

\page %---------------------------------------------------------| 

\hyphenpenalty=10000
\exhyphenpenalty=10000

Davi percebeu que havia ali {\bf UMA GRANDE POETA} e perguntou-lhe se ela tinha outros textos. Ela respondeu com um fichário repleto de poemas. Davi selecionou alguns deles e mostrou ao professor Antonio Candido, que também gostou muito do que leu.

\page %---------------------------------------------------------| 

\hyphenpenalty=10000
\exhyphenpenalty=10000

Com o apoio de Davi, Orides saiu de sua cidade natal e mudou-se para São Paulo, onde faria o curso de Filosofia na antiga FFLC, atual FFLCH da USP. O {\bf PRIMEIRO LIVRO DE ORIDES}, {\it Transposição} (1969), já nasceu consagrado no ambiente universitário, mas só recentemente a poeta tem alcançado o reconhecimento que merece.

\page %---------------------------------------------------------| 

\hyphenpenalty=10000
\exhyphenpenalty=10000

\MyPicture{./010.jpeg}

Neste mês de junho, {\it Helianto}, {\bf SEGUNDO LIVRO DA AUTORA}, publicado em 1973, está em pré-venda no site da Hedra. 

\page %---------------------------------------------------------|

\MyCover{./ORIDES_HELIANTO_THUMB.png}

\page %---------------------------------------------------------|

\Hedra

\stoptext %---------------------------------------------------------|