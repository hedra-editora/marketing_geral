% ORIDES_HELIANTO_CURIOSIDADES.tex
% Preencher com o nome das cor ou composição RGB (ex: [r=0.862, g=0.118, b=0.118]) 
\usecolors[crayola] 			   % Paleta de cores pré-definida: wiki.contextgarden.net/Color#Pre-defined_colors

% Cores definidas pelo designer:
% MyGreen		r=0.251, g=0.678, b=0.290 % 40ad4a
% MyCyan		r=0.188, g=0.749, b=0.741 % 30bfbd
% MyRed			r=0.820, g=0.141, b=0.161 % d12429
% MyPink		r=0.980, g=0.780, b=0.761 % fac7c2
% MyGray		r=0.812, g=0.788, b=0.780 % cfc9c7
% MyOrange		r=0.980, g=0.671, b=0.290 % faab4a

% Configuração de cores
\definecolor[MyColor][x=7ac766]      % ou ex: [r=0.862, g=0.118, b=0.118] % corresponde a RGB(220, 30, 30)
\definecolor[MyColorText][black]     % ou ex: [r=0.862, g=0.118, b=0.118] % corresponde a RGB(167, 169, 172)

% Classe para diagramação dos posts
\environment{marketing.env}		   

\starttext %---------------------------------------------------------|

\hyphenpenalty=10000
\exhyphenpenalty=10000

\Mensagem{EM CONTEXTO} %Sempre usar esse header

\startMyCampaign

\hyphenpenalty=10000
\exhyphenpenalty=10000


{\bf ORIDES FONTELA\\}
E O BUDISMO 
\stopMyCampaign

\page %---------------------------------------------------------| 
\hyphenpenalty=10000
\exhyphenpenalty=10000

{\bf ORIDES} esteve entre os primeiros praticantes regulares de zen-budismo no monastério Busshinji e foi um dos primeiros brasileiros oficialmente inciados na doutrina.

\page %---------------------------------------------------------|

Fascinada pela seriedade, disciplina e, principalmente, pelo {\bf SILÊNCIO} próprio da prática, Orides participa entusiasticamente de sessões de meditações, viagens e retiros.
\page

«Mestre Tokuda não tece comentários sobre a “praticante meio esquisita”. Entende que [Orides] mantém a inquietação interior, tentando {\bf DOMESTICAR A FÚRIA E O CAOS}. Sabe que ela preza o equilíbrio, embora --- e talvez porque --- viva o indomesticável.»

\vfill\scale[factor=7]{{\bf O enigma Orides}, Gustavo de Castro}
\page

Embora tenha servido para a poeta como um {\bf CATALISADOR DA ATIVIDADE POÉTICA}, em um entrevista Orides explica que, com relação ao zen-budismo, «procurava a {\bf ILUMINAÇÃO} mesmo. Mas só cheguei a um pisca-pisca». 

\page

\MyCover{ORIDES_HELIANTO_THUMB}

\page %---------------------------------------------------------|

\Hedra

\stoptext %---------------------------------------------------------|
