% CHOMSKY_ANARQUISMO_EFEMERIDE.tex
% Preencher com o nome das cor ou composição RGB (ex: [r=0.862, g=0.118, b=0.118]) 
\usecolors[crayola] 			   % Paleta de cores pré-definida: wiki.contextgarden.net/Color#Pre-defined_colors

% Cores definidas pelo designer:
% MyGreen		r=0.251, g=0.678, b=0.290 % 40ad4a
% MyCyan		r=0.188, g=0.749, b=0.741 % 30bfbd
% MyRed			r=0.820, g=0.141, b=0.161 % d12429
% MyPink		r=0.980, g=0.780, b=0.761 % fac7c2
% MyGray		r=0.812, g=0.788, b=0.780 % cfc9c7
% MyOrange		r=0.980, g=0.671, b=0.290 % faab4a

% Configuração de cores
\definecolor[MyColor][Blush]      % ou ex: [r=0.862, g=0.118, b=0.118] % corresponde a RGB(220, 30, 30)
\definecolor[MyColorText][white]  % ou ex: [r=0.862, g=0.118, b=0.118] % corresponde a RGB(167, 169, 172)

% Classe para diagramação dos posts
\environment{marketing.env}		   

\starttext %---------------------------------------------------------|

\hyphenpenalty=10000
\exhyphenpenalty=10000

\Mensagem{14 DE MAIO} %Sempre usar esse header

\MyPicture{THUMB_AUTOR.jpeg}

\vfill\scale[factor=6]{\Seta\,56 ANOS DO {\bf MAIO DE 1968}}

\page %---------------------------------------------------------| 

\hyphenpenalty=10000
\exhyphenpenalty=10000

O que veio a ser a {\bf MAIOR GREVE GERAL DA FRANÇA}, começou com protestos estudantis na Universidade de Nanterre contra a divisão dos dormitórios entre homens e mulheres. Aproveitando do incidente, outros universitários franceses e grupos político partidários também foram às ruas manifestar suas reivindicações.

\page

Logo, estudantes e operários estavam juntos lutando por ampliação dos direitos trabalhistas e reformas estudantis, e
 demonstrando que eram parte de nova geração que reivindicava o fim de estruturas conservadoras e autoritárias.


\page

 Equipados com capacetes e escudos, policiais entraram na prestigiosa Sorbonne para desocupá-la e detiveram cerca de 600 jovens.


\page

Com a cobertura televisiva, o episódio francês ficava conhecido pelo mundo e acabou inspirando outros movimentos ao redor do mundo, em prol da expansão dos direitos civis, de mais liberdade sexual e contra guerras como as do Vietnã.
 
 
 \page

\page %---------------------------------------------------------|

\MyCover{CHOMSKY_ANARQUISMO_THUMB}

\page %---------------------------------------------------------|

\Hedra

\stoptext %---------------------------------------------------------|






% marcada pela confluência entre um profundo mal-estar popular e desejos de mudança.


% Os primeiros confrontos ocorreram no entorno de Sorbonne. Logo, as universidades, as escolas e os teatros ocupados por jovens viraram locais de debates permanentes, onde todos queriam mudar o mundo. Nas paredes cobertas de pichações, era possível ler: "A imaginação no poder".

% A manifestação maciça de 13 de maio, em Paris, implicou a união dos movimentos estudantil e operário. As fábricas já tinham sido palco de vários conflitos sociais em 1967 e no começo de 1968. A automatização das tarefas, divididas e cronometradas, acelerou as cadências. Quando seus corpos já não conseguiam mais acompanhar o ritmo, os operários mais velhos começaram a ganhar menos.

% O setor agrícola também de manifestou, sobretudo no oeste do país, muito mobilizado.

% A partir de 14 de maio, com as fábricas ocupadas, os trens e os transportes paralisados e os postos de gasolina desabastecidos, a França viveu a maior greve geral de sua história, um movimento que durou semanas e teve a adesão de sete a dez milhões de trabalhadores.


% O movimento de Maio de 1968 na França teve uma relação significativa com o socialismo libertário e o anarquismo, embora não tenha sido exclusivamente liderado por essas ideologias. Aqui estão alguns pontos que destacam essa relação:

%     Antiautoritarismo: O movimento de Maio de 1968 era essencialmente antiautoritário, contestando não apenas o governo francês, mas também estruturas hierárquicas mais amplas na sociedade, incluindo instituições educacionais e corporativas. Essa atitude antiautoritária ressoa com os princípios do anarquismo, que rejeita o controle centralizado e defende a autonomia individual e coletiva.

%     Horizontalidade e autogestão: Durante os protestos, os participantes frequentemente organizavam-se de forma horizontal e tomavam decisões coletivas em assembleias gerais. Essa prática de autogestão e descentralização reflete os princípios do socialismo libertário e do anarquismo, que valorizam a organização horizontal e a tomada de decisões democráticas diretas.

%     Crítica à burocracia: Os manifestantes de Maio de 1968 criticavam fortemente a burocracia e a alienação nas instituições estatais e corporativas. Essa crítica ecoa as ideias anarquistas sobre a abolição do Estado e a organização da sociedade de forma não hierárquica.

%     Experimentação com formas de vida alternativas: Durante os eventos de Maio de 1968, houve uma explosão de experimentação com formas alternativas de vida, incluindo comunas e coletivos autogeridos. Essa busca por novas formas de organização social e política reflete os ideais anarquistas de construir uma sociedade baseada na solidariedade, cooperação e liberdade individual.

% Embora o movimento de Maio de 1968 não tenha sido estritamente anarquista em sua totalidade, muitos de seus elementos e valores fundamentais estão alinhados com os princípios do anarquismo e do socialismo libertário. Como resultado, é frequentemente considerado um momento importante na história do movimento anarquista e um exemplo de como essas ideologias podem ser mobilizadas em larga escala para desafiar as estruturas de poder existentes.



% Berço das liberdades individuais, da libertação da palavra e da emancipação das mulheres. Os acontecimentos de Maio de 68 — protestos realizados na França e que completam 50 anos em 2018 — foram responsáveis por uma verdadeira revolução moral, muito instigada por filósofos e escritores da época, como Jean-Paul Sartre e Simone de Beauvoir. Mais que intelectuais, com obras relevantes ainda hoje, esses nomes atuaram diretamente junto aos líderes do movimento estudantil.

%  filósofo e escritor foi um dos principais rostos do episódio histórico e viu suas ideias político-sociais reverberarem desde a publicação de Crítica da Razão Dialética, quatros anos antes do movimento.