% AUTOR_LIVRO_EFEMERIDE.tex
% Preencher com o nome das cor ou composição RGB (ex: [r=0.862, g=0.118, b=0.118]) 
\usecolors[crayola] 			   % Paleta de cores pré-definida: wiki.contextgarden.net/Color#Pre-defined_colors

% Cores definidas pelo designer:
% MyGreen		r=0.251, g=0.678, b=0.290 % 40ad4a
% MyCyan		r=0.188, g=0.749, b=0.741 % 30bfbd
% MyRed			r=0.820, g=0.141, b=0.161 % d12429
% MyPink		r=0.980, g=0.780, b=0.761 % fac7c2
% MyGray		r=0.812, g=0.788, b=0.780 % cfc9c7
% MyOrange		r=0.980, g=0.671, b=0.290 % faab4a

% Configuração de cores
\definecolor[MyColor][BrickRed]      % ou ex: [r=0.862, g=0.118, b=0.118] % corresponde a RGB(220, 30, 30)
\definecolor[MyColorText][white]     % ou ex: [r=0.862, g=0.118, b=0.118] % corresponde a RGB(167, 169, 172)

% Classe para diagramação dos posts
\environment{marketing.env}		   

\starttext %---------------------------------------------------------|

\hyphenpenalty=10000
\exhyphenpenalty=10000

\Mensagem{1º DE MAIO} %Sempre usar esse header

\MyPicture{CHOMSKY_ANARQUISMO_6}

\vfill\scale[factor=6]{\Seta\,DIA DO TRABALHO}

\page %---------------------------------------------------------| 

\hyphenpenalty=10000
\exhyphenpenalty=10000

Celebrado anualmente em quase todos os países do mundo, o {\bf DIA DO TRABALHO} remonta ao 1º de maio de 1886, quando sindicatos organizaram uma greve operária em Chigado para reivindicar melhores condições de trabalho, incluindo a redução da jornada diária para oito horas.

\page

A {\bf GREVE GERAL}, estimulada pelos anarquistas, conquistou ampla adesão, envolvendo cerca de 340\,000 trabalhadores em todo o país. Os protestos foram marcados por {\bf REPRESSÃO} e confrontos violentos entre grevistas e policiais.

\page

 Após a {\bf EXPLOSÃO DE UMA BOMBA} que resultou na morte de agentes da polícia, os sindicalistas anarquistas Albert Parsons, Adolph Fischer, George Engel e August Spies foram condenados à morte por enforcamento, apesar da falta de evidências. O episódio ficou conhecido como {\bf BLACK FRIDAY}.

\page

Em 1893, três outros trabalhadores condenados à prisão perpétua foram inocentados e reabilitados depois da confirmação de que teria sido o chefe da polícia quem organizara tudo, tendo inclusive encomendado o {\bf ATENTADO} para justificar a repressão que viria a seguir.

\page

A {\bf SEGUNDA INTERNACIONAL SOCIALISTA}, reunida em Paris em 1889, elegeu o 1º de maio para uma manifestação anual pela jornada de 8 horas, em homenagem às lutas sindicais de Chicago. Em 1919, o Senado francês tornou o 1º de maio um {\bf FERIADO NACIONAL} ao ratificar a jornada de 8 horas, o que foi posteriormente adotado pela União Soviética e outras nações. 


\page

\MyCover{CHOMSKY_ANARQUISMO_THUMB}

\page %---------------------------------------------------------|

\Hedra

\stoptext %---------------------------------------------------------|


% Origens operárias

% Nos Estados Unidos, durante o congresso de 1884, os sindicatos estabelecem o prazo de dois anos para conseguir impor aos empregadores a limitação da jornada de trabalho para oito horas. Eles iniciaram a campanha em 1 de maio, quando muitas empresas começavam seu ano contábil, os contratos de trabalho terminavam e os trabalhadores buscavam outros empregos. Estimulada pelos anarquistas, a adesão à greve geral de 1 de maio de 1886 foi ampla, envolvendo cerca de 340.000 trabalhadores em todo o país.
% Em Chicago, a greve atingiu várias empresas. No dia 3 de maio, durante uma manifestação, grevistas da fábrica McCormick saem em perseguição aos indivíduos contratados pela empresa para furar a greve. São recebidos pelos detetives da agência Pinkerton e policiais armados de rifles. O confronto resulta em três trabalhadores mortos. No dia seguinte, realiza-se uma marcha de protesto e, à noite, após a multidão se dispersar na Haymarket Square, restaram cerca de 200 manifestantes e o mesmo número de policiais. Foi quando uma bomba explodiu perto dos policiais, matando um deles. Sete outros foram mortos no confronto que se seguiu.
% Em consequência desses eventos, os sindicalistas anarquistas Albert Parsons, Adolph Fischer, George Engel, August Spies e Louis Lingg, foram condenados à forca, apesar da inexistência de provas. Louis Lingg cometeu suicídio na prisão, ingerindo uma cápsula explosiva. Os outros quatro foram enforcados em 11 de novembro de 1887, dia que ficou conhecido como Black Friday. Três outros foram condenados à prisão perpétua. Em 1893 eles foram inocentados e reabilitados pelo governador de Illinois, que confirmou ter sido o chefe da polícia quem organizara tudo, inclusive encomendando o atentado para justificar a repressão que viria a seguir.

% No 20 de junho de 1889, a segunda Internacional Socialista, reunida em Paris, decidiu convocar anualmente uma manifestação com o objetivo de lutar pela jornada de 8 horas de trabalho. A data escolhida foi o primeiro dia de maio, como homenagem às lutas sindicais de Chicago. Em 1º de maio de 1891, uma manifestação no norte de França foi dispersada pela polícia, resultando na morte de dez manifestantes. Esse novo drama serviu para reforçar o significado da data como um dia de luta dos trabalhadores. Meses depois, a Internacional Socialista de Bruxelas proclamou a data como dia internacional de reivindicação de condições laborais.
% Em 23 de abril de 1919, o senado francês ratificou a jornada de 8 horas e proclamou feriado o dia 1º de maio daquele ano. Em 1920, a então União Soviética adotou o 1º de maio como feriado nacional, sendo seguida por alguns países.
% Até hoje, o governo dos Estados Unidos se nega a reconhecer o primeiro dia de maio como o Dia do Trabalhador. Em 1890, a luta dos trabalhadores norte-americanos fez com que o Congresso aprovasse a redução da jornada de trabalho, de 16 horas para 8 horas diárias.

% Dia do Trabalhador no Brasil
% Com a chegada de imigrantes europeus no Brasil, as ideias de luta pelos direitos dos trabalhadores vieram junto. Em 1917, houve uma Greve geral. Com o crescimento do operariado, o dia 1 de maio foi declarado feriado pelo presidente Artur Bernardes, em 1925.
% Até o início da Era Vargas (1930–1945) certos tipos de agremiação dos trabalhadores fabris eram bastante comuns, embora não constituísse um grupo político muito forte, dada a incipiente industrialização do país. O movimento operário caracterizou-se, em um primeiro momento, teve influências do anarquismo e, mais tarde, do comunismo, mas com a chegada de Getúlio Vargas ao poder, essas influências foram gradativamente dissolvidas pelo chamado trabalhismo.
% Até então, o Dia do Trabalhador era considerado, no âmbito dos movimentos anarquistas e comunistas, como um momento de luta, protesto e crítica às estruturas socioeconômicas do país. A propaganda trabalhista de Vargas, sutilmente, transformou um dia destinado a celebrar o trabalhador em Dia do Trabalho. Tal mudança, aparentemente superficial, alterou profundamente as atividades realizadas no 1º de maio. Até então marcado por piquetes e passeatas, o Dia do Trabalhador passou a ser comemorado com festas populares, desfiles e celebrações similares.
% Aponta-se que o caráter massificador do Dia do Trabalhador, no Brasil, se expressa especialmente pelo costume que os governos têm de anunciar neste dia o aumento anual do salário mínimo. Outro ponto muito importante atribuído ao dia do trabalhador foi a criação da Consolidação das Leis do Trabalho – CLT, em 1º de maio de 1943.

