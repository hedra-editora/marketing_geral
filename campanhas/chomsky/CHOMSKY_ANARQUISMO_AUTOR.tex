% CHOMSKY_ANARQUISMO_AUTOR.tex
% quem é "autor"
% > "VIDA E OBRA"

% Preencher com o nome das cor ou composição RGB (ex: [r=0.862, g=0.118, b=0.118]) 
\usecolors[crayola] 			   % Paleta de cores pré-definida: wiki.contextgarden.net/Color#Pre-defined_colors

% Cores definidas pelo designer:
% MyGreen		r=0.251, g=0.678, b=0.290 % 40ad4a
% MyCyan		r=0.188, g=0.749, b=0.741 % 30bfbd
% MyRed			r=0.820, g=0.141, b=0.161 % d12429
% MyPink		r=0.980, g=0.780, b=0.761 % fac7c2
% MyGray		r=0.812, g=0.788, b=0.780 % cfc9c7
% MyOrange		r=0.980, g=0.671, b=0.290 % faab4a

% Configuração de cores
\definecolor[MyColor][x=c0e016]	  [r=0.862, g=0.118, b=0.118]      % ou ex: [r=0.862, g=0.118, b=0.118] % corresponde a RGB(220, 30, 30)
\definecolor[MyColorText][black] [r=0.655, g=0.663, b=0.675]      % ou ex: [r=0.862, g=0.118, b=0.118] % corresponde a RGB(167, 169, 172)

% Classe para diagramação dos posts
\environment{marketing.env}		   

% Cabeço e rodapé: Informações (caso queira trocar alguma coisa)
 		\def\MensagemSaibaMais  {SAIBA MAIS:}
 		\def\MensagemSite		{HEDRA.COM.BR}
 		\def\MensagemLink       {LINK NA BIO}

\starttext %--------------------------------------------------------|

\Mensagem{UMA VOZ DA ESQUERDA}

\hyphenpenalty=10000
\exhyphenpenalty=10000

%\startMyCampaign

\MyPicture{CHOMSKY_ANARQUISMO_2}

%\stopMyCampaign

\vfill\scale[factor=6]{\Seta\,NOAM CHOMSKY (1928--)}

\page %----------------------------------------------------------|

\hyphenpenalty=10000
\exhyphenpenalty=10000

Professor de linguística na Universidade do Arizona e 
professor emérito do Instituto de Tecnologia de Massachusetts ({\cap mit}), além do trabalho na área de linguística, Chomsky é reconhecido
internacionalmente como um dos maiores intelectuais vivos da esquerda.

\page

Publicou centenas de artigos e livros sobre temas como mídia, movimentos sociais, política e economia global e desenvolveu uma teoria sobre a “gramática
gerativa”, a qual teve um profundo impacto no campo dos estudos
linguísticos, fundamentalmente por meio da obra {\bf ESTRUTURAS SINTÁTICAS}
(Edições 70, 1980), de 1957. 


\page

Ganhou notoriedade com o artigo “A responsabilidade dos intelectuais”, publicado em 1969 no livro {\bf O PODER AMERICANO E OS NOVOS MANDARINS} (Record, 2006). Escreveu, também, sobre o papel propagandista da mídia, {\bf MANUFACTURING CONSENT [A MANIPULAÇÃO DO PÚBLICO: POLÍTICA E PODER ECONÔMICO NO USO DA MÍDIA]} (Futura, 2003). 

\page

Dentre seus livros publicados no Brasil, estão: {\bf 11 DE SETEMBRO} (Bertrand Brasil, 2003), {\bf CONTENDO A DEMOCRACIA} (Record, 2003), {\bf O
IMPÉRIO AMERICANO} (Campus, 2004), {\bf PARA ENTENDER O PODER} (Bertrand
Brasil, 2005), {\bf O LUCRO OU AS PESSOAS} (Bertrand Brasil, 2006), {\bf O
GOVERNO DO FUTURO} (Record, 2007) e {\bf RAZÕES DE ESTADO} (Record, 2008).

\page %----------------------------------------------------------|

\MyCover{THUMB_LIVRO.pdf}

\page %----------------------------------------------------------|

\Hedra

\stoptext %---------------------------------------------------------|