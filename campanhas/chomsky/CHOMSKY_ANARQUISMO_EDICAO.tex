% CHOMSKY_ANARQUISMO_EDICAO.tex
% Preencher com o nome das cor ou composição RGB (ex: [r=0.862, g=0.118, b=0.118]) 
\usecolors[crayola] 			   % Paleta de cores pré-definida: wiki.contextgarden.net/Color#Pre-defined_colors

% Cores definidas pelo designer:
% MyGreen		r=0.251, g=0.678, b=0.290 % 40ad4a
% MyCyan		r=0.188, g=0.749, b=0.741 % 30bfbd
% MyRed			r=0.820, g=0.141, b=0.161 % d12429
% MyPink		r=0.980, g=0.780, b=0.761 % fac7c2
% MyGray		r=0.812, g=0.788, b=0.780 % cfc9c7
% MyOrange		r=0.980, g=0.671, b=0.290 % faab4a

% Configuração de cores
\definecolor[MyColor][BrickRed]      % ou ex: [r=0.862, g=0.118, b=0.118] % corresponde a RGB(220, 30, 30)
\definecolor[MyColorText][white]  % ou ex: [r=0.862, g=0.118, b=0.118] % corresponde a RGB(167, 169, 172)

% Classe para diagramação dos posts
\environment{marketing.env}		   

\starttext %---------------------------------------------------------|

\Mensagem{EM DEFESA DA LIBERDADE}


\startMyCampaign

\hyphenpenalty=10000
\exhyphenpenalty=10000

\setupinterlinespace[line=-3ex]{A MAIOR \\ 
 COLETÂNEA}
DE {\bf NOAM CHOMSKY}
SOBRE 

ANARQUISMO

\stopMyCampaign

%\vfill\scale[lines=1.5]{\MyStar[MyColorText][none]}

\page %---------------------------------------------------------| 

\MyCover{CHOMSKY_ANARQUISMO_THUMB}

\page %---------------------------------------------------------| 

\hyphenpenalty=10000
\exhyphenpenalty=10000

{\bf NOTAS SOBRE ANARQUISMO} é uma compilação inédita, não apenas em português mas internacionalmente, de oito entrevistas e dois artigos. 

\page

Nesses textos, {\bf CHOMSKY} desafia conceitos arraigados e assume posições bastante ecléticas e
antidogmáticas, que se baseiam numa união entre o {\bf SOCIALISMO} e o {\bf LIBERALISMO}.


\page

Ele defende como princípio\\ fundamental o combate às \\{\bf ESTRUTURAS AUTORITÁRIAS DE PODER} responsáveis pela dominação em todos os níveis, além de discutir estratégias de lutas populares, que conciliam reformas de curto prazo com a busca de um horizonte revolucionário com ganhos reais em relação às empresas e ao Estado.

\page %---------------------------------------------------------|

\hyphenpenalty=10000
\exhyphenpenalty=10000

«Para o anarquista, a {\bf LIBERDADE} não é um conceito
abstrato e filosófico, mas a possibilidade concreta essencial para
todo ser humano desenvolver completamente todas as
faculdades, as capacidades e os talentos com os quais a natureza o dotou, e
convertê-los em {\bf VALOR SOCIAL}.»

\vfill\scale[factor=5]{{\bf Rudolf Rocker} citado por Noam Chomsky no capítulo}\setupinterlinespace[line=1.5ex]\scale[factor=5]{«Notas sobre o anarquismo».}
\page %---------------------------------------------------------|

\Hedra

\stoptext %---------------------------------------------------------|

