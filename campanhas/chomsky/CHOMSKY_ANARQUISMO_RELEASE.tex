\setuppapersize[A4]
\usecolors[crayola]
\setupbackgrounds[paper][background=color,backgroundcolor=Almond]
	
	\definefontfeature
		[default]
		[default]
		[expansion=quality,protrusion=quality,onum=yes]
	\setupalign[fullhz,hanging]
	\definefontfamily [mainface] [sf] [Formular]
	\setupbodyfont[mainface,11pt]

% Indenting [4.4 cont-enp.p.65]
			\setupindenting[yes, 3ex]  % none small medium big next first dimension
			\indenting[next]           % never not no yes always first next
			
			% [cont-ent.p.76]
			\setupspacing[broad]  %broad packed
			% O tamanho do espaço entre o ponto final e o começo de uma sentença. 


\startsetups[grid][mypenalties]
    \setdefaultpenalties
    \setpenalties\widowpenalties{2}{10000}
    \setpenalties\clubpenalties {2}{10000}
\stopsetups

\setuppagenumbering
  [location={}]            % Estilo dos números de páginat

\setuphead[subject]
[style=bfb]		

\setuplayout[
          location=middle,
          %
          leftedge=0mm,
          leftedgedistance=0mm,
          leftmargin=20mm,
          leftmargindistance=0mm,
          width=100mm,
          rightmargindistance=0mm,
          rightmargin=20mm,
          rightedgedistance=0mm,
          rightedge=0mm,
          backspace=20mm,
          %
          top=21mm,
          topdistance=0mm,
          header=0mm,
          headerdistance=0mm,
          height=250mm,
          footerdistance=0mm,
          footer=0mm,
          bottomdistance=0mm,
          bottom=21mm,
          topspace=21mm,
        setups=mypenalties,
]

\setupalign[right]

\starttext
{\bfb Coletânea inédita de Noam \\
Chomsky sobre anarquismo}

\blank[big]

\noindent{\it Maior compilação de Noam Chomsky já
publicada sobre o assunto,} Notas sobre anarquismo {\it conta com oito entrevistas e dois artigos,
expõe pontos de vista acerca das bases ideológicas esquerda
fundamentam sua análise e sua proposta estratégica de transformação
social.}

\blank[1cm]

\inoutermargin[width=60mm,hoffset=1cm,style=tfx,,voffset=3.5cm]{
\externalfigure[CHOMSKY_ANARQUISMO_THUMB][width=60mm]
}


\inoutermargin[width=70mm,hoffset=1cm,voffset=4.5cm,style=tfx]
{\noindent{\bf Título} {\em Notas sobre anarquismo}\\
{\bf Autor} Noam Chosmky\\
{\bf Organizadores} Felipe Corrêa, Rodrigo Rosa, Bruna Mantese, Pablo Ortellado, Arthur Dantas e Ruy Fernando Cavalheiro\\
{\bf Editora} Hedra\\
{\bf ISBN} 978-85-7715-752-5\\
{\bf Pág.} 224\\
{\bf Pré-venda} XXXX\\
{\bf Lançamento} XXXX\\
{\bf Preço} R\$\,XXXX
}




\noindent Em {\em Notas sobre anarquismo}, Noam Chomsky, um dos
maiores intelectuais vivos da esquerda, defende suas posições sobre o
anarquismo, e afirma uma concepção significativamente eclética e
antidogmática, cuja filiação ideológica seria proveniente de uma união
entre o socialismo e o liberalismo.

A partir de autores como
Mikhail Bakunin, Piotr Kropotkin e Rudolf Rocker, Noam Chomsky defende e
apresenta uma concepção anarquista eclética e antidogmática, proveniente da
união entre o socialismo e pensamento libertário.

Defendendo como princípio fundamental a luta e o combate às estruturas
autoritárias de poder responsáveis pela dominação em todos os níveis, Chomsky
critica o socialismo de Estado, de inspiração leninista, que restringiu os
espaços de liberdade e reforçou instituições repressoras como o partido
único. 

Chomsky discute ainda estratégias de lutas populares, que conciliam reformas de
curto prazo com a busca de um horizonte revolucionário com ganhos reais em
relação às empresas e ao Estado. E surpreende aqueles para quem o anarquismo é
essencialmente uma luta contra o poder, ao propor que o Estado, por vezes,
precisa ser reforçado, pois só ele pode impedir “tiranias ainda piores”,
estabelecidas pelos poderes privados das corporações capitalistas, “que vêm
atacando os progressos conquistados em benefício da democracia e dos direitos
humanos”.

A edição conta com a organização de Felipe Corrêa, Rodrigo Rosa, Bruna Mantese, Pablo Ortellado, Arthur Dantas e Ruy Fernando Cavalheiro, e introdução de Alexandre Samis.

\blank[big]

% \subject{Sobre os organizadores}

% \startitemize
%   \item
    
\page
\subject{Trechos do livro}

  \startitemize
    \item
    {\bf Capítulo {\em Anarquismo, marxismo e expectativas para o futuro}}

    \startitemize
    \item
      Eu me encantei pelo anarquismo ainda jovem, assim que comecei a
pensar no mundo para além de uma perspectiva bastante limitada, e não vi
muitos motivos, desde então, para modificar aquelas antigas atitudes.
Creio que o anarquismo só tem sentido ao buscar e identificar estruturas
autoritárias, hierarquia e dominação em todos aspectos da vida, e
questioná-las, e a não ser que se justifiquem, estas estruturas são
ilegítimas e devem ser desmanteladas.




% Um anarquista coerente deve se opor à propriedade privada dos meios de produção e à escravidão salarial, que é um componente desse sistema, considerando-os incompatíveis com o princípio de que o trabalho deve ser livremente empreendido e estar sob o controle dos produtores. 

    \stopitemize
  \item
    {\bf Capítulo {\em Metas e projetos}}

 Tendemos a considerar as estruturas resultantes do poder imutáveis, praticamente como partes da natureza. Elas são tudo menos isso.

    \startitemize
    \item
    \stopitemize
 
  \item
    {\bf Capítulo {\em }}

    \startitemize
    \item
    
    \stopitemize
  \stopitemize

\stoptext