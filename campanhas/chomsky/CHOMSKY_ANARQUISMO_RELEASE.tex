\setuppapersize[A4]
\usecolors[crayola]
\setupbackgrounds[paper][background=color,backgroundcolor=Almond]
	
	\definefontfeature
		[default]
		[default]
		[expansion=quality,protrusion=quality,onum=yes]
	\setupalign[fullhz,hanging]
	\definefontfamily [mainface] [sf] [Formular]
	\setupbodyfont[mainface,11pt]

% Indenting [4.4 cont-enp.p.65]
			\setupindenting[yes, 3ex]  % none small medium big next first dimension
			\indenting[next]           % never not no yes always first next
			
			% [cont-ent.p.76]
			\setupspacing[broad]  %broad packed
			% O tamanho do espaço entre o ponto final e o começo de uma sentença. 


\startsetups[grid][mypenalties]
    \setdefaultpenalties
    \setpenalties\widowpenalties{2}{10000}
    \setpenalties\clubpenalties {2}{10000}
\stopsetups

\setuppagenumbering
  [location={}]            % Estilo dos números de páginat

\setuphead[subject]
[style=bfb]		

\setuplayout[
          location=middle,
          %
          leftedge=0mm,
          leftedgedistance=0mm,
          leftmargin=20mm,
          leftmargindistance=0mm,
          width=100mm,
          rightmargindistance=0mm,
          rightmargin=20mm,
          rightedgedistance=0mm,
          rightedge=0mm,
          backspace=20mm,
          %
          top=21mm,
          topdistance=0mm,
          header=0mm,
          headerdistance=0mm,
          height=250mm,
          footerdistance=0mm,
          footer=0mm,
          bottomdistance=0mm,
          bottom=21mm,
          topspace=21mm,
        setups=mypenalties,
]

\setupalign[right]

\starttext
{\bfb Coletânea inédita de Noam \\
Chomsky sobre anarquismo}

\blank[big]

\noindent{\it Maior compilação de Noam Chomsky já
publicada sobre o assunto,} Notas sobre anarquismo {\it reune oito entrevistas e dois artigos em que o filósofo
expõe pontos de vista acerca das bases ideológicas da esquerda
que fundamentam sua análise e sua proposta estratégica de transformação
social.}

\blank[1cm]

\inoutermargin[width=60mm,hoffset=1cm,style=tfx,,voffset=3.5cm]{
\externalfigure[CHOMSKY_ANARQUISMO_THUMB][width=60mm]
}


\inoutermargin[width=70mm,hoffset=1cm,voffset=4.5cm,style=tfx]
{\noindent{\bf Título} {\em Notas sobre anarquismo}\\
{\bf Autor} Noam Chosmky\\
{\bf Organizadores} Felipe Corrêa, Rodrigo Rosa, Bruna Mantese, Pablo Ortellado, Arthur Dantas e Ruy Fernando Cavalheiro\\
{\bf Editora} Hedra\\
{\bf ISBN} 978-85-7715-752-5\\
{\bf Edição} 2ª\\
{\bf Pág.} 224\\
{\bf Pré-venda} XXXX\\
{\bf Lançamento} XXXX\\
{\bf Preço} R\$\,XXXX
}



\noindent Em {\em Notas sobre anarquismo}, Noam Chomsky, um dos maiores intelectuais vivos da esquerda, apresenta uma concepção anarquista eclética e antidogmática, proveniente de uma união entre o socialismo e o liberalismo. Partindo de autores como Mikhail Bakunin, Piotr Kropotkin e Rudolf Rocker, Chomsky desafia conceitos arraigados e propõe novas perspectivas.

O filósofo surpreende ao defender que instituições de poder ilegítimas, como o Estado, deveriam ser eventualmente reforçadas para impedir “sistemas de opressão ainda piores”, estabelecidos pelos poderes privados das corporações capitalistas, “dedicados a atacar os progressos que foram conquistados na extensão da democracia e dos direitos humanos”.
 
Assim, ao assumir posições pragmáticas que priorizam o bem-estar social, Chomsky desafia a noção convencional de que o anarquismo é uma luta indiscriminada contra o poder. No entanto, ele reafirma como princípio fundamental o combate às estruturas autoritárias de poder responsáveis pela dominação em todos os níveis, criticando especialmente o socialismo de Estado de inspiração leninista, que restringiu os espaços de liberdade e fortaleceu instituições repressoras, como o partido único.

Essa reunião inédita de textos proporciona uma visão única e esclarecedora das ideias do autor sobre a luta contra as estruturas autoritárias. Discutindo estratégias de lutas populares que conciliam reformas de curto prazo com a busca de um horizonte revolucionário, Chomsky evidencia seu compromisso com uma transformação social genuinamente libertadora.



% \noindent Em {\em Notas sobre anarquismo}, Noam Chomsky, um dos
% maiores intelectuais vivos da esquerda, defende suas posições sobre o
% anarquismo e, a partir de autores como Mikhail Bakunin, Piotr Kropotkin e Rudolf Rocker, apresenta uma concepção anarquista eclética e antidogmática, proveniente de uma união
% entre o socialismo e o liberalismo.
 
% Defendendo como princípio fundamental a luta e o combate às estruturas
% autoritárias de poder responsáveis pela dominação em todos os níveis, Chomsky
% critica o socialismo de Estado, de inspiração leninista, que restringiu os
% espaços de liberdade e reforçou instituições repressoras como o partido
% único. 

% Chomsky discute ainda estratégias de lutas populares, que conciliam reformas de
% curto prazo com a busca de um horizonte revolucionário com ganhos reais em
% relação às empresas e ao Estado. E surpreende aqueles para quem o anarquismo é
% essencialmente uma luta contra o poder, ao propor que o Estado, por vezes,
% precisa ser reforçado, pois só ele pode impedir “tiranias ainda piores”,
% estabelecidas pelos poderes privados das corporações capitalistas, “que vêm
% atacando os progressos conquistados em benefício da democracia e dos direitos
% humanos”.

% A edição conta com a organização de Felipe Corrêa, Rodrigo Rosa, Bruna Mantese, Pablo Ortellado, Arthur Dantas e Ruy Fernando Cavalheiro, e introdução de Alexandre Samis.

\blank[big]

% \subject{Sobre os organizadores}

% \startitemize
%   \item
    
\page
\subject{Trechos do livro}

  \startitemize
    \item
    {\bf Capítulo {\em Anarquismo, marxismo e expectativas para o futuro}}

    \startitemize
    \item
      Eu me encantei pelo anarquismo ainda jovem, assim que comecei a
pensar no mundo para além de uma perspectiva bastante limitada, e não vi
muitos motivos, desde então, para modificar aquelas antigas atitudes.
Creio que o anarquismo só tem sentido ao buscar e identificar estruturas
autoritárias, hierarquia e dominação em todos aspectos da vida, e
questioná-las, e a não ser que se justifiquem, estas estruturas são
ilegítimas e devem ser desmanteladas.


    \stopitemize
  
  \item
    {\bf Capítulo {\em Metas e projetos}}


    \startitemize
    \item  Tendemos a considerar as estruturas resultantes do poder imutáveis, praticamente como partes da natureza. Elas são tudo menos isso. Essas formas de tirania privada só chegaram a algo próximo de sua forma atual, com os direitos de pessoas imortais, no início deste século. 

    \stopitemize

  \item
    {\bf Capítulo {\em Metas e projetos }}

    \startitemize
    \item Minha meta de curto prazo é defender, e até reforçar
elementos da autoridade do Estado que, embora sejam ilegítimos em
seus fundamentos, são decisivamente necessários neste momento para
impedir os esforços que vêm atacando os progressos que foram
conseguidos em benefício da democracia e dos direitos humanos. A
autoridade de Estado está agora sob severo ataque nas sociedades mais
democráticas, mas isso não em benefício do projeto libertário.
Justamente o oposto: porque ela oferece (fraca) proteção a alguns
aspectos desse projeto. Os governos têm uma importante falha: diferente
das tiranias privadas, as instituições de poder e
a autoridade do Estado oferecem ao desprezado público uma oportunidade
de desempenhar algum papel, ainda que limitado, na gestão de seus
próprios assuntos.
    
    \stopitemize
  
% \item
%     {\bf Capítulo {\em Anarquismo, intelectuais e Estado}}

%     \startitemize
%     \item Do meu ponto de vista (e do ponto de vista de
% alguns outros), o Estado é uma instituição
% ilegítima. Mas disso não decorre que você não deva apoiar o Estado.
% Talvez haja uma instituição ainda mais ilegítima, que vai tomar conta
% se você não apoiar essa instituição ilegítima.
  


%   \stopitemize
  

% \item
%     {\bf Capítulo {\em Noam Chomsky o anarquismo}}

%     \startitemize
%     \item Pessoalmente, não confio em meus próprios projetos sobre o “caminho certo“ e não me impressiono com as presunçosas declarações de outros, incluindo aqueles que são bons amigos. Sinto que muito pouco é conhecido para termos a capacidade de dizer muito, com alguma confiança. Podemos tentar formular nossos projetos de longo prazo, nossos objetivos, nossos ideais; e podemos (e devemos) nos dedicar a trabalhar em questões de importância à humanidade. Mas a lacuna entre essas duas possibilidades é frequentemente considerável, e eu raramente vejo algum caminho para transpô-la, exceto em um nível vago e genérico.

  \stopitemize

\stoptext