% POVOS_LIVRO_EFEMERIDE.tex
% Preencher com o nome das cor ou composição RGB (ex: [r=0.862, g=0.118, b=0.118]) 
\usecolors[crayola] 			   % Paleta de cores pré-definida: wiki.contextgarden.net/Color#Pre-defined_colors

% Cores definidas pelo designer:
% MyGreen		r=0.251, g=0.678, b=0.290 % 40ad4a
% MyCyan		r=0.188, g=0.749, b=0.741 % 30bfbd
% MyRed			r=0.820, g=0.141, b=0.161 % d12429
% MyPink		r=0.980, g=0.780, b=0.761 % fac7c2
% MyGray		r=0.812, g=0.788, b=0.780 % cfc9c7
% MyOrange		r=0.980, g=0.671, b=0.290 % faab4a

% Configuração de cores
\definecolor[MyColor][MaximumBluePurple]      % ou ex: [r=0.862, g=0.118, b=0.118] % corresponde a RGB(220, 30, 30)
\definecolor[MyColorText][black]     % ou ex: [r=0.862, g=0.118, b=0.118] % corresponde a RGB(167, 169, 172)

% Classe para diagramação dos posts
\environment{marketing.env}		   

\starttext %---------------------------------------------------------|

\hyphenpenalty=10000
\exhyphenpenalty=10000

\Mensagem{9 DE AGOSTO} %Sempre usar esse header

\MyPicture{diadospovos}

\vfill\scale[factor=6]{\Seta\,DIA INTERNACIONAL DOS {\bf POVOS INDÍGENAS}}

\page %---------------------------------------------------------| 

\hyphenpenalty=10000
\exhyphenpenalty=10000

Criado pela {\cap ONU} em 1994, o {\bf DIA INTERNACIONAL DOS POVOS INDÍGENAS} busca homenagear e reconhecer as tradições dos povos originários, reafirmando as garantias previstas na Declaração das Nações Unidas sobre os Direitos dos Povos Indígenas. 

\page %---------------------------------------------------------|

A origem dessa celebração remonta à primeira reunião do {\bf GRUPO DE TRABALHO DAS NAÇÕES UNIDAS SOBRE POPULAÇÕES INDÍGENAS}, em Genebra, em 9 de agosto de 1982. 

\page

Esse órgão inclusive participou do processo de estudos e debates que levaria à formulação da {\bf DECLARAÇÃO SOBRE OS DIREITOS DOS POVOS INDÍGENAS} em 2007, além de mobilizar uma discussão entorno da garantia de melhores condições de vida e aplicação efetiva de seus direitos, como o do resgate de sua cultura e o direito à terra.

\page
\MyCover{XAMANISMOS_THUMB}

\page %---------------------------------------------------------|

\Hedra

\stoptext %---------------------------------------------------------|



