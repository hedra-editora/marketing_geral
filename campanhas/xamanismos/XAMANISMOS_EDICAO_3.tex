% XAMANISMOS_EDICAO_2.tex
% Preencher com o nome das cor ou composição RGB (ex: [r=0.862, g=0.118, b=0.118]) 
\usecolors[crayola] 			   % Paleta de cores pré-definida: wiki.contextgarden.net/Color#Pre-defined_colors

% Cores definidas pelo designer:
% MyGreen		r=0.251, g=0.678, b=0.290 % 40ad4a
% MyCyan		r=0.188, g=0.749, b=0.741 % 30bfbd
% MyRed			r=0.820, g=0.141, b=0.161 % d12429
% MyPink		r=0.980, g=0.780, b=0.761 % fac7c2
% MyGray		r=0.812, g=0.788, b=0.780 % cfc9c7
% MyOrange		r=0.980, g=0.671, b=0.290 % faab4a

% Configuração de cores

\definecolor[MyColor][MaximumBluePurple]      % ou ex: [r=0.862, g=0.118, b=0.118] % corresponde a RGB(220, 30, 30)
\definecolor[MyColorText][black]     % ou ex: [r=0.862, g=0.118, b=0.118] % corresponde a RGB(167, 169, 172)

% Classe para diagramação dos posts
\environment{marketing.env}		   

\def\MyPhoto#1{\startpositioning
                \position(0,0mm){\externalfigure[#1][width=\textwidth, factor=max]}
                \stoppositioning}



\starttext %---------------------------------------------------------|

\Mensagem{POR DENTRO DA EDIÇÃO}

\startMyCampaign

\hyphenpenalty=10000
\exhyphenpenalty=10000

{\bf XAMANISMOS}\ 
E AS SUAS 
IMAGENS\ 

\stopMyCampaign


\page %---------------------------------------------------------| 

\MyPhoto{XAMANISMOS_IMAGEM1}



\scale[factor=4]{{\it Onça-homem olmeca}. Figura de Claudia Ros, baseada no {\it Machado}}\setupinterlinespace[line=1.5ex]\scale[factor=4]{{\it cerimonial olmeca}.}

\page %---------------------------------------------------------|


\placefigure{}{\externalfigure[XAMANISMOS_IMAGEM_3][width=.8\textwidth, factor=max]}


\scale[factor=4]{Abe e Bʉhpo, “Sol” e “Trovão”, seres ancestrais do gaapi pahti, “mundo}\setupinterlinespace[line=1.5ex]\scale[factor=4]{de gaapi”.}


\page

\placefigure{}{\externalfigure[XAMANISMOS_IMAGEM_4][width=.8\textwidth, factor=max]}

\vfill\scale[factor=4]{{\it Gaapi wahro}, cuia especial para oferenda do gaapi.}

\page
\MyPhoto{XAMANISMOS_IMAGEM_5}
\vfill\scale[factor=4]{Xamã kaiowá portando os instrumentos {\it xiru}, a cruz, e {\it mbaraka},}\setupinterlinespace[line=1.5ex]\scale[factor=4]{o chocalho, durante um jeroky na Reserva Indígena de Amambai ({\cap MS}).}

\page


\MyPhoto{XAMANISMOS_IMAGEM_2}

\vfill\scale[factor=4]{Máscara atujuwá yanumaka do Apapaatai Iyãu de Itsakumã dançando}\setupinterlinespace[line=1.5ex]\scale[factor=4]{no centro da aldeia Piyulaga, 1991.} 
\page

\MyCover{XAMANISMOS_THUMB}

\page

\Hedra

\stoptext %---------------------------------------------------------|