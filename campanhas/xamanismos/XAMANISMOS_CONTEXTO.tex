% AUTOR_LIVRO_CURIOSIDADE.tex
% Vamos falar sobre isso "curiosidades"
% > "EM CONTEXTO"

% Preencher com o nome das cor ou composição RGB (ex: [r=0.862, g=0.118, b=0.118]) 
\usecolors[crayola] 			   % Paleta de cores pré-definida: wiki.contextgarden.net/Color#Pre-defined_colors

% Cores definidas pelo designer:
% MyGreen		r=0.251, g=0.678, b=0.290 % 40ad4a
% MyCyan		r=0.188, g=0.749, b=0.741 % 30bfbd
% MyRed			r=0.820, g=0.141, b=0.161 % d12429
% MyPink		r=0.980, g=0.780, b=0.761 % fac7c2
% MyGray		r=0.812, g=0.788, b=0.780 % cfc9c7
% MyOrange		r=0.980, g=0.671, b=0.290 % faab4a

% Configuração de cores
\definecolor[MyColor][MaximumBluePurple]      % ou ex: [r=0.862, g=0.118, b=0.118] % corresponde a RGB(220, 30, 30)
\definecolor[MyColorText][black]     % ou ex: [r=0.862, g=0.118, b=0.118] % corresponde a RGB(167, 169, 172)

% Classe para diagramação dos posts
\environment{marketing.env}		   

\setupnarrower[right=3cm,left=3cm]

\starttext %---------------------------------------------------------|

\hyphenpenalty=10000
\exhyphenpenalty=10000

\Mensagem{EM CONTEXTO} %Sempre usar esse header

\startMyCampaign

\hyphenpenalty=10000
\exhyphenpenalty=10000

%Contra Mircea Eliade
% O xamanismo para além de Mircea Eliade
{\bf COMO O XAMANISMO É~ATUAL?}

\blank[medium]

Para além das ideias ultrapassadas de {\bf MIRCEA ELIADE}

\stopMyCampaign

\page %---------------------------------------------------------| 

\hyphenpenalty=10000
\exhyphenpenalty=10000

\starttikzpicture[remember picture,overlay]
\node at (6,-5)
{\externalfigure
              [./IMAGETICAS_NEURATH_1.png]
              [width=.7\textwidth]};
\stoptikzpicture

\startnarrower[right]
Uma das definições mais clássicas do xamanismo vem do intelectual romeno Mircea Eliade. Para o autor de {\it História das Religiões}, o xamanismo seria caracterizado pelo uso das “técnicas arcaicas do êxtase” e por uma “cosmovisão xamanística”.
\stopnarrower

\page %---------------------------------------------------------|

\hyphenpenalty=10000
\exhyphenpenalty=10000

\starttikzpicture[remember picture,overlay]
\node at (2.5,-5)
{\externalfigure
              [./IMAGETICAS_NEURATH_2.png]
              [width=.7\textwidth]};

\node[anchor=east, text width=4cm,font=\tfxx] at (9,-6) {{\it O Feiticeiro de Trois Frères}. Figura de
Claudia Ros baseada em {\it O feiticeiro de Trois-Frères}, de Henri
Breuil.};
\stoptikzpicture

%\startnarrower[left]
Baseado numa pesquisa de gabinete, Eliade enfatizava o caráter {\bf UNIVERSAL} e {\bf ARCAICO} do xamanismo, que teria doutrinas e liturgias bem definidas.
%\stopnarrower

\page

Eliade, assim, imprime ao xamanismo uma ideia de arcaísmo e continuidade milenar.
No entanto, em vez de uma prática {\bf MILENAR IMÓVEL}, os xamanismos se alteram com o tempo, bem como as culturas indígenas. As {\bf PRÁTICAS XAMÂNICAS} se nutrem das relações com os outros. Na abertura ao outro, essas práticas se alteram em resposta ao mundo no qual estão inseridas.

\page

\MyCover{XAMANISMOS_THUMB}

\page %---------------------------------------------------------|

\Hedra

\stoptext %---------------------------------------------------------|