% Preencher com o nome das cor ou composição RGB (ex: [r=0.862, g=0.118, b=0.118]) 
\usecolors[crayola] 			   % Paleta de cores pré-definida: wiki.contextgarden.net/Color#Pre-defined_colors

% Cores definidas pelo designer:
% MyGreen		r=0.251, g=0.678, b=0.290 % 40ad4a
% MyCyan		r=0.188, g=0.749, b=0.741 % 30bfbd
% MyRed			r=0.820, g=0.141, b=0.161 % d12429
% MyPink		r=0.980, g=0.780, b=0.761 % fac7c2
% MyGray		r=0.812, g=0.788, b=0.780 % cfc9c7
% MyOrange		r=0.980, g=0.671, b=0.290 % faab4a

% Configuração de cores
\definecolor[MyColor][MaximumBluePurple]      % ou ex: [r=0.862, g=0.118, b=0.118] % corresponde a RGB(220, 30, 30)
\definecolor[MyColorText][black]  % ou ex: [r=0.862, g=0.118, b=0.118] % corresponde a RGB(167, 169, 172)

% Classe para diagramação dos posts
\environment{marketing.env}		   

\starttext %---------------------------------------------------------|

\hyphenpenalty=10000
\exhyphenpenalty=10000

\Mensagem{19 DE ABRIL} %Sempre usar esse header

\MyPicture{XAMANISMOS_EFEMERIDE_FOTO}

% {\scale[factor=4]{\Seta\,Foto do livro {\bf XAMANISMOS AMERÍNDIOS}, {capítulo «{\it Ohendu}, aprendendo a cantar.»}}

% \scale[]{Rezadoras} kaiowá acompanhadas de uma jovem aprendiz dançam
% seus cantos \emph{mborahéi} em recepção aos convidados da festa \emph{avatikyry} em
% Guyra Kambi'y, \textsc{ti} Panambi/\,Lagoa Rica, Douradina (\textsc{ms}). Foto: Tatiane
% Klein, 2018.}}

\vfill\scale[factor=6]{\Seta\,DIA DOS {\bf POVOS INDÍGENAS}}

\page %---------------------------------------------------------| 

\hyphenpenalty=10000
\exhyphenpenalty=10000

O dia 19 de abril é dedicado à questão indígena em diversos países do continente americano. Essa é a data da primeira discussão compartilhada entre lideranças indígenas e indigenistas no Congresso Indigenista Interamericano realizado no México, em 1940. Hoje, no Brasil, ele é designado {\bf DIA DOS POVOS INDÍGENAS}, em reconhecimento à diversidade das coletividades indígenas no Brasil, bem como aos seus direitos originários à terra.

\page %---------------------------------------------------------|

A Coleção Mundo Indígena da Hedra vem se construindo em colaboração com {\bf DIFERENTES LIDERANÇAS, PENSADORES E COLETIVIDADES INDÍGENAS}, bem como aliados não indígenas dedicados à garantia de seus direitos.

\page %---------------------------------------------------------|

O novo título da coleção,\\ {\bf XAMANISMOS AMERÍNDIOS}, apresenta um caleidoscópio de {\bf COSMOLOGIAS E ONTOLOGIAS DE POVOS INDÍGENAS} de norte a sul do continente, celebrando essa diversidade e a importância do direito dos povos indígenas à diferença em todas as suas dimensões.

\page %---------------------------------------------------------|


\MyCover{XAMANISMOS_THUMB}

\page %---------------------------------------------------------|

\Hedra

\stoptext %---------------------------------------------------------|

% A partir do ano de 2022, o dia 19 de abril passou a ser chamado de {\bf DIA DOS POVOS INDÍGENAS} em vez de Dia do Índio. Essa mudança promove uma abordagem mais inclusiva e respeitosa, quebrando com a ideia estereotipada de índio e reconhecendo as {\bf PLURALIDADES} étnicas e culturais dos povos originários.

% A coleção {\bf MUNDO INDÍGENA} enquadra-se nesse esforço de propagar os conhecimentos e modos de existência desses povos diversos, reunindo estudos {\bf ANTROPOLÓGICOS} sobre suas práticas e culturas, além de {\bf NARRATIVAS INDÍGENAS}. Essas integram um acervo destinado a apoiar os processos de educação, formação, e preservação de suas histórias dentro de suas comunidades.

% O mais novo lançamento da coleção é {\bf XAMANISMOS AMERÍNDIOS}, que contesta o conceito homogêneo de 
% {\it xamã}, confrontando-o com as multiplicidades e particulares dos líderes e práticas espirituais de diferentes {\bf POVOS NATIVOS} das Américas.
