\setuppapersize[A4]
\usecolors[crayola]
\setupbackgrounds[paper][background=color,backgroundcolor=Almond]
	
	\definefontfeature
		[default]
		[default]
		[expansion=quality,protrusion=quality,onum=yes]
	\setupalign[fullhz,hanging]
	\definefontfamily [mainface] [sf] [Formular]
	\setupbodyfont[mainface,11pt]

% Indenting [4.4 cont-enp.p.65]
			\setupindenting[yes, 3ex]  % none small medium big next first dimension
			\indenting[next]           % never not no yes always first next
			
			% [cont-ent.p.76]
			\setupspacing[broad]  %broad packed
			% O tamanho do espaço entre o ponto final e o começo de uma sentença. 


\startsetups[grid][mypenalties]
    \setdefaultpenalties
    \setpenalties\widowpenalties{2}{10000}
    \setpenalties\clubpenalties {2}{10000}
\stopsetups

\setuppagenumbering
  [location={}]            % Estilo dos números de páginat

\setuphead[subject]
[style=bfb]		

\setuplayout[
          location=middle,
          %
          leftedge=0mm,
          leftedgedistance=0mm,
          leftmargin=20mm,
          leftmargindistance=0mm,
          width=100mm,
          rightmargindistance=0mm,
          rightmargin=20mm,
          rightedgedistance=0mm,
          rightedge=0mm,
          backspace=20mm,
          %
          top=21mm,
          topdistance=0mm,
          header=0mm,
          headerdistance=0mm,
          height=250mm,
          footerdistance=0mm,
          footer=0mm,
          bottomdistance=0mm,
          bottom=21mm,
          topspace=21mm,
        setups=mypenalties,
]

\setupalign[right]

\starttext
{\bfb O xamanismo sob a perspectiva de 20 povos}

\blank[big]

\noindent{\it Primeira publicação no Brasil, depois de duas décadas, dedicada à reflexão sobre a prática xamânica de diferentes povos da América}, Xamanismos ameríndios {\it promove uma virada na compreensão dos modos de pensar e agir entendidos como xamanismos}

\blank[1cm]

\inoutermargin[width=60mm,hoffset=1cm,style=tfx,,voffset=3.5cm]{
\externalfigure[XAMANISMOS_THUMB.pdf][width=60mm]
}


\inoutermargin[width=70mm,hoffset=1cm,voffset=4.5cm,style=tfx]{
\noindent{\bf Título} {\em Xamanismos ameríndios: Expressões sensíveis e ações cosmopolíticas}\\
{\bf Autor} Aristoteles Barcelos Neto, Laura Pérez Gil e Danilo Paiva Ramos (orgs.)\\
{\bf Editora} Hedra\\
{\bf ISBN} 978-85-7715-986-4\\
{\bf Pág.} 504\\
{\bf Pré-venda} 08/04\\
{\bf Lançamento} 19/04\\
{\bf Preço} R\$\,117,00
}

\noindent{\em Xamanismos ameríndios} reúne artigos de
autores indígenas e não indígenas a respeito dos xamanismos de 20
povos das Américas, originários de diversas regiões da Amazônia,
Aridoamérica, Brasil Central, Canadá, Chaco, Llanos de Venezuela e
Mesoamérica. São 18 textos,
nos quais são explorados modos xamânicos particulares de conceber o
mundo e agir nele: da permeação estética --- a agência no mundo por
meio da beleza e da poesia --- associada à centralidade do sensível
através de sons, cheiros, imagens, texturas e movimentos à sua
{\em potência transformativa} de criar e transformar mundos. Explora
também as alterações das práticas xamânicas ao longo do tempo e sua
consequente abertura ao outro, em resposta ao mundo no qual estão
inseridas.

A coletânea abrange povos indígenas
da América do Norte --- Olmeca, Wixárika, Tepehuano, Masewal e Atikamekw
--- e da América do Sul --- Kaiowá, Guarani, Qom, Wauja, Huni Kuin, Yaminawa,
Kulina, Yuhupdëh, Hupd’äh, Desana, Tariano, Ye’pâmasa (Makuna), Pumé,
Tukano, Mebengokré (Xikrin). Ao reunir tal diversidade de trabalhos, o livro
estabelece ressonâncias, aberturas, articulações e transversalidades xamânicas
perpassadas pela diversidade linguística, geográfica, temporal e cosmológica.

O livro foi organizado a partir de dois grupos de trabalho, um no
VII Congresso da Associação Portuguesa de Antropologia e outro na 6ª
Reunião Equatorial de Antropologia, ambos em 2019. 

\blank[big]

\subject{Sobre os organizadores}

\startitemize
  \item
    {\bf Aristoteles Barcelos} é professor associado e diretor de curso
    na Sainsbury Research Unit for the Arts of Africa, Oceania and the
    Americas (University of East Anglia, Norwich). É também museólogo no
    Museu Indígena Ulupuwene do Povo Wauja do Alto Xingu, professor
    colaborador do Centro de Estudos Mesoamericanos e Andinos da
    Universidade de São Paulo (USP), {\em fellow} da British Higher
    Education Academy e membro do Conselho Internacional de Museus.
    Recebeu o Prêmio CNPq--Anpocs de melhor tese de doutorado em
    Ciências Sociais. Foi pesquisador visitante do CNPq no Laboratoire
    d'Anthropologie Sociale du Collège de France, da FAPESP na USP e do
    convênio Newton Fund/FAPESP na UNICAMP.
  \item
    {\bf Laura Pérez Gil} é professora de Antropologia Americana na
    Universidad Complutense de Madrid, do departamento da Antropologia
    (em excedência) e do programa de pós-graduação em Antropologia e
    Arqueologia da Universidade Federal do Paraná. Foi responsável pelas
    coleções de etnologia indígena, e posteriormente coordenadora do
    Museu de Arqueologia e Etnologia da mesma universidade. Atua na área
    de Etnologia Indígena das Terras Baixas da América do Sul com povos
    de língua pano, especialmente sobre xamanismo, corporalidade,
    organização social e processos históricos. Igualmente, desenvolve e
    orienta pesquisas sobre coleções etnográficas em museus.
  \item
    {\bf Danilo Paiva Ramos} é antropólogo, professor adjunto C do
    departamento de Ciências Humanas da Universidade Federal de Alfenas
    (UNIFAL--MG) e professor efetivo do programa de pós-graduação em
    Antropologia Social da Universidade Federal de São Carlos
    (PPGAS--UFSCar). É líder do grupo de pesquisa em Etnologia,
    Linguística e Saúde Indígena (ETNOLINSI, do CNPq). Desenvolve
    pesquisas em etnologia indígena, com ênfase em estudos sobre
    xamanismo, línguas indígenas, arte verbal e saúde indígena. É membro
    do Coletivo de Apoio aos Povos Yuhupdëh, Hupd'äh, Dâw e Nadëb
    (CAPYHDN) e da Associação Saúde Sem Limites (SSL).
  \stopitemize

\page
\subject{Trechos do livro}

  \startitemize
    \item
    {\bf Capítulo {\em A flauta-jaguar e o xamanismo wauja}}

    \startitemize
    \item
      A jaguaridade é um tema complexo na Amazônia e sua indexicalidade
      é amplamente heterogênea {[}...{]} não é completamente capturada
      por indexes materiais ou sensíveis; ela é constituída, antes de
      tudo, por ideias de poder e de transformação {[}...{]} Seguindo
      essa referência da identificação do jaguar como cachorro, retomo
      um trecho da narrativa de Itsakumã: “ele (o jaguar) veio andando
      na minha direção, como se eu fosse o dono dele, como se ele fosse
      o meu cachorro”. Essa aparição, de natureza claramente xamânica,
      era o prenúncio de que Itsakumã se tonaria, de fato, dono dos
      jaguares. A mansidão indicava uma aproximação entre iguais, pois,
      logo na noite depois do encontro na estrada, a alma de Itsakumã
      seria novamente levada pelos jaguares para suas aldeias para comer
      carne humana. Essa ideia de poder contida na {\em jaguaridade}
      raramente opera desvinculada de uma presença material e sensível.
    \stopitemize
  \item
    {\bf Capítulo {\em Gaapi, a bebida cósmica dos desana}}

    \startitemize
    \item
      {\bf Os seres primordiais}: Antes do mundo existia Ʉmũsĩ Dihtaru,
      onde viviam Abe, “Sol”, e Omẽ Mahʉ, “Trovão”. Nesse lago celeste
      existiam três tipos de linhas de forças: a linha branca de nuvens,
      a linha preta de nuvens e a linha aquosa de nuvens. Além delas, no
      centro do lago, existia um pilar hiperbrilhante. Meu avô Toramü
      Bayaru, Venâncio Ramos, relatou que esse pilar era de breu, e que
      representava o próprio Abe, “Sol”. Por ser muito luminoso, nenhum
      ser podia aproximar-se dele. No ápice do pilar, ainda, havia um
      pedaço circular de pedra, do qual saíam três chamas de fogo: fogo
      de chama vermelha intensa, fogo de chama amarela intensa e fogo de
      chama verde intensa. Ao redor do pilar de breu havia sete cuias em
      círculo. Essas cuias representavam a fonte de vida e a força dos
      seres humanos. O líquido derretido do breu caía na forma de linhas
      e elas se juntavam dentro das cuias. Esse evento significava a
      reconstrução da vida de Abe, e o breu derretido simbolizava a sua
      força essencial. A mancha escura simboliza, por outro lado, a
      força de Bʉhpo, que domina as nuvens --- a força que emana do
      cigarro do {\em kumu}, “conhecedor” ou “xamã".
    \stopitemize
 
  \item
    {\bf Capítulo {\em Ohendu: aprendendo a cantar entre os Kaiowá e
    Guarani~}}

    \startitemize
    \item
      Como explicam os {\em opuraheiva}, os cantos atravessam os muitos
      domínios da vida guarani e são absolutamente fundamentais para a
      sustentação do mundo, por isso o cantar emerge não só como o
      principal fazer dos humanos. Os Kaiowá e Guarani aprendem com as
      vozes dos pássaros, das corredeiras, árvores e pedras, em suma,
      tudo que tem {\em ñe'ẽ}, “linguagem”, de modo que alguns objetos
      de seu xamanismo, como o chocalho {\em mbaraka}, o bastão de ritmo
      {\em takuapu}, a flauta {\em mimby}, e o arco de boca
      {\em guyrapa'i}, ainda que categorizados como instrumentos
      musicais, são dotados de estatutos ontológicos próprios.
    \stopitemize
  \stopitemize

\stoptext