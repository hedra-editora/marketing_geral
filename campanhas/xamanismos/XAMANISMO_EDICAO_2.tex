% XAMANISMOS_EDICAO.tex
% Preencher com o nome das cor ou composição RGB (ex: [r=0.862, g=0.118, b=0.118]) 
\usecolors[crayola] 			   % Paleta de cores pré-definida: wiki.contextgarden.net/Color#Pre-defined_colors

% Cores definidas pelo designer:
% MyGreen		r=0.251, g=0.678, b=0.290 % 40ad4a
% MyCyan		r=0.188, g=0.749, b=0.741 % 30bfbd
% MyRed			r=0.820, g=0.141, b=0.161 % d12429
% MyPink		r=0.980, g=0.780, b=0.761 % fac7c2
% MyGray		r=0.812, g=0.788, b=0.780 % cfc9c7
% MyOrange		r=0.980, g=0.671, b=0.290 % faab4a

% Configuração de cores
\definecolor[MyColor][BlueI]      % ou ex: [r=0.862, g=0.118, b=0.118] % corresponde a RGB(220, 30, 30)
\definecolor[MyColorText][white]     % ou ex: [r=0.862, g=0.118, b=0.118] % corresponde a RGB(167, 169, 172)

% Classe para diagramação dos posts
\environment{marketing.env}		   

\starttext %---------------------------------------------------------|

\Mensagem{xamanismos ameríndios}

\startMyCampaign

\hyphenpenalty=10000
\exhyphenpenalty=10000

\starttikzpicture[remember picture,overlay]
\node at (6.9,-4)
{\externalfigure
              [./XAMANISMOS_EDICAO2_01A.png]
              [width=.5\textwidth]};
\stoptikzpicture


\startnarrower[2*right]
\kern-2ex A ORIGEM ESPIRITUAL DO {\bf GAAPI}, BEBIDA RITUAL PSICOATIVA DO POVO DESANA
\stopnarrower

\vfill
\startnarrower[3*left]
\setupinterlinespace[line=1.4ex]
\setupbodyfont[9pt]\tfxx
{\it Umũkori Mahsu kũũ kahtida},\\ «Fontes de vida, Gente do dia».\\ Desenho: Jaime Diakara (2018)
\stopnarrower

\stopMyCampaign

%\vfill\scale[lines=1.5]{\MyStar[MyColorText][none]}

\page %---------------------------------------------------------| 

\starttikzpicture[remember picture,overlay]
\node at (2,-4)
{\externalfigure
              [./XAMANISMOS_EDICAO2_01B.png]
              [width=.5\textwidth]};
\stoptikzpicture

\hyphenpenalty=10000
\exhyphenpenalty=10000

\startnarrower[5*left]
Os velhos contam que em {\it Diawi}, uma maloca ancestral, nasceu o menino chamado
Gaapi, revestido do poder de {\it Umukori Mahsu}, a «Gente do dia». No momento de seu
nascimento, todos os seres sentiram o poder da bebida.
\stopnarrower

\page %---------------------------------------------------------|
\hyphenpenalty=10000
\exhyphenpenalty=10000

Logo em seguida, a mãe de Gaapi ofereceu a bebida em uma enorme cuia a Bohtari Wõãku, «ser-potência dos esteios da Casa da Emergência», que, imediatamente, sentiu o efeito da bebida. O bebê Gaapi tinha a forma humana e brilhava como o sol. Seu corpo era constituído por vários tipos da planta de {\it gaapi}.

\page

\externalfigure[XAMANISMOS_EDICAO2_02.png][width=\textwidth]


{\vfill\scale[factor=fit]{\Seta\,{\it Gaapi Mahsu}, «Gente de {\it gaapi}», divisão de {\it gaapi} no {\it Diawi}. Desenho: Jaime Diakara (2018)}}

\page
\hyphenpenalty=10000
\exhyphenpenalty=10000

Em {\it Xamanismos ameríndios}, o antropólogo indígena Jaime Diakara escreve também sobre os diferentes tipos de planta {\it gaapi}; sua preparação, efeitos e cuidados; e conta como, sob o efeito do {\it gaapi} agenciado, os participantes acessavam outros mundos de conhecimentos.

\page


\MyCover{XAMANISMOS_THUMB.pdf}

\page %---------------------------------------------------------| 

\Hedra

\stoptext %---------------------------------------------------------|