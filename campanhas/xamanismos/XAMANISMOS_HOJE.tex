% Preencher com o nome das cor ou composição RGB (ex: [r=0.862, g=0.118, b=0.118]) 
\usecolors[crayola] 			   % Paleta de cores pré-definida: wiki.contextgarden.net/Color#Pre-defined_colors

% Cores definidas pelo designer:
% MyGreen		r=0.251, g=0.678, b=0.290 % 40ad4a
% MyCyan		r=0.188, g=0.749, b=0.741 % 30bfbd
% MyRed			r=0.820, g=0.141, b=0.161 % d12429
% MyPink		r=0.980, g=0.780, b=0.761 % fac7c2
% MyGray		r=0.812, g=0.788, b=0.780 % cfc9c7
% MyOrange		r=0.980, g=0.671, b=0.290 % faab4a

% Configuração de cores
\definecolor[MyColor][MaximumBluePurple]      % ou ex: [r=0.862, g=0.118, b=0.118] % corresponde a RGB(220, 30, 30)
\definecolor[MyColorText][black]     % ou ex: [r=0.862, g=0.118, b=0.118] % corresponde a RGB(167, 169, 172)

% Classe para diagramação dos posts
\environment{marketing.env}		   

% Comandos & Instruções %%%%%%%%%%%%%%%%%%%%%%%%%%%%%%%%%%%%%%%%%%%%%%%%%%%%%%%%%%%%%%%|

% Cabeço e rodapé: Informações (caso queira trocar alguma coisa)
 		\def\MensagemSaibaMais 	{SAIBA MAIS:}
 		\def\MensagemSite		{HEDRA.COM.BR}
 		\def\MensagemLink		{LINK NA BIO}

\starttext %---------------------------------------------------------|

\Mensagem{LANÇAMENTO}

\startMyCampaign

\hyphenpenalty=10000
\exhyphenpenalty=10000

\scale[factor=8]{É HOJE}\\
\blank[0.1em]
{\bf XAMANISMOS AMERÍNDIOS}

\stopMyCampaign

\vfill

\scale[factor=fit]{\MyPortrait{XAMANISMOS_PAIVA_FOTO_PB}\quad\MyPortrait{XAMANISMOS_RIVELINO_FOTO_PB}\quad\MyPortrait{XAMANISMOS_KLEIN_FOTO_PB}\quad\MyPortrait{XAMANISMOS_LOLLI_FOTO_PB.jpg}\quad\MyPortrait{XAMANISMOS_TIMOTEO_FOTO_PB}}


\vfill

\scale[factor=6]{\Seta\,{\bf CONVIDADOS} Danilo Paiva Ramos, João}
\setupinterlinespace[line=1.7ex]\scale[factor=6]{Rivelino Rezende Barreto, Tatiane Maíra Klein,}
\setupinterlinespace[line=1.7ex]\scale[factor=6]{Pedro Lolli e Timóteo Popyguá}
\setupinterlinespace[line=2ex]\scale[factor=6]{\Seta\,{\bf 19 DE ABRIL 19H} na Livraria Megafauna}

\page %---------------------------------------------------------|

\hyphenpenalty=10000
\exhyphenpenalty=10000

O lançamento contará com a presença de Danilo Paiva Ramos, João Rivelino Rezende Barreto, Tatiane Maíra Klein, Pedro Lolli e Timóteo Popyguá e será mediado por {\bf LUÍSA VALENTINI}, antropóloga e coordenadora da coleção {\bf MUNDO INDÍGENA}.

{\vfill\scale[factor=5]{\Seta\,A {\bf Livraria Megafauna} fica na Av.\,Ipiranga 200,}\setupinterlinespace[line=1.7ex]\scale[factor=5]{no bairro da República.}}

\page %---------------------------------------------------------|

\hyphenpenalty=10000
\exhyphenpenalty=10000

O livro é organizado por\\ {\bf ARISTOTELES BARCELOS NETO}, {\bf LAURA PÉREZ GIL} e {\bf DANILO\\ PAIVA RAMOS}. Reúne 18 textos, que exploram modos xamânicos particulares de conceber o mundo e agir nele. O posfácio é de {\bf ESTHER JEAN LANGDON}, pesquisadora de narrativas xamânicas e performance.

\page %---------------------------------------------------------|

\MyCover{XAMANISMOS_THUMB}

\page %---------------------------------------------------------|

\Hedra

\stoptext %---------------------------------------------------------|
