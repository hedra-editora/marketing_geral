% AUTOR_LIVRO_AUTOR.tex
% Preencher com o nome das cor ou composição RGB (ex: [r=0.862, g=0.118, b=0.118]) 
\usecolors[crayola] 			   % Paleta de cores pré-definida: wiki.contextgarden.net/Color#Pre-defined_colors

% Cores definidas pelo designer:
% MyGreen		r=0.251, g=0.678, b=0.290 % 40ad4a
% MyCyan		r=0.188, g=0.749, b=0.741 % 30bfbd
% MyRed			r=0.820, g=0.141, b=0.161 % d12429
% MyPink		r=0.980, g=0.780, b=0.761 % fac7c2
% MyGray		r=0.812, g=0.788, b=0.780 % cfc9c7
% MyOrange		r=0.980, g=0.671, b=0.290 % faab4a

% Configuração de cores
\definecolor[MyColor][MaximumBluePurple]      % ou ex: [r=0.862, g=0.118, b=0.118] % corresponde a RGB(220, 30, 30)
\definecolor[MyColorText][black]  % ou ex: [r=0.862, g=0.118, b=0.118] % corresponde a RGB(167, 169, 172)

% Classe para diagramação dos posts
\environment{marketing.env}		   

% Cabeço e rodapé: Informações (caso queira trocar alguma coisa)
 		\def\MensagemSaibaMais  {SAIBA MAIS:}
 		\def\MensagemSite		{HEDRA.COM.BR}
 		\def\MensagemLink       {LINK NA BIO}

\starttext %--------------------------------------------------------|

\Mensagem{SOBRE OS ORGANIZADORES}

\startMyCampaign

\hyphenpenalty=10000
\exhyphenpenalty=10000
{\bf XAMANISMOS \\
AMERÍNDIOS}

\vfill
\scale[factor=5]{\Seta\,Aristoteles Barcelos Neto}\\
\scale[factor=5]{\Seta\,Laura Pérez Gil}\\
\scale[factor=5]{\Seta\,Danilo Paiva Ramos}\\
% \scale[factor=fit]{\MyPortrait{ARANTES_MANES_AUTOR1}\quad\MyPortrait{ARANTES_MANES_AUTOR4.jpg}\quad\MyPortrait{ARANTES_MANES_AUTOR3.png}}

\stopMyCampaign

\page %----------------------------------------------------------|

\hyphenpenalty=10000
\exhyphenpenalty=10000

\MyPicture{XAMANISMOS_ARIS_FOTO_PB}

{\vfill\scale[factor=6]{\Seta\,{\bf ARISTOTELES BARCELOS NETO} é}
\setupinterlinespace[line=2ex]\scale[factor=6]{professor e museólogo. Recebeu o Prêmio}
\setupinterlinespace[line=2ex]\scale[factor=6]{{\cap CNP}q--Anpocs de melhor tese de doutorado}
\setupinterlinespace[line=2ex]\scale[factor=6]{{em Ciências Sociais.}} 

\page %----------------------------------------------------------|

\MyPicture{XAMANISMOS_LAURA_FOTO_PB}

\vfill\scale[factor=6]{\Seta\,{\bf LAURA PEREZ GIL} é professora de}\setupinterlinespace[line=2ex]\scale[factor=6]{Antropologia Americana e responsável pelas}\setupinterlinespace[line=2ex]\scale[factor=6]{coleções de etnologia indígena na {\cap UCM}.}

% \vfill\scale[factor=6]{\Seta\,{\bf LAURA PÉREZ GIL} é professora de Antropologia Americana}\setupinterlinespace[line=2ex]\scale[factor=6]{e responsável pelas coleções de etnologia}\setupinterlinespace[line=2ex]\scale[factor=6]{indígena na Universidad}\setupinterlinespace[line=2ex]\scale[factor=6]{Complutense de Madrid.}}

% \textls[10]{\textbf{Laura Pérez Gil} é professora de Antropologia Americana na Universidad Complutense de Madrid, do departamento da Antropologia (em excedência) e do programa de pós-graduação em Antropologia e Arqueologia da Universidade Federal do Paraná. Foi responsável pelas coleções de etnologia indígena, e posteriormente coordenadora do Museu de Arqueologia e Etnologia da mesma universidade. Atua na área de Etnologia Indígena das Terras Baixas da América do Sul com povos de língua pano, especialmente sobre xamanismo, corporalidade, organização social e processos históricos. Igualmente, desenvolve e orienta pesquisas sobre coleções etnográficas em museus.}

\page

\MyPicture{XAMANISMOS_PAIVA_FOTO_PB2}


\vfill\scale[factor=6]{\Seta\,{\bf DANILO PAIVA RAMOS} é antropólogo e}\setupinterlinespace[line=2ex]\scale[factor=6]{professor da UNIFAL-MG e PPGAS-UFSCAR.}\setupinterlinespace[line=2ex]\scale[factor=6]{Pesquisa etnologia indígena, com ênfase em}\setupinterlinespace[line=2ex]\scale[factor=6]{estudos sobre xamanismo e saúde indígena.}

%foi curadora da exposição {\it Travessias do Moderno em Mauá} (2022) e 
%é graduanda em História da Arte (Unifesp) e responsável pelos projetos Pinacoteca Digital Mauá (2019) e Falando em Arte (2020). 

\page

\MyCover{XAMANISMOS_THUMB}

\page %----------------------------------------------------------|

\Hedra

\stoptext %---------------------------------------------------------|