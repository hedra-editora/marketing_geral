% Preencher com o nome das cor ou composição RGB (ex: [r=0.862, g=0.118, b=0.118]) 
\usecolors[crayola] 			   % Paleta de cores pré-definida: wiki.contextgarden.net/Color#Pre-defined_colors

% Cores definidas pelo designer:
% MyGreen		r=0.251, g=0.678, b=0.290 % 40ad4a
% MyCyan		r=0.188, g=0.749, b=0.741 % 30bfbd
% MyRed			r=0.820, g=0.141, b=0.161 % d12429
% MyPink		r=0.980, g=0.780, b=0.761 % fac7c2
% MyGray		r=0.812, g=0.788, b=0.780 % cfc9c7
% MyOrange		r=0.980, g=0.671, b=0.290 % faab4a

% Configuração de cores
\definecolor[MyColor][x=c0e016]      % ou ex: [r=0.862, g=0.118, b=0.118] % corresponde a RGB(220, 30, 30)
\definecolor[MyColorText][black]  % ou ex: [r=0.862, g=0.118, b=0.118] % corresponde a RGB(167, 169, 172)

% Classe para diagramação dos posts
\environment{marketing.env}		   


% Comandos & Instruções %%%%%%%%%%%%%%%%%%%%%%%%%%%%%%%%%%%%%%%%%%%%%%%%%%%%%%%%%%%%%%%|

% Cabeço e rodabé: Informações (caso queira trocar alguma coisa)
% 		\def\MensagemSaibaMais{SAIBA MAIS:}
% 		\def\MensagemSite{HEDRA.COM.BR}
% 		\def\MensagemLink{LINK NA BIO}

% Pesos para os títulos:
%		\startMyCampaign...		 \stopMyCampaign
%		\stopMyCampaignSection...   \stopMyCampaignSection

% Aplicação de imagens: 
% 		\MyCover{capa.pdf}  	% Aplicação de capa de livro com sombra
%		\MyPicture{Imagem.png}  % Imagem com aplicação de filtro segundo cor MyColorText
%		\MyPhoto{}			    % Aplicação simples de imagem com tamamho \textwidth

% Aplicação de imagem com legenda:		
% 		\placefigure{Legenda}{\externalfigure[drop2-1.png][width=\textwidth]}

% Cabeço e rodabé: Opções
% 		\Mensagem{AGORA É QUE SÃO ELAS}
% 		\Hashtag{campanha de natal}
% 		\Mensagem{campanha de natal}

% Alteração de várias cores de background:
% \setupbackgrounds[page][background=color,backgroundcolor=MyGray]

% Estrela: 
% \vfill\scale[lines=2]{\MyStar[MyColorText][none]} 					% Estrela pequena  
% \startpositioning 											% Estrela grande
%  \position(-1,-.3){\scale[scale=980]{\MyStar[white][none]}}
% \stoppositioning

% Logos e selos: 				
% \Hedra
% \HedraAyllon	% Não está pronto
% \HedraAcorde	% Não está pronto
% \Ayllon		% Não está pronto
% \Acorde		% Não está pronto

% Atalhos: 						
% 		\Seta  % Seta para baixo

%%%%%%%%%%%%%%%%%%%%%%%%%%%%%%%%%%%%%%%%%%%%%%%%%%%%%%%%%%%%%%%%%%%%%%%%%%%%%%%%%%%%%%%|

\starttext
%\showframe  %Para mostrar somente as linhas.

\Mensagem{DESTAQUES}

\MyCover{BENJAMIN_DIARIO_THUMB.pdf}

\page %---------------------------------------------------------|

{\MyPicture{BENJAMIN_CONTADOR_4.jpeg}}

\vfill
\scale[factor=fit]{Tradução do alemão de {\bf Carla Milani \&
Pedro Hussak}}

\page 

«Não está
a rememoração involuntária, a {\it mémoire
involontaire} de Proust, muito mais
próxima do esquecimento que daquilo
que na maioria das vezes se chama
lembrança? E essa obra de rememoração
espontânea, na qual a lembrança é a
trama, e a urdidura o esquecer, não é
muito mais o oposto à obra de Penélope
do que sua imagem semelhante? Pois
aqui o dia desfaz o que a noite tecia.
Toda manhã, despertos, seguramos em
nossas mãos, quase sempre de maneira
fraca e solta, apenas algumas franjas
do tapete da existência vivida, como o
esquecer o teceu em nós. Mas cada dia
desfaz o entrelaçamento, os ornamentos
do esquecer por meio da ação vinculada
a um objetivo e, mais ainda, por meio
do lembrar preso a um objetivo.» W.\,B.
\page

{\MyPicture{BENJAMIN_CONTADOR_2.jpeg}}

{\it Entre 1926 e 1936, o período que compreende a
redação, as anotações e a publicação dos
escritos que
compõem o {\bf Diário parisiense}, Walter Benjamin refinou seus 
instrumentos críticos, durante um período conturbado politicamente, entre
o processo de derrocada dos ideais que
presidiram o advento da República de Weimar
e a crescente ascensão do nazismo, ao
mesmo tempo, atravessado pelos ventos que
sopravam da União Soviética.} \blank[.4ex]

{\hfill\tf ---Ernani Chaves}


\page

\Hedra

\stoptext

% Sabemos que Proust não descreveu em
% sua obra uma vida como ela foi, mas uma
% vida como aquele que a viveu lembra
% dessa vida. Entretanto, isso ainda está
% expresso de maneira imprecisa e, de
% longe, de forma muito grosseira. Pois
% aqui, para o autor que lembra, aquilo
% que ele vivenciou não desempenha de
% forma alguma o papel principal, mas
% sim o tecer de suas lembranças, o
% trabalho de Penélope da rememoração.
% Ou deveríamos falar de uma obra de
% Penélope do esquecimento? Não está
% a rememoração involuntária, a {\it mémoire
% involontaire} de Proust, muito mais
% próxima do esquecimento que daquilo
% que na maioria das vezes se chama
% lembrança? E essa obra de rememoração
% espontânea, na qual a lembrança é a
% trama, e a urdidura o esquecer, não é
% muito mais o oposto à obra de Penélope
% do que sua imagem semelhante? Pois
% aqui o dia desfaz o que a noite tecia.
% Toda manhã, despertos, seguramos em
% nossas mãos, quase sempre de maneira
% fraca e solta, apenas algumas franjas
% do tapete da existência vivida, como o
% esquecer o teceu em nós. Mas cada dia
% desfaz o entrelaçamento, os ornamentos
% do esquecer por meio da ação vinculada
% a um objetivo e, mais ainda, por meio
% do lembrar preso a um objetivo. Por
% isso, Proust, ao final, transformou
% seus dias em noite para dedicar
% imperturbavelmente todas as suas horas
% à obra, sob luz artificial em seu quarto
% escurecido, para não perder nenhum de
% seus enredados arabescos.
%
% Walter Benjamin, “Imagem de Proust”


% Post
% ====
% «Diário parisiense e outros escritos» reúne quinze textos de Walter
% Benjamin, dos anos de 1926 a 1936, dentre eles um diário redigido em
% Paris, que dá nome ao livro e, assim como outros títulos nele coligidos,
% é inédito em português. A seleta de textos remete fundamentalmente
% ao trânsito entre Alemanha e França, realizado por Benjamin em
% vários sentidos: literário-crítico, filosófico, artístico, político e biográfico. 
% Trânsito a partir do qual constituíram-se tanto as análises
% originais de Benjamin acerca de Marcel Proust, André Gide e Paul
% Valéry, o “triângulo equilátero da nova literatura francesa”, quanto o
% seu “lugar” como crítico literário autêntico, judeu-alemão e refugiado
% político, inserido no debate literário francês do período entreguerras.


%Carla Milani Damião é professora da Faculdade de Filosofia e dos
%Programas de Pós-graduação em Filosofia e em Arte e Cultura Visual
%da Universidade Federal de Goiás (ufg). Entre outras publicações, é
%autora do livro Sobre o declínio da “sinceridade”: Filosofia e autobio-
%grafia de Jean-Jacques Rousseau a Walter Benjamin (2006) e organi-
%zadora de coletâneas, entre as quais: Confluindo tradições estéticas
%(2016), Estética em preto e branco (2018) e Estéticas indígenas (2019).

%Pedro Hussak van Velthen Ramos é professor de Estética na Uni-
%versidade Federal Rural do Rio de Janeiro (ufrrj), onde atua nos
%cursos de graduação e pós-graduação em Filosofia. Colabora também
%no Programa de Pós-Graduação em Estudos Contemporâneos das
%Artes da Universidade Federal Fluminense (uff). Entre outros títulos,
%publicou como organizador Educação Estética: de Schiller a Marcuse
%(2011) e foi editor de dossiês temáticos sobre Jacques Rancière e arte
%contemporânea.

% Coleção Walter Benjamin é um projeto acadêmico-editorial que
% envolve pesquisa, tradução e publicação de obras e textos seletos
% desse importante filósofo, crítico literário e historiador da cultura
% judeu-alemão, em volumes organizados por estudiosos versados em
% diferentes aspectos de sua obra, vida e pensamento. (Amon Pinho \& 
% Francisco De Ambrosis Pinheiro Machado)




% Entre 1926 e 1936, o período que compreende a
% redação, as anotações e a publicação dos
% escritos de Walter Benjamin (1892–1940) que
% compõem este volume, nosso autor refinou e
% afinou os seus instrumentos críticos. Trata-se
% de um período conturbado politicamente, entre
% o processo de derrocada dos ideais que
% presidiram o advento da República de Weimar
% e a crescente ascensão do nazismo. Mas, ao
% mesmo tempo, atravessado pelos ventos que
% sopravam da União Soviética. Não é por acaso,
% portanto, que a crítica militante de Walter
% Benjamin nessa época tome como centro de
% referência três cidades: Berlim, Moscou e Paris.
% Delas emergiam, em diferentes nuances e
% escalas, formas de experimentação artística
% imbricadas com os processos políticos — o
% teatro de Brecht, o cinema soviético e o
% surrealismo, por exemplo — sobre os quais
% Benjamin se debruçou com atenção, cuidado e,
% principalmente, com um senso crítico acurado
% para perceber as tensões, os conflitos, os
% paradoxos e as dificuldades que o encontro ou
% o desencontro entre estética e política acabava
% por proporcionar.
% 
% Nesses textos, Benjamin, na esteira do
% Primeiro Romantismo, torna indissociável a
% reflexão sobre as obras e a atividade do crítico.
% Entretanto, gostaria de destacar uma das
% singularidades de seu trabalho crítico, que
% se mostra de forma tão eloquente, nos textos
% sobre Valéry, Gide e Proust aqui reunidos: a
% sua concepção de crítica que, se por um lado,
% propugna pela imersão, por um mergulho nas
% obras, por outro, como suplemento necessário,
% pensa na necessária distância que devemos
% tomar em relação aos objetos da crítica.
% 
% A imersão, o mergulho nas profundezas
% do objeto não pode terminar num processo
% identificatório, empático, com o objeto. A
% empatia, dirá ele um pouco mais tarde, nas
% famosas teses Sobre o conceito de história, é
% sempre com o vencedor. Ao contrário, é preciso
% também como que se desvencilhar do objeto,
% não em nome de uma pretensa neutralidade,
% mas sim das inúmeras possibilidades de
% interpretação que a distância certa acaba por
% oferecer. Em oposição a esses processos de
% identificação, os quais requerem e exigem uma
% unidade entre sujeito e objeto e, portanto, um
% fechamento do sentido, a atividade da crítica
% deve pressupor o sentido inacabado, do qual o
% procedimento alegórico é testemunha.
% 
% Nessa perspectiva, os ensaios sobre
% esses autores, que ele chama de o triângulo
% equilátero da nova literatura francesa — todos
% eles se movimentando no campo aberto por
% Baudelaire —, são exemplares não apenas para
% a compreensão da situação da cultura francesa
% no entreguerras, em meio ao conturbado
% momento político e à efervescência das
% vanguardas, mas também das perspectivas
% estético-políticas do próprio Benjamin. O
% público brasileiro, seja ele leitor e estudioso
% de Benjamin ou não, passa a ter, com essa
% coletânea, um precioso instrumento de
% trabalho, que certamente abrirá novas
% perspectivas de pesquisa.
% 
% Ernani Chaves