% AUTOR_LIVRO_EDICAO.tex
% Preencher com o nome das cor ou composição RGB (ex: [r=0.862, g=0.118, b=0.118]) 
\usecolors[crayola] 			   % Paleta de cores pré-definida: wiki.contextgarden.net/Color#Pre-defined_colors

% Cores definidas pelo designer:
% MyGreen		r=0.251, g=0.678, b=0.290 % 40ad4a
% MyCyan		r=0.188, g=0.749, b=0.741 % 30bfbd
% MyRed			r=0.820, g=0.141, b=0.161 % d12429
% MyPink		r=0.980, g=0.780, b=0.761 % fac7c2
% MyGray		r=0.812, g=0.788, b=0.780 % cfc9c7
% MyOrange		r=0.980, g=0.671, b=0.290 % faab4a

% Configuração de cores
\definecolor[MyColor][x=f9d3b9]      % ou ex: [r=0.862, g=0.118, b=0.118] % corresponde a RGB(220, 30, 30)
\definecolor[MyColorText][Manatee]  % ou ex: [r=0.862, g=0.118, b=0.118] % corresponde a RGB(167, 169, 172)

% Classe para diagramação dos posts
\environment{marketing.env}		   

\starttext %---------------------------------------------------------|

\Mensagem{DESVENDANDO BENJAMIN}


\startMyCampaign

\hyphenpenalty=10000
\exhyphenpenalty=10000

{\bf 
 WALTER 

 BENJAMIN}

O NARRADOR 
OU 
O CONTADOR DE HISTÓRIAS?
\stopMyCampaign

\page %---------------------------------------------------------| 

\MyCover{BENJAMIN_CONTADOR_THUMB}

\page %---------------------------------------------------------| 

\hyphenpenalty=10000
\exhyphenpenalty=10000


 Normalmente traduzido como «O narrador», {\bf O CONTADOR DE HISTÓRIAS E OUTROS TEXTOS} propõe uma nova tradução anotada do clássico ensaio no qual {\bf WALTER BENJAMIN} esboça a figura do contador de histórias a partir de um comentário crítico do contista russo {\bf NIKOLAI LESKOV}.

\page

  O volume inclui também a pouco conhecida {\bf PRODUÇÃO FICCIONAL} do próprio ensaísta, trazendo o conjunto de seus contos, alguns {\bf INÉDITOS EM PORTUGUÊS}, além de peças que {\bf BENJAMIN} produziu para o rádio.

\page


\hyphenpenalty=10000
\exhyphenpenalty=10000

A organização é de {\bf PATRÍCIA LAVELLE}, professora do\\
Departamento de Letras da PUC-Rio, pesquisadora e autora do livro {\bf RELIGION ET HISTOIRE. SUR LE CONCEPT D’EXPÉRIENCE CHEZ WALTER BENJAMIN} (2008).

\page

O livro integra a {\bf COLEÇÃO WALTER BENJAMIN}, um projeto que envolve pesquisa, tradução e publicação de obras e textos seletos desse importante filósofo, crítico literário e historiador da cultura judeu-alemão. A direção da coleção é de {\bf AMON PINHO} e {\bf FRANCISCO DE AMBROSIS PINHEIRO MACHADO}.

\page %---------------------------------------------------------|

\Hedra

\stoptext %---------------------------------------------------------|