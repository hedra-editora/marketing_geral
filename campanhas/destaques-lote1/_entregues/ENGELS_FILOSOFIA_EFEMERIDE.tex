%AUTOR_LIVRO_CURIOSIDADES.tex
% Preencher com o nome das cor ou composição RGB (ex: [r=0.862, g=0.118, b=0.118]) 
\usecolors[crayola]                % Paleta de cores pré-definida: wiki.contextgarden.net/Color#Pre-defined_colors

% Cores definidas pelo designer:
% MyGreen       r=0.251, g=0.678, b=0.290 % 40ad4a
% MyCyan        r=0.188, g=0.749, b=0.741 % 30bfbd
% MyRed         r=0.820, g=0.141, b=0.161 % d12429
% MyPink        r=0.980, g=0.780, b=0.761 % fac7c2
% MyGray        r=0.812, g=0.788, b=0.780 % cfc9c7
% MyOrange      r=0.980, g=0.671, b=0.290 % faab4a

% Configuração de cores
\definecolor[MyColor][BrickRed]      % ou ex: [r=0.862, g=0.118, b=0.118] % corresponde a RGB(220, 30, 30)
\definecolor[MyColorText][white]     % ou ex: [r=0.862, g=0.118, b=0.118] % corresponde a RGB(167, 169, 172)


% Classe para diagramação dos posts
\environment{marketing.env}        


\def\startMyCampaign{\bgroup
            \FormularMI
            \switchtobodyfont[26.5pt]
            \setupinterlinespace[line=1.9ex]
            \setcharacterkerning[packed]}
\def\stopMyCampaign{\par\egroup}
\starttext %---------------------------------------------------------|

\Mensagem{EM CONTEXTO} %Sempre usar esse header

\startMyCampaign

\hyphenpenalty=10000
\exhyphenpenalty=10000

«ÉRAMOS TODOS {\bf FEUERBACHIANOS}»
\blank[1ex]
\Seta\,FRIEDRICH ENGELS
\stopMyCampaign

\page %---------------------------------------------------------| 

\hyphenpenalty=10000
\exhyphenpenalty=10000

{\bf LUDWIG FEUERBACH} rompeu com idealismo hegeliano e colocou a «natureza e o ser humano» no centro de sua filosofia. 
\page

Sua principal obra, {\bf A ESSÊNCIA DO CRISTIANISMO} (1841), argumenta que Deus é uma projeção dos desejos humanos e que a religião aliena as pessoas de sua verdadeira essência humana. Ao reorientar a crítica da religião e da filosofia e colocar «o materialismo novamente no trono», Feuerbach tornou-se uma referência indispensável a {\bf MARX E ENGELS}. 

\page

Em {\bf LUDWIG FEUERBACH E O FIM DA FILOSOFIA CLÁSSICA ALEMÃ}, um dos textos mais conhecidos de seu período tardio, Engels destaca a influência que o filósofo alemão teve no pensamento marxista.

\page

O texto reelabora a crítica à filosofia alemã desenvolvida em seu período de juventude, junto com Marx, ao mesmo tempo em que defende a potência crítica do {\bf MATERIALISMO-HISTÓRICO MARXISTA} em oposição aos outros materialismos, idealismos e positivismos no contexto histórico alemão pós-revolução de 1848.

\page %---------------------------------------------------------|

Para Engels, «Feuerbach, em mais de um aspecto, estabelece {\bf UM MEIO DE LIGAÇÃO} entre a filosofia de Hegel e nossa concepção» e seria necessário reconhecer a influência que, mais do que todos os outros filósofos pós-hegelianos, Feuerbach teve sobre Marx e ele durante o período de {\bf STRUMUND DRANG} [tempestade e ímpeto].

\page

\page
\MyCover{ENGELS_FILOSOFIA_THUMB}

\page %---------------------------------------------------------|

\Hedra

\stoptext %---------------------------------------------------------|



% Ludwig Feuerbach e o fim da filosofia clássica alemã, ou o ponto de saída da filosofia clássica alemã, é um dos textos mais conhecidos do período tardio de Engels, com grande influência nas discussões marxistas posteriores.Foi publicado pela primeira vez nos volumes 4 e 5 da revista Die Neue Zeit(1886). Uma versão brevemente estendida sai em 1888 (Dietz, Stuttgart), juntamente com o anexo da primeira aparição dasTeses sobre Feuerbach de 1845. O texto reelabora a crítica à filosofia alemã desenvolvida em seu período de juventude, junto com Marx, ao mesmo tempo em que defende a potência crítica do materialismo-histórico marxista em oposição aos outros materialismos, idealismos e positivismos no contexto histórico alemão pós-revolução de 1848.


% Ludwig Andreas Feuerbach (Landshut, 28 de julho de 1804 – Rechenberg, Nuremberg, 13 de setembro de 1872) foi um filósofo alemão.[1] Feuerbach é reconhecido pelo ateísmo humanista antropológico e pela influência que o seu pensamento exerce sobre Karl Marx e Sigmund Freud. 

% Abandonou os estudos de Teologia para tornar-se aluno do filósofo Hegel, durante dois anos, em Berlim. Em 1828, passa a estudar ciências naturais em Erlangen e dois anos depois publica anonimamente o primeiro livro, “Pensamentos sobre Morte e Imortalidade”. Nesse trabalho, ataca a ideia da imortalidade, sustentando que, após a morte, as qualidades humanas são absorvidas pela natureza. Escreve “Abelardo e Heloísa” (1834), “Piere Bayle” (1838) e “Sobre Filosofia e Cristianismo” (1839). Na primeira parte desta última obra, que influencia Marx, discute a "essência verdadeira ou antropológica da religião". Na segunda parte, analisa a "essência falsa ou teológica". De acordo com esta filosofia, a religião é uma forma de patologia psíquica[2] que transforma em objeto os conceitos do ideal humano em um ser supremo que é totalmente estranho ao humano. Ao atacar religiosos ortodoxos entre 1848 e 1849, anos de turbulência política, é considerado um herói por muitos revolucionários. 

% O seu posicionamento filosófico é uma transição entre o Idealismo Alemão, de uma parte e, de outra, o materialismo histórico de Marx e o materialismo cientificista da segunda metade do século XIX. Este posicionamento é caracterizado pela inflexão antropológica que Feuerbach imprime a algumas categorias herdadas de Hegel. Suas principais obras são: Da razão, una, universal, infinita (1828); Pensamentos sobre morte e imortalidade (1830); Sobre a crítica da filosofia positiva (1838); Crítica da filosofia hegeliana (1839); A essência do cristianismo (1841); Sobre a apreciação do escrito “A essência do cristianismo” (1842); Princípios da filosofia do futuro (1843); Teses provisórias para a reforma da filosofia (1843); Lutero como árbitro entre Strauss e Feuerbach (1843); A essência da religião (1846); Fragmentos para a caracterização de meu Curriculum vitae (1846); Preleções sobre a essência da religião (1851) e Teogonia (1857).

% Para Feuerbach, a religião segue-se dentro de uma teoria a qual busca o sentido e a essência do homem no mundo, mas o homem é essencialmente antropológico na característica humana, pois adquire sentimentos e sensibilidade. É desta forma que Feuerbach observa essa subversão decorrente em cada indivíduo que busca uma relação substancial entre Homem e Deus.

% O que proporcionou esse pensamento de Feuerbach foi a influência da filosofia de Hegel e, mais tarde, a teoria de Marx. Posteriormente, nessas duas linhas de pensamento, uma teórica, a outra prática, Feuerbach busca a fórmula do Homem vs. Deus vs. Religião.

% Intermediar essas teorias, no entanto, não foi fácil para Feuerbach, pois a Alemanha passava por uma forte mudança cultural; daí a forte crítica ao seu pensamento. Dentro desse contexto histórico, observa-se a teoria de Feuerbach voltada para a “teoria”, e a teoria de Marx, onde a lógica é a prática. Porém, não é a teoria que busca a essência do homem, mas é na prática que os indivíduos se relacionam, afirma Marx mais tarde, com sua crítica a Feuerbach.

% Faleceu em 13 de setembro de 1872. Encontra-se sepultado em Johannisfriedhof, Nuremberga, Baviera na Alemanha.[3] A influência de Feuerbach depois de sua morte é notória. Filósofos como Enrique Dussel, Sigmund Freud, Guy Debord, Emil Cioran e René Girard são altamente influenciados por sua filosofia que tem como foco o caráter reconstitucional e real do ser humano.[4] 





% E também me pareceu necessário um reconhecimento completo da influência que, acima de todos os outros filósofos pós-hegeli-anos, Feuerbach teve sobre nós durante nosso período deStrumund Drang[tempestade e ímpeto], como uma dívida de honranão quitada.

% Aí surgiu aEssência do cristianismode Feuerbach. Comum só golpe pulverizou a contradição ao colocar, sem rodeios,o materialismo novamente no trono.





% Feuerbach, Hegel, Marx e Engels estão conectados por suas contribuições para a filosofia alemã e, em particular, pela influência que exerceram uns sobre os outros na formação do materialismo dialético e do pensamento revolucionário.

% 1. **Hegel**: Georg Wilhelm Friedrich Hegel (1770-1831) é a figura central na filosofia idealista alemã. Ele desenvolveu o sistema do **idealismo absoluto**, que propunha que a realidade se desenvolve de acordo com uma lógica dialética, um processo de tese, antítese e síntese. Para Hegel, a realidade última era a ideia ou espírito absoluto, e ele via a história como a manifestação progressiva do espírito.

% 2. **Feuerbach**: Ludwig Feuerbach (1804-1872), um jovem hegeliano, rompeu com Hegel, criticando o idealismo hegeliano e colocando a **natureza e o ser humano** no centro de sua filosofia. Sua principal obra, *A Essência do Cristianismo* (1841), argumenta que Deus é uma projeção dos desejos humanos e que a religião aliena as pessoas de sua verdadeira essência humana. Feuerbach influenciou Marx e Engels ao reorientar a crítica da religião e da filosofia para uma crítica mais materialista e humanista.

% 3. **Marx e Engels**: Karl Marx (1818-1883) e Friedrich Engels (1820-1895) começaram como jovens hegelianos, mas logo se afastaram do idealismo de Hegel e do materialismo humanista de Feuerbach, que consideravam insuficientemente radical. Eles criticaram Feuerbach por focar nas ideias sem levar em conta as **condições materiais e econômicas** da vida humana. Marx e Engels desenvolveram uma filosofia do **materialismo histórico**, que vê as relações econômicas como a base de todas as estruturas sociais e políticas. Em obras como *A Ideologia Alemã* (1846), eles combinaram a dialética de Hegel com o materialismo de Feuerbach para criar o **materialismo dialético**, que foi a base de sua crítica ao capitalismo e ao desenvolvimento de suas teorias revolucionárias.

% A ligação entre os quatro filósofos, portanto, envolve um processo de desenvolvimento e crítica: Marx e Engels partiram da dialética de Hegel, passaram pela crítica humanista de Feuerbach, e chegaram ao materialismo histórico e dialético, que funde a análise histórica e econômica com a dialética.