% BASKIN_ISRAEL_EFEMERIDE.tex
% Preencher com o nome das cor ou composição RGB (ex: [r=0.862, g=0.118, b=0.118]) 
\usecolors[crayola] 			   % Paleta de cores pré-definida: wiki.contextgarden.net/Color#Pre-defined_colors

% Cores definidas pelo designer:
% MyGreen		r=0.251, g=0.678, b=0.290 % 40ad4a
% MyCyan		r=0.188, g=0.749, b=0.741 % 30bfbd
% MyRed			r=0.820, g=0.141, b=0.161 % d12429
% MyPink		r=0.980, g=0.780, b=0.761 % fac7c2
% MyGray		r=0.812, g=0.788, b=0.780 % cfc9c7
% MyOrange		r=0.980, g=0.671, b=0.290 % faab4a

% Configuração de cores
\definecolor[MyColor][x=778cc7]      % ou ex: [r=0.862, g=0.118, b=0.118] % corresponde a RGB(220, 30, 30)
\definecolor[MyColorText][black]     % ou ex: [r=0.862, g=0.118, b=0.118] % corresponde a RGB(167, 169, 172)

% Classe para diagramação dos posts
\environment{marketing.env}		   

\starttext %---------------------------------------------------------|

\hyphenpenalty=10000
\exhyphenpenalty=10000

\Mensagem{2 DE MAIO} %Sempre usar esse header

\MyPicture{BASKIN_ISRAEL_2}

\vfill\scale[factor=6]{\Seta\,68 ANOS DE {\bf GERSHON BASKIN}}

\page %---------------------------------------------------------| 

\hyphenpenalty=10000
\exhyphenpenalty=10000

{\bf GERSHON BASKIN} é colunista, ativista social e pesquisador do conflito israelo-palestino.


\page

Nascido em Nova York, Baskin se envolveu no movimento pelos {\bf DIREITOS CIVIS} e contra a Guerra do Vietnã durante seus anos de estudante. Em 1978, recebeu seu bacharelado pela New York University em política e história do Oriente Médio, e no mesmo ano {\bf IMIGROU PARA ISRAEL}.


\page %---------------------------------------------------------|
 
 Trabalhou como assistente comunitário em Kafr Qara, uma vila árabe-palestina em Israel, além de atuar no Ministério da Educação de Israel como coordenador da educação para a coexistência entre os sistemas escolares judeus e árabes. Em 1983, Baskin fundou e dirigiu o {\bf INSTITUTO DE EDUCAÇÃO PARA A COEXISTÊNCIA JUDAICO-ÁRABE}.

\page

\MyCover{BASKIN_ISRAEL_THUMB}

\page %---------------------------------------------------------|

\Hedra

\stoptext %---------------------------------------------------------|