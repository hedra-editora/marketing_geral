% Preencher com o nome das cor ou composição RGB (ex: [r=0.862, g=0.118, b=0.118]) 
\usecolors[crayola] 			   % Paleta de cores pré-definida: wiki.contextgarden.net/Color#Pre-defined_colors

% Cores definidas pelo designer:
% MyGreen		r=0.251, g=0.678, b=0.290 % 40ad4a
% MyCyan		r=0.188, g=0.749, b=0.741 % 30bfbd
% MyRed			r=0.820, g=0.141, b=0.161 % d12429
% MyPink		r=0.980, g=0.780, b=0.761 % fac7c2
% MyGray		r=0.812, g=0.788, b=0.780 % cfc9c7
% MyOrange		r=0.980, g=0.671, b=0.290 % faab4a

% Configuração de cores
\definecolor[MyColor][SpringGreen]      % ou ex: [r=0.862, g=0.118, b=0.118] % corresponde a RGB(220, 30, 30)
\definecolor[MyColorText][black]     % ou ex: [r=0.862, g=0.118, b=0.118] % corresponde a RGB(167, 169, 172)

% Classe para diagramação dos posts
\environment{marketing.env}		   

\starttext %---------------------------------------------------------|

\Mensagem{PROJETO LUIZ GAMA}

\startMyCampaign

\hyphenpenalty=10000
\exhyphenpenalty=10000

«A religião, a moral, o direito e a liberdade são gêneros deteriorados que não têm cotação nos mercados\ldots{}» 

\vfill\hfill →

\stopMyCampaign

\page %---------------------------------------------------------| 

\startMyCampaign

\hyphenpenalty=10000
\exhyphenpenalty=10000

«Aí compra-se e vende-se o homem; açoitam-se-lhe as carnes, monetariza-se-lhe o suor e o sangue, o homem é o escravo\ldots{}»

\vfill\hfill →

\stopMyCampaign

\page %---------------------------------------------------------| 

\startMyCampaign

\hyphenpenalty=10000
\exhyphenpenalty=10000

«o escravo é o dinheiro e a questão é essencialmente econômica!»

\stopMyCampaign

{\vfill\scale[factor=6]{\Seta\,Trecho do livro {\bf Direito}, de Luiz Gama,}\setupinterlinespace[line=1.5ex]\scale[factor=6]{símbolo do movimento abolicionista e da luta}\setupinterlinespace[line=1.5ex]\scale[factor=6]{contra a escravidão no Brasil. Com mais de}\setupinterlinespace[line=1.5ex]\scale[factor=6]{oitenta por cento de textos inéditos, as obras}\setupinterlinespace[line=1.5ex]\scale[factor=6]{completas de Luiz Gama serão publicadas}\setupinterlinespace[line=1.5ex]\scale[factor=6]{pela Hedra em onze volumes.}}

\page %---------------------------------------------------------| 

\MyCover{GAMA_DIREITO_THUMB.png}

\page %---------------------------------------------------------|

\Hedra

\stoptext %---------------------------------------------------------|