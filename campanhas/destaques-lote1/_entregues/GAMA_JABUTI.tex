% AUTOR_LIVRO_EDICAO.tex
% Preencher com o nome das cor ou composição RGB (ex: [r=0.862, g=0.118, b=0.118]) 
\usecolors[crayola]                   % Paleta de cores pré-definida: wiki.contextgarden.net/Color#Pre-defined_colors

% Cores definidas pelo designer:
% MyGreen          r=0.251, g=0.678, b=0.290 % 40ad4a
% MyCyan          r=0.188, g=0.749, b=0.741 % 30bfbd
% MyRed               r=0.820, g=0.141, b=0.161 % d12429
% MyPink          r=0.980, g=0.780, b=0.761 % fac7c2
% MyGray          r=0.812, g=0.788, b=0.780 % cfc9c7
% MyOrange          r=0.980, g=0.671, b=0.290 % faab4a

% Configuração de cores
\definecolor[MyColor][VividViolet]      % ou ex: [r=0.862, g=0.118, b=0.118] % corresponde a RGB(220, 30, 30)
\definecolor[MyColorText][white]     % ou ex: [r=0.862, g=0.118, b=0.118] % corresponde a RGB(167, 169, 172)


\def\MyFirstCover#1{\starttikzpicture[overlay, remember picture]
              \node
                   [draw=none, 
                    blur shadow={shadow xshift=5.5pt,shadow yshift=-1.5pt, shadow scale=0.93},
                    shadow opacity=50, 
                    shadow blur extra rounding] 
                    at (0.25  \textwidth,-.3\textheight) {\externalfigure[#1][width=2.6cm]};
                \stoptikzpicture}


\def\MySecondCover#1{\starttikzpicture[overlay, remember picture]
              \node
                   [draw=none, 
                    blur shadow={shadow xshift=5.5pt,shadow yshift=-1.5pt, shadow scale=0.93},
                    shadow opacity=50, 
                    shadow blur extra rounding] 
                    at (0.70  \textwidth,-.3\textheight) {\externalfigure[#1][width=2.6cm]};
                \stoptikzpicture}

% Classe para diagramação dos posts
\environment{marketing.env}             

\starttext %---------------------------------------------------------|

\Mensagem{PROJETO LUIZ GAMA}

\startMyCampaign

\hyphenpenalty=10000
\exhyphenpenalty=10000

{\bf SEMIFINALISTA\\ DO JABUTI ACADÊMICO}
\stopMyCampaign

\MyFirstCover{GAMA_CRIME_THUMB} \MySecondCover{GAMA_DIREITO_THUMB}
%\vfill\scale[lines=1.5]{\MyStar[MyColorText][none]}
\starttikzpicture[remember picture,overlay]
\node[text=white] at (3.3,0) {Bruno Rodrigues de Lima (org.)};
\stoptikzpicture

 \page %---------------------------------------------------------| 


\hyphenpenalty=10000
\exhyphenpenalty=10000

Os livros {\bf CRIME} e {\bf DIREITO}, de Luiz Gama, dois dos onze volumes das
suas Obras Completas, que serão publicadas na íntegra pela Hedra, estão entre os {\bf SEMIFINALISTAS DO JABUTI}.

\page

\MyPhoto{GAMA_JABUTI_1}
\page


{\bf CRIME} apresenta mais de quarenta artigos que abordam, por exemplo, a discussão teórica acerca da {\bf DEFESA CONCRETA DE ESCRAVIZADOS}.
São textos sobre a importância da identificação dos limites de uma jurisdição.

\page %---------------------------------------------------------|

 {\bf DIREITO} reúne {\bf TEXTOS\\ PUBLICADOS PELO
 AUTOR NA IMPRENSA, ALÉM DE CARTAS PESSOAIS}, logo depois de sua demissão do cargo de amanuense
da Secretaria de Polícia de São Paulo.
 
\page

\hyphenpenalty=10000
\exhyphenpenalty=10000

«Será trabalho de gerações, como efetivamente tem sido, {\bf RECUPERAR O LEGADO DE LUIZ GAMA} e reinseri-lo no lugar que merece ocupar nas letras, no jornalismo, na política, no direito e na história.»
\blank[1ex]


\page %---------------------------------------------------------|

\MyPhoto{GAMA_JABUTI_BRUNO}
\vfill\scale[factor=8]{\Seta\,{\bf Bruno Rodrigues de Lima}}

\page

\Hedra

\stoptext %---------------------------------------------------------|



 % Crime é o sétimo volume dasObras Completas de Luiz Gama. Os textos aqui recolhidos foram divididos em oito partes. A primeira contém três artigos acerca da importância da identificação dos limites de uma jurisdição, discussão teórica cuja implicação é a defesa concreta de escravizados. A seguir, em “A injúria”, Gama debate, em três textos, questões de natureza jurídico-criminal e defende a imprensa como fórum de salvaguarda de direitos — tema que retorna na histórica defesa a seu cliente Justiniano Silva em “O abuso da liberdade de opinião e de imprensa”, da terceira parte. A abusiva batida policial ao estúdio de fotografia de Victor Telles, seguida de sua prisão e de mais cinco artistas, é o assunto de “A falsificação de moeda”. Em “O roubo”, quinta parte,o leitor acompanha a exaustiva investigação empreendida por Luiz Gama para a defesa do tesoureiro da alfândega de Santos, acusado do roubo milionário de 185 mil contos de réis do cofre da instituição. Nesta parte, encontra-se “O misterioso roubo da alfândega de Santos”, o mais longo texto já escrito porGama, em que se destaca não apenas seu já conhecido talento no direito, mas sobretudo sua personalidade detetivesca, encarnada no gosto sherloquiano pelo enigma a desvendar. “O homicídio” contém seis textos em que Gama discute na imprensa casos que tratavam do crime de homicídio ou tentativa de homicídio. Em “Ladrão que rouba ladrão”, Gama trata da demanda de liberdade de uma escravizada, mediante o pagamento de alta quantia. Finalmente, na oitava parte, o leitor encontra cartas de Gama escritas nesse período.

 % Luiz Gonzaga Pinto da Gama nasceu livre em Salvador da Bahia no dia 21de junho de 1830 e morreu na cidade de São Paulo, como herói da liberdade,em 24 de agosto de 1882. Filho de Luiza Mahin, africana livre, e de um fidalgo baiano cujo nome nunca revelou, Gama foi escravizado pelo próprio pai, naausência da mãe, e vendido para o sul do país no dia 10 de novembro de1840. Dos dez aos dezoito anos de idade, Gama viveu escravizado em SãoPaulo e, após conseguir provas de sua liberdade, fugiu do cativeiro e assentoupraça como soldado (1848). Depois de seis anos de serviço militar (1854),Gama tornou-se escrivão de polícia e, em 1859, publicou suasPrimeiras trovasburlescas, livro de poesias escrito sob o pseudônimo Getulino, que marcariao seu ingresso na história da literatura brasileira. Desde o período em queera funcionário público, Gama redigiu, fundou e contribuiu com veículos deimprensa, tornando-se um dos principais jornalistas de seu tempo. Mas foicomo advogado, posição que conquistou em dezembro de 1869, que escreveu asua obra magna, a luta contra a escravidão por dentro do direito, que resultou nofeito assombroso — sem precedentes no abolicionismo mundial — de conferira liberdade para aproximadamente 750 pessoas através das lutas nos tribunais.


 % Bruno Rodrigues de Limaé advogado e historiador do direito, graduadoem Direito pela Universidade do Estado da Bahia (uneb-Cabula), mestre emDireito, Estado e Constituição pela Universidade de Brasília (unb) e doutorem História do Direito pela Universidade de Frankfurt, Alemanha, com tesesobre a obra jurídica de Luiz Gama. Em 2022, ganhou o Prêmio Walter Kolbde melhor tese de doutorado da Universidade de Frankfurt. Atualmente, épesquisador de pós-doutorado no Instituto Max Planck de História do Direito e Teoria do Direito. Pela edufba, publicou o livro Lama & Sangue – Bahia 1926(2018)


 % Direito é o quinto volume dasObras Completas de Luiz Gama e contém textos publicados pelo autor na imprensa, além de algumas cartas pessoais, logo depois de sua demissão do cargo de amanuense da Secretaria de Polícia de SãoPaulo, no qual aprofundou o conhecimento dos ritos processuais, da casuística local e do repertório normativo da justiça brasileira. Dividido em três grandes blocos contendo um total de 24 partes, este volume colige escritos do advogado que com o tempo se torna jurista. Isto é, que passa de causídico militante para o magistério do direito na esfera pública. Nestes 70 textos, é possível acompanhar os primeiros tempos de advocacia de Gama e as primeiras causas que ele, já como advogado, debateu na imprensa. De um juízo a outro, de uma comarca a outra, Gama modula estratégias e formula argumentos, contrainimigos que o instigavam a escrever em momentos delicados