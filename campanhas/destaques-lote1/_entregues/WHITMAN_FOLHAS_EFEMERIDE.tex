% WHITMAN_FOLHAS_EFEMERIDE.tex
% Preencher com o nome das cor ou composição RGB (ex: [r=0.862, g=0.118, b=0.118]) 
\usecolors[crayola] 			   % Paleta de cores pré-definida: wiki.contextgarden.net/Color#Pre-defined_colors

% Cores definidas pelo designer:
% MyGreen		r=0.251, g=0.678, b=0.290 % 40ad4a
% MyCyan		r=0.188, g=0.749, b=0.741 % 30bfbd
% MyRed			r=0.820, g=0.141, b=0.161 % d12429
% MyPink		r=0.980, g=0.780, b=0.761 % fac7c2
% MyGray		r=0.812, g=0.788, b=0.780 % cfc9c7
% MyOrange		r=0.980, g=0.671, b=0.290 % faab4a

% Configuração de cores
\definecolor[MyColor][black]      % ou ex: [r=0.862, g=0.118, b=0.118] % corresponde a RGB(220, 30, 30)
\definecolor[MyColorText][white]     % ou ex: [r=0.862, g=0.118, b=0.118] % corresponde a RGB(167, 169, 172)

% Classe para diagramação dos posts
\environment{marketing.env}		   

\starttext %---------------------------------------------------------|

\hyphenpenalty=10000
\exhyphenpenalty=10000

\Mensagem{31 DE MAIO} %Sempre usar esse header

\MyPhoto{WHITMAN_FOLHAS_3.png}

\vfill\scale[factor=6]{\Seta\,205 ANOS DE {\bf WALT WHITMAN}}

\page %---------------------------------------------------------| 

\hyphenpenalty=10000
\exhyphenpenalty=10000

 Tido como um dos mais importantes poetas norte-americano, {\bf WALT WHITMAN} ficou conhecido por romper com as convenções poéticas de sua época.

\page

\MyPhoto{WHITMAN_FOLHAS_5}

  Sua poesia não alcançou \\ reconhecimento público enquanto ele era vivo, justamente por conta de suas {\bf INOVAÇÕES ESTILÍSTICAS} e temáticas polêmicas. 

\page

A busca por uma {\bf POESIA NACIONAL}, dotada de uma visão crítica da modernização e do imperialismo estadunidense, fez com {\bf WHITMAN} se firmasse como o primeiro escritor a realizar uma poesia verdadeiramente norte-americana.

\page %---------------------------------------------------------|

\MyCover{WHITMAN_FOLHAS_THUMB.pdf}

\page %---------------------------------------------------------|

\Hedra

\stoptext %---------------------------------------------------------|

