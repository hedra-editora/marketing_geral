% AUTOR_LIVRO_EFEMERIDE.tex
% Preencher com o nome das cor ou composição RGB (ex: [r=0.862, g=0.118, b=0.118]) 
\usecolors[crayola] 			   % Paleta de cores pré-definida: wiki.contextgarden.net/Color#Pre-defined_colors

% Cores definidas pelo designer:
% MyGreen		r=0.251, g=0.678, b=0.290 % 40ad4a
% MyCyan		r=0.188, g=0.749, b=0.741 % 30bfbd
% MyRed			r=0.820, g=0.141, b=0.161 % d12429
% MyPink		r=0.980, g=0.780, b=0.761 % fac7c2
% MyGray		r=0.812, g=0.788, b=0.780 % cfc9c7
% MyOrange		r=0.980, g=0.671, b=0.290 % faab4a

% Configuração de cores
\definecolor[MyColor][x=41b23b]      % ou ex: [r=0.862, g=0.118, b=0.118] % corresponde a RGB(220, 30, 30)
\definecolor[MyColorText][black]     % ou ex: [r=0.862, g=0.118, b=0.118] % corresponde a RGB(167, 169, 172)

% Classe para diagramação dos posts
\environment{marketing.env}		   

\starttext %---------------------------------------------------------|

\hyphenpenalty=10000
\exhyphenpenalty=10000

\Mensagem{JOSÉ DE ALENCAR} %Sempre usar esse header

\MyPortrait{THUMB_ALENCAR.png}

\vfill\scale[factor=6]{\Seta\,1º DE MAIO: 195 ANOS DE {\bf JOSÉ DE ALENCAR}}

\page %---------------------------------------------------------| 

\hyphenpenalty=10000
\exhyphenpenalty=10000

{\bf José de Alencar}, romancista, dramaturgo e político, destacou-se como um dos mais {\bf BRILHANTES HOMENS DE LETRAS} do Brasil no século 19. Amigo pessoal de Machado de Assis, que o considerava um mestre, Alencar publicou {\it O Guarani} (1857), {\it Iracema} (1865) e {\it Senhora} (1875), entre muitos outros.

\page %---------------------------------------------------------|

\hyphenpenalty=10000
\exhyphenpenalty=10000

Dedicou quase um terço de sua vida à atividade parlamentar, elegendo-se quatro vezes deputado e ocupando por três anos o cargo de ministro da Justiça (1868--1870). Alencar era conhecido pelas {\bf POLÊMICAS PÚBLICAS NA IMPRENSA}, em especial com Joaquim Nabuco e D.\,Pedro II.

\page %---------------------------------------------------------|

\hyphenpenalty=10000
\exhyphenpenalty=10000

O livro {\bf CARTAS A FAVOR DA ESCRAVIDÃO} reúne sete textos políticos que José de Alencar publicou em franca oposição a D.\,Pedro II. \mbox{O propósito} central da obra era a defesa política da escravidão brasileira, que vinha sofrendo intensa pressão internacional e doméstica após a abolição nos Estados Unidos (1865).

\page %---------------------------------------------------------|

Por ter abordado um tema controverso para os padrões contemporâneos, as cartas foram excluídas das obras completas do autor. A presente edição procura suprir a lacuna, fornecendo um precioso texto ao público interessado nos atuais {\bf DEBATES SOBRE RELAÇÕES RACIAIS} no país.

\page %---------------------------------------------------------|

\MyCover{ALENCAR_CARTAS_THUMB.pdf}

\page %---------------------------------------------------------|

\Hedra

\stoptext %---------------------------------------------------------|