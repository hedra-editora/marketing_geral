% RUSKIN_PRAETERITA_EDICAO.tex
% Preencher com o nome das cor ou composição RGB (ex: [r=0.862, g=0.118, b=0.118]) 
\usecolors[crayola] 			   % Paleta de cores pré-definida: wiki.contextgarden.net/Color#Pre-defined_colors

% Cores definidas pelo designer:
% MyGreen		r=0.251, g=0.678, b=0.290 % 40ad4a
% MyCyan		r=0.188, g=0.749, b=0.741 % 30bfbd
% MyRed			r=0.820, g=0.141, b=0.161 % d12429
% MyPink		r=0.980, g=0.780, b=0.761 % fac7c2
% MyGray		r=0.812, g=0.788, b=0.780 % cfc9c7
% MyOrange		r=0.980, g=0.671, b=0.290 % faab4a

% Configuração de cores
\definecolor[MyColor][MountainMeadow]      % ou ex: [r=0.862, g=0.118, b=0.118] % corresponde a RGB(220, 30, 30)
\definecolor[MyColorText][black]     % ou ex: [r=0.862, g=0.118, b=0.118] % corresponde a RGB(167, 169, 172)

% Classe para diagramação dos posts
\environment{marketing.env}		   

\starttext %---------------------------------------------------------|

\Mensagem{O MESTRE DE PROUST}

\startMyCampaign

\hyphenpenalty=10000
\exhyphenpenalty=10000


{\bf JOHN RUSKIN}
UM DOS {\bf MAIORES AUTORES}
DO SÉCULO XIX
POUCO CONHECIDO DO
GRANDE PÚBLICO

\stopMyCampaign

%\vfill\scale[lines=1.5]{\MyStar[MyColorText][none]}

\page %---------------------------------------------------------| 

\MyCover{RUSKIN_PRAETERITA_THUMB}

\page %---------------------------------------------------------| 

\hyphenpenalty=10000
\exhyphenpenalty=10000

Entre narrativa de viagem, elegia e coleção de excertos de diário, {\bf PRAETERITA} é a autobiografia de um dos mais importantes intelectuais da era vitoriana, {\bf JOHN RUSKIN}.

\page

\MyPhoto{RUSKIN_PRAETERITA_1.jpg}

\page

De caráter experimental, esta é a última obra de Ruskin, escrita ao longo de 27 anos. O livro discorre sobre essa figura excêntrica e contraditória cujos escritos inspiraram {\bf GANDHI}, {\bf TOLSTÓI}, e {\bf PROUST}.

\page

A influência de Ruskin sobre Proust se manifesta essencialmente em dois aspectos: a {\bf NARRATIVA NÃO LINEAR}, mas cíclica, na qual os temas são retomados e retrabalhados ao longo do texto; e uma espécie de {\bf SOLIPSISMO LITERÁRIO}, no qual cada ente, animado ou inanimado, que entra no campo existencial do escritos ganha importância e transcendência.

\page %---------------------------------------------------------|

\hyphenpenalty=10000
\exhyphenpenalty=10000

«Escrevi de modo franco e gárrulo, ao correr da pena; estendendo-me o quanto quisesse sobre o que me dava prazer recordar e guardando total silêncio sobre coisas que não me davam qualquer prazer recordar, e cujo relato não interessaria ao leitor. A descrição de minha vida revelou-se mais divertida do que eu esperava, à medida que evoquei cenas do passado longínquo para {\bf SUBMETÊ-LAS AO ESCRUTÍNIO DO PRESENTE.}»

{\vfill\scale[factor=7]{\Seta,{\bf Praeterita}, John Ruskin}}

\page %---------------------------------------------------------|

\Hedra

\stoptext %---------------------------------------------------------|