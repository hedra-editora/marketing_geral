% AUTOR_LIVRO_EFEMERIDE.tex
% Preencher com o nome das cor ou composição RGB (ex: [r=0.862, g=0.118, b=0.118]) 
\usecolors[crayola] 			   % Paleta de cores pré-definida: wiki.contextgarden.net/Color#Pre-defined_colors

% Cores definidas pelo designer:
% MyGreen		r=0.251, g=0.678, b=0.290 % 40ad4a
% MyCyan		r=0.188, g=0.749, b=0.741 % 30bfbd
% MyRed			r=0.820, g=0.141, b=0.161 % d12429
% MyPink		r=0.980, g=0.780, b=0.761 % fac7c2
% MyGray		r=0.812, g=0.788, b=0.780 % cfc9c7
% MyOrange		r=0.980, g=0.671, b=0.290 % faab4a

% Configuração de cores
\definecolor[MyColor][x=ec248b]      % ou ex: [r=0.862, g=0.118, b=0.118] % corresponde a RGB(220, 30, 30)
\definecolor[MyColorText][black]     % ou ex: [r=0.862, g=0.118, b=0.118] % corresponde a RGB(167, 169, 172)

% Classe para diagramação dos posts
\environment{marketing.env}		   

\starttext %---------------------------------------------------------|

\hyphenpenalty=10000
\exhyphenpenalty=10000

\Mensagem{30 DE MAIO} %Sempre usar esse header

\MyPicture{THUMB_JULIA_LOPES.jpeg}

\vfill\scale[factor=6]{\Seta\,90 ANOS SEM {\bf JÚLIA LOPES DE ALMEIDA}}

\page %---------------------------------------------------------| 

\hyphenpenalty=10000
\exhyphenpenalty=10000

Noventa anos atrás, a literatura brasileira perdia Júlia Lopes de Almeida.
Nascida no Rio de Janeiro, em 1862, ela foi um verdadeiro {\bf FENÔMENO 
LITERÁRIO} de sua época: escreveu romances, contos, novelas, peças teatrais, 
crônicas, relatos de viagem, ensaios, livros didáticos e infantis, alcançando
sucesso de público e crítica. 

\page

\hyphenpenalty=10000
\exhyphenpenalty=10000

Em seu casarão no bairro de Santa Teresa, Júlia oferecia celebrados saraus nos 
jardins, então conhecidos como {\bf SALÃO VERDE}. Ela foi uma das idealizadoras 
da Academia Brasileira de Letras e deveria ter ocupado uma cadeira na entidade, 
mas foi afastada da lista oficial dos primeiros “imortais” por ser mulher, 
posicionamento machista que contribuiu para seu subsequente apagamento. 

\page

\hyphenpenalty=10000
\exhyphenpenalty=10000

Ela lutou pela {\bf EMANCIPAÇÃO FEMININA}, criticou filósofos misóginos e 
contestou severamente o tipo de educação que as mulheres recebiam, 
destinada apenas ao casamento e à futilidade. Desde seu falecimento, em 1934, 
foi gradativa e injustamente alijada da memória e história literárias. Recentemente, sua obra tem sido considerada uma das mais relevantes da literatura
brasileira.   

\page %---------------------------------------------------------|

\MyCover{JULIALOPES_CONTOS_THUMB.jpeg}

\page %---------------------------------------------------------|

\Hedra

\stoptext %---------------------------------------------------------|]]></content>
