% AUTOR_LIVRO_CURIOSIDADES.tex
% Preencher com o nome das cor ou composição RGB (ex: [r=0.862, g=0.118, b=0.118]) 
\usecolors[crayola] 			   % Paleta de cores pré-definida: wiki.contextgarden.net/Color#Pre-defined_colors

% Cores definidas pelo designer:
% MyGreen		r=0.251, g=0.678, b=0.290 % 40ad4a
% MyCyan		r=0.188, g=0.749, b=0.741 % 30bfbd
% MyRed			r=0.820, g=0.141, b=0.161 % d12429
% MyPink		r=0.980, g=0.780, b=0.761 % fac7c2
% MyGray		r=0.812, g=0.788, b=0.780 % cfc9c7
% MyOrange		r=0.980, g=0.671, b=0.290 % faab4a

% Configuração de cores
\definecolor[MyColor][MaximumRed]      % ou ex: [r=0.862, g=0.118, b=0.118] % corresponde a RGB(220, 30, 30)
\definecolor[MyColorText][black]     % ou ex: [r=0.862, g=0.118, b=0.118] % corresponde a RGB(167, 169, 

% Classe para diagramação dos posts
\environment{marketing.env}		   

\starttext %---------------------------------------------------------|

\hyphenpenalty=10000
\exhyphenpenalty=10000

\Mensagem{160 ANOS DA 1ª INTERNACIONAL} %Sempre usar esse header

\startMyCampaign

\hyphenpenalty=10000
\exhyphenpenalty=10000

«ESTIVESSE {\bf MARX} AINDA AO MEU LADO, PARA VER ISSO COM OS PRÓPRIOS OLHOS!»

\stopMyCampaign

\page %---------------------------------------------------------| 

\hyphenpenalty=10000
\exhyphenpenalty=10000

Com essas as palavras Engels encerra o prefácio do {\bf MANIFESTO COMUNISTA}, escrito no dia 1º maio de 1890.

\page %---------------------------------------------------------|

Esse foi um dia marcado pela greve geral dos operários estadunidenses, que reivindicavam a redução da jornada de trabalho de 16 horas para 8 horas. Mais tarde, essa data se firmaria como {\bf DIA DO TRABALHO}, visando homenagear a mobilização dos trabalhadores e os mortos durante a repressão.

\page

Apesar do fim da {\bf ASSOCIAÇÃO INTERNACIONAL DOS \\TRABALHADORES}, Engels reconhece nos eventos de 1º de maio a continuidade da «eterna aliança dos proletários de todos os países, fundada por ela».

\page

«{\bf “PROLETÁRIOS DE TODOS OS PAÍSES, UNI-VOS!”} Somente poucas vozes
responderam quando lançamos essas palavras ao mundo, já há 42 anos. Mas, em 28 de
setembro de 1864, proletários da maioria dos países da Europa ocidental
uniam-se na Associação Internacional dos Trabalhadores, de gloriosa
memória...
\vfill\hfill →

\page

E o espetáculo do presente dia haverá de abrir os olhos dos capitalistas e
proprietários fundiários de todos os países para o fato de que hoje {\bf OS
PROLETÁRIOS DE TODOS OS PAÍSES ESTÃO EFETIVAMENTE UNIDOS}.»
\page

\MyCover{MARX_MANIFESTO_THUMB}

\page %---------------------------------------------------------|

\Hedra

\stoptext %---------------------------------------------------------|




«A Internacional viveu apenas nove anos. Mas, que a
eterna aliança dos proletários de todos os países, fundada por ela,
ainda vive, e com mais força do que nunca --- para este fato não há
melhor testemunho do que, exatamente, o presente dia.»

Pois hoje, enquanto
eu escrevo estas linhas, o proletariado europeu e americano passa em
revista suas forças de combate mobilizadas pela primeira vez ---
mobilizadas como um exército, sob uma bandeira e para um objetivo
próximo: a implantação legal da jornada de trabalho de oito horas,
proclamada já em 1866 pelo Congresso da Internacional em Genebra e,
reiteradamente, pelo Congresso Operário parisiense em 1889. E o
espetáculo do presente dia haverá de abrir os olhos dos capitalistas e
proprietários fundiários de todos os países para o fato de que hoje os
proletários de todos os países estão efetivamente unidos.\looseness=-1



