% AUTOR_LIVRO_CURIOSIDADE.tex
% Vamos falar sobre isso "curiosidades"
% > "EM CONTEXTO"

% Preencher com o nome das cor ou composição RGB (ex: [r=0.862, g=0.118, b=0.118]) 
\usecolors[crayola] 			   % Paleta de cores pré-definida: wiki.contextgarden.net/Color#Pre-defined_colors

% Cores definidas pelo designer:
% MyGreen		r=0.251, g=0.678, b=0.290 % 40ad4a
% MyCyan		r=0.188, g=0.749, b=0.741 % 30bfbd
% MyRed			r=0.820, g=0.141, b=0.161 % d12429
% MyPink		r=0.980, g=0.780, b=0.761 % fac7c2
% MyGray		r=0.812, g=0.788, b=0.780 % cfc9c7
% MyOrange		r=0.980, g=0.671, b=0.290 % faab4a

% Configuração de cores
\definecolor[MyColor][MiddleRed]      % ou ex: [r=0.862, g=0.118, b=0.118] % corresponde a RGB(220, 30, 30)
\definecolor[MyColorText][black]     % ou ex: [r=0.862, g=0.118, b=0.118] % corresponde a RGB(167, 169, 172)

% Classe para diagramação dos posts
\environment{marketing.env}		   

\starttext %---------------------------------------------------------|

\hyphenpenalty=10000
\exhyphenpenalty=10000

\Mensagem{EM CONTEXTO} %Sempre usar esse header

\startMyCampaign

\hyphenpenalty=10000
\exhyphenpenalty=10000

O PASSADO ANARQUISTA E RADICAL DE {\bf MARY SHELLEY}, AUTORA DE {\bf FRANKENSTEIN}

\stopMyCampaign

\page %---------------------------------------------------------| 

\hyphenpenalty=10000
\exhyphenpenalty=10000

Mary Shelley, filha de duas grandes figuras literárias da Inglaterra do século {\cap XVIII} --- Mary Wollstonecraft e William Godwin, será sempre lembrada por seu romance de estreia e mais conhecida criação, o clássico {\bf FRANKENSTEIN, OU O PROMETEU MODERNO} (1818), uma das obras-primas da literatura.


\page

 Dentre os intelectuais presentes nos encontros na casa de Godwin, estava {\bf PERCY BYSSHE SHELLEY}, que, da admiração ao pai, passaria ao casamento com a filha. 

\page %---------------------------------------------------------|

A relação entre Shelley e Mary seria importantíssima para o desenvolvimento literário da escritora: sem o encorajamento, a pena e os cuidados editoriais do poeta, é possível que «Frankenstein» --- escrito durante viagem do casal à Suíça, onde frequentavam a casa de {\bf LORD BYRON} --- jamais tivesse existido. 

\page %---------------------------------------------------------|

Já sem Shelley, morto em um naufrágio na Itália em 1822, e decidida a viver da própria escrita, Mary produziria {\bf OUTRAS CINCO OBRAS}, como o romance apocalíptico e precursor da ficção científica «The Last Man» (1825), forjadas no mesmo caldeirão político de «Frankenstein», além de literatura de viagem e trabalhos editoriais, como a organização dos escritos do pai. 

\page %---------------------------------------------------------|

Mas nenhuma de suas outras obras se equiparou a {\bf FRANKENSTEIN}, que a partir da década de 1820 seria levado aos teatros. Isso foi fundamental para a transformação futura do romance em obra fílmica, e à apreciação do grande público. 

\page %---------------------------------------------------------|

%\MyCover{SHELLEY_FRANKENSTEIN_THUMB.jpg}

%\page %---------------------------------------------------------|

\Hedra

\stoptext %---------------------------------------------------------|