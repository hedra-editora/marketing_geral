% AUTOR_LIVRO_EFEMERIDE.tex
% Preencher com o nome das cor ou composição RGB (ex: [r=0.862, g=0.118, b=0.118]) 
\usecolors[crayola] 			   % Paleta de cores pré-definida: wiki.contextgarden.net/Color#Pre-defined_colors

% Cores definidas pelo designer:
% MyGreen		r=0.251, g=0.678, b=0.290 % 40ad4a
% MyCyan		r=0.188, g=0.749, b=0.741 % 30bfbd
% MyRed			r=0.820, g=0.141, b=0.161 % d12429
% MyPink		r=0.980, g=0.780, b=0.761 % fac7c2
% MyGray		r=0.812, g=0.788, b=0.780 % cfc9c7
% MyOrange		r=0.980, g=0.671, b=0.290 % faab4a

% Configuração de cores
\definecolor[MyColor][MaximumRed]      % ou ex: [r=0.862, g=0.118, b=0.118] % corresponde a RGB(220, 30, 30)
\definecolor[MyColorText][black]     % ou ex: [r=0.862, g=0.118, b=0.118] % corresponde a RGB(167, 169, 172)

% Classe para diagramação dos posts
\environment{marketing.env}		   

\starttext %---------------------------------------------------------|

\hyphenpenalty=10000
\exhyphenpenalty=10000

\Mensagem{206 ANOS DE MARX} %Sempre usar esse header

\MyPicture{MARX6b}

%\vfill\scale[factor=6]{\Seta\,DIA DO TRABALHO}

\page %---------------------------------------------------------| 

\hyphenpenalty=10000
\exhyphenpenalty=10000

No processo de superação da sociedade capitalista, «desaparecidas as diferenças de classes no curso do desenvolvimento e concentrada toda a produção nas mãos dos indivíduos associados, então {\bf o poder público perde o caráter político}»

%\page

%A {\bf GREVE GERAL}, estimulada por anarquistas e sindicalistas, conquistou ampla adesão, envolvendo cerca de 340\,000 trabalhadores em todo o país. Os protestos foram marcados por {\bf REPRESSÃO} e confrontos violentos entre grevistas e policiais.

% \page

%  Após a {\bf EXPLOSÃO DE UMA BOMBA} que resultou na morte de agentes da polícia, os sindicalistas anarquistas Albert Parsons, Adolph Fischer, George Engel e August Spies foram condenados à morte por enforcamento, apesar da falta de evidências. O episódio ficou conhecido como {\bf BLACK FRIDAY}.

% \page

% Três outros trabalhadores condenados à {\bf PRISÃO PERPÉTUA} foram inocentados e reabilitados, em 1893, depois da confirmação de que teria sido o chefe da polícia quem encomendou o {\bf ATENTADO} para justificar a repressão que viria a seguir.

% \page

% Em 1889, a {\bf SEGUNDA \\INTERNACIONAL SOCIALISTA}, elegeu o 1º de maio para uma manifestação anual pelos direitos trabalhistas. Em 1919, o Senado francês tornou a data um {\bf FERIADO NACIONAL} ao ratificar a jornada de oito horas, o que foi posteriormente implementado por outras nações.

\page



  «No lugar da velha sociedade burguesa com as suas classes e os seus antagonismos de classes surge uma associação na qual {\bf o livre desenvolvimento de cada um é a condição para o livre desenvolvimento de todos}.»

\vfill\scale[factor=5]{{\bf Karl Marx}, no «Manifesto comunista».}
\page


\MyCover{MARX_MANIFESTO_THUMB}


\page %---------------------------------------------------------|

\Hedra

\stoptext %---------------------------------------------------------|

