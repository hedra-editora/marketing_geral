% FIRMINA_ESCRAVA_EDICAO.tex
% Preencher com o nome das cor ou composição RGB (ex: [r=0.862, g=0.118, b=0.118]) 
\usecolors[crayola] 			   % Paleta de cores pré-definida: wiki.contextgarden.net/Color#Pre-defined_colors

% Cores definidas pelo designer:
% MyGreen		r=0.251, g=0.678, b=0.290 % 40ad4a
% MyCyan		r=0.188, g=0.749, b=0.741 % 30bfbd
% MyRed			r=0.820, g=0.141, b=0.161 % d12429
% MyPink		r=0.980, g=0.780, b=0.761 % fac7c2
% MyGray		r=0.812, g=0.788, b=0.780 % cfc9c7
% MyOrange		r=0.980, g=0.671, b=0.290 % faab4a

% Configuração de cores
\definecolor[MyColor][TealBlue]      % ou ex: [r=0.862, g=0.118, b=0.118] % corresponde a RGB(220, 30, 30)
\definecolor[MyColorText][white]     % ou ex: [r=0.862, g=0.118, b=0.118] % corresponde a RGB(167, 169, 172)

% Classe para diagramação dos posts
\environment{marketing.env}		   

\starttext %---------------------------------------------------------|

\Mensagem{LITERATURA E ESCRAVIDÃO}

\startMyCampaign

\hyphenpenalty=10000
\exhyphenpenalty=10000

{\bf 
AS DIVERSAS FACETAS DE MARIA FIRMINA}

\stopMyCampaign

%\vfill\scale[lines=1.5]{\MyStar[MyColorText][none]}

\page %---------------------------------------------------------| 

\MyCover{HEDRA_FIRMINA_THUMB.pdf}

\page %---------------------------------------------------------| 

\hyphenpenalty=10000
\exhyphenpenalty=10000

Através dessa antologia, entramos em contato com as diversas facetas da produção literária de {\bf MARIA FIRMINA DOS REIS} --- conto, novela e poesia --- e percebemos as {\bf LINHAS DE FORÇA} que atravessaram toda a sua obra.

\page

 O fio que perpassa toda a antologia é não só o relato da escravidão, mas {\bf A HISTÓRIA DE RESISTÊNCIA E LIBERDADE DOS ESCRAVIZADOS}.

\page %---------------------------------------------------------|

O livro se inicia com o conto «A escrava», de nítido caráter abolicionista, e se encerra com «Hino à liberdade dos escravos», composto na ocasião da {\bf ABOLIÇÃO DA ESCRAVATURA} em 1888.

\page
\hyphenpenalty=10000
\exhyphenpenalty=10000

«Quebrou-se enfim a cadeia\\
Da nefanda {\bf ESCRAVIDÃO}!\\
Aqueles que antes oprimias,\\
Hoje terás como irmão!»


{\vfill\scale[factor=5]{{\bf Hino à liberdade dos escravos}, Maria Firmina}}

% {\vfill\scale[factor=5]{{\bf A escrava}, Maria Firmina}}
% «Por qualquer modo que encaremos a {\bf ESCRAVIDÃO}, ela é, e será sempre um
% grande mal. Dela a decadência do comércio; porque o comércio e a lavoura
% caminham de mãos dadas, e o escravo não pode fazer florescer a lavoura;
% porque o seu trabalho é forçado. Ele não tem futuro.»

% {\vfill\scale[factor=5]{{\bf A escrava}, Maria Firmina}}
\page %---------------------------------------------------------|

\Hedra

\stoptext %---------------------------------------------------------|

