% AUTOR_LIVRO_CURIOSIDADE.tex
% Vamos falar sobre isso "curiosidades"
% > "EM CONTEXTO"

% Preencher com o nome das cor ou composição RGB (ex: [r=0.862, g=0.118, b=0.118]) 
\usecolors[crayola] 			   % Paleta de cores pré-definida: wiki.contextgarden.net/Color#Pre-defined_colors

% Cores definidas pelo designer:
% MyGreen		r=0.251, g=0.678, b=0.290 % 40ad4a
% MyCyan		r=0.188, g=0.749, b=0.741 % 30bfbd
% MyRed			r=0.820, g=0.141, b=0.161 % d12429
% MyPink		r=0.980, g=0.780, b=0.761 % fac7c2
% MyGray		r=0.812, g=0.788, b=0.780 % cfc9c7
% MyOrange		r=0.980, g=0.671, b=0.290 % faab4a

% Configuração de cores
\definecolor[MyColor][MagicMint]      % ou ex: [r=0.862, g=0.118, b=0.118] % corresponde a RGB(220, 30, 30)
\definecolor[MyColorText][black]     % ou ex: [r=0.862, g=0.118, b=0.118] % corresponde a RGB(167, 169, 172)

% Classe para diagramação dos posts
\environment{marketing.env}		   

\starttext %---------------------------------------------------------|

\hyphenpenalty=10000
\exhyphenpenalty=10000

\Mensagem{EM CONTEXTO} %Sempre usar esse header

\startMyCampaign

\hyphenpenalty=10000
\exhyphenpenalty=10000

PARA SABER MAIS SOBRE {\bf HEGEL,\\ 
FIGURA CENTRAL}\\ 
DA FILOSOFIA MODERNA\\ 


\stopMyCampaign

\page %---------------------------------------------------------| 

\hyphenpenalty=10000
\exhyphenpenalty=10000

Hegel (1770--1831), nascido Georg Wilhelm Friedrich Hegel, foi um dos principais representantes do {\bf IDEALISMO ALEMÃO. }


\page

Nascido em Stuttgart, no Sacro Império Romano-Germânico, suas principais influências foram o {\bf ILUMINISMO E A FILOSOFIA ANTIGA}, além de ter sido\\ profundamente impactado pela Revolução Francesa e pelas guerras napoleônicas.

\page %---------------------------------------------------------|

Aos dezoito anos, Hegel entrou na Stift de Tobinger, um seminário protestante ligado à Universidade de Tübingen, onde teve como companheiros de quarto {\bf O POETA FRIEDRICH HÖLDERLIN E O FILÓSOFO FRIEDRICH SCHELLING}, com quem compartilhava uma antipatia pelo ambiente restritivo do Seminário. Os três se tornaram amigos íntimos e influenciaram mutuamente as ideias um do outro. 

\page %---------------------------------------------------------|

Hegel também atuou como tutor em casas de famílias aristocratas e, mais tarde, lecionou nas universidades de Jena, Heidelberg, e Berlim, onde suas palestras atraíram um público considerável. 


\page

Obras como «Fenomenologia do Espírito», «Enciclopédia das Ciências Filosóficas» e «Filosofia do Direito», tiveram ampla influência na filosofia ocidental, e inspiraram movimentos como o {\bf MARXISMO, O EXISTENCIALISMO, E A TEORIA CRÍTICA}.

\page %---------------------------------------------------------|

\Hedra

\stoptext %---------------------------------------------------------|