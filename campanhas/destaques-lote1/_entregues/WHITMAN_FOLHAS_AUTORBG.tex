% AUTOR_LIVRO_AUTOR.tex
\usecolors[crayola]

% Configuração de cores TickleMePink
\definecolor[MyColor][black]      % ou ex: [r=0.862, g=0.118, b=0.118] % corresponde a RGB(220, 30, 30)
\definecolor[MyColorText][white]     % ou ex: [r=0.862, g=0.118, b=0.118] % corresponde a RGB(167, 169, 172)

% Classe para diagramação dos posts
\environment{marketing.env}        


% Cabeço e rodapé: Informações (caso queira trocar alguma coisa)
        \def\MensagemSaibaMais  {SAIBA MAIS:}
        \def\MensagemSite       {HEDRA.COM.BR}
        \def\MensagemLink       {LINK NA BIO}
      
\environment{extra.env}

\starttext  %---------------------------------------------------------|

\def\MyBackgroundMessage{PAI DO VERSO LIVRE}

\MyBackground{WHITMAN_FOLHAS_2}

\startMyCampaign
\hyphenpenalty=10000
\exhyphenpenalty=10000
%{\bf NOAN CHOMSKY} O ANARQUISTA DO NOSSO SÉCULO
\position(0,7.8){\scale[factor=4]{\Seta\,WALT WHITMAN (1819--1892)}}
\stopMyCampaign

\page 

\Mensagem{PAI DO VERSO LIVRE}
\setupbackgrounds[page][background=color,backgroundcolor=MyColor]

\hyphenpenalty=10000
\exhyphenpenalty=10000

 Um dos poetas mais influentes da literatura americana do século {\cap XIX}, {\bf WALT WHITMAN} nasceu em Long Island e cresceu no Brooklyn, em Nova York.

 \page

 Por ser de família humilde, aos onze anos abandonou os estudos para trabalhar, tendo atuado como tipógrafo, professor e editor. O poeta passou parte de sua juventude viajando pelos Estados Unidos, o que serviu de inspiração sua mais famosa obra, {\bf FOLHAS DE RELVA}.

 \page

\MyPicture{WHITMAN_FOLHAS_3.png}

Conhecido por {\bf ROMPER COM AS CONVENÇÕES POÉTICAS} de sua época, Whitman não teve amplo {\bf RECONHECIMENTO PÚBLICO} enquanto vivo.

\page

Contudo, deixou um legado duradouro como um dos poetas mais inovadores da literatura americana, tendo influenciado grandes nomes como {\bf EZRA POUND, WILLIAM CARLOS WILLIAMS} e {\bf ALLEN GINSBERG}.


\page %----------------------------------------------------------|

\MyCover{WHITMAN_FOLHAS_THUMB.pdf}

\page %----------------------------------------------------------|

\Hedra

\stoptext %---------------------------------------------------------|


