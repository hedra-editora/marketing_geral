% AKUTAGAWA_RASHOMON_DETALHE.tex
% Detalhes da edição ou por dentro da edição: ex: sumário
% > "POR DENTRO DA EDIÇÃO"

% Preencher com o nome das cor ou composição RGB (ex: [r=0.862, g=0.118, b=0.118]) 
\usecolors[crayola] 			   % Paleta de cores pré-definida: wiki.contextgarden.net/Color#Pre-defined_colors

% Cores definidas pelo designer:
% MyGreen		r=0.251, g=0.678, b=0.290 % 40ad4a
% MyCyan		r=0.188, g=0.749, b=0.741 % 30bfbd
% MyRed			r=0.820, g=0.141, b=0.161 % d12429
% MyPink		r=0.980, g=0.780, b=0.761 % fac7c2
% MyGray		r=0.812, g=0.788, b=0.780 % cfc9c7
% MyOrange		r=0.980, g=0.671, b=0.290 % faab4a

% Configuração de cores
\definecolor[MyColor][x=e3ee5c]      % ou ex: [r=0.862, g=0.118, b=0.118] % corresponde a RGB(220, 30, 30)
\definecolor[MyColorText][x=d22027]  % ou ex: [r=0.862, g=0.118, b=0.118] % corresponde a RGB(167, 169, 172)

% Classe para diagramação dos posts
\environment{marketing.env}	
	   
\starttext %---------------------------------------------------------|

\Mensagem{POR DENTRO DA EDIÇÃO}

\MyCover{AKUTAGAWA_RASHOMON_THUMB}

\page %---------------------------------------------------------|

{\bf RASHÔMON E OUTROS CONTOS} reúne dez contos de diversos períodos da breve
existência de Ryûnosuke Akutagawa. As temáticas abordadas vão desde a cultura local, a ética cristã, a abertura do Japão ao Ocidente, até a própria biografia do autor. 


\page

\startpositioning
                \position(-2cm,-5mm){
                \clip[height=7cm]{
                \externalfigure[AKUTAGAWA_RASHOMON_6][height=2\textheight]
                }}	
\stoppositioning


\page

Com organização e tradução do japonês de {\bf MADALENA HASHIMOTO CORDARO} e {\bf JUNKO OTA}, esta nova edição, com texto revisto pelas tradutoras, conta ainda com nova introdução e acréscimo de notas.

\page %---------------------------------------------------------|

{\bf MADALENA HASHIMOTO} é docente de literatura japonesa na Universidade de São Paulo
({\cap usp}). Dedica-se à pesquisa de arte e literatura japonesas do período Edo (1603--1867), 
à tradução de autores japoneses e à produção de obras visuais.


{\bf JUNKO OTA} é tradutora e professora de língua japonesa na {\cap USP}.

\page

\Hedra

\stoptext