% AUTOR_LIVRO_DETALHE.tex
% Detalhes da edição ou por dentro da edição: ex: sumário
% > "POR DENTRO DA EDIÇÃO"

% Preencher com o nome das cor ou composição RGB (ex: [r=0.862, g=0.118, b=0.118]) 
\usecolors[crayola] 			   % Paleta de cores pré-definida: wiki.contextgarden.net/Color#Pre-defined_colors

% Cores definidas pelo designer:
% MyGreen		r=0.251, g=0.678, b=0.290 % 40ad4a
% MyCyan		r=0.188, g=0.749, b=0.741 % 30bfbd
% MyRed			r=0.820, g=0.141, b=0.161 % d12429
% MyPink		r=0.980, g=0.780, b=0.761 % fac7c2
% MyGray		r=0.812, g=0.788, b=0.780 % cfc9c7
% MyOrange		r=0.980, g=0.671, b=0.290 % faab4a

% Configuração de cores
\definecolor[MyColor][BananaMania]      % ou ex: [r=0.862, g=0.118, b=0.118] % corresponde a RGB(220, 30, 30)
\definecolor[MyColorText][BurntUmber]  % ou ex: [r=0.862, g=0.118, b=0.118] % corresponde a RGB(167, 169, 172)

% Classe para diagramação dos posts
\environment{marketing.env}	
	   
\starttext %---------------------------------------------------------|

\Mensagem{POR DENTRO DA EDIÇÃO}

\MyCover{HESIODO_TEOGONIA_THUMB}

\page 

\hyphenpenalty=10000
\exhyphenpenalty=10000

{\bf TEOGONIA} é um poema de 1022 versos hexâmetros datílicos que descreve a
origem e a genealogia dos deuses. Muito do que sabemos sobre os antigos mitos
gregos é graças a esse poema que, pela narração em primeira pessoa do próprio
poeta, sistematiza e organiza as histórias da criação do mundo e do nascimento
dos deuses, com ênfase especial em Zeus e em suas façanhas até chegar ao
poder.

\page %---------------------------------------------------------|

\MyPicture{HESIODO_TRABALHOS_2}

\page %---------------------------------------------------------|

\Hedra

\stoptext