% AUTOR_LIVRO_AUTOR.tex
% Preencher com o nome das cor ou composição RGB (ex: [r=0.862, g=0.118, b=0.118]) 
\usecolors[crayola]                % Paleta de cores pré-definida: wiki.contextgarden.net/Color#Pre-defined_colors

% Cores definidas pelo designer:
% MyGreen       r=0.251, g=0.678, b=0.290 % 40ad4a
% MyCyan        r=0.188, g=0.749, b=0.741 % 30bfbd
% MyRed         r=0.820, g=0.141, b=0.161 % d12429
% MyPink        r=0.980, g=0.780, b=0.761 % fac7c2
% MyGray        r=0.812, g=0.788, b=0.780 % cfc9c7
% MyOrange      r=0.980, g=0.671, b=0.290 % faab4a

% Configuração de cores

\definecolor[MyColor][Eggplant]      % ou ex: [r=0.862, g=0.118, b=0.118] % corresponde a RGB(220, 30, 30)
\definecolor[MyColorText][Canary]  % ou ex: [r=0.862, g=0.118, b=0.118] % corresponde a RGB(167, 169, 172)

% Classe para diagramação dos posts
\environment{marketing.env}        

% Cabeço e rodapé: Informações (caso queira trocar alguma coisa)
        \def\MensagemSaibaMais  {SAIBA MAIS:}
        \def\MensagemSite           {HEDRA.COM.BR}
        \def\MensagemLink           {LINK NA BIO}

\def\MyBackground#1{
\defineoverlay
  [backgroundimage]
  [{\externalfigure[#1][height=\overlayheight]}]
}


\starttext  %---------------------------------------------------------|
\Mensagem{MESTRE DAS ENTRELINHAS}

% Foto para background
\MyBackground{METABIBLIOTECA_MACHADO_PAI_7.jpeg}

\startstandardmakeup[background=backgroundimage]
\startMyCampaign
\vfill\scale[factor=4]{\Seta\,MACHADO DE ASSIS (1839--1908)}
\stopMyCampaign
\stopstandardmakeup

\page 
\Mensagem{MESTRE DAS ENTRELINHAS}


\hyphenpenalty=10000
\exhyphenpenalty=10000


{\bf MACHADO DE ASSIS} foi um renomado escritor do século {\cap XIX}, conhecido pelo retrato pungente que fez da elite brasileira.

\page %----------------------------------------------------------|

Considerado um dos maiores escritores brasileiros de todos os tempos, {\bf MACHADO} transpôs para a forma das suas narrativas os {\bf DILEMAS} do país.

\page

Neto de escravos alforriados, o escritor ganhou a vida em atividades ligadas ao {\bf MUNDO LETRADO}. Conviveu intensamente com a elite artístico-intelectual de sua época, tendo inclusive sido eleito como presidente da {\bf ACADEMIA BRASILEIRA DE LETRAS}, em 1897.

\page
\MyCover{METABIBLIOTECA_MACHADO_PAI_THUMB}

\page %----------------------------------------------------------|

\Hedra

\stoptext %---------------------------------------------------------|


