% Preencher com o nome das cor ou composição RGB (ex: [r=0.862, g=0.118, b=0.118]) 
\usecolors[crayola] 			   % Paleta de cores pré-definida: wiki.contextgarden.net/Color#Pre-defined_colors

% Cores definidas pelo designer:
% MyGreen		r=0.251, g=0.678, b=0.290 % 40ad4a
% MyCyan		r=0.188, g=0.749, b=0.741 % 30bfbd
% MyRed			r=0.820, g=0.141, b=0.161 % d12429
% MyPink		r=0.980, g=0.780, b=0.761 % fac7c2
% MyGray		r=0.812, g=0.788, b=0.780 % cfc9c7
% MyOrange		r=0.980, g=0.671, b=0.290 % faab4a

% Configuração de cores
\definecolor[MyColor][EnglishVermilion]      % ou ex: [r=0.862, g=0.118, b=0.118] % corresponde a RGB(220, 30, 30)
\definecolor[MyColorText][black]     % ou ex: [r=0.862, g=0.118, b=0.118] % corresponde a RGB(167, 169, 172)

% Classe para diagramação dos posts
\environment{marketing.env}		   

\starttext %---------------------------------------------------------|

\Mensagem{EMMA GOLDMAN}

\startMyCampaign

\hyphenpenalty=10000
\exhyphenpenalty=10000
«Todo estímulo que desperta a imaginação e eleva o espírito é tão necessário à vida quanto o ar\ldots{}» 

\vfill\hfill →

\page

\hyphenpenalty=10000
\exhyphenpenalty=10000

«\ldots{}Sem estímulos, o trabalho criativo é impossível, como é também impossível o espírito de bondade e generosidade.»

\stopMyCampaign

{\vfill\scale[factor=6]{\Seta\,Trecho de {\bf Sobre anarquismo, sexo}}\setupinterlinespace[line=1.5ex]\scale[factor=6]{{\bf e casamento}, de Emma Goldman. Excerto}\setupinterlinespace[line=1.5ex]\scale[factor=6]{ do capítulo {\bf A hipocrisia do puritanismo},}\setupinterlinespace[line=1.5ex]\scale[factor=6]{publicado em 1910 por Goldman na revista}\setupinterlinespace[line=1.5ex]\scale[factor=6]{que editava, «Mother Earth».}}

\page %---------------------------------------------------------| 

\MyCover{./GOLDMAN_CASAMENTO_IMAGEM.jpg}

{\vfill\scale[factor=6]{\Seta\,Edição n.\,10 de «Mother Earth».}}
% \setupinterlinespace[line=1.5ex]\scale[factor=6]{{\bf e casamento}, de Emma Goldman. O excerto}\setupinterlinespace[line=1.5ex]\scale[factor=6]{sai do capítulo {\bf A hipocrisia do puritanismo},}

\page %---------------------------------------------------------| 

\MyCover{./GOLDMAN_CASAMENTO_THUMB.pdf}

\page %---------------------------------------------------------|

\Hedra

\stoptext %---------------------------------------------------------|