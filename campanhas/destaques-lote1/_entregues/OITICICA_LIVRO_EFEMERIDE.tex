% OITICICA_EFEMERIDE_2.tex
% Preencher com o nome das cor ou composição RGB (ex: [r=0.862, g=0.118, b=0.118]) 
\usecolors[crayola] 			   % Paleta de cores pré-definida: wiki.contextgarden.net/Color#Pre-defined_colors

% Cores definidas pelo designer:
% MyGreen		r=0.251, g=0.678, b=0.290 % 40ad4a
% MyCyan		r=0.188, g=0.749, b=0.741 % 30bfbd
% MyRed			r=0.820, g=0.141, b=0.161 % d12429
% MyPink		r=0.980, g=0.780, b=0.761 % fac7c2
% MyGray		r=0.812, g=0.788, b=0.780 % cfc9c7
% MyOrange		r=0.980, g=0.671, b=0.290 % faab4a

% Configuração de cores
\definecolor[MyColor][PermanentGeraniumLake]      % ou ex: [r=0.862, g=0.118, b=0.118] % corresponde a RGB(220, 30, 30)
\definecolor[MyColorText][black]     % ou ex: [r=0.862, g=0.118, b=0.118] % corresponde a RGB(167, 169, 172)

% Classe para diagramação dos posts
\environment{marketing.env}		   

\starttext %---------------------------------------------------------|

\hyphenpenalty=10000
\exhyphenpenalty=10000

\Mensagem{POETA E ANARQUISTA} %Sempre usar esse header

\MyPicture{OITICICA_EFEMERIDE_1}

\vfill\scale[factor=6]{\Seta\,{\bf JOSÉ OITICICA} (1882--1957)}

\page %---------------------------------------------------------| 

\hyphenpenalty=10000
\exhyphenpenalty=10000

Anarquista, poeta, filólogo e ativista vegetariano, {\bf JOSÉ OITICICA} se dividia entre a literatura, o magistério e a {\bf MILITÂNCIA ANARQUISTA}.
\page

Oiticica participou ativamente de {\bf MOVIMENTOS SOCIAIS} e escreveu diversas obras sobre linguística e anarquismo.  

\page

Entre os anos de 1911 e 1955, lançou 14 livros de poesia, teoria anarquista e filologia, entre os quais {\bf PRINCÍPIOS E FINS DO PROGRAMA COMUNISTA-ANARQUISTA} (1919) e {\bf A DOUTRINA ANARQUISTA AO ALCANCE DE TODOS} (1945). Em 1946, fundou o jornal {\bf AÇÃO DIRETA}.
\page

Foi um dos grandes articuladores da {\bf INSURREIÇÃO ANARQUISTA DE 1918} que, inspirada pela Revolução Russa, pretendia derrubar o governo central na capital do país. Oiticica foi preso diversas vezes ao longo de sua vida por suas atividades políticas. 

\page %---------------------------------------------------------|

\Hedra

\stoptext %---------------------------------------------------------|

% https://diccionario.cedinci.org/oiticica-jose/
% http://www.estelnegre.org/documents/joseoiticica/joseoiticica.html
% https://vegazeta.com.br/jose-oiticica-o-consumo-de-carne-como-vicio-social/16]      % ou ex: [r=0.862, g=0.118, b=0.118] % corresponde a RGB(220, 30, 30)
