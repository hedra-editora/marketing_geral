% Preencher com o nome das cor ou composição RGB (ex: [r=0.862, g=0.118, b=0.118]) 
\usecolors[crayola] 			   % Paleta de cores pré-definida: wiki.contextgarden.net/Color#Pre-defined_colors

% Cores definidas pelo designer:
% MyGreen		r=0.251, g=0.678, b=0.290 % 40ad4a
% MyCyan		r=0.188, g=0.749, b=0.741 % 30bfbd
% MyRed			r=0.820, g=0.141, b=0.161 % d12429
% MyPink		r=0.980, g=0.780, b=0.761 % fac7c2
% MyGray		r=0.812, g=0.788, b=0.780 % cfc9c7
% MyOrange		r=0.980, g=0.671, b=0.290 % faab4a

% Configuração de cores
\definecolor[MyColor][Mahogany]      % ou ex: [r=0.862, g=0.118, b=0.118] % corresponde a RGB(220, 30, 30)
\definecolor[MyColorText][black]     % ou ex: [r=0.862, g=0.118, b=0.118] % corresponde a RGB(167, 169, 172)

% Classe para diagramação dos posts
\environment{marketing.env}		   

\starttext %---------------------------------------------------------|

% Legenda escrita:
% No livro do Gênesis, é de lábios abertos que vêm o sopro da vida, as palavras da tentação e o anúncio da queda. A fala cria a existência e sua finitude, gerando assim nossa própria humanidade.

% Nossa linguagem, portanto, não está somente a serviço do que é útil, belo e justo. Ela pode, ao contrário, servir ao que há de mais nefasto na condição humana: o ódio por seu se-melhante, visto como seu pior inimigo. Depois de assistir de perto às atrocidades do nazismo e aos terríveis usos de sua linguagem, Klemperer não têm dúvidas em afirmar que as
% «palavras podem ser como minúsculas doses de arsênico: são engolidas de maneira despercebida e parecem ser inofensi-vas; passado um tempo, o efeito do veneno se faz notar».

\Mensagem{A LINGUAGEM FASCISTA}

\startMyCampaign

\hyphenpenalty=10000
\exhyphenpenalty=10000
«Ao abrir a boca, podemos conquistar ou arruinar nossa liberdade, podemos ganhar a vida ou sucumbir à morte\ldots{}» 

\vfill\hfill →

\page

\hyphenpenalty=10000
\exhyphenpenalty=10000

«\ldots{}Vários mitos repetem essa ideia, porque ao fazê-lo os seres humanos respiram, se alimentam e falam. Quando falamos\ldots{}» 

\vfill\hfill →

\page

\hyphenpenalty=10000
\exhyphenpenalty=10000

«\ldots{}nossas palavras podem abrir ou fechar portas, podem ampliar nossos horizontes ou aniquilar nossos sonhos.»

\stopMyCampaign

{\vfill\scale[factor=6]{\Seta\,Trecho da introdução do livro {\bf A linguagem}}\setupinterlinespace[line=1.5ex]\scale[factor=6]{{\bf fascista}, de Carlos Piovezani e Emilio Gentile.}}

\page %---------------------------------------------------------| 

\MyCover{./PIOVEZANI_LINGUAGEM_THUMB.pdf}

\page %---------------------------------------------------------|

\Hedra

\stoptext %---------------------------------------------------------|