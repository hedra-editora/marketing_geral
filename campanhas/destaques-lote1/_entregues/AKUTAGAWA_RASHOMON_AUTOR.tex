% AKUTAGAWA_RASHOMON_AUTOR.tex
% Preencher com o nome das cor ou composição RGB (ex: [r=0.862, g=0.118, b=0.118]) 
\usecolors[crayola] 			   % Paleta de cores pré-definida: wiki.contextgarden.net/Color#Pre-defined_colors

% Cores definidas pelo designer:
% MyGreen		r=0.251, g=0.678, b=0.290 % 40ad4a
% MyCyan		r=0.188, g=0.749, b=0.741 % 30bfbd
% MyRed			r=0.820, g=0.141, b=0.161 % d12429
% MyPink		r=0.980, g=0.780, b=0.761 % fac7c2
% MyGray		r=0.812, g=0.788, b=0.780 % cfc9c7
% MyOrange		r=0.980, g=0.671, b=0.290 % faab4a

% Configuração de cores
\definecolor[MyColor][x=e3ee5c]      % ou ex: [r=0.862, g=0.118, b=0.118] % corresponde a RGB(220, 30, 30)
\definecolor[MyColorText][x=d22027]  % ou ex: [r=0.862, g=0.118, b=0.118] % corresponde a RGB(167, 169, 172)

% Classe para diagramação dos posts
\environment{marketing.env}		   

% Cabeço e rodapé: Informações (caso queira trocar alguma coisa)
 		\def\MensagemSaibaMais  {SAIBA MAIS:}
 		\def\MensagemSite		{HEDRA.COM.BR}
 		\def\MensagemLink       {LINK NA BIO}


\starttext %--------------------------------------------------------|

\Mensagem{O MESTRE DO CONTO JAPONÊS}

\MyPicture{AKUTAGAWA_RASHOMON_3.jpeg}

\vfill
\Seta Ryûnosuke Akutagawa (1892--1927)

\page
% 芥川龍之介
\hyphenpenalty=10000
\exhyphenpenalty=10000

{\bf RYÛNOSUKE AKUTAGAWA}, um dos maiores nomes da
literatura moderna japonesa, nasceu em Tóquio no
fim do século {\cap XIX}, durante o período Meiji, quando
o país se abria à influência da cultura 
ocidental. Mesclou tradição e modernidade, reinterpretando temas japoneses do século {\cap XII} e autores e filósofos ocidentais. Escreveu também contos autobiográficos, que relembram o contexto familiar tremendamente conturbado.

%, como {\bf PASSAGENS DO CADERNO DE NOTAS DE YASUKICHI} e {\bf A VIDA DE UM IDIOTA}.

% Leitor precoce, ainda criança lê com 
% entusiasmo traduções do dramaturgo norueguês {\bf HENRIK IBSEN} e do escritor francês {\bf ANATOLE FRANCE}.
% Na juventude, traduz {\bf W.\,B.\,YEATS} e se especializa em literatura inglesa na Universidade Imperial de Tóquio.

% \page

% A instabilidade psíquico-emocional de sua mãe, então considerada louca, perseguiu-o como um  fantasma durante a vida inteira, e há quem diga que foi isso que o levou ao suicídio, em julho de 1927.


\page

\MyCover{AKUTAGAWA_RASHOMON_THUMB}

\page

\Hedra

\stoptext %---------------------------------------------------------|
			