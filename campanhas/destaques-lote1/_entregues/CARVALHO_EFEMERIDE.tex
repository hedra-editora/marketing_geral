% FCARVALHO_CURIOSIDADES.tex
% Preencher com o nome das cor ou composição RGB (ex: [r=0.862, g=0.118, b=0.118]) 
\usecolors[crayola] 			   % Paleta de cores pré-definida: wiki.contextgarden.net/Color#Pre-defined_colors

% Cores definidas pelo designer:
% MyGreen		r=0.251, g=0.678, b=0.290 % 40ad4a
% MyCyan		r=0.188, g=0.749, b=0.741 % 30bfbd
% MyRed			r=0.820, g=0.141, b=0.161 % d12429
% MyPink		r=0.980, g=0.780, b=0.761 % fac7c2
% MyGray		r=0.812, g=0.788, b=0.780 % cfc9c7
% MyOrange		r=0.980, g=0.671, b=0.290 % faab4a

% Configuração de cores
\definecolor[MyColor][Carmine]      % ou ex: [r=0.862, g=0.118, b=0.118] % corresponde a RGB(220, 30, 30)
\definecolor[MyColorText][black]     % ou ex: [r=0.862, g=0.118, b=0.118] % corresponde a RGB(167, 169, 172)

% Classe para diagramação dos posts
\environment{marketing.env}		   

\starttext %---------------------------------------------------------|

\hyphenpenalty=10000
\exhyphenpenalty=10000

\Mensagem{EM CONTEXTO} %Sempre usar esse header

\startMyCampaign

\hyphenpenalty=10000
\exhyphenpenalty=10000


O DIA QUE {\bf FLÁVIO DE CARVALHO}
DESFILOU DE MINISSAIA %Aqui a manchete pode ser mais longa

\stopMyCampaign

\page %---------------------------------------------------------| 

\hyphenpenalty=10000
\exhyphenpenalty=10000

\MyPhoto{flavio5}

\page

{\bf FLÁVIO DE CARVALHO} foi um dos grandes nomes da geração modernista brasileira, atuando como arquiteto, dramaturgo, pintor, escritor, filósofo e músico. Além disso, o seu {\it Experiência nº 3} o consolidou como {\bf PERCUSSOR DA PERFORMANCE} no país. 

\page %---------------------------------------------------------|

A {\it Experiência} aconteceu em 1956, quando o arquiteto escrevia no Diário de São Paulo. Seu editor pediu que fizesse um modelo de roupa masculino. Então, Flávio criou um traje composto por uma minissaia, uma camisa bufante, um chapéu e uma meia arrastão com sandálias de couro. 

\page

Vestindo essas peças que compunham seu {\it new look}, fruto de pesquisas que dedicou à moda na década 1950, Flávio desfilou pelo centro de São Paulo em direção ao Cine Marrocos, onde o uso de terno e gravata era obrigatório para os homens.

\page

\MyPhoto{flavio2}

\page


Chamado ainda de «traje tropical», o {\it new look} seria uma alternativa apropriada aos climas tropicais. Como vantagens do conjunto, o artista listou: «maior circulação de ar, meias arrastão opcionais para esconder varizes e cores vivas que substituem desejos de agressão».


\page %---------------------------------------------------------|

\MyPhoto{flavio4}

\page

Em breve a editora Hedra publicará, em conjunto com a editora Circuito, uma seleção da obra de Flávio de Carvalho em três volumes.

\page

\Hedra

\stoptext %---------------------------------------------------------|




