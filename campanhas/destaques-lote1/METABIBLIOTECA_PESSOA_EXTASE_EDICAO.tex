% AUTOR_LIVRO_EDICAO.tex
% Preencher com o nome das cor ou composição RGB (ex: [r=0.862, g=0.118, b=0.118]) 
\usecolors[crayola] 			   % Paleta de cores pré-definida: wiki.contextgarden.net/Color#Pre-defined_colors

% Cores definidas pelo designer:
% MyGreen		r=0.251, g=0.678, b=0.290 % 40ad4a
% MyCyan		r=0.188, g=0.749, b=0.741 % 30bfbd
% MyRed			r=0.820, g=0.141, b=0.161 % d12429
% MyPink		r=0.980, g=0.780, b=0.761 % fac7c2
% MyGray		r=0.812, g=0.788, b=0.780 % cfc9c7
% MyOrange		r=0.980, g=0.671, b=0.290 % faab4a

% Configuração de cores% Configuração de cores
\definecolor[MyColor][MyCyan]      % ou ex: [r=0.862, g=0.118, b=0.118] % corresponde a RGB(220, 30, 30)
\definecolor[MyColorText][black]  % ou ex: [r=0.862, g=0.118, b=0.118] % corresponde a RGB(167, 169, 172)



% Classe para diagramação dos posts
\environment{marketing.env}		   


\starttext %---------------------------------------------------------|

\Mensagem{TEATRO}

\startMyCampaign

\hyphenpenalty=10000
\exhyphenpenalty=10000

CINCO\ 
POEMAS\ 
DRAMÁTICOS\
DE {\bf FERNANDO\ 
PESSOA}

\stopMyCampaign

%\vfill\scale[lines=1.5]{\MyStar[MyColorText][none]}

\page %---------------------------------------------------------| 

\MyCover{METABIBLIOTECA_PESSOA_EXTASE_THUMB}

\page %---------------------------------------------------------| 

\hyphenpenalty=10000
\exhyphenpenalty=10000

{\bf TEATRO DO ÊXTASE} reúne cinco peças de Fernando Pessoa, concebidas 
como {\bf POEMAS DRAMÁTICOS} e destinadas mais à leitura do que à encenação. 

\page

As {\bf PEÇAS} reunidas --- «O marinheiro», «A morte do príncipe»,  «Diálogo no jardim do palácio», «Salomé» e  «Sakyamuni» --- são provavelmente as mais acabadas dentre os muitos {\bf FRAGMENTOS} deixados por seu autor, e apresentam como  eixo comum a concepção pessoana \\
de êxtase.

\page %---------------------------------------------------------|

\hyphenpenalty=10000
\exhyphenpenalty=10000

«a palavra {\it êxtase} identifica uma característica comum às [peças] que aqui
estão reunidas: nelas, há sempre um momento em que as personagens
 parecem encarnar a figura do {\bf SONHADOR VISIONÁRIO}, que viaja, através de
 conjecturas, para além do {\bf REAL IMEDIATO}, deixando-se absorver por um
 estado de consciência independente de toda e qualquer ação externa.»

{\vfill\scale[factor=5]{{\bf Caio Gagliardi}, professor da Universidade de São Paulo,}\setupinterlinespace[line=1.5ex]\setupinterlinespace[line=1.5ex]\scale[factor=5]{na introdução de {\bf Teatro do êxtase}}}

\page %---------------------------------------------------------|

\Hedra

\stoptext %---------------------------------------------------------|




% %
% \textit{A morte do príncipe} remonta a \textit{Hamlet}, de Shakespeare. 
% Trata de um príncipe que
% alcança, através de sua viagem delirante pelos arcanos da própria alma,
% uma espécie de êxtase visionário, que o leva a afirmar que a única
% realidade reside no sonho, isto é, não na própria vida, mas no teatro
% da vida. 
% %
% \textit{Diálogo no jardim do palácio} guarda referências platônicas,
% no que diz respeito à reflexão sobre o amor e à dicotomia
% entre corpo e alma.
% %
% \textit{Salomé} insere"-se na rica tradição de leituras do
% tema bíblico da \textit{mulher fatal}, ao apresentar o delírio 
% da executora de São João Batista diante de sua cabeça decepada. 
% %
% \textit{Sakyamuni}, por sua vez, representa a ascensão de Siddhartha Gautama ao estado de
% iluminação, em que passa a ser reconhecido como Buda. 
% %


% \textbf{Caio Gagliardi} é professor da Universidade de São Paulo na área de Literatura Portuguesa, onde coordena o grupo de pesquisas Estudos Pessoanos; mestre e doutor em Teoria e História Literária pela \textsc{unicamp}
% e pós"-doutor em Teoria Literária pela \textsc{usp}. É autor de \textit{O renascimento do autor: autoria, heteronímia e \textit{fake memoirs}} (Hedra, 2019) e organizador de \textit{Fernando Pessoa \& Cia. não heterônima} (Mundaréu, 2019), entre outras publicações.


% INTRO GAGLIARDI

