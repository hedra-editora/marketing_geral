% AUTOR_LIVRO_CURIOSIDADE.tex
% Vamos falar sobre isso "curiosidades"
% > "EM CONTEXTO"

% Preencher com o nome das cor ou composição RGB (ex: [r=0.862, g=0.118, b=0.118]) 
\usecolors[crayola] 			   % Paleta de cores pré-definida: wiki.contextgarden.net/Color#Pre-defined_colors

% Cores definidas pelo designer:
% MyGreen		r=0.251, g=0.678, b=0.290 % 40ad4a
% MyCyan		r=0.188, g=0.749, b=0.741 % 30bfbd
% MyRed			r=0.820, g=0.141, b=0.161 % d12429
% MyPink		r=0.980, g=0.780, b=0.761 % fac7c2
% MyGray		r=0.812, g=0.788, b=0.780 % cfc9c7
% MyOrange		r=0.980, g=0.671, b=0.290 % faab4a

% Configuração de cores
\definecolor[MyColor][black]      % ou ex: [r=0.862, g=0.118, b=0.118] % corresponde a RGB(220, 30, 30)
\definecolor[MyColorText][white]     % ou ex: [r=0.862, g=0.118, b=0.118] % corresponde a RGB(167, 169, 172)

% Classe para diagramação dos posts
\environment{marketing.env}		   

\starttext %---------------------------------------------------------|

\hyphenpenalty=10000
\exhyphenpenalty=10000

\Mensagem{EM CONTEXTO}

\startMyCampaign
\hyphenpenalty=10000
\exhyphenpenalty=10000

QUEM FOI O {\bf ESTAGIÁRIO} QUE
DERRUBOU O CÉREBRO DE {\bf WALT WHITMAN}?

%ou QUEM DERRUBOU O CÉREBRO DE {\bf WALT WHITMAN}?


%\vfill\scale[lines=2]{\MyStar[MyColorText][none]} 					% Estrela pequena  

\stopMyCampaign

\page %---------------------------------------------------------| 

\hyphenpenalty=10000
\exhyphenpenalty=10000

Pouco antes de morrer, o poeta {\bf WALT WHITMAN}, adepto da frenologia, estabeleceu como desejo final que o seu {\bf CÉREBRO} fosse conservado e estudado para revelar os segredos de sua mente e as {\bf ORIGENS DA INTELIGÊNCIA HUMANA}.

\page

A extração e o estudo foram confiados ao {\bf DR. HENRY CATTELL}, membro do Instituto Antropométrico, onde os cérebros de outros {\bf AMERICANOS NOTÁVEIS} eram guardados. 

\page

Depois de quinze anos de silêncio sobre os resultados do estudo, foi revelado que, após um desafortunado acidente, {\bf NADA RESTAVA DO CÉREBRO DO POETA}.

\page

Embora a versão mais popular fosse a de que um {\bf DESASTRADO ASSISTENTE}, ao manusear o cérebro, teria o derrubado no chão, a {\bf VERDADE VEIO À TONA} nos diários de Dr. Cattell: ele mesmo foi o responsável pelo acidente e não seu assistente a quem foi atribuída a culpa.

\page %---------------------------------------------------------|

\MyCover{WHITMAN_FOLHAS_THUMB}

\page %---------------------------------------------------------|

\Hedra

\stoptext %---------------------------------------------------------|
	


 % Como em vida ele havia mostrado certa curiosidade pela frenologia - uma pseudociência que relaciona a atividade cerebral com os acidentes ósseos do crânio - pareceu natural para Whitman doar seu cérebro para a ciência para revelar os segredos de sua mente. Houve alguma expectativa sobre os resultados deste estudo em uma figura tão notável. A extração e o estudo foram confiados ao Dr. Henry Cattell, médico da Sociedade Americana de Antropometria. No entanto, os resultados foram adiados e, após quinze anos de silêncio, o Dr. Edward Spitzka (então uma celebridade em estudos neuroanatômicos) declarou que após um "desafortunado acidente", nada restava do cérebro do poeta, encerrando as expectativas científicas de saber onde reside o centro da inspiração lírica. A história do cérebro do gênio, envolto em um frasco com formol, destruído pela negligência de um assistente, penetrou profundamente na cultura americana, a ponto de ser refletida em um filme de Frankenstein em 1931. Investigações posteriores indicaram que não houve um assistente descuidado, mas sim que o próprio Dr. Cattell (então professor na Universidade da Pensilvânia) havia cometido tal erro. O professor Cattell, como já mencionamos, era membro do Instituto Antropométrico, onde os cérebros de outros americanos notáveis eram guardados. Em 1899, os mesmos membros desse instituto fundaram o "Clube do Cérebro" para o estudo anatomopatológico da matéria cinzenta de figuras notáveis... No entanto, o acidente imprevisto pôs fim às especulações. O segredo de como o "acidente" aconteceu pode nunca ter sido conhecido se não fosse pelo diário do Dr. Cattell, onde ele anotava todos os detalhes de sua existência, até mesmo os relacionados a esportes (ele era um entusiasta de beisebol). Em maio de 1893, ele confessou em seu diário: "Sou um tolo, um maldito tolo, sem consciência ou memória, nem sou adequado para uma posição acadêmica. Deixei o cérebro de Whitman exposto sem cobrir o frasco que o continha. Descobri isso esta manhã. Isso vai me arruinar aos olhos dos outros membros da Sociedade Antropométrica..." Por esse motivo, e para evitar retaliações, ele decidiu culpar seu assistente Edward, que não trabalhava mais com ele. O assunto continuou a pesar na consciência do atribulado México. No mesmo diário, em setembro de 1893, ele escreveu: "Se não fosse por meus pais, eu teria ido para a África ou morrido." Nos meses seguintes, os comentários suicidas continuaram, com altos e baixos típicos de uma mente bipolar. Whitman bem sabia que a carne se corrompe, e ele o expressa em seu poema "O tempo que virá": "O curioso molde humano / Não é compartilhado por todos / O cérebro e o coração / Assim como tudo, se corrompem". E isso aconteceu com seu cérebro, que ele deixou como um troféu para a ciência e terminou no lixo por um esquecimento. Se soubesse desse fim sem glória ou epopeia, talvez Whitman tivesse contemplado o pôr do sol, observado as folhas mortas sobre a grama verde e, enquanto fumava seu cachimbo, pensado: "O futuro não é mais incerto do que o presente, aceito a realidade e não me atrevo a questioná-la". A vida, apesar do desfile interminável de deslealdades, tolices e desprezos, merece ser homenageada com rimas e cantos.