% AUTOR_LIVRO_EDICAO.tex
% Preencher com o nome das cor ou composição RGB (ex: [r=0.862, g=0.118, b=0.118]) 
\usecolors[crayola] 			   % Paleta de cores pré-definida: wiki.contextgarden.net/Color#Pre-defined_colors

% Cores definidas pelo designer:
% MyGreen		r=0.251, g=0.678, b=0.290 % 40ad4a
% MyCyan		r=0.188, g=0.749, b=0.741 % 30bfbd
% MyRed			r=0.820, g=0.141, b=0.161 % d12429
% MyPink		r=0.980, g=0.780, b=0.761 % fac7c2
% MyGray		r=0.812, g=0.788, b=0.780 % cfc9c7
% MyOrange		r=0.980, g=0.671, b=0.290 % faab4a

% Configuração de cores
\definecolor[MyColor][PewterBlue]      % ou ex: [r=0.862, g=0.118, b=0.118] % corresponde a RGB(220, 30, 30)
\definecolor[MyColorText][TigersEye]  % ou ex: [r=0.862, g=0.118, b=0.118] % corresponde a RGB(167, 169, 172)

% Classe para diagramação dos posts
\environment{marketing.env}		   

\starttext %---------------------------------------------------------|

\Mensagem{EXPLORANDO FRAGMENTOS}

\startMyCampaign

\hyphenpenalty=10000
\exhyphenpenalty=10000

{\bf 
EDIÇÃO BILÍNGUE
E ANOTADA
DO CONJUNTO DE TEXTOS
DA MÉLICA SÁFICA}

\stopMyCampaign

%\vfill\scale[lines=1.5]{\MyStar[MyColorText][none]}

\page %---------------------------------------------------------| 

\MyCover{SAFO_HINO_THUMB}

\page %---------------------------------------------------------| 

\hyphenpenalty=10000
\exhyphenpenalty=10000

{\bf HINO A AFRODITE E OUTROS POEMAS} reúne os textos traduzidos e anotados remanescentes da mélica sáfica, ou seja, de suas canções para {\bf PERFORMANCE} ao som da lira, em solo ou em coro. 


\page

Mais precisamente, dessa poesia de {\bf TRADIÇÃO ORAL}, foram selecionados a única canção completa e os fragmentos mais legíveis de canções do {\it corpus} de Safo, que sobreviveram ao tempo. 


\page

Anotações de leitura buscam lançar luz sobre elementos relevantes da {\bf ESTRUTURA}, {\bf CONTEÚDO} ou transmissão dos fragmentos organizados tematicamente. 

\page

A edição conta também com uma introdução sobre {\bf SAFO}, sua poesia e o contexto em que se produziu e circulou, o {\bf GÊNERO MÉLICO}, e a fortuna crítica sobre a poeta.

\page

A organização, tradução e notas explicativas são de {\bf GIULIANA RAGUSA}, professora livre-docente de Língua e Literatura Grega na Faculdade de Filosofia, Letras e Ciências Humanas da ({\cap usp}), e autora dos livros {\bf FRAGMENTOS DE UMA DEUSA: A REPRESENTAÇÃO DE AFRODITE NA LÍRICA DE SAFO} (Editora da Unicamp, 2005) e {\bf LIRA GREGA: ANTOLOGIA DE POESIA ARCAICA} (Hedra, 2013).

\page %---------------------------------------------------------|

% \hyphenpenalty=10000
% \exhyphenpenalty=10000

% «Pulvinar ante, a ultricies magna {\bf TRECHO EM DESTAQUE, MAS PODE HAVER MAIS DE UM}, sempre em negrito e caixa alta. Aqui entra um trecho cativante do texto.»

% {\vfill\scale[factor=5]{{\bf Nome de quem escreveu a análise}, qualificação de}\setupinterlinespace[line=1.5ex]\scale[factor=5]{XPTO professora na Universidade de Nova York. Lembre}\setupinterlinespace[line=1.5ex]\scale[factor=5]{de quebrar as linhas nos códigos.}}

% \page %---------------------------------------------------------|

\Hedra

\stoptext %---------------------------------------------------------|