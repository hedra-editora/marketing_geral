% AUTOR_LIVRO_TRECHO.tex
% Preencher com o nome das cor ou composição RGB (ex: [r=0.862, g=0.118, b=0.118]) 
\usecolors[crayola] 			   % Paleta de cores pré-definida: wiki.contextgarden.net/Color#Pre-defined_colors

% Cores definidas pelo designer:
% MyGreen		r=0.251, g=0.678, b=0.290 % 40ad4a
% MyCyan		r=0.188, g=0.749, b=0.741 % 30bfbd
% MyRed			r=0.820, g=0.141, b=0.161 % d12429
% MyPink		r=0.980, g=0.780, b=0.761 % fac7c2
% MyGray		r=0.812, g=0.788, b=0.780 % cfc9c7
% MyOrange		r=0.980, g=0.671, b=0.290 % faab4a

% Configuração de cores
\definecolor[MyColor][RedOrange]      % ou ex: [r=0.862, g=0.118, b=0.118] % corresponde a RGB(220, 30, 30)
\definecolor[MyColorText][black]     % ou ex: [r=0.86

% Classe para diagramação dos posts
\environment{marketing.env}		   

\starttext %---------------------------------------------------------|

\Mensagem{DESTAQUE}

\startMyCampaign

\hyphenpenalty=10000
\exhyphenpenalty=10000

«Sobre a rica personalidade de Dostoiévski, seria o caso
de evidenciar quatro facetas: a do escritor, a do neurótico,
a do ético e a do pecador.»

\stopMyCampaign

{\vfill\scale[factor=6]{\Seta\,Trecho do livro {\bf Escritos sobre literatura},}\setupinterlinespace[line=1.5ex]\scale[factor=6]{ de Sigmund Freud, do capítulo «Dostoiévski}\setupinterlinespace[line=1.5ex]\scale[factor=6]{e o parricídio».}}

\page %---------------------------------------------------------| 

\MyCover{FREUD_LITERATURA_THUMB}

\page %---------------------------------------------------------|

\Hedra

\stoptext %---------------------------------------------------------|


% «Sobre a rica personalidade de Dostoiévski, seria o caso
% de evidenciar quatro facetas: a do escritor, a do neurótico,
% a do ético e a do pecador. Mas como poderemos nos encontrar 
% em meio a essa desconcertante complicação? Quanto
% ao escritor, há poucas dúvidas de que seu lugar é não muito 
% atrás de Shakespeare. Dos romances escritos, {\em Os irmãos
% Karamázov} é o de maior envergadura e o episódio do grande
% inquisidor, das mais altas realizações da literatura mundial,
% quase não se pode superestimar. Já no que diz respeito a
% seus problemas pessoais, infelizmente a análise deve depor
% armas.»   S.\,F.
