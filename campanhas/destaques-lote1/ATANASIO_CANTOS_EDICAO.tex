% ATANASIO_CANTOS_EDICAO.tex
% Preencher com o nome das cor ou composição RGB (ex: [r=0.862, g=0.118, b=0.118]) 
\usecolors[crayola] 			   % Paleta de cores pré-definida: wiki.contextgarden.net/Color#Pre-defined_colors

% Cores definidas pelo designer:
% MyGreen		r=0.251, g=0.678, b=0.290 % 40ad4a
% MyCyan		r=0.188, g=0.749, b=0.741 % 30bfbd
% MyRed			r=0.820, g=0.141, b=0.161 % d12429
% MyPink		r=0.980, g=0.780, b=0.761 % fac7c2
% MyGray		r=0.812, g=0.788, b=0.780 % cfc9c7
% MyOrange		r=0.980, g=0.671, b=0.290 % faab4a

% Configuração de cores
\definecolor[MyColor][x=a92d5b]      % ou ex: [r=0.862, g=0.118, b=0.118] % corresponde a RGB(220, 30, 30)
\definecolor[MyColorText][white]     % ou ex: [r=0.862, g=0.118, b=0.118] % corresponde a RGB(167, 169, 172)

% Classe para diagramação dos posts
\environment{marketing.env}		   

\starttext %---------------------------------------------------------|

\Mensagem{MUNDO INDÍGENA}

\startMyCampaign

\hyphenpenalty=10000
\exhyphenpenalty=10000

{\bf 
A ORIGEM DOS SERES
SEGUNDO OS KAIOWÁ E GUARANI}

\stopMyCampaign

%\vfill\scale[lines=1.5]{\MyStar[MyColorText][none]}

\page %---------------------------------------------------------| 

\MyCover{ATANASIO_CANTOS_THUMB}

\page %---------------------------------------------------------| 

\hyphenpenalty=10000
\exhyphenpenalty=10000

{\bf CANTOS DOS ANIMAIS \\PRIMORDIAIS} apresenta 26 histórias sobre animais da mata, acompanhados pelos cantos {\it guahu} que cantam sua história desde o princípio dos tempos. Esses {\it guahu} fazem parte de um
 conjunto maior de “cantos míticos”. 

\page %---------------------------------------------------------|

As narrativas e explicações que acompanham os cantos foram elaboradas por Izaque João, a partir de falas e orientações de {\bf ATANÁSIO TEIXEIRA}, um dos mais importantes {\it ñanderu} ou “rezador” do povo Kaiowá em atividade.

\page
\hyphenpenalty=10000
\exhyphenpenalty=10000

«No princípio, todas
as aves e animais da mata conversavam uns com os outros, eram
considerados humanos. Um dia, seguindo os irmãos Pa’i Kuara
e Jasy, Sol e Lua, na travessia de um rio, as aves e animais da
mata foram derrubados na água pelo irmão mais novo, Jasy, e
foram transformados: nunca mais voltaram a se entender. Até
hoje, cada um deles {\bf CONTA SUAS ORIGENS} e seu modo de ser por meio
de um canto {\it guahu}.»

\page %---------------------------------------------------------|

\Hedra

\stoptext %---------------------------------------------------------|


%  ao
% longo dos últimos seis anos. Os processos de seleção, transcrição e
% tradução para esta edição bilíngue também foram feitos em diálogo com o
% xamã e as versões em português dos textos e cantos \textit{guahu} são um
% exercício de aproximação a suas belas palavras.

% \textbf{Ava Ñomoandyja Atanásio Teixeira} (1922) é um dos mais importantes \textit{ñanderu} ou ``rezador'' do povo Kaiowá em atividade. Nascido em 1922, Ataná é chamado de \textit{ñamoῖ}, avô, por lideranças e rezadores de diferentes comunidades kaiowá, pelos quais é reconhecido como mestre. É um dos precursores dos \textit{jeroky guasu}, as ``grandes danças'' dos anos 1980, e do movimento histórico pela recuperação dos territórios kaiowá e guarani em Mato Grosso do Sul, a \textit{Aty Guasu}, ``grande reunião'', além de ser reconhecido como um grande xamã também pelos Guarani. O prestígio de Atanásio está, entre outros motivos, ligado ao fato de dominar as mais variadas técnicas ligadas ao xamanismo kaiowá: os \textit{ñembo'e}, fórmulas verbais de proteção pessoal ou coletiva; os \textit{mborahei} e \textit{guahu}, cantos coletivos ligados aos rituais; os diversos tipos de gestos conhecidos como \textit{jehovasa} --- que podem ser utilizados para influenciar as condições climáticas, desviando tempestades, por exemplo; para curar doenças físicas e espirituais; para garantir a sanidade das lavouras e colheitas etc.
