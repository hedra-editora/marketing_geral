% INDÍGENAS_EFEMERIDE.tex
% Preencher com o nome das cor ou composição RGB (ex: [r=0.862, g=0.118, b=0.118]) 
\usecolors[crayola] 			   % Paleta de cores pré-definida: wiki.contextgarden.net/Color#Pre-defined_colors

% Cores definidas pelo designer:
% MyGreen		r=0.251, g=0.678, b=0.290 % 40ad4a
% MyCyan		r=0.188, g=0.749, b=0.741 % 30bfbd
% MyRed			r=0.820, g=0.141, b=0.161 % d12429
% MyPink		r=0.980, g=0.780, b=0.761 % fac7c2
% MyGray		r=0.812, g=0.788, b=0.780 % cfc9c7
% MyOrange		r=0.980, g=0.671, b=0.290 % faab4a

% Configuração de cores
\definecolor[MyColor][Aquamarine]      % ou ex: [r=0.862, g=0.118, b=0.118] % corresponde a RGB(220, 30, 30)
\definecolor[MyColorText][black]     % ou ex: [r=0.862, g=0.118, b=0.118] % corresponde a RGB(167, 169, 172)

% Classe para diagramação dos posts
\environment{marketing.env}		   

\def\startMyCampaign{\bgroup
            \FormularMI
            \switchtobodyfont[28pt]
            \setupinterlinespace[line=1.9ex]
            \setcharacterkerning[packed]}
\def\stopMyCampaign{\par\egroup}

\starttext %---------------------------------------------------------|

\hyphenpenalty=10000
\exhyphenpenalty=10000

\Mensagem{OS XETÁ} %Sempre usar esse header

\startMyCampaign

\hyphenpenalty=10000
\exhyphenpenalty=10000

O QUE PODEMOS COMEMORAR \\NO {\bf DIA \\INTERNACIONAL 
DOS POVOS INDÍGENAS?}

\stopMyCampaign

\page %---------------------------------------------------------| 

\hyphenpenalty=10000
\exhyphenpenalty=10000

No dia 9 de agosto, dia em que comemoramos os povos indígenas, é bom relembrar a triste história do {\bf GENOCÍDIO INDÍGENA}. Mais especificamente o do povo Xetá, a última etnia do estado do Paraná a entrar em contato com a sociedade nacional.

\page %---------------------------------------------------------|

\MyPhoto{xetá}


\vfill{\scale[factor=3.5]{Grupo familiar em acampamento próximo a Fazenda Santa Rosa. Segunda}\setupinterlinespace[line=1.2ex]\scale[factor=3.5]{expedição científica e de contato, novembro de 1955 (Governo do Estado do Paraná,}\setupinterlinespace[line=1.2ex]\scale[factor=3.5]{Secretaria da Comunicação Social e da Cultura e Museu Paranaense/Arquivo).}}

\page

Falantes de uma língua do tronco Tupi-Guarani, os Xetá foram praticamente {\bf DIZIMADOS} – restando, até onde se sabe, apenas oito crianças do povo –, em decorrência do avanço da frente cafeeira sobre o seu território, entre as décadas de 1940 e 1960. Esse caso foi, inclusive, reconhecido como genocídio pelos relatórios da Comissão Nacional da Verdade ({\cap CNV}).

\page

Sem território, hoje {\bf OS XETÁ VIVEM DISPERSOS} nos estados do Paraná, Santa Catarina e São Paulo. Do período da colonização, ainda estão vivos cinco indígenas do povo Xetá: Kuein Manhaa’ei Nhaguakã, Ã Maria Rosa – Moko na língua Xetá –, Tiguá Maria Rosa Brasil, Tiguá Ana Maria e Moha’ay Rondon Xetá.

\page %---------------------------------------------------------|

Nem mesmo as décadas de violações foram capazes de interromper a luta do povo Xetá pela {\bf RETOMADA DO TERRITÓRIO ANCESTRAL}. Apesar de fisicamente distantes uns dos outros, os indígenas ainda se sentem conectados e alegam que a volta para a terra de origem, localizada na Serra dos Dourados ({\cap PR}), seria a maior realização de suas vidas.

\page

\Hedra

\stoptext %---------------------------------------------------------|





