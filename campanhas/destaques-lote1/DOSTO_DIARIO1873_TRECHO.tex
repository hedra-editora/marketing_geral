% DOSTO_DIARIO1873_TRECHO.tex
% >> destaques "um trecho do livro:"
% >> "OP.CIT."

% Preencher com o nome das cor ou composição RGB (ex: [r=0.862, g=0.118, b=0.118]) 
\usecolors[crayola] 			   % Paleta de cores pré-definida: wiki.contextgarden.net/Color#Pre-defined_colors

% Cores definidas pelo designer:
% MyGreen		r=0.251, g=0.678, b=0.290 % 40ad4a
% MyCyan		r=0.188, g=0.749, b=0.741 % 30bfbd
% MyRed			r=0.820, g=0.141, b=0.161 % d12429
% MyPink		r=0.980, g=0.780, b=0.761 % fac7c2
% MyGray		r=0.812, g=0.788, b=0.780 % cfc9c7
% MyOrange		r=0.980, g=0.671, b=0.290 % faab4a

% Configuração de cores
\definecolor[MyColor][SteelTeal]      % ou ex: [r=0.862, g=0.118, b=0.118] % corresponde a RGB(220, 30, 30)
\definecolor[MyColorText][SunnyPearl]  % ou ex: [r=0.862, g=0.118, b=0.118] % corresponde a RGB(167, 169, 172)

% Classe para diagramação dos posts
\environment{marketing.env}		   


% Cabeço e rodapé: Informações (caso queira trocar alguma coisa)
 		\def\MensagemSaibaMais  {SAIBA MAIS:}
 		\def\MensagemSite		{HEDRA.COM.BR}
 		\def\MensagemLink       {LINK NA BIO}

\starttext  %-----------------------------------------------------|

\Mensagem{OP.\,CIT.}

\startMyCampaign
\hyphenpenalty=10000
\exhyphenpenalty=10000
«A PALAVRA DITA É DE PRATA, MAS A NÃO DITA É DE OURO.»\blank[.5ex]

 {\bf \Seta F. DOSTOIÉVSKI}  
\stopMyCampaign


\page  %--------------------------------------------------------|

\MyCover{DOSTO_DIARIO1873_THUMB}

\page %---------------------------------------------------------|

\hyphenpenalty=10000
\exhyphenpenalty=10000

 «Um dos piores equívocos do jovem poeta consiste em considerar
o desmascaramento do vício (ou do que o liberalismo toma por vício) e a 
incitação ao ódio e à vingança como o único caminho possível para atingir seu
alvo. No entanto, para um talento forte, até desse caminho estreito é possível
escapar, sem, com isso, anular-se no começo de sua vida artística; ele
se valerá de uma regra de ouro: a palavra dita é de prata, mas a não dita é de ouro.»

{\hfill\bf A propósito de uma exposição}  

\page  

{\MyPicture{DOSTO_DIARIO1873_3}

\vfill
\scale[factor=fit]{Tradução do alemão de {\bf Patrícia Lavelle}}

\page 

\page %---------------------------------------------------------|

%{\MyPicture{AUTOR_LIVRO_4.jpeg}}

{\it Obra caudalosa e heterogênea, transitando nas difusas fronteiras entre ficção e realidade e entre jornalismo e literatura, o {\bf Diário} traz algumas das mais reveladoras páginas de Dostoiévski} \blank[1ex]

{\hfill\tf ---Irineu Franco Perpétuo}

\page  %---------------------------------------------------------|

{\bf IRINEU FRANCO PERPÉTUO}, autor da apresentação desta edição, é tradutor e jornalista. Dentre as suas traduções, está {\it Vida e destino} (Editora Alfaguara), de Vassili Grossman, com a qual
conquistou o segundo lugar no Prêmio Jabuti 2015.

\page

\Hedra

\stoptext  %-----------------------------------------------------|