% GAMA_CLIPPING.tex
% Preencher com o nome das cor ou composição RGB (ex: [r=0.862, g=0.118, b=0.118]) 
\usecolors[crayola] 			   % Paleta de cores pré-definida: wiki.contextgarden.net/Color#Pre-defined_colors

% Cores definidas pelo designer:
% MyGreen		r=0.251, g=0.678, b=0.290 % 40ad4a
% MyCyan		r=0.188, g=0.749, b=0.741 % 30bfbd
% MyRed			r=0.820, g=0.141, b=0.161 % d12429
% MyPink		r=0.980, g=0.780, b=0.761 % fac7c2
% MyGray		r=0.812, g=0.788, b=0.780 % cfc9c7
% MyOrange		r=0.980, g=0.671, b=0.290 % faab4a

% Configuração de cores
\definecolor[MyColor][VividViolet]      % ou ex: [r=0.862, g=0.118, b=0.118] % corresponde a RGB(220, 30, 30)
\definecolor[MyColorText][white]     % ou ex: [r=0.862, g=0.118, b=0.118] % corresponde a RGB(167, 169, 172)

% Classe para diagramação dos posts
\environment{marketing.env}		   

% Comandos & Instruções %%%%%%%%%%%%%%%%%%%%%%%%%%%%%%%%%%%%%%%%%%%%%%%%%%%%%%%%%%%%%%%|

% Cabeço e rodapé: Informações (caso queira trocar alguma coisa)
 		\def\MensagemSaibaMais 	{SAIBA MAIS:}
 		\def\MensagemSite		{HEDRA.COM.BR}
 		\def\MensagemLink		{LINK NA BIO}

\def\MyCover#1{\starttikzpicture[overlay, remember picture]
              \node
                   [draw=none, 
                    blur shadow={shadow xshift=5.5pt,shadow yshift=-1.5pt, shadow scale=0.93},
                    shadow opacity=50, 
                    shadow blur extra rounding] 
                    at (0.12  \textwidth,-.4\textheight) {\externalfigure[#1][width=2cm]};
                \stoptikzpicture}

\def\MySecondCover#1{\starttikzpicture[overlay, remember picture]
              \node
                   [draw=none, 
                    blur shadow={shadow xshift=5.5pt,shadow yshift=-1.5pt, shadow scale=0.93},
                    shadow opacity=50, 
                    shadow blur extra rounding] 
                    at (0.37  \textwidth,-.33\textheight) {\externalfigure[#1][width=2.05cm]};
                \stoptikzpicture}

\def\MyThirdCover#1{\starttikzpicture[overlay, remember picture]
              \node
                   [draw=none, 
                    blur shadow={shadow xshift=5.5pt,shadow yshift=-1.5pt, shadow scale=0.93},
                    shadow opacity=50, 
                    shadow blur extra rounding] 
                    at (0.62  \textwidth,-.259\textheight) {\externalfigure[#1][width=2cm]};
                \stoptikzpicture}

\def\MyFourthCover#1{\starttikzpicture[overlay, remember picture]
              \node
                   [draw=none, 
                    blur shadow={shadow xshift=5.5pt,shadow yshift=-1.5pt, shadow scale=0.93},
                    shadow opacity=50, 
                    shadow blur extra rounding] 
                    at (0.87  \textwidth,-.187\textheight) {\externalfigure[#1][width=2.05cm]};
                \stoptikzpicture}


\starttext %---------------------------------------------------------|

\Mensagem{NA IMPRENSA}

\starttikzpicture[remember picture,overlay]
        \node at (4.5,-3.8) {\externalfigure[GAMA_IMPRENSA_2][width=0.7\textwidth]};
\stoptikzpicture

\page %---------------------------------------------------------|

\hyphenpenalty=10000
\exhyphenpenalty=10000


\starttikzpicture[remember picture,overlay]
        \node at (4.5,-3.8) {\externalfigure[GAMA_IMPRENSA_TRECHO][width=\textwidth]};
\stoptikzpicture
{\vfill\scale[factor=5]{\Seta\,Trecho da coluna de {\bf Tom Farias}, da Folha de São}\setupinterlinespace[line=1.5ex]\scale[factor=5]{Paulo, em 24 de julho.}}

% «A ideia de publicação das {\bf OBRAS COMPLETAS DE LUIZ GAMA} é genial, corajosa e altamente oportuna. Bruno Lima se esmera no trato
% editorial dos artigos, abrindo cada um deles com resumo explicativo que nos orienta e nos insere no seu contexto, no tempo em que
% foi produzido.»

% {\vfill\scale[factor=5]{\Seta\,Trecho da coluna de {\bf Tom Farias}, da Folha de São}\setupinterlinespace[line=1.5ex]\scale[factor=5]{Paulo, em 24 de julho.}}

\page %---------------------------------------------------------|

\MyCover{GAMA_LIBERDADE_THUMB}

\MySecondCover{GAMA_DIREITO_THUMB}

\MyThirdCover{GAMA_CRIME_THUMB}

\MyFourthCover{GAMA_DEMOCRACIA_THUMB}

\page %---------------------------------------------------------|

\Hedra

\stoptext %---------------------------------------------------------|

