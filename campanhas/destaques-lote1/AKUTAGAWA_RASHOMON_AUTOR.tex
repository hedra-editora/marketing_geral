% AKUTAGAWA_RASHOMON_AUTOR.tex
% quem é "autor"
% > "VIDA & OBRA"

% Preencher com o nome das cor ou composição RGB (ex: [r=0.862, g=0.118, b=0.118]) 
\usecolors[crayola] 			   % Paleta de cores pré-definida: wiki.contextgarden.net/Color#Pre-defined_colors

% Cores definidas pelo designer:
% MyGreen		r=0.251, g=0.678, b=0.290 % 40ad4a
% MyCyan		r=0.188, g=0.749, b=0.741 % 30bfbd
% MyRed			r=0.820, g=0.141, b=0.161 % d12429
% MyPink		r=0.980, g=0.780, b=0.761 % fac7c2
% MyGray		r=0.812, g=0.788, b=0.780 % cfc9c7
% MyOrange		r=0.980, g=0.671, b=0.290 % faab4a

% Configuração de cores
\definecolor[MyColor][x=e3ee5c]      % ou ex: [r=0.862, g=0.118, b=0.118] % corresponde a RGB(220, 30, 30)
\definecolor[MyColorText][x=d22027]  % ou ex: [r=0.862, g=0.118, b=0.118] % corresponde a RGB(167, 169, 172)

% Classe para diagramação dos posts
\environment{marketing.env}		   

% Cabeço e rodapé: Informações (caso queira trocar alguma coisa)
 		\def\MensagemSaibaMais  {SAIBA MAIS:}
 		\def\MensagemSite		{HEDRA.COM.BR}
 		\def\MensagemLink       {LINK NA BIO}


\starttext %--------------------------------------------------------|

\Mensagem{VIDA \& OBRA}

\MyPicture{AKUTAGAWA_RASHOMON_1.jpeg}

\page %----------------------------------------------------------|


%芥川龍之介

{\bf RYÛNOSUKE AKUTAGAWA}, um dos maiores nomes da
literatura moderna japonesa, nasceu em Tóquio no
fim do século {\cap XIX}, durante o período Meiji, quando
o país se abria por completo à influência da cultura 
ocidental.

\page %----------------------------------------------------------|


Leitor precoce, ainda criança lê com 
entusiasmo traduções do dramaturgo norueguês {\bf Henrik Ibsen} e do escritor francês {\bf Anatole France}.
Na juventude, traduz {\bf W.\,B.\,Yeats} e se especializa em literatura inglesa na Universidade Imperial de Tóquio, na mesma época em que passa a se interessar pela ética cristã.

\page %----------------------------------------------------------|

Assim, soube mesclar tradição e modernidade como nenhum outro contemporâneo seu, reinterpretando temas japoneses de narrativas do século {\cap XII} e autores e filósofos ocidentais. Escreveu ainda contos autobiográficos, sobretudo a partir da década de 1920, que relembram  o contexto familiar tremendamente conturbado, como {\bf Passagens do caderno de notas de Yasukichi} e {\bf A vida de um idiota}.

\page

 A instabilidade psíquico-emocional de sua mãe, então considerada louca, perseguiu-o como um  fantasma durante a vida inteira, e há quem diga que foi isso que o levou ao suicídio, em julho de 1927.

 \page
 
O fato é que {\bf Ryûnosuke Akutagawa} maneja uma pluralidade de gêneros literários e desenvolve temas 
demasiadamente humanos e universais, como o egoísmo e o valor da arte enquanto redentora da miséria da vida cotidiana, além da incessante busca por um equilíbrio moral entre o tradicional e o moderno.

\page

\Hedra

\stoptext %---------------------------------------------------------|
			