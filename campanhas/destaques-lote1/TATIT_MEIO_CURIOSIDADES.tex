% AUTOR_LIVRO_CURIOSIDADES.tex
% Preencher com o nome das cor ou composição RGB (ex: [r=0.862, g=0.118, b=0.118]) 
\usecolors[crayola] 			   % Paleta de cores pré-definida: wiki.contextgarden.net/Color#Pre-defined_colors

% Cores definidas pelo designer:
% MyGreen		r=0.251, g=0.678, b=0.290 % 40ad4a
% MyCyan		r=0.188, g=0.749, b=0.741 % 30bfbd
% MyRed			r=0.820, g=0.141, b=0.161 % d12429
% MyPink		r=0.980, g=0.780, b=0.761 % fac7c2
% MyGray		r=0.812, g=0.788, b=0.780 % cfc9c7
% MyOrange		r=0.980, g=0.671, b=0.290 % faab4a

% Configuração de cores
\definecolor[MyColor][x=10ae7f]      % ou ex: [r=0.862, g=0.118, b=0.118] % corresponde a RGB(220, 30, 30)
\definecolor[MyColorText][black]     % ou ex: [r=0.862, g=0.118, b=0.118] % corresponde a RGB(167, 169, 172)

% Classe para diagramação dos posts
\environment{marketing.env}		   

\starttext %---------------------------------------------------------|
\hyphenation{distraidamente}
\hyphenation{esforço}
\hyphenation{melodia}
\hyphenation{cancionistas}
\hyphenation{maneira}
\hyphenation{base}
\hyphenation{importa}
\hyphenation{essencialmente}

\hyphenpenalty=10000
\exhyphenpenalty=10000

\Mensagem{EM CONTEXTO} %Sempre usar esse header

\startMyCampaign

\hyphenpenalty=10000
\exhyphenpenalty=10000

5 FATOS SOBRE OS {\bf CANCIONISTAS}
SEGUNDO 

{\bf LUIZ TATIT}
\stopMyCampaign

\page %---------------------------------------------------------| 


\setupinterlinespace[line=3ex]
% \null
% \vskip 1em
\hyphenpenalty=10000
\exhyphenpenalty=10000

{\bfd 1.}  {\tfb «O cancionista mais parece um malabarista. Tem um controle de 
 atividade que permite equilibrar a melodia no texto e o texto
 na melodia, distraidamente, como se para isso não despendesse
 qualquer esforço. Só habilidade, manha e improviso.»}

\page %---------------------------------------------------------|

\hyphenpenalty=10000
\exhyphenpenalty=10000

{\bfd 2.}  {\tfb «No mundo dos cancionistas não importa tanto o que é dito mas
a maneira de dizer, e a maneira é essencialmente melódica. Sobre
essa base, o que é dito 

torna-se, muitas vezes, grandioso.»}

\page

\hyphenpenalty=10000
\exhyphenpenalty=10000

{\bfd 3.}  {\tfb «E na junção da sequência melódica com as unidades linguísticas, 
ponto nevrálgico de tensividade, o cancionista tem sempre
um gesto oral elegante, no sentido de aparar as arestas e eliminar
os resíduos que poderiam quebrar a naturalidade da canção.»}

\page

\hyphenpenalty=10000
\exhyphenpenalty=10000

{\bfd 4.}  {\tfb «A grandeza do gesto oral do cancionista está em criar uma
obra perene com os mesmos recursos utilizados para a produção
efêmera da fala cotidiana.»}

\page

\hyphenpenalty=10000
\exhyphenpenalty=10000

{\bfd 5.}  {\tfb «Os cancionistas são, em geral, pessoas sintonizadas com a modernidade, sensíveis às questões humanas, às relações interpessoais e com grande pendor para mesclar fatos de diferentes universos de experiência num único discurso: a canção.»}

\page

\MyCover{ACORDE_TATIT_THUMB}

\page %---------------------------------------------------------|

\Hedra

\stoptext %---------------------------------------------------------|


10ae7f
f37c9d







