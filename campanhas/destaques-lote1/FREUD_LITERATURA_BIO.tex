% Preencher com o nome das cor ou composição RGB (ex: [r=0.862, g=0.118, b=0.118]) 
\usecolors[crayola] 			   % Paleta de cores pré-definida: wiki.contextgarden.net/Color#Pre-defined_colors

% Cores definidas pelo designer:
% MyGreen		r=0.251, g=0.678, b=0.290 % 40ad4a
% MyCyan		r=0.188, g=0.749, b=0.741 % 30bfbd
% MyRed			r=0.820, g=0.141, b=0.161 % d12429
% MyPink		r=0.980, g=0.780, b=0.761 % fac7c2
% MyGray		r=0.812, g=0.788, b=0.780 % cfc9c7
% MyOrange		r=0.980, g=0.671, b=0.290 % faab4a

% Configuração de cores
\definecolor[MyColor][Salmon]      % ou ex: [r=0.862, g=0.118, b=0.118] % corresponde a RGB(220, 30, 30)
\definecolor[MyColorText][black]  % ou ex: [r=0.862, g=0.118, b=0.118] % corresponde a RGB(167, 169, 172)

% Classe para diagramação dos posts
\environment{marketing.env}		   


% Comandos & Instruções %%%%%%%%%%%%%%%%%%%%%%%%%%%%%%%%%%%%%%%%%%%%%%%%%%%%%%%%%%%%%%%|

% Cabeço e rodabé: Informações (caso queira trocar alguma coisa)
% 		\def\MensagemSaibaMais{SAIBA MAIS:}
% 		\def\MensagemSite{HEDRA.COM.BR}
% 		\def\MensagemLink{LINK NA BIO}

% Pesos para os títulos:
%		\startMyCampaign...		 \stopMyCampaign
%		\stopMyCampaignSection...   \stopMyCampaignSection

% Aplicação de imagens: 
% 		\MyCover{capa.pdf}  	% Aplicação de capa de livro com sombra
%		\MyPicture{Imagem.png}  % Imagem com aplicação de filtro segundo cor MyColorText
%		\MyPhoto{}			    % Aplicação simples de imagem com tamamho \textwidth

% Aplicação de imagem com legenda:		
% 		\placefigure{Legenda}{\externalfigure[drop2-1.png][width=\textwidth]}

% Cabeço e rodabé: Opções
% 		\Mensagem{AGORA É QUE SÃO ELAS}
% 		\Hashtag{campanha de natal}
% 		\Mensagem{campanha de natal}

% Alteração de várias cores de background:
% \setupbackgrounds[page][background=color,backgroundcolor=MyGray]

% Estrela: 
% \vfill\scale[lines=2]{\MyStar[MyColorText][none]} 					% Estrela pequena  
% \startpositioning 											% Estrela grande
%  \position(-1,-.3){\scale[scale=980]{\MyStar[white][none]}}
% \stoppositioning

% Logos e selos: 				
% \Hedra
% \HedraAyllon	% Não está pronto
% \HedraAcorde	% Não está pronto
% \Ayllon		% Não está pronto
% \Acorde		% Não está pronto

% Atalhos: 						
% 		\Seta  % Seta para baixo

%%%%%%%%%%%%%%%%%%%%%%%%%%%%%%%%%%%%%%%%%%%%%%%%%%%%%%%%%%%%%%%%%%%%%%%%%%%%%%%%%%%%%%%|

\starttext
%\showframe  %Para mostrar somente as linhas.

\Mensagem{SOBRE O AUTOR}

\MyCover{FREUD_LITERATURA_THUMB.pdf}

\page %---------------------------------------------------------|

\MyPicture{FREUD_LITERATURA_2.jpeg}

\hyphenpenalty=10000
\exhyphenpenalty=10000

«Sobre a rica personalidade de Dostoiévski, seria o caso
de evidenciar quatro facetas: a do escritor, a do neurótico,
a do ético e a do pecador. Mas como poderemos nos encontrar 
em meio a essa desconcertante complicação? Quanto
ao escritor, há poucas dúvidas de que seu lugar é não muito 
atrás de Shakespeare. Dos romances escritos, {\em Os irmãos
Karamázov} é o de maior envergadura e o episódio do grande
inquisidor, das mais altas realizações da literatura mundial,
quase não se pode superestimar. Já no que diz respeito a
seus problemas pessoais, infelizmente a análise deve depor
armas.»   F.\,D.

\page

\MyPicture{FREUD_LITERATURA_1.jpeg}

\vfill
\scale[factor=fit]{Tradução de {\bf Saulo Krieger}, org. de {\bf Iuri Pereira}}


\page

Psicanálise e literatura nunca estiveram distantes. Mas
em nenhuma parte da vasta obra de Freud estão mais
próximas do que em {\bf Escritos sobre literatura}, em que
as ferramentas analíticas que a própria literatura ajudou
a lapidar agora servem para reinterpretá-la. A personalidade 
e a obra de Dostoiévski, a sensação do estranho
familiar a partir do escritor “fantástico” E. T. A. Hoffman,
os mecanismos de significação da memória sugeridos por
Goethe e o próprio impulso criativo, que Freud relaciona
à pulsão erótica, ganham novas luzes − além de sombras
mais profundas.

\page

\Hedra

\stoptext

% Legenda:
% Escritos sobre literatura reúne textos que, de certa forma, resgatam 
% o “débito” de Freud com a história literária. A psicanálise se
% baseia em discursos simbólicos: narrativas pessoais, que contam e
% contêm mais do que o narrador-personagem percebe. Daí a relação
% da obra de Freud com as artes em geral e com a literatura em particular 
% ter sido forte e fecunda desde o início: sua hipótese mais famosa
% e fundamental, o “complexo de Édipo”, remete à estrutura narrativa
% de uma peça de Sófocles. Aqui, as posições se invertem: a psicanálise 
% se volta para a literatura, na chamada “psicanálise aplicada”,
% usando seus novos recursos para reinterpretá-la: “Dostoiévski e o
% parricídio” (que relaciona o escritor russo e sua obra ao “complexo
% de Édipo”); “O estranho” (que discute a surpreendente sensação do
% estranho familiar a partir do conto “O homem de areia”, do escritor
% “fantástico” E. T. A Hoffman); “O poeta e o fantasiar” (que apresenta
% a relação freudiana entre criação e pulsão erótica) e “Uma lembrança
% infantil de Poesia e Verdade” (que especula sobre os mecanismos da
% memória a partir de uma passagem autobiográfica de Goethe).