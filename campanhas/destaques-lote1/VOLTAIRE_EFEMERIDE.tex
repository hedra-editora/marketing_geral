%VOLTAIRE_EFEMERIDE.tex
% Preencher com o nome das cor ou composição RGB (ex: [r=0.862, g=0.118, b=0.118]) 
\usecolors[crayola] 			   % Paleta de cores pré-definida: wiki.contextgarden.net/Color#Pre-defined_colors

% Cores definidas pelo designer:
% MyGreen		r=0.251, g=0.678, b=0.290 % 40ad4a
% MyCyan		r=0.188, g=0.749, b=0.741 % 30bfbd
% MyRed			r=0.820, g=0.141, b=0.161 % d12429
% MyPink		r=0.980, g=0.780, b=0.761 % fac7c2
% MyGray		r=0.812, g=0.788, b=0.780 % cfc9c7
% MyOrange		r=0.980, g=0.671, b=0.290 % faab4a

% Configuração de cores
\definecolor[MyColor][Maize]      % ou ex: [r=0.862, g=0.118, b=0.118] % corresponde a RGB(220, 30, 30)
\definecolor[MyColorText][black]     % ou ex: [r=0.862, g=0.118, b=0.118] % corresponde a RGB(167, 169, 172)

% Classe para diagramação dos posts
\environment{marketing.env}		   

\starttext %---------------------------------------------------------|

\hyphenpenalty=10000
\exhyphenpenalty=10000

\Mensagem{30 DE MAIO} %Sempre usar esse header

\MyPicture{VOLTAIRE_EFEMERIDE_1}

\vfill\scale[factor=6]{\Seta\,246 ANOS DA MORTE DE {\bf VOLTAIRE}}

\page %---------------------------------------------------------| 

\hyphenpenalty=10000
\exhyphenpenalty=10000

François-Marie Arouet, mais conhecido pelo seu pseudônimo, {\bf VOLTAIRE}, foi um escritor, historiador e filósofo iluminista francês. Foi um escritor versátil, tendo produzido peças de teatro, poemas, romances e ensaios.

\page %---------------------------------------------------------|

Famoso pela sagacidade de suas críticas, voltadas sobretudo ao cristianismo e à escravidão, Voltaire foi um dos primeiros autores a se tornar conhecido e comercialmente {\bf BEM-SUCEDIDO INTERNACIONALMENTE}.

\page

Defensor declarado das {\bf LIBERDADES CIVIS}, estava em constante atrito com as rígidas leis de censura da monarquia francesa. Seus escritos satirizavam o dogma religioso e as instituições francesas de sua época, como é o caso de sua obra mais célebre, {\bf CÂNDIDO, OU O OTIMISMO}, um verdadeiro retrato satírico que critica e ridiculariza eventos, pensadores e filosofias de seu tempo.

% \page

% \MyCover{THUMB_LIVRO.pdf}

\page %---------------------------------------------------------|

\Hedra

\stoptext %---------------------------------------------------------|

