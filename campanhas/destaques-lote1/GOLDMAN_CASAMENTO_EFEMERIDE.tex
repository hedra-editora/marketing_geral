% GOLDMAN_INDIVIDUO_EFEMERIDE.tex
% Preencher com o nome das cor ou composição RGB (ex: [r=0.862, g=0.118, b=0.118]) 
\usecolors[crayola] 			   % Paleta de cores pré-definida: wiki.contextgarden.net/Color#Pre-defined_colors

% Cores definidas pelo designer:
% MyGreen		r=0.251, g=0.678, b=0.290 % 40ad4a
% MyCyan		r=0.188, g=0.749, b=0.741 % 30bfbd
% MyRed			r=0.820, g=0.141, b=0.161 % d12429
% MyPink		r=0.980, g=0.780, b=0.761 % fac7c2
% MyGray		r=0.812, g=0.788, b=0.780 % cfc9c7
% MyOrange		r=0.980, g=0.671, b=0.290 % faab4a

% Configuração de cores
\definecolor[MyColor][Mahogany]      % ou ex: [r=0.862, g=0.118, b=0.118] % corresponde a RGB(220, 30, 30)
\definecolor[MyColorText][white]     % ou ex: [r=0.862, g=0.118, b=0.118] % corresponde a RGB(167, 169, 172)

% Classe para diagramação dos posts
\environment{marketing.env}		   

\starttext %---------------------------------------------------------|

\hyphenpenalty=10000
\exhyphenpenalty=10000

\Mensagem{14 DE MAIO} %Sempre usar esse header

\MyPicture{GOLDMAN_CASAMENTO_1}

\vfill\scale[factor=6]{\Seta\,84 ANOS SEM {\bf EMMA GOLDMAN}}

\page %---------------------------------------------------------| 

\hyphenpenalty=10000
\exhyphenpenalty=10000

{\bf EMMA GOLDMAN} foi uma \\
revolucionária anarquista nascida em um gueto judeu na província russa do Kovno.
Aos 17 anos tornou-se operária em
 São Petersburgo e em 1886 emigrou para os {\cap eua}.


\page %---------------------------------------------------------|

Ativista dos {\bf DIREITOS DA MULHER}, engajou-se na luta pelo
controle de natalidade e deu palestras por todos os Estados Unidos, um dos
motivos que levaram à sua perseguição constante pelo {\cap fbi}.

\page

Suas práticas como oradora e organizadora de lutas operárias fizeram com que fosse presa inúmeras vezes, e, em 1919, foi deportada para a Rússia com seu companheiro {\bf ALEXANDER BERKMAN} e centenas de outros revolucionários.

\page

\MyCover{GOLDMAN_CASAMENTO_THUMB}

\page %---------------------------------------------------------|

\Hedra

\stoptext %---------------------------------------------------------|



% \paragraph{Goldman, Emma (1868--1940)}  Tanto o enforcamento
% dos quatro anarquistas de Chicago, em 11 de novembro de 1887, quanto
% seu contato com Johann Most e Voltarine de Cleyre (1866--1912),
% fazem-na aderir ao anarquismo. Em Nova York inicia uma relação
% amorosa com Alexander Berkman (ver Berkman, Alexander). Suas práticas
% como oradora e organizadora de lutas de trabalhadores fazem com que
% seja mantida no cárcere ao longo de 1893. Após inúmeras ações e
% publicações, é deportada com centenas de revolucionários para a
% Rússia, em 1919. Ao deixar o país, escreve \emph{My Disillusionment
% in Russia} (1923) e \emph{My Further Disillusionment in Russia}
% (1924), obras nas quais desmascara o assassínio bolchevique. Após
% realizar algumas atividades políticas no Canadá, muda-se para França
% e escreve sua autobiografia, \emph{Living My Life} (1931). Abala-se
% profundamente com a morte do companheiro Berkman. Participa das
% atividades da Espanha revolucionária. Morre em 1940, em Toronto,
% Canadá. Após ser autorizado pelo governo dos \textsc{eua}, seu corpo é
% sepultado em Chicago, junto aos militantes operários assassinados no
% século \textsc{xix}.


