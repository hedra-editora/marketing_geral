% AKUTAGAWA_RASHOMON_TRECHO.tex
% >> destaques "um trecho do livro:"
% >> "OP.CIT."

% Preencher com o nome das cor ou composição RGB (ex: [r=0.862, g=0.118, b=0.118]) 
\usecolors[crayola] 			   % Paleta de cores pré-definida: wiki.contextgarden.net/Color#Pre-defined_colors

% Cores definidas pelo designer:
% MyGreen		r=0.251, g=0.678, b=0.290 % 40ad4a
% MyCyan		r=0.188, g=0.749, b=0.741 % 30bfbd
% MyRed			r=0.820, g=0.141, b=0.161 % d12429
% MyPink		r=0.980, g=0.780, b=0.761 % fac7c2
% MyGray		r=0.812, g=0.788, b=0.780 % cfc9c7
% MyOrange		r=0.980, g=0.671, b=0.290 % faab4a

% Configuração de cores
\definecolor[MyColor][x=e3ee5c]      % ou ex: [r=0.862, g=0.118, b=0.118] % corresponde a RGB(220, 30, 30)
\definecolor[MyColorText][x=d22027]  % ou ex: [r=0.862, g=0.118, b=0.118] % corresponde a RGB(167, 169, 172)

% Classe para diagramação dos posts
\environment{marketing.env}		   


% Cabeço e rodapé: Informações (caso queira trocar alguma coisa)
 		\def\MensagemSaibaMais  {SAIBA MAIS:}
 		\def\MensagemSite		{HEDRA.COM.BR}
 		\def\MensagemLink       {LINK NA BIO}

\starttext  %-----------------------------------------------------|

\Mensagem{OP.\,CIT.}

\startMyCampaign
\hyphenpenalty=10000
\exhyphenpenalty=10000

«UMA BORBOLETA VOLTEAVA NO VENTO»\blank[.5ex]

 {\bf \Seta AKUGATAWA}  
\stopMyCampaign

\page %---------------------------------------------------------|

\MyPicture{AKUTAGAWA_RASHOMON_4}

\vfill
\scale[factor=fit]{\tfxx Tradução do Japonês de {\bf Madalena Hashimoto e Junko Ota}.}

\page 
\hyphenpenalty=10000
\exhyphenpenalty=10000

 «Uma borboleta volteava no vento impregnado por um cheiro
de ervas aquáticas. Durante apenas um ínfimo segundo, ele
sentiu o roçar de suas asas sobre os lábios ressecados. Mas
o pó das asas que assim fora espalhado sobre seus lábios
continuou a brilhar, mesmo muitos anos depois.» 

\page %---------------------------------------------------------|

{\MyPicture{AKUTAGAWA_RASHOMON_2}}

{\it [\unknown] os escritos de
Akutagawa são disseminações de seu conflito profundo entre uma vida de
sucesso material e uma tendência profundamente melancólica e
sedenta de uma linguagem moral.} \blank[1ex]

{\hfill\tf ---Madalena Hashimoto Cordaro}

\page  %--------------------------------------------------------|

\MyCover{AKUTAGAWA_RASHOMON_THUMB}

\page

{\bf Rashômon e outros contos} reúne dez contos de diversos períodos da breve
existência do autor. Dentre eles estão “Dentro do bosque” (1922), “O mártir” (1918), “Terra morta” (1918) e “A vida de um idiota” (1927). As temáticas abordadas vão desde a cultura de Heian e Edo (atuais Quioto e Tóquio), a ética cristã, a abertura do Japão ao Ocidente, até a própria biografia do autor. Esta nova edição, com texto revisto pelas tradutoras, conta ainda com nova introdução e acréscimo de notas.

\page  %---------------------------------------------------------|

\Hedra

\stoptext  %-----------------------------------------------------|