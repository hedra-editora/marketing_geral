% CARVALHO_EFEMERIDE.tex
% Preencher com o nome das cor ou composição RGB (ex: [r=0.862, g=0.118, b=0.118]) 
\usecolors[crayola] 			   % Paleta de cores pré-definida: wiki.contextgarden.net/Color#Pre-defined_colors

% Cores definidas pelo designer:
% MyGreen		r=0.251, g=0.678, b=0.290 % 40ad4a
% MyCyan		r=0.188, g=0.749, b=0.741 % 30bfbd
% MyRed			r=0.820, g=0.141, b=0.161 % d12429
% MyPink		r=0.980, g=0.780, b=0.761 % fac7c2
% MyGray		r=0.812, g=0.788, b=0.780 % cfc9c7
% MyOrange		r=0.980, g=0.671, b=0.290 % faab4a

% Configuração de cores
\definecolor[MyColor][Carmine]      % ou ex: [r=0.862, g=0.118, b=0.118] % corresponde a RGB(220, 30, 30)
\definecolor[MyColorText][black]     % ou ex: [r=0.862, g=0.118, b=0.118] % corresponde a RGB(167, 169, 172)

% Classe para diagramação dos posts
\environment{marketing.env}		   

\starttext %---------------------------------------------------------|

\hyphenpenalty=10000
\exhyphenpenalty=10000

\Mensagem{EXPOENTE MODERNISTA} %Sempre usar esse header

\MyPhoto{flavio}

\vfill\scale[factor=6]{\Seta\,125 ANOS DE {\bf FLÁVIO DE CARVALHO}}

\page %---------------------------------------------------------| 

\hyphenpenalty=10000
\exhyphenpenalty=10000

{\bf FLÁVIO DE CARVALHO} foi um dos grandes nomes da geração modernista brasileira, atuando como arquiteto, dramaturgo, pintor, escritor, filósofo e músico.  

\page %---------------------------------------------------------|

Oriundo de uma família abastada, Flávio morou na França e na Inglaterra, onde estudou engenharia civil e belas artes. Em 1939, foi indicado ao {\bf PRÊMIO NOBEL DE LITERATURA}.

\page

Em 1956, quando escrevia sobre arquitetura no Diário de São Paulo, seu editor pediu que fizesse um modelo de roupa masculino. Flávio criou uma minissaia, uma camisa bufante, um chapéu e uma meia de modelo arrastão com sandálias de couro e deu o nome de Experiência nº 3 ao seu projeto. Ele mesmo desfilou com seu protótipo naquele mesmo ano em São Paulo.
\page

\MyPhoto{flavio2}

{\scale[factor=4]{Imagens da performance de Flávio de Carvalho, entitulada {\it Experiência}}\setupinterlinespace[line=1.5ex]\scale[factor=4]{nº 3.}}

\hyphenpenalty=10000
\exhyphenpenalty=10000



\page %---------------------------------------------------------|

\Hedra

\stoptext %---------------------------------------------------------|





