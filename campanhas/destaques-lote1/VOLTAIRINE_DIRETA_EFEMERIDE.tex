% VOLTAIRINE_DIRETA_EFEMERIDE.tex
% Preencher com o nome das cor ou composição RGB (ex: [r=0.862, g=0.118, b=0.118]) 
\usecolors[crayola] 			   % Paleta de cores pré-definida: wiki.contextgarden.net/Color#Pre-defined_colors

% Cores definidas pelo designer:
% MyGreen		r=0.251, g=0.678, b=0.290 % 40ad4a
% MyCyan		r=0.188, g=0.749, b=0.741 % 30bfbd
% MyRed			r=0.820, g=0.141, b=0.161 % d12429
% MyPink		r=0.980, g=0.780, b=0.761 % fac7c2
% MyGray		r=0.812, g=0.788, b=0.780 % cfc9c7
% MyOrange		r=0.980, g=0.671, b=0.290 % faab4a

% Configuração de cores
\definecolor[MyColor][RustyRed]      % ou ex: [r=0.862, g=0.118, b=0.118] % corresponde a RGB(220, 30, 30)
\definecolor[MyColorText][black]     % ou ex: [r=0.862, g=0.118, b=0.118] % corresponde a RGB(167, 169, 172)

% Classe para diagramação dos posts
\environment{marketing.env}		   

\starttext %---------------------------------------------------------|

\hyphenpenalty=10000
\exhyphenpenalty=10000

\Mensagem{20 de JUNHO} %Sempre usar esse header

\MyPicture{VOLTAIRINE_DIRETA_1}

\vfill\scale[factor=6]{\Seta\,112 ANOS SEM A ANARQUISTA {\bf VOLTAIRINE}}

\page %---------------------------------------------------------| 

\hyphenpenalty=10000
\exhyphenpenalty=10000

Anarquista, poeta, conferencista e linguista, {\bf VOLTAIRINE DE CLEYRE} é uma figura singular na história das lutas anarquistas, apesar de pouco conhecida. Militante de uma geração que viveu em meio à intensa proliferação do movimento anarquista nos {\cap eua}, Voltairine organizou sua vida e sua obra a partir do valor da dignidade humana e do desejo apaixonado pela {\bf LIBERDADE}.
 
\page %---------------------------------------------------------|

Embora tenha iniciado a carreira de militante no pacifismo, o \\desenvolvimento acelerado do capitalismo nos {\cap EUA} e eventos marcantes como a Revolução de 1905 na Rússia e a Revolução Mexicana, em 1910, alteraram-lhe a compreensão acerca dos métodos e a levaram a abraçar a {\bf AÇÃO DIRETA}.

\page

 Em 1892, ajudou a fundar a \emph{Ladie's Liberal League}, depois unida à \emph{Radical Library}, fornecendo palestras sobre sexo, anarquismo e revolução. Inspirada pelo libertário inglês William Godwin, escreveu sobre o {\bf AMOR LIVRE} e contra o casamento e a exploração da individualidade. 

\page

Foi contemporânea e associada de figuras como {\bf EMMA GOLDMAN} e {\bf ALEXANDER BERKMAN}, {\bf BENJAMIN
 TUCKER}, {\bf JOHANN MOST} e {\bf LUCY PARSONS.} 

\page

\MyCover{VOLTAIRINE_DIRETA_THUMB.png}

\page %---------------------------------------------------------|

\Hedra

\stoptext %---------------------------------------------------------|

% poucos os escritos dela (e sobre ela) que circulam em língua portuguesa:
% artigos traduzidos na revista \textit{verve}, do Nu-Sol, como o belíssimo escrito
% biográfico de Emma Goldman sobre a amiga,\footnote{Ver \textit{verve} 36, 2019.} a edição
% muito cuidadosa de uma coletânea de ensaios pela editora portuguesa
% % Barricada de Livros, alguns zines de coletivos militantes e reproduções
% % em sites anarquistas. Esta antologia,
% % portanto, é um esforço de fazer conhecer as palavras de fogo da
% % mulher que viveu os embates pela liberdade em si mesma, numa
% % \emph{pequena guerra permanente} contra o mundo da autoridade
% % democrática que, ao longo do século \textsc{xx}, se tornaria imperial: os Estados
% % Unidos da América.


% \textls[-10]{Voltairine foi uma professora das mais dedicadas, apesar de
% limitações físicas e sérios problemas de saúde que lhe causavam fortes
% dores. Além do inglês, lecionava a língua francesa, oferecia aula de
% música e caligrafia para ganhar a vida e cuidar de sua mãe adoecida.
% Aprendeu ídiche ao lecionar sobre anarquismo para judeus imigrantes no
% gueto da Filadélfia, a cidade do amor fraternal que era extremamente
% hostil aos anarquistas. Além das aulas, publicou centenas de poemas,
% ensaios e artigos em revistas anarquistas como \emph{Lucifer},
% \emph{Free Society}, \emph{Mother Earth}, \emph{Les Temps Nouveaux.}
% Traduziu para o inglês os livros de Jean Grave e Francisco Ferrer,
% deixando a tradução da autobiografia de Louise Michel não concluída.
% George Brown, orador anarquista na Filadélfia e contemporâneo de
% Voltairine, a considerava a mulher mais intelectual que já conhecera e,
% também, a mais paciente, corajosa e amorosa.}\looseness=-1

% Voltairine de Cleyre nasceu em 17 de novembro de 1866 numa vila de
% Leslie, Michigan, passou a maior parte de sua vida na Filadélfia e
% morreu em Chicago, em 1912, onde morou por quase dois anos. Seu avô
% materno foi membro da \emph{Ferrovia Clandestina}, uma associação que ajudava
% pessoas escravizadas a fugirem para o Canadá, e para quem, segundo
% Voltairine, a lei estava geralmente distante da vida efetiva, e a ação
% direta era um imperativo. Seu pai foi um liberal e livre-pensador,
% admirador dos escritos de Voltaire, de onde retirou o nome da filha como
% uma homenagem ao escritor e filósofo francês. Ele havia emigrado da
% França para os Estados Unidos da América aos 18 anos e lutou na Guerra
% Civil pelos estados do Norte. Segundo a biografia escrita por Paul
% Avrich,\footnote{\textsc{avrich}, Paul. \emph{An American Anarchist: The life of Voltairine de Cleyre}. California: \textsc{ak} Press, 2018.} Voltairine foi uma criança rebelde e brilhante que escreveu seu
% primeiro poema, ``I wish'', aos 6 anos de idade. Seu pai, que havia
% se convertido ao catolicismo, a enviou para ser criada em um convento de
% freiras no Canadá. Para ela, foi uma experiência extremamente dolorosa e
% marcante; ela considerava a vida no convento um encarceramento, do qual
% tentou fugir atravessando um rio a nado e, sem dinheiro algum, acabou na
% casa de amigos dos seus pais que a enviaram de volta ao enclausuramento.
% No entanto, ela viveu a experiência da formação religiosa de forma
% ambígua, pois, ao mesmo tempo que recusava a clausura e a austeridade da
% vida monástica, se sentia atraída pelos ideais fraternais e
% dadivosos professados pelo catolicismo, como a ajuda aos pobres e às
% pessoas em desalento. No entanto, ao perceber a hipocrisia de tais
% ideias e ter sua mente de criança povoada pelos fantasmas da religião,
% recusou a fé cristã e se proclamou uma livre-pensadora, dedicando-se,
% segundo suas palavras, ``não a Deus, mas ao homem''.


% Dessa recusa à religião emerge o interesse pela anarquia. Como sua amiga
% Emma Goldman, Voltairine assiste às mobilizações grevistas que culminaram na
% Batalha do Haymarket e fica impactada com o que vê. Sua conclusão diante
% das armações da polícia e da execução dos nove companheiros anarquistas
% é inequívoca: ``Sim, está crescendo. Sua palavra de medo, nossa palavra
% de fogo, \textsc{anarquia}''.\footnote{Voltairine \textit{apud} Avrich, 2018, p.\,90.} Essa frase
% resume não apenas o sentimento de Voltairine após a revolta de
% Haymarket, em 4 maio de 1886, em Chicago, como sintetiza sua posição em
% meio às lutas anarquistas. Uma posição decidida e corajosa, pois naquele
% momento muitos julgavam que o movimento anarquista também morria com as
% execuções, extradições e perseguições das autoridades, especialmente da
% polícia e seus juízes.\looseness=-1


% Essas diferentes
% experimentações dos anarquismos preenchiam os \textit{meetings} e periódicos
% libertários da época com discussões sobre o pacifismo e revolução, o uso
% da violência e sua relação com a ação direta, o autoritarismo dos
% marxistas e as armadilhas das democracias liberais. Num ponto, em meio a
% essas discussões, os anarquistas estavam de acordo e se distanciavam, em
% bloco, especialmente dos outros socialistas: afirmavam que as lutas para
% a abolição da propriedade e da exploração do capital não se apartam das
% práticas de liberdade e não seriam alcançadas por meios autoritários,
% como defendiam os partidários de uma ditadura do proletariado.

% Voltairine passou a vida na pobreza, tendo um estilo de vida austero,
% resistindo ao ``culto das coisas'', com saúde frágil, enfrentando
% diversas doenças, mas com obstinação e atividade incessante, nunca se
% aproximando de uma existência miserável. Nos \textsc{eua}, foi contemporânea e
% associada de anarquistas como Emma Goldman e Alexander Berkman, Benjamin
% Tucker, Johann Most, Josiah Warren, Lucy Parsons. O historiador
% anarquista Paul Avrich a chamou de \emph{figura menor}. Talvez por isso
% concentre essa enorme potência que salta dos textos aqui reunidos. Pouco
% conhecida, por motivos incertos, tratava-se de uma pessoa visceral e
% delicada, elegante e corajosa, avessa ao luxo e muito generosa. Era
% descrita por seus camaradas como uma pessoa simples e sofisticada,
% afeita à inteligência e à rejeição de dogmas. Era uma profunda
% conhecedora da história dos \textsc{eua}, que dominava com precisão, sempre pela
% perspectiva libertária dos embates e lutas, não das formalizações
% institucionais e legais. Por isso, era uma ferrenha opositora da
% centralização e da violência de Estado, da sujeição individual, do
% capitalismo, da exploração da mulher e da opressão do casamento.

% Voltairine passou a vida oferecendo palestras sobre anarquismo. Viajou a
% Ohio e Pensilvânia, palestrando em nome da \emph{American Secular
% Union}, à Escócia e à Grã-Bretanha, onde conheceu e tornou-se amiga de
% Kropotkin, Max Netllau, Louise Michel e diversos anarquistas espanhóis,
% russos e franceses. Foi próxima de Jean Grave, que conheceu em Paris por
% ocasião de uma visita a Sébastian Faure, quando conheceu de perto suas
% experimentações em educação libertária na La Ruche. Sempre em movimento
% e atenta às lutas e conversações, Voltairine transitou pelo
% individualismo de Benjamin Tucker, pelo mutualismo de Proudhon e pelo
% pacifismo de Tolstói, afastando-se de rótulos e identidades
% pré-estabelecidas. Recusava sobretudo as concepções de sociedade do
% comunismo e do socialismo autoritário que, segundo ela, culminavam numa
% regulamentação redutora de possibilidades e experiências de liberdade,
% ou seja, eram apenas, em suas palavras, a ``futura escravidão''. No lugar
% da fórmula marxista de que os homens são o que as circunstâncias
% fazem deles, afirmou que ``as circunstâncias são o que o homem faz delas''.\footnote{Voltairine \textit{apud} Avrich, 2018, p.\,141.}

% \textls[-10]{A elegância de Voltairine é sempre destacada nos depoimentos de seus
% alunos e camaradas. Sua delicadeza era acompanhada de seu ardor pela
% anarquia, a quem respeitava assim como a própria vida. Por isso
% abominava a forma com que certos intelectuais transformavam o anarquismo
% em moda e frequentavam, se encantando, os mesmos lugares que a
% burguesia. Sobre Samuel Gordon, companheiro com quem rompeu por lhe
% exigir o ``programa'' regular da vida conjugal, disse a Kropotkin certa
% vez que a paixão pela anarquia havia esfriado devido à conquista do
% sucesso material.}\looseness=-1


%Após ler Henry David Thoreau, escreveu sobre a
% exploração da terra. A exemplo de Errico Malatesta e Louise Michel, 
% recusou-se a denunciar um ex-aluno que lhe acertou um tiro, produzindo como
% reposta ao atentado que sofreu um discurso contra o sistema penal: ``os
% maiores crimes são cometidos pelo próprio Estado. Mas este chefe dos
% assassinos, com suas próprias mãos vermelhas com o sangue de centenas de
% milhares, assume a correção de infrações individuais, decretando
% milhares de leis para definir os vários graus de ofensa e punição, e
% assim empilha belas pedras com o propósito de enjaulá-los e
% atormentá-los. O Estado pune por milhares de anos e não nos livramos do
% crime, não o diminuímos''.\footnote{Voltairine \textit{apud} Avrich, 2018, p.\,154.} Assim,
% encontra-se na vida e nos escritos de Voltairine o que é inegociável
% para os anarquistas: a recusa de uma educação pautada nos castigos e
% recompensas que tem no sistema de justiça criminal e no Direito sua
% expressão terminal.
