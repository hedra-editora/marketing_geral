% Preencher com o nome das cor ou composição RGB (ex: [r=0.862, g=0.118, b=0.118]) 
\usecolors[crayola] 			   % Paleta de cores pré-definida: wiki.contextgarden.net/Color#Pre-defined_colors

% Cores definidas pelo designer:
% MyGreen		r=0.251, g=0.678, b=0.290 % 40ad4a
% MyCyan		r=0.188, g=0.749, b=0.741 % 30bfbd
% MyRed			r=0.820, g=0.141, b=0.161 % d12429
% MyPink		r=0.980, g=0.780, b=0.761 % fac7c2
% MyGray		r=0.812, g=0.788, b=0.780 % cfc9c7
% MyOrange		r=0.980, g=0.671, b=0.290 % faab4a

% Configuração de cores
\definecolor[MyColor][MyPink]      % ou ex: [r=0.862, g=0.118, b=0.118] % corresponde a RGB(220, 30, 30)
\definecolor[MyColorText][black]  % ou ex: [r=0.862, g=0.118, b=0.118] % corresponde a RGB(167, 169, 172)

% Classe para diagramação dos posts
\environment{marketing.env}		   


% Comandos & Instruções %%%%%%%%%%%%%%%%%%%%%%%%%%%%%%%%%%%%%%%%%%%%%%%%%%%%%%%%%%%%%%%|

% Cabeço e rodabé: Informações (caso queira trocar alguma coisa)
% 		\def\MensagemSaibaMais{SAIBA MAIS:}
% 		\def\MensagemSite{HEDRA.COM.BR}
% 		\def\MensagemLink{LINK NA BIO}

% Pesos para os títulos:
%		\startMyCampaign...		 \stopMyCampaign
%		\stopMyCampaignSection...   \stopMyCampaignSection

% Aplicação de imagens: 
% 		\MyCover{capa.pdf}  	% Aplicação de capa de livro com sombra
%		\MyPicture{Imagem.png}  % Imagem com aplicação de filtro segundo cor MyColorText
%		\MyPhoto{}			    % Aplicação simples de imagem com tamamho \textwidth

% Aplicação de imagem com legenda:		
% 		\placefigure{Legenda}{\externalfigure[drop2-1.png][width=\textwidth]}

% Cabeço e rodabé: Opções
% 		\Mensagem{AGORA É QUE SÃO ELAS}
% 		\Hashtag{campanha de natal}
% 		\Mensagem{campanha de natal}

% Alteração de várias cores de background:
% \setupbackgrounds[page][background=color,backgroundcolor=MyGray]

% Estrela: 
% \vfill\scale[lines=2]{\MyStar[MyColorText][none]} 					% Estrela pequena  
% \startpositioning 											% Estrela grande
%  \position(-1,-.3){\scale[scale=980]{\MyStar[white][none]}}
% \stoppositioning

% Logos e selos: 				
% \Hedra
% \HedraAyllon	% Não está pronto
% \HedraAcorde	% Não está pronto
% \Ayllon		% Não está pronto
% \Acorde		% Não está pronto

% Atalhos: 						
% 		\Seta  % Seta para baixo

%%%%%%%%%%%%%%%%%%%%%%%%%%%%%%%%%%%%%%%%%%%%%%%%%%%%%%%%%%%%%%%%%%%%%%%%%%%%%%%%%%%%%%%|

\starttext
%\showframe  %Para mostrar somente as linhas.

\Mensagem{QUEM FOI O FILÓSOFO LAOZI?}

\MyCover{LAOZI_DAODEJING_THUMB.pdf}

\page %---------------------------------------------------------|



\page

\MyPicture{LAOZI_DAODEJING_3.jpeg}


\page
\hyphenpenalty=10000
\exhyphenpenalty=10000


{\bf Laozi}, também conhecido como Lao Tzu, Laotze e Lao Tsé, é o primeiro e
mais importante autor do taoísmo chinês. Acredita-se que seu verdadeiro
nome era Li Er. Quase nada se sabe sobre sua vida, e mesmo sua real existência 
individual é posta em dúvida por especialistas. As biografias tradicionais
o apontam como o mantenedor dos arquivos da corte de Zhou. Também
mencionam o fato de ele haver se encontrado com Confúcio (Kong Fu Zi),
ou, literalmente, “Mestre Kong” −, que teria ficado impressionado com os
ensinamentos de Laozi. O {\bf Dao De Jing} teria sido escrito por Laozi a pedido 
de um guarda de fronteira, que o instou a registrar seus ensinamentos
como condição para partir, quando ele deixava o reino de Zhou, então em
decadência.


\page

\Hedra

\stoptext