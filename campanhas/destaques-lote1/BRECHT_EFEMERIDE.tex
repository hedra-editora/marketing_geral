% BRECHT_EFEMERIDE.tex
\usecolors[crayola]

% Configuração de cores TickleMePink
\definecolor[MyColor][Emerald]      % ou ex: [r=0.862, g=0.118, b=0.118] % corresponde a RGB(220, 30, 30)
\definecolor[MyColorText][black]  % ou ex: [r=0.862, g=0.118, b=0.118] % corresponde a RGB(167, 169, 172)

% Classe para diagramação dos posts
\environment{marketing.env}        

% Cabeço e rodapé: Informações (caso queira trocar alguma coisa)
        \def\MensagemSaibaMais  {SAIBA MAIS:}
        \def\MensagemSite       {HEDRA.COM.BR}
        \def\MensagemLink       {LINK NA BIO}
      
\environment{extra.env}

\starttext  %---------------------------------------------------------|

\def\MyBackgroundMessage{TEATRO ANTIBURGUÊS}
\MyBackground{brecht}

\startMyCampaign
\hyphenpenalty=10000
\exhyphenpenalty=10000
%{\bf NOAN CHOMSKY} O ANARQUISTA DO NOSSO SÉCULO
\position(0,7.8){\scale[factor=4]{\Seta\,BERTOLT BRECHT (1898--1956)}}
\stopMyCampaign

\page 

\Mensagem{MANCHETE CATIVANTE}
\setupbackgrounds[page][background=color,backgroundcolor=MyColor]


\page %----------------------------------------------------------|


\hyphenpenalty=10000
\exhyphenpenalty=10000

{\bf BERTOLT BRECHT} foi um importante dramaturgo e poeta alemão do século {\cap XX}. Seus trabalhos artísticos e teóricos influenciaram profundamente o teatro contemporâneo, tornando-o mundialmente conhecido.


\page

\starttikzpicture[remember picture,overlay]
\node at (4.8,-3.2) {\externalfigure[brecht2][width=10cm]};
\stoptikzpicture


\vfill{\scale[factor=4]{Bertolt Brecht e Walter Benjamin jogando xadrez na Dinamarca}\setupinterlinespace[line=1.5ex]\scale[factor=4]{em 1934.}}

\page
Durante o regime nazista, Brecht deixou a Alemanha, vivendo na Escandinavia e {\cap EUA}. Com o fim da guerra, o escritor retornou a Berlim Oriental e fundou a Berliner Ensemble com sua esposa e colaboradora de longa data, a atriz Helene Weigel.


\page %----------------------------------------------------------|

A sua concepção brechtiana de teatro implica um amplo conjunto de {\bf CRÍTICAS AO MUNDO BURGUÊS E CAPITALISTA}. Diante de uma sociedade que reproduzia acriticamente suas desigualdades, Brecht acredita que o teatro precisaria fazer com que o homem enxergasse os mecanismos perversos dessa reprodução, despertando a atenção do espectador para o seu {\bf POTENCIAL TRANSFORMADOR}. 

\page

É nesse sentido que o dramaturgo adota {\bf MECANISMOS DE DISTANCIAMENTO} --- a ruptura com os cenários realistas, a exposição da teatralidade, o canto --- como meios de afastar o espectador daquilo que lhe é familiar e tirá-lo de sua relação hipnótica de identificação com a cena e ensiná-lo a {\bf OBSERVAR O MUNDO CRITICAMENTE.}

\page

Seus textos, montagens e\\ inovações teóricas o consolidaram como {\bf UM DOS ESCRITORES FUNDAMENTAIS DESTE SÉCULO}.

\page

\Hedra

\stoptext %---------------------------------------------------------|



\page

