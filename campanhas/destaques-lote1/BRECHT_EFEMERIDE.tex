% AUTOR_LIVRO_AUTOR.tex
% Preencher com o nome das cor ou composição RGB (ex: [r=0.862, g=0.118, b=0.118]) 
\usecolors[crayola] 			   % Paleta de cores pré-definida: wiki.contextgarden.net/Color#Pre-defined_colors

% Cores definidas pelo designer:
% MyGreen		r=0.251, g=0.678, b=0.290 % 40ad4a
% MyCyan		r=0.188, g=0.749, b=0.741 % 30bfbd
% MyRed			r=0.820, g=0.141, b=0.161 % d12429
% MyPink		r=0.980, g=0.780, b=0.761 % fac7c2
% MyGray		r=0.812, g=0.788, b=0.780 % cfc9c7
% MyOrange		r=0.980, g=0.671, b=0.290 % faab4a

% Configuração de cores
\definecolor[MyColor][Emerald]      % ou ex: [r=0.862, g=0.118, b=0.118] % corresponde a RGB(220, 30, 30)
\definecolor[MyColorText][black]  % ou ex: [r=0.862, g=0.118, b=0.118] % corresponde a RGB(167, 169, 172)

% Classe para diagramação dos posts
\environment{marketing.env}		   

% Cabeço e rodapé: Informações (caso queira trocar alguma coisa)
 		\def\MensagemSaibaMais  {SAIBA MAIS:}
 		\def\MensagemSite		{HEDRA.COM.BR}
 		\def\MensagemLink       {LINK NA BIO}

\starttext %--------------------------------------------------------|

\Mensagem{TEATRO ANTIBURGUÊS}

\hyphenpenalty=10000
\exhyphenpenalty=10000

%\startMyCampaign

\MyPhoto{brecht}

%\stopMyCampaign

\vfill\scale[factor=6]{\Seta\,BERTOLT BRECHT (1898--1956)}

\page %----------------------------------------------------------|

{\bf BERTOLT BRECHT} foi um importante dramaturgo e poeta alemão do século {\cap XX}. Seus trabalhos artísticos e teóricos influenciaram profundamente o teatro contemporâneo, tornando-o mundialmente conhecido.


% \page

% \starttikzpicture[remember picture,overlay]
% \node at (4.8,-3.2) {\externalfigure[brecht2][width=10cm]};
% \stoptikzpicture


% \vfill{\scale[factor=4]{Bertolt Brecht e Walter Benjamin jogando xadrez na Dinamarca}\setupinterlinespace[line=1.5ex]\scale[factor=4]{em 1934.}}

\page
Durante o regime nazista, Brecht deixou a Alemanha, vivendo na Escandinávia e {\cap EUA}. Com o fim da guerra, o escritor retornou a Berlim Oriental e fundou a {\bf BERLINER ENSEMBLE} com sua esposa e colaboradora de longa data, a atriz Helene Weigel.


\page %----------------------------------------------------------|

\MyPhoto{brecht3}

A concepção brechtiana de teatro implica um amplo conjunto de {\bf CRÍTICAS AO MUNDO BURGUÊS E CAPITALISTA}.


\page
 Diante de uma sociedade que reproduzia acriticamente suas desigualdades, o teatro precisaria fazer com que o homem enxergasse os mecanismos perversos dessa reprodução, despertando a atenção do espectador para o seu {\bf POTENCIAL TRANSFORMADOR}. 

\page

É nesse sentido que o dramaturgo adota {\bf MECANISMOS DE\\ DISTANCIAMENTO} --- a ruptura com os cenários realistas, a exposição da teatralidade, o canto --- como meios de afastar o espectador daquilo que lhe é familiar e tirá-lo de sua relação hipnótica de identificação com a cena e ensiná-lo a {\bf OBSERVAR O MUNDO CRITICAMENTE.}

\page

Seus textos, montagens e\\ inovações teóricas o consolidaram como {\bf UM DOS ESCRITORES FUNDAMENTAIS DESTE SÉCULO}.

\page

\Hedra

\stoptext %---------------------------------------------------------|



