% AUTOR_LIVRO_AUTOR.tex
\usecolors[crayola]

% Configuração de cores TickleMePink
\definecolor[MyColor][x=88DB00]      % ou ex: [r=0.862, g=0.118, b=0.118] % corresponde a RGB(220, 30, 30)
\definecolor[MyColorText][black]  % ou ex: [r=0.862, g=0.118, b=0.118] % corresponde a RGB(167, 169, 172)

% Classe para diagramação dos posts
\environment{marketing.env}        

% Cabeço e rodapé: Informações (caso queira trocar alguma coisa)
        \def\MensagemSaibaMais  {SAIBA MAIS:}
        \def\MensagemSite       {HEDRA.COM.BR}
        \def\MensagemLink       {LINK NA BIO}
      
\environment{extra.env}

\starttext  %---------------------------------------------------------|

\def\MyBackgroundMessage{MENSAGEM CATIVANTE}
\MyBackground{METABIBLIOTECA_ALENCAR_CARTAS_1}

\startMyCampaign
\hyphenpenalty=10000
\exhyphenpenalty=10000
%{\bf NOAN CHOMSKY} O ANARQUISTA DO NOSSO SÉCULO
\position(0,7.8){\scale[factor=4]{\Seta\,NOME DO AUTOR (1900--2000)}}
\stopMyCampaign

\page 

\Mensagem{ÍCONE DO ROMANTISMO}
\setupbackgrounds[page][background=color,backgroundcolor=MyColor]

\hyphenpenalty=10000
\exhyphenpenalty=10000

{\bf JOSÉ DE ALENCAR}, um dos maiores escritores brasileiros do século {\cap XIX}, atravessou diversos gêneros, do romance ao teatro, deixando um {\bf LEGADO ÚNICO} na literatura nacional.

\page

Figura central do romantismo brasileiro, entre seus romances figuram {\bf O GUARANI} e {\bf IRACEMA}, que demonstram sua habilidade em criar tramas envolventes e personagens inesquecíveis.
 \page

 Além de sua contribuição literária, {\bf ALENCAR} foi figura ativa na política e no jornalismo de sua época, assumindo {\bf POSIÇÕES BASTANTE CONSERVADORAS} --- como a defesa da escravidão.

\page %----------------------------------------------------------|

\MyCover{METABIBLIOTECA_ALENCAR_CARTAS_THUMB}

\page %----------------------------------------------------------|

\Hedra

\stoptext %---------------------------------------------------------|