% BLECHER_ACONTECIMENTOS_EFEMERIDE.tex
% Preencher com o nome das cor ou composição RGB (ex: [r=0.862, g=0.118, b=0.118]) 
\usecolors[crayola] 			   % Paleta de cores pré-definida: wiki.contextgarden.net/Color#Pre-defined_colors

% Cores definidas pelo designer:
% MyGreen		r=0.251, g=0.678, b=0.290 % 40ad4a
% MyCyan		r=0.188, g=0.749, b=0.741 % 30bfbd
% MyRed			r=0.820, g=0.141, b=0.161 % d12429
% MyPink		r=0.980, g=0.780, b=0.761 % fac7c2
% MyGray		r=0.812, g=0.788, b=0.780 % cfc9c7
% MyOrange		r=0.980, g=0.671, b=0.290 % faab4a

% Configuração de cores
\definecolor[MyColor][ScreaminGreen]      % ou ex: [r=0.862, g=0.118, b=0.118] % corresponde a RGB(220, 30, 30)
\definecolor[MyColorText][black]     % ou ex: [r=0.862, g=0.118, b=0.118] % corresponde a RGB(167, 169, 172)

% Classe para diagramação dos posts
\environment{marketing.env}		   

\starttext %---------------------------------------------------------|

\hyphenpenalty=10000
\exhyphenpenalty=10000

\Mensagem{LITERATURA ROMENA EM FOCO} %Sempre usar esse header

\MyPhoto{BLECHER_ACONTECIMENTOS_2}

\vfill\scale[factor=6]{\Seta\,{\bf MAX BLECHER} (1909--1938)}\\
\page %---------------------------------------------------------| 

\hyphenpenalty=10000
\exhyphenpenalty=10000

\page %---------------------------------------------------------|

{\bf MAX BLECHER} é conhecido por suas obras introspectivas e exploratórias da condição humana. Frequentemente comparado a Franz Kafka e Marcel Proust pela intensidade e estilo de sua escrita, sua obra mais conhecida, {\bf ACONTECIMENTOS NA IRREALIDADE IMEDIATA}, é considerada um marco na literatura romena moderna pela sua originalidade e profundidade psicológica.

\page

Blecher pretendia cursar medicina em Paris, mas foi obrigado a abandonar os estudos por conta de sua saúde. Diagnosticado com um quadro grave de {\bf TUBERCULOSE ÓSSEA} aos 19 anos de idade, foi despachado para um sanatório e, ante um quadro clínico cada vez mais grave e sem perspectivas de melhora, passou o restante de sua vida em uma sequência de {\bf INTERNAÇÕES HOSPITALARES}.

\page

Por conta da sua experiência pessoal, a {\bf TEMÁTICA DOS SANATÓRIOS} perpassa a maior parte de sua obra. Passada uma década de seu diagnóstico, Max Blecher faleceu em  31 de maio de 1938 com apenas 28 anos.


\page

Os anos de sanatório lhe renderam muitos escritos e correspondências, como as {\bf CARTAS TROCADAS COM ANDRÉ BRETON}, líder do movimento surrealista francês, e os livros {\bf CORPO TRANSPARENTE}, {\bf CORAÇÕES CICATRIZADOS}. A Hedra publicou o {\bf ACONTECIMENTOS NA \\IRREALIDADE IMEDIATA}, de 1933, e em breve publicará {\bf A TOCA ILUMINADA}, de 1971, uma publicação póstuma.

\page %---------------------------------------------------------|

\MyCover{BLECHER_ACONTECIMENTOS_THUMB}

\page %---------------------------------------------------------|

\Hedra

\stoptext %---------------------------------------------------------|


