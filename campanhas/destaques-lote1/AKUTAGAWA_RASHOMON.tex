% Preencher com o nome das cor ou composição RGB (ex: [r=0.862, g=0.118, b=0.118]) 
\usecolors[crayola] 			   % Paleta de cores pré-definida: wiki.contextgarden.net/Color#Pre-defined_colors


% Cores definidas pelo designer:
% MyGreen		r=0.251, g=0.678, b=0.290 % 40ad4a
% MyCyan		r=0.188, g=0.749, b=0.741 % 30bfbd
% MyRed			r=0.820, g=0.141, b=0.161 % d12429
% MyPink		r=0.980, g=0.780, b=0.761 % fac7c2
% MyGray		r=0.812, g=0.788, b=0.780 % cfc9c7
% MyOrange		r=0.980, g=0.671, b=0.290 % faab4a

% Configuração de cores
\definecolor[MyColor][x=e3ee5c]      % ou ex: [r=0.862, g=0.118, b=0.118] % corresponde a RGB(220, 30, 30)
\definecolor[MyColorText][x=d22027]  % ou ex: [r=0.862, g=0.118, b=0.118] % corresponde a RGB(167, 169, 172)

% Classe para diagramação dos posts
\environment{marketing.env}		   


% Comandos & Instruções %%%%%%%%%%%%%%%%%%%%%%%%%%%%%%%%%%%%%%%%%%%%%%%%%%%%%%%%%%%%%%%|

% Cabeço e rodabé: Informações (caso queira trocar alguma coisa)
% 		\def\MensagemSaibaMais{SAIBA MAIS:}
% 		\def\MensagemSite{HEDRA.COM.BR}
% 		\def\MensagemLink{LINK NA BIO}

% Pesos para os títulos:
%		\startMyCampaign...		 \stopMyCampaign
%		\stopMyCampaignSection...   \stopMyCampaignSection

% Aplicação de imagens: 
% 		\MyCover{capa.pdf}  	% Aplicação de capa de livro com sombra
%		\MyPicture{Imagem.png}  % Imagem com aplicação de filtro segundo cor MyColorText
%		\MyPhoto{}			    % Aplicação simples de imagem com tamamho \textwidth

% Aplicação de imagem com legenda:		
% 		\placefigure{Legenda}{\externalfigure[drop2-1.png][width=\textwidth]}

% Cabeço e rodabé: Opções
% 		\Mensagem{AGORA É QUE SÃO ELAS}
% 		\Hashtag{campanha de natal}
% 		\Mensagem{campanha de natal}

% Alteração de várias cores de background:
% \setupbackgrounds[page][background=color,backgroundcolor=MyGray]

% Estrela: 
% \vfill\scale[lines=2]{\MyStar[MyColorText][none]} 					% Estrela pequena  
% \startpositioning 											% Estrela grande
%  \position(-1,-.3){\scale[scale=980]{\MyStar[white][none]}}
% \stoppositioning

% Logos e selos: 				
% \Hedra
% \HedraAyllon	% Não está pronto
% \HedraAcorde	% Não está pronto
% \Ayllon		% Não está pronto
% \Acorde		% Não está pronto

% Atalhos: 						
% 		\Seta  % Seta para baixo

%%%%%%%%%%%%%%%%%%%%%%%%%%%%%%%%%%%%%%%%%%%%%%%%%%%%%%%%%%%%%%%%%%%%%%%%%%%%%%%%%%%%%%%|



\starttext

%\showframe  %Para mostrar somente as linhas.

\Mensagem{DESTAQUES}

\MyCover{AKUTAGAWA_RASHOMON_THUMB.pdf}

\page %---------------------------------------------------------|

\hyphenpenalty=10000
\exhyphenpenalty=10000


«Uma borboleta volteava no vento impregnado por um cheiro
de ervas aquáticas. Durante apenas um ínfimo segundo, ele
sentiu o roçar de suas asas sobre os lábios ressecados. Mas
o pó das asas que assim fora espalhado sobre seus lábios
continuou a brilhar, mesmo muitos anos depois.»

\vfill
\scale[factor=fit]{\tfxx Tradução do Japonês de {\bf Madalena Hashimoto e Junko Ota}.}

\page 

\MyPicture{AKUTAGAWA_RASHOMON_1.jpeg}


\page


Ryûnosuke Akutagawa 芥川龍之介, um dos maiores nomes da
literatura moderna japonesa, nasceu em Tóquio no
fim do século XIX, durante o período Meiji, quando
o país se abria por completo à influência da cultura 
ocidental. 

% Leitor precoce, ainda criança lê com
% entusiasmo traduções do dramaturgo norueguês
% Henrik Ibsen e do escritor francês Anatole France.
% Na juventude, traduz W. B. Yeats e se especializa
% em literatura inglesa na Universidade Imperial de
% Tóquio, na mesma época em que passa a se interessar 
% pela ética cristã. Assim, soube mesclar tradição
% e modernidade como nenhum outro contemporâneo 
% seu, reinterpretando temas japoneses de
% narrativas do século XII e autores e filósofos ocidentais.
%  Escreveu ainda contos autobiográficos,
% sobretudo a partir da década de 1920, que relembram 
% o contexto familiar tremendamente conturbado,
%  como Passagens do caderno de notas de
% Yasukichi e A vida de um idiota.
% A instabilidade psíquico-emocional de sua mãe,
% então considerada louca, perseguiu-o como um
% fantasma durante a vida inteira, e há quem diga que
% foi isso que o levou ao suicídio, em julho de 1927.
% O fato é que Ryûnosuke Akutagawa maneja uma
% pluralidade de gêneros literários e desenvolve temas 
% demasiadamente humanos e universais, como
% o egoísmo e o valor da arte enquanto redentora
% da miséria da vida cotidiana, além da incessante
% busca por um equilíbrio moral entre o tradicional e
% o moderno.


\page

\MyPicture{AKUTAGAWA_RASHOMON_2.jpeg}

\page 

\MyPicture{AKUTAGAWA_RASHOMON_3.jpeg}

\page 

\MyPicture{AKUTAGAWA_RASHOMON_4.png}



\page

\Hedra

\stoptext

% Para texto no post
% ==================
% Ryûnosuke Akutagawa, um dos maiores nomes da
% literatura moderna japonesa, nasceu em Tóquio no
% fim do século XIX, durante o período Meiji, quando
% o país se abria por completo à influência da cultura 
% ocidental. Leitor precoce, ainda criança lê com
% entusiasmo traduções do dramaturgo norueguês
% Henrik Ibsen e do escritor francês Anatole France.
% Na juventude, traduz W. B. Yeats e se especializa
% em literatura inglesa na Universidade Imperial de
% Tóquio, na mesma época em que passa a se interessar 
% pela ética cristã. Assim, soube mesclar tradição
% e modernidade como nenhum outro contemporâneo 
% seu, reinterpretando temas japoneses de
% narrativas do século XII e autores e filósofos ocidentais.
%  Escreveu ainda contos autobiográficos,
% sobretudo a partir da década de 1920, que relembram 
% o contexto familiar tremendamente conturbado,
%  como Passagens do caderno de notas de
% Yasukichi e A vida de um idiota.
% A instabilidade psíquico-emocional de sua mãe,
% então considerada louca, perseguiu-o como um
% fantasma durante a vida inteira, e há quem diga que
% foi isso que o levou ao suicídio, em julho de 1927.
% O fato é que Ryûnosuke Akutagawa maneja uma
% pluralidade de gêneros literários e desenvolve temas 
% demasiadamente humanos e universais, como
% o egoísmo e o valor da arte enquanto redentora
% da miséria da vida cotidiana, além da incessante
% busca por um equilíbrio moral entre o tradicional e
% o moderno.

% OU 

% Rashômon e outros contos reúne dez contos de diversos períodos da breve
% existência do autor. Rashômon (1915) e Dentro do bosque (1922) retratam a
% cultura de Heian (atual Quioto). Em Memorando «Ryôsai Ogata» (1917), Ogin
% (1923) e O mártir (1918), a temática cristã é o fio condutor. Devoção à literatura
% popular (1917) e Terra morta (1918) têm como pano de fundo a cultura de Edo,
% atual Tóquio. A abertura do Japão para o Ocidente no período Meiji compõe
% o enredo de O baile (1912). Por fim, dois contos de caráter autobiográfico,
% do final da vida de Akutagawa: Passagens do caderno de notas de Yasukichi
% (1923) e A vida de um idiota (1927). Esta nova edição, com texto revisto pelas
% tradutoras, conta ainda com nova introdução e acréscimo de notas.

% Hashtags
% 1. #RyunosukeAkutagawa
% 2. #LiteraturaJaponesa
% 3. #ModernidadeETradição
% 4. #PeríodoMeiji
% 5. #CulturaOcidentalNoJapão

% Ideias para futuros posts
% =========================
% * 芥川龍之介 - Ryūnosuke Akutagawa (o que significa o nome de Akutagawa em japonês)
% * Filme e livro
% * Sua mãe 
% * Suicídio
% * Mangá (para o tiktok)
% * Entrevistas com as tradutoras