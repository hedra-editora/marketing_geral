% AKUTAGAWA_RASHOMON_CURIOSIDADE.tex
% Vamos falar sobre isso "curiosidades"
% > "EM CONTEXTO"

% Preencher com o nome das cor ou composição RGB (ex: [r=0.862, g=0.118, b=0.118]) 
\usecolors[crayola] 			   % Paleta de cores pré-definida: wiki.contextgarden.net/Color#Pre-defined_colors

% Cores definidas pelo designer:
% MyGreen		r=0.251, g=0.678, b=0.290 % 40ad4a
% MyCyan		r=0.188, g=0.749, b=0.741 % 30bfbd
% MyRed			r=0.820, g=0.141, b=0.161 % d12429
% MyPink		r=0.980, g=0.780, b=0.761 % fac7c2
% MyGray		r=0.812, g=0.788, b=0.780 % cfc9c7
% MyOrange		r=0.980, g=0.671, b=0.290 % faab4a

% Configuração de cores
\definecolor[MyColor][MaximumBluePurple]      % ou ex: [r=0.862, g=0.118, b=0.118] % corresponde a RGB(220, 30, 30)
\definecolor[MyColorText][black]  % ou ex: [r=0.862, g=0.118, b=0.118] % corresponde a RGB(167, 169, 172)

% Classe para diagramação dos posts
\environment{marketing.env}		   

\starttext %---------------------------------------------------------|

\hyphenpenalty=10000
\exhyphenpenalty=10000

\Mensagem{EM CONTEXTO}

\startMyCampaign
\hyphenpenalty=10000
\exhyphenpenalty=10000

A {\bf VERDADE 
E A NATUREZA HUMANA}
EM RASHÔMON
%\vfill\scale[lines=2]{\MyStar[MyColorText][none]} 					% Estrela pequena  

\stopMyCampaign

\page %---------------------------------------------------------| 

\hyphenpenalty=10000
\exhyphenpenalty=10000

Dentre os contos de Ryûnosuke Akutagawa, «Rashômon» e «Dentro do bosque» estão, sem dúvidas, dentre os mais conhecidos. O sucesso dessas obras se deve, em grande parte, devido ao filme {\it Rashômon}, de 1950, {\bf ADAPTAÇÃO PREMIADA DE AKIRA KUROSAWA}.

\page

\MyPicture{AKUTAGAWA_RASHOMON_5}

\vfill
\scale[factor=5]{{\bf Rashômon}, filme de Akira Kurosawa, 1950, Japão.}

\page

\hyphenpenalty=10000
\exhyphenpenalty=10000

«Embora nomeie o filme, o primeiro conto é pouco utilizado, servindo, entretanto, como poderoso espaço simbólico da ação, sendo que a discussão fundamental encontra-se nos vários depoimentos em primeira pessoa de envolvidos num
crime ocorrido {\it dentro do bosque}.»

\vfill
\scale[factor=5]{{\bf Madalena Hashimoto Cordaro}, tradutora da edição da
Hedra de {\it Rashômon e outros contos}.} 
% \page

% {\it Ora, um
% dos recursos máximos da tradição literária japonesa é  {\it honkadori}, “tirar de um poema original”, alusão ou reinterpretação de trechos, versos, trama de autor respeitado. Akutagawa, quando retira de uma
% coletânea compilada oito séculos antes pequenas cenas ou relatos
% sucintos, estes são deliberadamente transformados em obra sua, sendo 
% utilizadas como motor para sua discussão contemporânea acerca da ética;} 

% \page

% {\it 
% e Kurosawa, ao retomar e adaptar dois contos seus, também se apropria
% das reflexões desenvolvidas, mas, criando novos personagens, reitera
% sua confiança final no ser humano, sentido ausente nos originais.} 

% {\hfill\tf ---Madalena Hashimoto Cordaro}

% \page

% Apesar das diferenças, ambas as obras constrõem-se a partir da apresentação de múltiplas versões conflitantes de um mesmo evento, levantando questões sobre a relatividade da verdade e a complexidade da natureza humana.

\page %---------------------------------------------------------|

\MyCover{AKUTAGAWA_RASHOMON_THUMB}

% \page

% {\bf Rashômon e outros contos} reúne dez contos de diversos períodos da breve
% existência do autor. Dentre eles estão “Dentro do bosque” (1922), “O mártir” (1918), “Terra morta” (1918) e “A vida de um idiota” (1927). As temáticas abordadas vão desde a cultura de Heian e Edo (atuais Quioto e Tóquio), a ética cristã, a abertura do Japão ao Ocidente, até a própria biografia do autor. Esta nova edição, com texto revisto pelas tradutoras, conta ainda com nova introdução e acréscimo de notas.

\page

\Hedra

\stoptext %---------------------------------------------------------|
	