% HESIODO_TRABALHOS_CURIOSIDADE_2.tex
% Vamos falar sobre isso "curiosidades"
% > "EM CONTEXTO"

% Preencher com o nome das cor ou composição RGB (ex: [r=0.862, g=0.118, b=0.118]) 
\usecolors[crayola] 			   % Paleta de cores pré-definida: wiki.contextgarden.net/Color#Pre-defined_colors

% Cores definidas pelo designer:
% MyGreen		r=0.251, g=0.678, b=0.290 % 40ad4a
% MyCyan		r=0.188, g=0.749, b=0.741 % 30bfbd
% MyRed			r=0.820, g=0.141, b=0.161 % d12429
% MyPink		r=0.980, g=0.780, b=0.761 % fac7c2
% MyGray		r=0.812, g=0.788, b=0.780 % cfc9c7
% MyOrange		r=0.980, g=0.671, b=0.290 % faab4a

% Configuração de cores
\definecolor[MyColor][Almond]      % ou ex: [r=0.862, g=0.118, b=0.118] % corresponde a RGB(220, 30, 30)
\definecolor[MyColorText][MountainMeadow]  % ou ex: [r=0.862, g=0.118, b=0.118] % corresponde a RGB(167, 169, 172)

% Classe para diagramação dos posts
\environment{marketing.env}		   

\starttext %---------------------------------------------------------|

\hyphenpenalty=10000
\exhyphenpenalty=10000

\Mensagem{EM CONTEXTO}

\startMyCampaign
\hyphenpenalty=10000
\exhyphenpenalty=10000

O MITO 
DE PANDORA
SEGUNDO
{\bf HESÍODO}

%\vfill\scale[lines=2]{\MyStar[MyColorText][none]} 					% Estrela pequena  

\stopMyCampaign

\page %---------------------------------------------------------|

{\MyPicture{AUTOR_LIVRO_1.jpeg}}

\page %---------------------------------------------------------| 

Em 1940, de posse de um visto transitório para os Estados Unidos, {\bf
Benjamin} deixou Paris e dirigiu-se para a fronteira franco-espanhola. Chegou
em Lourdes no dia 24 de Setembro e, após uma complicada viagem de autocarro
até a aldeia de Banyuls-sur-Mer, perto de Portbou, {\bf Benjamin} iniciou a
subida dos Pirinéus a pé.

\page %---------------------------------------------------------|

No dia 26 de Setembro, após horas de uma árdua caminhada, {\bf Benjamin} e um
pequeno grupo de refugiados que viajavam com ele, incluíndo a fotógrafa Henny
Gurland e seu filho, finalmente chegaram a Portbou, do lado espanhol.
Entretanto, ao tentar cruzar a fronteira na Espanha, foi informado de que a
política espanhola havia mudado de repente e que eles seriam deportados de
volta à França na manhã seguinte.

\page %---------------------------------------------------------|

Temendo ser entregue nas mãos dos nazistas, Walter Benjamin tomou uma decisão
trágica. Na noite de 26 de Setembro de 1940, em um quarto do Hotel de Francia,
em Portbou, Benjamin cometeu suicídio por overdose de morfina.

\page %---------------------------------------------------------|

Os outros que viajavam com ele tiveram permissão de passagem no dia seguinte e
chegaram em segurança à Lisboa em 30 de setembro de 1940.


\page %---------------------------------------------------------|

A sua tentativa de fuga e suicídio são retratadas na série {\bf
Transatlântico} (2023), que relata a história de um jornalista americano que,
entre os anos de 1940--41, coordena fugas para os {\cap  eua} para mais de
dois mil refugiados que correm o risco de vida.

\page %---------------------------------------------------------|

\MyCover{HESIODO_TRABALHOS_THUMB.pdf}

\page

{\bf Trabalhos e dias} é um poema épico de 828 versos
em que são contados alguns dos mitos gregos mais
conhecidos, como o de Prometeu e o de
Pandora. Com a ajuda das Musas, o poeta narra em primeira pessoa e se dirige 
a seu irmão Perses, na tentativa de ensinar a ele
verdades divinas a respeito das práticas humanas.

\page

\Hedra


\stoptext
%Trabalhos e dias é o poema grego no qual se
% mencionam a caixa de Pandora — na verdade uma
% ânfora —, as linhagens, raças ou idades do homem
% e uma poética representação das estações do
% ano e das atividades agrícolas associadas a elas.
% Nesse livro, porém, não é de quase-super-homens
% como Aquiles e Odisseu que se fala, mas de
% outros tipos de heróis: o poeta que de tudo sabe;
% o bom rei, que zela pela justiça em sua comu-
% nidade; e o agricultor bem-sucedido que, para
% produzir riqueza por meio de sua propriedade ou
% fazenda, deve não só trabalhar arduamente, mas
% atentar a uma série enorme de regras climáticas,
% morais e religiosas, aquilo que nós chamamos de
% acaso também espreita.
% 
% Além de Trabalhos e dias, somente chegaram
% inteiros até nós os poemas Teogonia e Escudo de
% Héracles, entre aqueles atribuídos na Antiguidade
% ao grego Hesíodo, poeta que teria vivido por volta
% dos séculos VIII e VII a.C., ou seja, mais ou menos
% na mesma época que Homero, tido pelos antigos
% como o autor da Ilíada e da Odisseia.
% 
% É possível enumerar vários aspectos relacio-
% nados à cultura grega do mesmo período — como
% a introdução e a expansão do uso da escrita — que
% fazem muitos pesquisadores duvidar que tenha
% havido um poeta histórico chamado Hesíodo e que
% ele tenha composto por escrito os poemas asso-
% ciados a seu nome. Mas, para entender um poema
% como Trabalhos e dias, o próprio texto ainda é
% nossa principal ferramenta, de sorte que muitas
% das questões a ele pertinentes precisarão con-
% tinuar sem uma resposta categórica, como, por
% exemplo, quando