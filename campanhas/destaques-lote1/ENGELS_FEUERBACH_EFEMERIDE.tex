% AUTOR_LIVRO_AUTOR.tex
% Preencher com o nome das cor ou composição RGB (ex: [r=0.862, g=0.118, b=0.118]) 
\usecolors[crayola] 			   % Paleta de cores pré-definida: wiki.contextgarden.net/Color#Pre-defined_colors

% Cores definidas pelo designer:
% MyGreen		r=0.251, g=0.678, b=0.290 % 40ad4a
% MyCyan		r=0.188, g=0.749, b=0.741 % 30bfbd
% MyRed			r=0.820, g=0.141, b=0.161 % d12429
% MyPink		r=0.980, g=0.780, b=0.761 % fac7c2
% MyGray		r=0.812, g=0.788, b=0.780 % cfc9c7
% MyOrange		r=0.980, g=0.671, b=0.290 % faab4a

% Configuração de cores
\definecolor[MyColor][Sepia]	  [r=0.862, g=0.118, b=0.118]      % ou ex: [r=0.862, g=0.118, b=0.118] % corresponde a RGB(220, 30, 30)
\definecolor[MyColorText][black] [r=0.655, g=0.663, b=0.675]      % ou ex: [r=0.862, g=0.118, b=0.118] % corresponde a RGB(167, 169, 172)

% Classe para diagramação dos posts
\environment{marketing.env}		   

% Cabeço e rodapé: Informações (caso queira trocar alguma coisa)
 		\def\MensagemSaibaMais  {SAIBA MAIS:}
 		\def\MensagemSite		{HEDRA.COM.BR}
 		\def\MensagemLink       {LINK NA BIO}

\starttext %--------------------------------------------------------|

\Mensagem{MANCHETE CATIVANTE}

\hyphenpenalty=10000
\exhyphenpenalty=10000

%\startMyCampaign

\MyPicture{engels}

%\stopMyCampaign

\vfill\scale[factor=6]{\Seta\,FRIEDRICH ENGELS (1820--1895)}

\page %----------------------------------------------------------|

\hyphenpenalty=10000
\exhyphenpenalty=10000

Hoje se completam 129 anos da morte do revolucionário alemão Friedrich Engels, responsável por desenvolver, com Marx, o chamado {\bf SOCIALISMO CIENTÍFICO}. 

\page
\hyphenpenalty=10000
\exhyphenpenalty=10000

Primogênito de um industrial da tecelagem, viveu em Berlim e depois em Manchester, onde conheceu {\bf MARX}, em 1842, ao se envolver com o jornalismo radical e política. 

\page
\hyphenpenalty=10000
\exhyphenpenalty=10000

Em Bruxelas, auxiliou na formação da {\bf LIGA DOS COMUNISTAS}. Em 1849, tomou parte de um levante no sul da Alemanha e, com seu fracasso, voltou à Inglaterra. 

\page
\hyphenpenalty=10000
\exhyphenpenalty=10000

Com a morte de Marx, trabalhou na preparação e na publicação dos dois últimos volumes de {\bf O CAPITAL}. Investiu seu tempo em outras produções teóricas e teve significativa influência na social-democracia alemã.

\page %----------------------------------------------------------|

\MyCover{ENGELS_FILOSOFIA_THUMB.pdf}

\page %----------------------------------------------------------|

\Hedra

\stoptext %---------------------------------------------------------|
