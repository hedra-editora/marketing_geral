% ENGELS_FEUERBACH_EFEMERIDE.tex
\usecolors[crayola]

% Configuração de cores TickleMePink
\definecolor[MyColor][x=d22027]      % ou ex: [r=0.862, g=0.118, b=0.118] % corresponde a RGB(220, 30, 30)
\definecolor[MyColorText][black]  % ou ex: [r=0.862, g=0.118, b=0.118] % corresponde a RGB(167, 169, 172)

% Classe para diagramação dos posts
\environment{marketing.env}        

% Cabeço e rodapé: Informações (caso queira trocar alguma coisa)
        \def\MensagemSaibaMais  {SAIBA MAIS:}
        \def\MensagemSite       {HEDRA.COM.BR}
        \def\MensagemLink       {LINK NA BIO}
      
\environment{extra.env}

\starttext  %---------------------------------------------------------|

\def\MyBackgroundMessage{MENSAGEM CATIVANTE}
\MyBackground{engels}

\startMyCampaign
\hyphenpenalty=10000
\exhyphenpenalty=10000
%{\bf NOAN CHOMSKY} O ANARQUISTA DO NOSSO SÉCULO
\position(0,7.8){\scale[factor=4]{\Seta\,FRIEDRICH ENGELS (1820--1895)}}
\stopMyCampaign

\page 

\Mensagem{MANCHETE CATIVANTE}
\setupbackgrounds[page][background=color,backgroundcolor=MyColor]
\hyphenpenalty=10000
\exhyphenpenalty=10000

Hoje se completam 129 anos da morte do revolucionário alemão Friedrich Engels, responsável por desenvolver, com Marx, o chamado {\bf SOCIALISMO CIENTÍFICO}. 

\page
\hyphenpenalty=10000
\exhyphenpenalty=10000

Filho mais velho de um industrial da tecelagem, viveu em Berlim e depois em Manchester, onde conheceu {\bf MARX}, em 1842, ao se envolver com o jornalismo radical e a política. 

\page
\hyphenpenalty=10000
\exhyphenpenalty=10000

Em Bruxelas, auxiliou na formação da {\bf LIGA DOS COMUNISTAS}. Em 1849, tomou parte de um levante no sul da Alemanha e, com seu fracasso, voltou à Inglaterra. 

\page
\hyphenpenalty=10000
\exhyphenpenalty=10000

Com a morte de Marx, trabalhou na preparação e na publicação dos dois últimos volumes de {\bf O CAPITAL}. Investiu seu tempo em outras produções teóricas e teve significativa influência na social-democracia alemã.

\page %----------------------------------------------------------|

\MyCover{ENGELS_FILOSOFIA_THUMB.pdf}

\page %----------------------------------------------------------|

\Hedra

\stoptext %---------------------------------------------------------|
