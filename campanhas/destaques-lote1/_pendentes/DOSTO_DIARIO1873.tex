% Preencher com o nome das cor ou composição RGB (ex: [r=0.862, g=0.118, b=0.118]) 
\usecolors[crayola] 			   % Paleta de cores pré-definida: wiki.contextgarden.net/Color#Pre-defined_colors

% Cores definidas pelo designer:
% MyGreen		r=0.251, g=0.678, b=0.290 % 40ad4a
% MyCyan		r=0.188, g=0.749, b=0.741 % 30bfbd
% MyRed			r=0.820, g=0.141, b=0.161 % d12429
% MyPink		r=0.980, g=0.780, b=0.761 % fac7c2
% MyGray		r=0.812, g=0.788, b=0.780 % cfc9c7
% MyOrange		r=0.980, g=0.671, b=0.290 % faab4a

% Configuração de cores
\definecolor[MyColor][MyGray]      % ou ex: [r=0.862, g=0.118, b=0.118] % corresponde a RGB(220, 30, 30)
\definecolor[MyColorText][black]  % ou ex: [r=0.862, g=0.118, b=0.118] % corresponde a RGB(167, 169, 172)

% Classe para diagramação dos posts
\environment{marketing.env}		   


% Comandos & Instruções %%%%%%%%%%%%%%%%%%%%%%%%%%%%%%%%%%%%%%%%%%%%%%%%%%%%%%%%%%%%%%%|

% Cabeço e rodabé: Informações (caso queira trocar alguma coisa)
% 		\def\MensagemSaibaMais{SAIBA MAIS:}
% 		\def\MensagemSite{HEDRA.COM.BR}
% 		\def\MensagemLink{LINK NA BIO}

% Pesos para os títulos:
%		\startMyCampaign...		 \stopMyCampaign
%		\stopMyCampaignSection...   \stopMyCampaignSection

% Aplicação de imagens: 
% 		\MyCover{capa.pdf}  	% Aplicação de capa de livro com sombra
%		\MyPicture{Imagem.png}  % Imagem com aplicação de filtro segundo cor MyColorText
%		\MyPhoto{}			    % Aplicação simples de imagem com tamamho \textwidth

% Aplicação de imagem com legenda:		
% 		\placefigure{Legenda}{\externalfigure[drop2-1.png][width=\textwidth]}

% Cabeço e rodabé: Opções
% 		\Mensagem{AGORA É QUE SÃO ELAS}
% 		\Hashtag{campanha de natal}
% 		\Mensagem{campanha de natal}

% Alteração de várias cores de background:
% \setupbackgrounds[page][background=color,backgroundcolor=MyGray]

% Estrela: 
% \vfill\scale[lines=2]{\MyStar[MyColorText][none]} 					% Estrela pequena  
% \startpositioning 											% Estrela grande
%  \position(-1,-.3){\scale[scale=980]{\MyStar[white][none]}}
% \stoppositioning

% Logos e selos: 				
% \Hedra
% \HedraAyllon	% Não está pronto
% \HedraAcorde	% Não está pronto
% \Ayllon		% Não está pronto
% \Acorde		% Não está pronto

% Atalhos: 						
% 		\Seta  % Seta para baixo

%%%%%%%%%%%%%%%%%%%%%%%%%%%%%%%%%%%%%%%%%%%%%%%%%%%%%%%%%%%%%%%%%%%%%%%%%%%%%%%%%%%%%%%|

\starttext
%\showframe  %Para mostrar somente as linhas.

\Mensagem{DESTAQUES}

\MyCover{DOSTO_DIARIO1873_THUMB.pdf}

\page %---------------------------------------------------------|


\starttikzpicture
                    \node[yshift=-5mm, inner sep=0] (image) at (0,0) {\clip[height=72mm,voffset=3mm]{\externalfigure[DOSTO_DIARIO1873_1.jpeg][width=\textwidth]}};
                    \fill[MyColorText,opacity=0.3] (image.south west) rectangle (image.north east);
\stoptikzpicture

\page 


«Um dos piores equívocos do jovem poeta consiste em considerar
o desmascaramento do vício (ou do que o liberalismo toma por vício) e a 
incitação ao ódio e à vingança como o único caminho possível para atingir seu
alvo. No entanto, para um talento forte, até desse caminho estreito é possível
escapar, sem, com isso, anular-se no começo de sua vida artística; ele
se valerá de uma regra de ouro: a palavra dita é de prata, mas a não dita é de ouro.»

\vfill
\scale[factor=fit]{Tradução do russo de {\bf Moissei e Daniela Mountian}}

\page

\MyPicture{DOSTO_DIARIO1873_2.jpeg}

{\it
Inclassificável, o {\bf Diário} é único no conjunto da
obra de Dostoiévski, espécie de síntese entre teoria
e prática, entre crítica e literatura.}\blank[1ex]

{\hfill\tf ---Irineu Franco Perpétuo}



\page

\Hedra

\stoptext




% Para post

% O livro reúne escritos publicados em revistas literárias 
% em que Dostoiévski, o gênio das letras russas, comentava 
% com paixão os temas políticos e estéticos de seu
% tempo. À primeira vista reacionárias e eslavófilas,
% suas posições seriam talvez melhor descritas como
% crítica da cultura de um conservador {\em sui generis}.
% Se observamos Dostoiévski defender uma incerta
% moral na relação servo-senhor e flertar com clichês
% antissemitas, também constatamos sua sensibilidade 
% para causas sociais como o investimento
% público em educação e a boa remuneração dos
% trabalhadores. 

% (...)

% “O escritor começa a escrever um romance e, logo que um comerciante ou
% um clérigo se põem a falar em suas páginas, escolhe uma linguagem usando
% as anotações do caderninho. Os leitores caem na risada e fazem elogios,
% e tudo parece tão exato: escrito, palavra por palavra, com naturalidade, mas
% acontece que isso é pior do que a mentira, justamente porque o comerciante
% ou o soldado no romance falam por essências, quer dizer, como nem um
% nem outro jamais falaria naturalmente.”

% Segundo texto

% Diário de um escritor reúne mais de mil páginas de ensaios, crônicas e contos
% que foram produzidos por Fiódor Dostoiévski entre 1873 e 1881 (ano de sua morte).
% Originalmente, o Diário era o título de uma coluna assinada por Dostoiévski na
% revista de política e literatura O cidadão, que a partir de 1876, passou a ser publi-
% cada como uma nova revista. Foi com a atividade de jornalista e polemista, e não
% como escritor, que Dostoiévski conquistou notoriedade. O interesse pelo Diário
% não reside, no entanto, apenas nas polêmicas de seu tempo: em suas páginas, é
% possível acompanhar o próprio processo criativo do autor, que “constrói uma teo-
% ria estética ao mesmo tempo que a aplica”, como observa Irineu Franco Perpetuo
% na “Apresentação” deste volume.