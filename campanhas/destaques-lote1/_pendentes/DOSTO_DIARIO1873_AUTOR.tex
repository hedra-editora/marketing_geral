% DOSTO_DIARIO1873_AUTOR.tex
% quem é "autor"
% > "VIDA & OBRA"

% Preencher com o nome das cor ou composição RGB (ex: [r=0.862, g=0.118, b=0.118]) 
\usecolors[crayola] 			   % Paleta de cores pré-definida: wiki.contextgarden.net/Color#Pre-defined_colors

% Cores definidas pelo designer:
% MyGreen		r=0.251, g=0.678, b=0.290 % 40ad4a
% MyCyan		r=0.188, g=0.749, b=0.741 % 30bfbd
% MyRed			r=0.820, g=0.141, b=0.161 % d12429
% MyPink		r=0.980, g=0.780, b=0.761 % fac7c2
% MyGray		r=0.812, g=0.788, b=0.780 % cfc9c7
% MyOrange		r=0.980, g=0.671, b=0.290 % faab4a

% Configuração de cores
\definecolor[MyColor][SteelTeal]      % ou ex: [r=0.862, g=0.118, b=0.118] % corresponde a RGB(220, 30, 30)
\definecolor[MyColorText][SunnyPearl]  % ou ex: [r=0.862, g=0.118, b=0.118] % corresponde a RGB(167, 169, 172)


% Classe para diagramação dos posts
\environment{marketing.env}		   

% Cabeço e rodapé: Informações (caso queira trocar alguma coisa)
 		\def\MensagemSaibaMais  {SAIBA MAIS:}
 		\def\MensagemSite		{HEDRA.COM.BR}
 		\def\MensagemLink       {LINK NA BIO}


\starttext %--------------------------------------------------------|

\Mensagem{VIDA \& OBRA}

\starttikzpicture
                    \node[yshift=-5mm, inner sep=0] (image) at (0,0) {\clip[height=72mm,voffset=3mm]{\externalfigure[DOSTO_DIARIO1873_1.jpeg][width=\textwidth]}};
                    \fill[MyColorText,opacity=0.3] (image.south west) rectangle (image.north east);
\stoptikzpicture

\page %----------------------------------------------------------|


\hyphenpenalty=10000
\exhyphenpenalty=10000


{\bf FIODOR DOSTOIÉVSKI} foi um renomado escritor russo do século {\cap XIX}, conhecido por suas obras profundas e exploratórias da psicologia humana. Nascido em Moscou em 1821, Dostoiévski sofreu grandes desafios ao longo de sua vida, incluindo a morte do pai na infância e sua prisão por participar de um círculo de intelectuais críticos ao regime czarista. 

\page

Durante seu tempo na prisão na Sibéria, {\bf Dostoiévski} experimentou profundamente a condição humana e refletiu sobre temas como o bem e o mal, a liberdade e a moralidade. Essas experiências moldaram sua visão de mundo e influenciaram sua escrita ao longo de sua carreira literária.

\page 

Dostoiévski é amplamente reconhecido por obras como {\bf Crime e castigo}, {\bf Os irmãos Karamázov} e {\bf O idiota}, as quais exploram questões existenciais, dilemas éticos e as complexidades da alma humana. Sua escrita profundamente psicológica e existencial fez dele um dos mais proeminentes autores da literatura russa e mundial.

\page

\MyCover{DOSTO_DIARIO1873_THUMB}

\page

{\bf Diário de um escritor (1873)} é o primeiro dos {\it Diários} de Dostoiévski, os quais reunem mais de mil páginas de ensaios, crônicas e contos produzidos pelo autor entre 1873 e 1881, originalmente para sua coluna jornalística de mesmo nome. Com esta obra, o leitor terá a chance de acompanhar o próprio processo criativo do autor, que constrói uma teoria estética ao mesmo tempo que a aplica, como observa Irineu Franco Perpetuo na apresentação desta edição.  

\page

\Hedra

\stoptext %---------------------------------------------------------|
			