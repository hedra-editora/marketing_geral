% Preencher com o nome das cor ou composição RGB (ex: [r=0.862, g=0.118, b=0.118]) 
\usecolors[crayola] 			   % Paleta de cores pré-definida: wiki.contextgarden.net/Color#Pre-defined_colors

% Cores definidas pelo designer:
% MyGreen		r=0.251, g=0.678, b=0.290 % 40ad4a
% MyCyan		r=0.188, g=0.749, b=0.741 % 30bfbd
% MyRed			r=0.820, g=0.141, b=0.161 % d12429
% MyPink		r=0.980, g=0.780, b=0.761 % fac7c2
% MyGray		r=0.812, g=0.788, b=0.780 % cfc9c7
% MyOrange		r=0.980, g=0.671, b=0.290 % faab4a

% Configuração de cores
\definecolor[MyColor][MyPink]      % ou ex: [r=0.862, g=0.118, b=0.118] % corresponde a RGB(220, 30, 30)
\definecolor[MyColorText][black]  % ou ex: [r=0.862, g=0.118, b=0.118] % corresponde a RGB(167, 169, 172)

% Classe para diagramação dos posts
\environment{marketing.env}		   


\starttext

\page %---------------------------------------------------------|
\Mensagem{OP.\,CIT.}

\startMyCampaign
\hyphenpenalty=10000
\exhyphenpenalty=10000
O {\bf CURSO} É UM VASO VAZIO\\
O USO NUNCA O REPLENA\blank[1ex]\vfill

 {\bf \Seta LAOZI}  
\stopMyCampaign




\page  %--------------------------------------------------------|

\MyPicture{LAOZI_DAODEJING_3.jpeg}

\page

o {\bf curso} é um vaso vazio\\
o uso nunca o replena\\\blank[1ex]
abismal!\\
parece o progenitor das dez-mil-coisas\\\blank[1ex]
abranda o cume\\
desfaz o emaranhado\\
harmoniza a luz\\
congloba o pó

%\vfill\scale[lines=2]{\MyStar[MyColorText][none]}

\page 

\MyCover{LAOZI_DAODEJING_THUMB.pdf}

\page %---------------------------------------------------------|

{\bf DAO DE JING} é uma obra
clássica da literatura chinesa antiga. Atribuída ao sábio filósofo Laozi, a
obra oferece uma profunda reflexão sobre temas como o {\bf DAO}
e a virtude, mas também pode ser entendida como uma arte da guerra. Em
tradução direta do chinês por Mário Sproviero, a edição é bilíngue e inclui
comentário a cada um dos capítulos. {\bf hedra.com.br/r/dao}

\page

\Hedra

\stoptext