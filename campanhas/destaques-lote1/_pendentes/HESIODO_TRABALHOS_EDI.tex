% Preencher com o nome das cor ou composição RGB (ex: [r=0.862, g=0.118, b=0.118]) 
\usecolors[crayola] 			   % Paleta de cores pré-definida: wiki.contextgarden.net/Color#Pre-defined_colors

% Cores definidas pelo designer:
% MyGreen		r=0.251, g=0.678, b=0.290 % 40ad4a
% MyCyan		r=0.188, g=0.749, b=0.741 % 30bfbd
% MyRed			r=0.820, g=0.141, b=0.161 % d12429
% MyPink		r=0.980, g=0.780, b=0.761 % fac7c2
% MyGray		r=0.812, g=0.788, b=0.780 % cfc9c7
% MyOrange		r=0.980, g=0.671, b=0.290 % faab4a

% Configuração de cores
\definecolor[MyColor][Almond]      % ou ex: [r=0.862, g=0.118, b=0.118] % corresponde a RGB(220, 30, 30)
\definecolor[MyColorText][MountainMeadow]  % ou ex: [r=0.862, g=0.118, b=0.118] % corresponde a RGB(167, 169, 172)

% Classe para diagramação dos posts
\environment{marketing.env}		   


% Comandos & Instruções %%%%%%%%%%%%%%%%%%%%%%%%%%%%%%%%%%%%%%%%%%%%%%%%%%%%%%%%%%%%%%%|

% Cabeço e rodabé: Informações (caso queira trocar alguma coisa)
% 		\def\MensagemSaibaMais{SAIBA MAIS:}
% 		\def\MensagemSite{HEDRA.COM.BR}
% 		\def\MensagemLink{LINK NA BIO}

% Pesos para os títulos:
%		\startMyCampaign...		 \stopMyCampaign
%		\stopMyCampaignSection...   \stopMyCampaignSection

% Aplicação de imagens: 
% 		\MyCover{capa.pdf}  	% Aplicação de capa de livro com sombra
%		\MyPicture{Imagem.png}  % Imagem com aplicação de filtro segundo cor MyColorText
%		\MyPhoto{}			    % Aplicação simples de imagem com tamamho \textwidth

% Aplicação de imagem com legenda:		
% 		\placefigure{Legenda}{\externalfigure[drop2-1.png][width=\textwidth]}

% Cabeço e rodabé: Opções
% 		\Mensagem{AGORA É QUE SÃO ELAS}
% 		\Hashtag{campanha de natal}
% 		\Mensagem{campanha de natal}

% Alteração de várias cores de background:
% \setupbackgrounds[page][background=color,backgroundcolor=MyGray]

% Estrela: 
% \vfill\scale[lines=2]{\MyStar[MyColorText][none]} 					% Estrela pequena  
% \startpositioning 											% Estrela grande
%  \position(-1,-.3){\scale[scale=980]{\MyStar[white][none]}}
% \stoppositioning

% Logos e selos: 				
% \Hedra
% \HedraAyllon	% Não está pronto
% \HedraAcorde	% Não está pronto
% \Ayllon		% Não está pronto
% \Acorde		% Não está pronto

% Atalhos: 						
% 		\Seta  % Seta para baixo

%%%%%%%%%%%%%%%%%%%%%%%%%%%%%%%%%%%%%%%%%%%%%%%%%%%%%%%%%%%%%%%%%%%%%%%%%%%%%%%%%%%%%%%|

\starttext
%\showframe  %Para mostrar somente as linhas.

\Mensagem{DESTAQUES}

\MyCover{HESIODO_TRABALHOS_THUMB.pdf}

\page %---------------------------------------------------------|

\MyPicture{HESIODO_TRABALHOS_4.jpeg}

\vfill\scale[factor=fit]{Tradução do grego de {\bf Christian Werner}}

\page
\hyphenpenalty=10000
\exhyphenpenalty=10000

(...) «doenças para os homens de dia, outras, de noite, / 
espontâneas, vagam, levando males aos homens /
em silêncio, pois tirou-lhes a voz o astucioso Zeus. /
Assim, é impossível escapar da mente de Zeus.»



\page

\MyPicture{HESIODO_TRABALHOS_2.jpeg}

\page

{\bf Trabalhos e dias} é um poema épico de 828 versos
em que são contados alguns dos mitos gregos mais
conhecidos até hoje, como o de Prometeu e o de
Pandora. Este poema é voltado para a condição dos
homens, explicitando suas necessidades e limitações,
com foco na justiça necessária para o florescimento
das comunidades ou das cidades e no trabalho
agrícola baseado nas estações do ano. Com a ajuda
das Musas, o poeta narra em primeira pessoa e se dirige 
a seu irmão Perses, na tentativa de ensinar a ele
verdades divinas a respeito das práticas humanas.

\page

\Hedra

\stoptext


% Trabalhos e dias é o poema grego no qual se
% mencionam a caixa de Pandora — na verdade uma
% ânfora —, as linhagens, raças ou idades do homem
% e uma poética representação das estações do
% ano e das atividades agrícolas associadas a elas.
% Nesse livro, porém, não é de quase-super-homens
% como Aquiles e Odisseu que se fala, mas de
% outros tipos de heróis: o poeta que de tudo sabe;
% o bom rei, que zela pela justiça em sua comu-
% nidade; e o agricultor bem-sucedido que, para
% produzir riqueza por meio de sua propriedade ou
% fazenda, deve não só trabalhar arduamente, mas
% atentar a uma série enorme de regras climáticas,
% morais e religiosas, aquilo que nós chamamos de
% acaso também espreita.
% 
% Além de Trabalhos e dias, somente chegaram
% inteiros até nós os poemas Teogonia e Escudo de
% Héracles, entre aqueles atribuídos na Antiguidade
% ao grego Hesíodo, poeta que teria vivido por volta
% dos séculos VIII e VII a.C., ou seja, mais ou menos
% na mesma época que Homero, tido pelos antigos
% como o autor da Ilíada e da Odisseia.
% 
% É possível enumerar vários aspectos relacio-
% nados à cultura grega do mesmo período — como
% a introdução e a expansão do uso da escrita — que
% fazem muitos pesquisadores duvidar que tenha
% havido um poeta histórico chamado Hesíodo e que
% ele tenha composto por escrito os poemas asso-
% ciados a seu nome. Mas, para entender um poema
% como Trabalhos e dias, o próprio texto ainda é
% nossa principal ferramenta, de sorte que muitas
% das questões a ele pertinentes precisarão con-
% tinuar sem uma resposta categórica, como, por
% exemplo, quando