% Preencher com o nome das cor ou composição RGB (ex: [r=0.862, g=0.118, b=0.118]) 
\usecolors[crayola] 			   % Paleta de cores pré-definida: wiki.contextgarden.net/Color#Pre-defined_colors

% Cores definidas pelo designer:
% MyGreen		r=0.251, g=0.678, b=0.290 % 40ad4a
% MyCyan		r=0.188, g=0.749, b=0.741 % 30bfbd
% MyRed			r=0.820, g=0.141, b=0.161 % d12429
% MyPink		r=0.980, g=0.780, b=0.761 % fac7c2
% MyGray		r=0.812, g=0.788, b=0.780 % cfc9c7
% MyOrange		r=0.980, g=0.671, b=0.290 % faab4a

% Configuração de cores
\definecolor[MyColor][MyRed]      % ou ex: [r=0.862, g=0.118, b=0.118] % corresponde a RGB(220, 30, 30)
\definecolor[MyColorText][Salmon]  % ou ex: [r=0.862, g=0.118, b=0.118] % corresponde a RGB(167, 169, 172)

% Classe para diagramação dos posts
\environment{marketing.env}		   


% Comandos & Instruções %%%%%%%%%%%%%%%%%%%%%%%%%%%%%%%%%%%%%%%%%%%%%%%%%%%%%%%%%%%%%%%|

% Cabeço e rodabé: Informações (caso queira trocar alguma coisa)
% 		\def\MensagemSaibaMais{SAIBA MAIS:}
% 		\def\MensagemSite{HEDRA.COM.BR}
% 		\def\MensagemLink{LINK NA BIO}

% Pesos para os títulos:
%		\startMyCampaign...		 \stopMyCampaign
%		\stopMyCampaignSection...   \stopMyCampaignSection

% Aplicação de imagens: 
% 		\MyCover{capa.pdf}  	% Aplicação de capa de livro com sombra
%		\MyPicture{Imagem.png}  % Imagem com aplicação de filtro segundo cor MyColorText
%		\MyPhoto{}			    % Aplicação simples de imagem com tamamho \textwidth

% Aplicação de imagem com legenda:		
% 		\placefigure{Legenda}{\externalfigure[drop2-1.png][width=\textwidth]}

% Cabeço e rodabé: Opções
% 		\Mensagem{AGORA É QUE SÃO ELAS}
% 		\Hashtag{campanha de natal}
% 		\Mensagem{campanha de natal}

% Alteração de várias cores de background:
% \setupbackgrounds[page][background=color,backgroundcolor=MyGray]

% Estrela: 
% \vfill\scale[lines=2]{\MyStar[MyColorText][none]} 					% Estrela pequena  
% \startpositioning 											% Estrela grande
%  \position(-1,-.3){\scale[scale=980]{\MyStar[white][none]}}
% \stoppositioning

% Logos e selos: 				
% \Hedra
% \HedraAyllon	% Não está pronto
% \HedraAcorde	% Não está pronto
% \Ayllon		% Não está pronto
% \Acorde		% Não está pronto

% Atalhos: 						
% 		\Seta  % Seta para baixo

%%%%%%%%%%%%%%%%%%%%%%%%%%%%%%%%%%%%%%%%%%%%%%%%%%%%%%%%%%%%%%%%%%%%%%%%%%%%%%%%%%%%%%%|

\starttext
%\showframe  %Para mostrar somente as linhas.

\Mensagem{DESTAQUES}

\MyCover{PASSETTI_ECOPOLÍTICA_THUMB.pdf}

\page

\startMyCampaign
{\bf Edson Passetti\\
Acácio Augusto\\
Beatriz S. Carneiro\\
Salete Oliveira\\
Thiago Rodrigues}
\Seta Organização
\stopMyCampaign

\vfill\scale[lines=2]{\MyStar[MyColorText][none]} 

\page %---------------------------------------------------------|

«A racionalidade neoliberal reconfigura
todas as dimensões da vida social em bases
econômicas e o governo das condutas se es-
tende à própria produção da subjetividade do
indivíduo tido como ‘empresário de si mesmo’.

\page

Esse instigante livro também nos apresenta
uma cartografia das liberdades, das lutas e re-
sistências que se abrem com a ação direta, a
antipolítica e as infinitas práticas da liberdade,
de que ele próprio é testemunho.»\blank[1ex]

\hfill ---Margareth Rago

\page

\Hedra

\stoptext



% Ousado e desafiador, esse livro enfrenta a difícil tarefa de di-
% agnosticar o presente, apontando para as rupturas profun-
% das nas formas pelas quais se manifesta o poder, no governo
% da vida e do planeta. Ecopolítica remete à ilimitada expansão
% e à progressiva transformação do controle biopolítico das
% populações para o governo de todo o sistema planetário,
% em nome do pluralismo democrático, da inclusão social, da
% boa governança e da salvação das espécies. Como aqui se
% mostra enfaticamente, a racionalidade neoliberal reconfigu-
% ra todas as dimensões da vida social em bases econômicas
% e o governo das condutas se estende à própria produção
% da subjetividade do indivíduo tido como “empresário de si
% mesmo”. Esse instigante livro também nos apresenta uma
% cartografia das liberdades, das lutas e resistências que se
% abrem com a ação direta, a antipolítica e as infinitas práticas
% da liberdade, de que ele próprio é testemunho. Assim, é tam-
% bém da transformação libertária que tratam essas pesqui-
% sas, das linhas de fuga e das resistências possíveis na so-
% ciedade de controle, que coloca no centro das atenções os
% temas da segurança, dos direitos e das penalizações, am-
% pliando consideravelmente seu domínio. Se as formas de
% controle e sujeição na era planetária se tornam muito mais
% complexas, abrangentes, sofisticadas e violentas, é preciso
% perceber por que emaranhados irrompem fluxos, fissuras e
% devires em direção à construção de outros espaços, hetero-
% topias de si, do mundo e da vida, no aqui e agora.
% 
% Margareth Rago