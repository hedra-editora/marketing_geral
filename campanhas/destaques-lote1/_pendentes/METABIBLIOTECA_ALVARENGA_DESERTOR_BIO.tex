% Preencher com o nome das cor ou composição RGB (ex: [r=0.862, g=0.118, b=0.118]) 
\usecolors[crayola] 			   % Paleta de cores pré-definida: wiki.contextgarden.net/Color#Pre-defined_colors

% Cores definidas pelo designer:
% MyGreen		r=0.251, g=0.678, b=0.290 % 40ad4a
% MyCyan		r=0.188, g=0.749, b=0.741 % 30bfbd
% MyRed			r=0.820, g=0.141, b=0.161 % d12429
% MyPink		r=0.980, g=0.780, b=0.761 % fac7c2
% MyGray		r=0.812, g=0.788, b=0.780 % cfc9c7
% MyOrange		r=0.980, g=0.671, b=0.290 % faab4a

% Configuração de cores
\definecolor[MyColor][x=54c7d8]      % ou ex: [r=0.862, g=0.118, b=0.118] % corresponde a RGB(220, 30, 30)
\definecolor[MyColorText][white]  % ou ex: [r=0.862, g=0.118, b=0.118] % corresponde a RGB(167, 169, 172)

% Classe para diagramação dos posts
\environment{marketing.env}		   


% Comandos & Instruções %%%%%%%%%%%%%%%%%%%%%%%%%%%%%%%%%%%%%%%%%%%%%%%%%%%%%%%%%%%%%%%|

% Cabeço e rodabé: Informações (caso queira trocar alguma coisa)
% 		\def\MensagemSaibaMais{SAIBA MAIS:}
% 		\def\MensagemSite{HEDRA.COM.BR}
% 		\def\MensagemLink{LINK NA BIO}

% Pesos para os títulos:
%		\startMyCampaign...		 \stopMyCampaign
%		\stopMyCampaignSection...   \stopMyCampaignSection

% Aplicação de imagens: 
% 		\MyCover{capa.pdf}  	% Aplicação de capa de livro com sombra
%		\MyPicture{Imagem.png}  % Imagem com aplicação de filtro segundo cor MyColorText
%		\MyPhoto{}			    % Aplicação simples de imagem com tamamho \textwidth

% Aplicação de imagem com legenda:		
% 		\placefigure{Legenda}{\externalfigure[drop2-1.png][width=\textwidth]}

% Cabeço e rodabé: Opções
% 		\Mensagem{AGORA É QUE SÃO ELAS}
% 		\Hashtag{campanha de natal}
% 		\Mensagem{campanha de natal}

% Alteração de várias cores de background:
% \setupbackgrounds[page][background=color,backgroundcolor=MyGray]

% Estrela: 
% \vfill\scale[lines=2]{\MyStar[MyColorText][none]} 					% Estrela pequena  
% \startpositioning 											% Estrela grande
%  \position(-1,-.3){\scale[scale=980]{\MyStar[white][none]}}
% \stoppositioning

% Logos e selos: 				
% \Hedra
% \HedraAyllon	% Não está pronto
% \HedraAcorde	% Não está pronto
% \Ayllon		% Não está pronto
% \Acorde		% Não está pronto

% Atalhos: 						
% 		\Seta  % Seta para baixo

%%%%%%%%%%%%%%%%%%%%%%%%%%%%%%%%%%%%%%%%%%%%%%%%%%%%%%%%%%%%%%%%%%%%%%%%%%%%%%%%%%%%%%%|

\starttext
%\showframe  %Para mostrar somente as linhas.

\Mensagem{DESTAQUES}

\MyCover{METABIBLIOTECA_ALVARENGA_DESERTOR_THUMB.pdf}

\page %---------------------------------------------------------|




\MyCover{METABIBLIOTECA_ALVARENGA_DESERTOR_1.jpeg}

\page 

\MyPicture{METABIBLIOTECA_ALVARENGA_DESERTOR_2.jpeg}

\page 


\MyPicture{METABIBLIOTECA_ALVARENGA_DESERTOR_3}

\page 


\page 

\hyphenpenalty=10000
\exhyphenpenalty=10000

Mussum Ipsum, cacilds vidis litro abertis.  Suco de cevadiss, é um leite
divinis, qui tem lupuliz, matis, aguis e fermentis. Cevadis im ampola
pa arma uma pindureta. Mauris nec dolor in eros commodo tempor. Aenean
aliquam molestie leo, vitae iaculis nisl. Paisis, filhis, espiritis
santis. Nullam volutpat risus nec leo commodo, ut interdum diam
laoreet. Sed non consequat odio. 

\page

\MyCover{METABIBLIOTECA_ALVARENGA_DESERTOR_1}

\page

\Hedra

\stoptext


% O desertor (1774), de Silva Alvarenga,
% é um poema heroi-cômico que narra as
% desventuras de Gonçalo, que desiste dos
% estudos e retorna a sua cidade natal,
% Mioselha, onde será recebido de forma
% desonrosa por não trazer na bagagem o
% que fora buscar, um status de bacharel.
% Gonçalo forma um grupo de desertores
% da vida acadêmica que, guiados pelo
% professor Tibúrcio, personificação da
% Ignorância, trilham juntos o caminho para
% a obscuridade da vida da aldeia, onde
% poderão empanturrar-se de queijo e
% tremoços. Neste volume, além do poema
% de Alvarenga com ortografia e pontuação
% atualizadas, o leitor encontrará uma
% introdução crítica, um glossário de
% termos específicos e um resumo da ação
% de cada um dos cinco cantos do poema,
% constituindo assim uma edição cujos
% aparatos de leitura garantem a fruição e
% compreensão do texto.
% 
% Organizadores
% Clara Carolina Santos
% Ricardo Martins Valle

% Brasileiro de nascimento, Manuel Inácio da Silva Alvaren-
% ga era filho de família relativamente humilde, e mesmo
% assim seguiu um caminho associado às elites coloniais:
% foi-se a Portugal estudar na Universidade de Coimbra.
% Sua maior herança está, pois, intimamente associada à
% cultura lusitana: este O desertor é uma das maiores apolo-
% gias literárias do pombalismo e do Portugal moderno.
% Trata-se de um épico-satírico, ou antes de um antiépi-
% co, cuja graça funda-se na representação de distinções
% ajustadas entre melhores e piores, narrando como se
% fosse grande coisa, e com palavras infladas, as bra-
% vatas irrisórias e os ânimos mesquinhos das personagens
% da trama — pela dissociação entre o baixo da invenção da
% matéria e o alto da elocução ornada com palavras graves
% dignas de grandes feitos. O vitupério se justifica como
% eficácia didática que exorta à virtude, uma vez que re-
% presenta as coisas dignas de vergonha com as mesmas
% palavras e proporcionais figuras com que se louvam os
% grandes feitos.
% 
% O tema heroico da sátira de O desertor é a ação do
% Marquês de Pombal sobre o ensino na Universidade, mas
% o relato recai sobre as ações vis de Gonçalo, o jovem
% estudante que desertou da carreira das letras para inglo-
% riamente retornar à província de onde saíra, agora mais
% obscuro do que antes. O jovem Gonçalo derroca na vida
% desde quando pela primeira vez não acorda para as aulas,
% e se precipita irrefreavelmente ao decidir deixar a Univer-
% sidade que se reformava.
% 
% Gonçalo retorna ao campo, mas não para um remanso
% da virtude, para um remédio dos vícios, como se lê nas
% tópicas árcades do século XVIII (e não apenas). A busca
% pela aldeia é antes uma fuga para o obscuro, que, por
% oposição às luzes de Pombal, remete sobretudo à Com-
% panhia de Jesus, sua grande adversária: eis o tabuleiro
% político do tempo de Silva Alvarenga.

% Primeira edição de O desertor, publicada em 1774. O poema, escrito no gênero
% herói-cômico, foi composto por Silva Alvarenga enquanto estudante da Faculdade
% de Cânones da Universidade de Coimbra. Gonçalo, aluno na instituição de ensino
% coimbrã, é a personagem principal do poema. Acompanhado de alguns amigos,
% Gonçalo abandona os bancos da Universidade, em virtude da dureza dos desafios
% que a vida estudantil apresenta ao estudante da recém-reformada instituição. O
% poema gira em torno da jornada de Gonçalo e seus amigos em direção a Miosélia,
% onde reside o tio de Gonçalo. O desertor teria marcado a entrada de Silva
% Alvarenga no círculo de poetas luso-americanos protegidos pelo marquês de
% Pombal, valido de D. José I e importante articulador da Reforma da
% Universidade de Coimbra, instituída em 1772.