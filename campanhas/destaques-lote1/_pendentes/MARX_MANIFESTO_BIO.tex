% Preencher com o nome das cor ou composição RGB (ex: [r=0.862, g=0.118, b=0.118]) 
\usecolors[crayola] 			   % Paleta de cores pré-definida: wiki.contextgarden.net/Color#Pre-defined_colors

% Cores definidas pelo designer:
% MyGreen		r=0.251, g=0.678, b=0.290 % 40ad4a
% MyCyan		r=0.188, g=0.749, b=0.741 % 30bfbd
% MyRed			r=0.820, g=0.141, b=0.161 % d12429
% MyPink		r=0.980, g=0.780, b=0.761 % fac7c2
% MyGray		r=0.812, g=0.788, b=0.780 % cfc9c7
% MyOrange		r=0.980, g=0.671, b=0.290 % faab4a

% Configuração de cores
\definecolor[MyColor][MyRed]      % ou ex: [r=0.862, g=0.118, b=0.118] % corresponde a RGB(220, 30, 30)
\definecolor[MyColorText][black]  % ou ex: [r=0.862, g=0.118, b=0.118] % corresponde a RGB(167, 169, 172)

% Classe para diagramação dos posts
\environment{marketing.env}		   


% Comandos & Instruções %%%%%%%%%%%%%%%%%%%%%%%%%%%%%%%%%%%%%%%%%%%%%%%%%%%%%%%%%%%%%%%|

% Cabeço e rodabé: Informações (caso queira trocar alguma coisa)
% 		\def\MensagemSaibaMais{SAIBA MAIS:}
% 		\def\MensagemSite{HEDRA.COM.BR}
% 		\def\MensagemLink{LINK NA BIO}

% Pesos para os títulos:
%		\startMyCampaign...		 \stopMyCampaign
%		\stopMyCampaignSection...   \stopMyCampaignSection

% Aplicação de imagens: 
% 		\MyCover{capa.pdf}  	% Aplicação de capa de livro com sombra
%		\MyPicture{Imagem.png}  % Imagem com aplicação de filtro segundo cor MyColorText
%		\MyPhoto{}			    % Aplicação simples de imagem com tamamho \textwidth

% Aplicação de imagem com legenda:		
% 		\placefigure{Legenda}{\externalfigure[drop2-1.png][width=\textwidth]}

% Cabeço e rodabé: Opções
% 		\Mensagem{AGORA É QUE SÃO ELAS}
% 		\Hashtag{campanha de natal}
% 		\Mensagem{campanha de natal}

% Alteração de várias cores de background:
% \setupbackgrounds[page][background=color,backgroundcolor=MyGray]

% Estrela: 
% \vfill\scale[lines=2]{\MyStar[MyColorText][none]} 					% Estrela pequena  
% \startpositioning 											% Estrela grande
%  \position(-1,-.3){\scale[scale=980]{\MyStar[white][none]}}
% \stoppositioning

% Logos e selos: 				
% \Hedra
% \HedraAyllon	% Não está pronto
% \HedraAcorde	% Não está pronto
% \Ayllon		% Não está pronto
% \Acorde		% Não está pronto

% Atalhos: 						
% 		\Seta  % Seta para baixo

%%%%%%%%%%%%%%%%%%%%%%%%%%%%%%%%%%%%%%%%%%%%%%%%%%%%%%%%%%%%%%%%%%%%%%%%%%%%%%%%%%%%%%%|

\starttext
%\showframe  %Para mostrar somente as linhas.

\Mensagem{DESTAQUES}

\MyCover{MARX_MANIFESTO_THUMB.pdf}

\page %---------------------------------------------------------|

\MyPicture{MARX_MANIFESTO_1.jpeg}

\vfill\scale[factor=fit]{\tfxx Tradução do alemão de {\bf Marcus Mazzari}.}


\page 

\hyphenpenalty=10000
\exhyphenpenalty=10000

«Que tremam as classes dominantes em face de uma
revolução comunista. Nela, os proletários nada têm a
perder senão as suas cadeias. Eles têm um mundo a
ganhar. Proletários de todos os países, uni-vos!»\blank[1ex]

\hfill ---Karl Marx

\page


\MyPhoto{MARX_MANIFESTO_3.jpeg}

\page 

Encomendado pela Liga dos Comunistas, com
o objetivo de tornar público ao trabalhador
um programa político claro voltado para a
dissolução da ordem existente, o {\bf Manifesto
comunista} convoca a classe trabalhadora
mundial a se unir contra o modo de produção
capitalista e suas estruturas, com vistas a
estabelecer uma forma de sociedade organi-
zada pelos produtores associados, na qual o
livre desenvolvimento de cada um é a condição
para o livre desenvolvimento de todos.
Sabemos que o Manifesto foi redigido por
Marx e Engels às vésperas das revoluções de
1848 e apresenta uma pioneira descrição do
mundo moderno, a qual, na contramão dos
prognósticos mais otimistas da época — que
anteviam uma ampliação progressiva da liberdade 
até a realização efetiva da paz perpétua —,
compreende a atual forma de sociedade como
estruturada por conflitos sociais insuperáveis.

\page 

\Hedra

\stoptext


% O leitor desavisado, porém, pode se sur-
% preender com a exaltação dos autores às
% maravilhas do capitalismo, que, segundo a
% letra do Manifesto, desempenhou na história
% um papel extremamente revolucionário, ala-
% vancou um desenvolvimento sem precedentes
% das forças produtivas humanas e, sobretudo,
% deu provas daquilo que a atividade dos homens
% é capaz de levar a cabo.

% Longe de apresentar uma rejeição uni-
% lateral e simplista da atual ordem existente, o
% Manifesto pretende libertar as possibilidades
% abertas pela modernidade de seu sufoca-
% mento pela estreiteza da forma burguesa,
% cuja pretensão de restringir as forças produ-
% tivas gestadas em seu seio aos parâmetros
% capitalistas se assemelha ao feiticeiro que já
% não consegue mais dominar os poderes subter-
% râneos que invocou.
% Se, de um lado, o capitalismo alavancou
% de modo inédito a capacidade humana para a
% produção de riqueza social, de outro, produz
% a miséria do trabalhador numa velocidade
% ainda maior, criando as condições para sua
% própria derrocada.


% Manifesto comunista, publicado originalmente como Manifesto do Par-
% tido Comunista, foi encomendado pela Liga dos Comunistas e publicado
% em 21 de fevereiro de 1848. É um dos textos mais influentes do mundo,
% expondo o programa da Liga, e contando com uma análise da luta de
% classes, tanto a partir de uma perspectiva histórica, quanto contem-
% porânea, e que trata do período em que se estabelecia o capitalismo e,
% consequentemente, a burguesia como classe dominante e sua luta perma-
% nente com o proletariado, na Europa do século xix. O Manifesto, ainda
% que tenha incorporado elementos de outros pensadores, constituiu as
% bases da teoria sobre as classes sociais no capitalismo e a luta de classes,
% fundamentando os princípios do marxismo. Ainda que a autoria do
% Manifesto seja historicamente atribuída a Marx e Engels, este último foi
% responsável somente pela elaboração de seus primeiros rascunhos e a
% redação foi realizada por Marx.
% 
% Marcus Vinicius Mazzari é professor de Teoria Literária na Universi-
% dade de São Paulo (usp). Traduziu para o português textos de Walter
% Benjamin, Adelbert von Chamisso, Gottfried Keller, Jeremias Gotthelf
% e outros. Tem diversas publicações no Brasil e na Alemanha, entre as
% quais Labirintos da aprendizagem (Editora 34, 2010), A dupla noite das
% tílias. História e natureza no Fausto de Goethe (Editora 34, 2019), Versu-
% chung und Widerstand in A Máquina do Mundo [Die Weltmaschine] von
% C. D. de Andrade (Metzler, 2022). Contemplado com Goldene Goethe-
% Medaille [Medalha de Ouro Goethe] de 2023, pela Goethe-Gesellschaft
% de Weimar.