% Preencher com o nome das cor ou composição RGB (ex: [r=0.862, g=0.118, b=0.118]) 
\usecolors[crayola] 			   % Paleta de cores pré-definida: wiki.contextgarden.net/Color#Pre-defined_colors

% Cores definidas pelo designer:
% MyGreen		r=0.251, g=0.678, b=0.290 % 40ad4a
% MyCyan		r=0.188, g=0.749, b=0.741 % 30bfbd
% MyRed			r=0.820, g=0.141, b=0.161 % d12429
% MyPink		r=0.980, g=0.780, b=0.761 % fac7c2
% MyGray		r=0.812, g=0.788, b=0.780 % cfc9c7
% MyOrange		r=0.980, g=0.671, b=0.290 % faab4a

% Configuração de cores
\definecolor[MyColor][MyPink]      % ou ex: [r=0.862, g=0.118, b=0.118] % corresponde a RGB(220, 30, 30)
\definecolor[MyColorText][black]  % ou ex: [r=0.862, g=0.118, b=0.118] % corresponde a RGB(167, 169, 172)

% Classe para diagramação dos posts
\environment{marketing.env}		   


\starttext

\Mensagem{Vida \& Obra}


\page

\MyPicture{LAOZI_DAODEJING_3.jpeg}


\page
\hyphenpenalty=10000
\exhyphenpenalty=10000


{\bf LAOZI}, sábio filósofo chinês era também conhecido como Lao Tzu, Laotze e
Lao Tsé. Primeiro e mais importante autor do taoísmo chinês, seu verdadeiro
nome era, segundo a tradição, Li Er. Quase nada se sabe sobre sua vida, e
muitos acreditam que ele não tenha existido. Contam que teria se encontrado
com Confúcio, ou “Mestre Kong”, que teria ficado impressionado com os
ensinamentos. {\bf DAO DE JING} teria sido escrito por Laozi a pedido de um
guarda de fronteira, que o instou a registrar seus ensinamentos como condição
para partir, quando ele deixava o reino de Zhou, então em decadência.


\page

\MyCover{LAOZI_DAODEJING_THUMB.pdf}

\page %---------------------------------------------------------|

{\bf DAO DE JING} é uma obra
clássica da literatura chinesa antiga. Atribuída ao sábio filósofo Laozi, a
obra oferece uma profunda reflexão sobre temas como o {\bf DAO}
e a virtude, mas também pode ser entendida como uma arte da guerra. Em
tradução direta do chinês por Mário Sproviero, a edição é bilíngue e inclui
comentário a cada um dos capítulos. {\bf hedra.com.br/r/dao}

\page

\Hedra

\stoptext