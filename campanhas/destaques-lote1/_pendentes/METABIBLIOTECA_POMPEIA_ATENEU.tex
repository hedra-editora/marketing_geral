% Preencher com o nome das cor ou composição RGB (ex: [r=0.862, g=0.118, b=0.118]) 
\usecolors[crayola] 			   % Paleta de cores pré-definida: wiki.contextgarden.net/Color#Pre-defined_colors

% Cores definidas pelo designer:
% MyGreen		r=0.251, g=0.678, b=0.290 % 40ad4a
% MyCyan		r=0.188, g=0.749, b=0.741 % 30bfbd
% MyRed			r=0.820, g=0.141, b=0.161 % d12429
% MyPink		r=0.980, g=0.780, b=0.761 % fac7c2
% MyGray		r=0.812, g=0.788, b=0.780 % cfc9c7
% MyOrange		r=0.980, g=0.671, b=0.290 % faab4a

% Configuração de cores
\definecolor[MyColor][Almond]      % ou ex: [r=0.862, g=0.118, b=0.118] % corresponde a RGB(220, 30, 30)
\definecolor[MyColorText][Gray]  % ou ex: [r=0.862, g=0.118, b=0.118] % corresponde a RGB(167, 169, 172)

% Classe para diagramação dos posts
\environment{marketing.env}		   


% Comandos & Instruções %%%%%%%%%%%%%%%%%%%%%%%%%%%%%%%%%%%%%%%%%%%%%%%%%%%%%%%%%%%%%%%|

% Cabeço e rodabé: Informações (caso queira trocar alguma coisa)
% 		\def\MensagemSaibaMais{SAIBA MAIS:}
% 		\def\MensagemSite{HEDRA.COM.BR}
% 		\def\MensagemLink{LINK NA BIO}

% Pesos para os títulos:
%		\startMyCampaign...		 \stopMyCampaign
%		\stopMyCampaignSection...   \stopMyCampaignSection

% Aplicação de imagens: 
% 		\MyCover{capa.pdf}  	% Aplicação de capa de livro com sombra
%		\MyPicture{Imagem.png}  % Imagem com aplicação de filtro segundo cor MyColorText
%		\MyPhoto{}			    % Aplicação simples de imagem com tamamho \textwidth

% Aplicação de imagem com legenda:		
% 		\placefigure{Legenda}{\externalfigure[drop2-1.png][width=\textwidth]}

% Cabeço e rodabé: Opções
% 		\Mensagem{AGORA É QUE SÃO ELAS}
% 		\Hashtag{campanha de natal}
% 		\Mensagem{campanha de natal}

% Alteração de várias cores de background:
% \setupbackgrounds[page][background=color,backgroundcolor=MyGray]

% Estrela: 
% \vfill\scale[lines=2]{\MyStar[MyColorText][none]} 					% Estrela pequena  
% \startpositioning 											% Estrela grande
%  \position(-1,-.3){\scale[scale=980]{\MyStar[white][none]}}
% \stoppositioning

% Logos e selos: 				
% \Hedra
% \HedraAyllon	% Não está pronto
% \HedraAcorde	% Não está pronto
% \Ayllon		% Não está pronto
% \Acorde		% Não está pronto

% Atalhos: 						
% 		\Seta  % Seta para baixo

%%%%%%%%%%%%%%%%%%%%%%%%%%%%%%%%%%%%%%%%%%%%%%%%%%%%%%%%%%%%%%%%%%%%%%%%%%%%%%%%%%%%%%%|

\starttext
%\showframe  %Para mostrar somente as linhas.

\Mensagem{DESTAQUES}

\MyCover{METABIBLIOTECA_POMPEIA_ATENEU_THUMB.pdf}

\page %---------------------------------------------------------|

\MyPicture{METABIBLIOTECA_POMPEIA_ATENEU_1}

\vfill\scale[factor=fit]{Introdução e organização de {\bf Caio Gagliardi}.}

\page 
\hyphenpenalty=10000
\exhyphenpenalty=10000


«Eu tinha onze anos. Frequentara como
externo, durante alguns meses, uma
escola familiar do Caminho Novo, onde
algumas senhoras inglesas, sob a direção
do pai, distribuíam educação à infância
como melhor lhes parecia.»

\page 

«Entrava às
nove horas, timidamente, ignorando
as lições com a maior regularidade, e
bocejava até as duas, torcendo-me de insipidez sobre os carcomidos
bancos que o colégio comprara, de
pinho e usados, lustrosos do contato
da malandragem de não sei quantas
gerações de pequenos.» 


% Ao meio-dia, davam-nos pão com manteiga.
% Esta recordação gulosa é o que mais
% pronunciadamente me ficou dos meses
% de externato; com a lembrança de alguns
% companheiro


\page 

\MyPicture{METABIBLIOTECA_POMPEIA_ATENEU_2}


\page

{\it Considerado nosso maior romancista impressionista
depois de Machado de Assis, Raul Pompeia tinha apenas 25 anos
quando publicou um dos mais significativos livros da literatura 
brasileira, {\bf O Ateneu}, em 1888. 


\page
Mistura de realismo,
impressionismo e romance psicológico, o livro percorre a
trajetória do narrador-personagem Sérgio durante os dois
anos em que viveu no colégio interno Ateneu.}

\page

\Hedra

\stoptext



% A narrativa é marcada pelas angústias diante do
% ambiente hostil da escola, por suas descrições satíricas
% e devaneios de alta voltagem lírica e de crítica social. O
% ensino, encarado aqui como didática do controle e da
% uniformização da diferença, explicitamente de inspiração
% militar, é também símbolo da sociedade brasileira do 2º
% Reinado. Uma sociedade que, como o Ateneu da história,
% é corrompida e corruptora, ambiente em que se passa de
% oprimido a opressor.
% Principal romance de formação da literatura brasileira,
% O Ateneu é profundamente moderno pelo seu regime de
% constante transição de tons: do impressionista, de pene-
% tração psicológica, ao expressionista, de alcance social.
% No chão social sobre o qual se vive no Ateneu, Sérgio
% nunca está em repouso, mas em constante processo de
% adaptação. Por isso, o realismo atmosférico do romance
% toca incessantemente a esfera mais íntima do psicológico.


% Raul dÁvila Pompeia (Jacuecanga, Angra dos Reis, 1863–Rio de Ja-
% neiro, 1895), polemista radical, abolicionista e republicano, foi um dos
% maiores escritores brasileiros do século xix, e também um dos mais
% singulares. Formou-se em direito em Recife, depois de ter se trans-
% ferido do Largo São Francisco, cujo corpo docente era basicamente
% composto por escravocratas e monarquistas ultramontanos. Além
% de O Ateneu, sua obra maior, legou-nos os romances Uma tragédia
% no Amazonas (1880), As joias da coroa (1882), os contos Microscópicos
% (1881), os poemas em prosa Canções sem metro (1900), as páginas de
% crônicas e reflexões recolhidas em Alma morta (1888) e Prosas espar-
% sas de Raul Pompeia (1920–21). Além de ativista político, romancista
% e cronista, foi professor da Escola Nacional de Belas-Artes, diretor da
% Biblioteca Nacional, desenhista e pintor. Os desenhos deste volume
% são de autoria do próprio autor. Raul Pompeia suicidou-se no Rio de
% Janeiro, no quarto de sua casa, na noite de Natal de 1895.
% 
% O Ateneu foi publicado em capítulos, no jornal carioca A gazeta de
% notícias, entre 8 de abril e 18 de maio de 1888, e, devido ao reconhe-
% cimento imediato, foi editado em livro no mesmo ano. Escrito em
% apenas três meses, é considerado o maior romance brasileiro do sé-
% culo xix depois dos romances realistas de Machado de Assis. Seu
% enredo consiste na recordação do período de dois anos em que o
% narrador, Sérgio, passa num tradicional colégio interno do Rio de
% Janeiro. O ingresso no Ateneu marca as descobertas amargas que
% acompanharão o narrador daí em diante, os sentimentos de desilusão,
% opressão e desconfiança, componentes da profunda solidão humana.
% Seu sentido é o de um ritual de passagem, em que o convívio com os
% colegas, os professores e o diretor definem a afirmação moral, sexual
% e intelectual de um menino de 11 anos. Difíceis de definir, o estilo e
% o significado do romance geraram uma das mais profícuas polêmi-
% cas da história da nossa literatura, aqui apresentada e antologizada
% cronologicamente no final do volume.
% Caio Gagliardi é professor da Universidade de São Paulo na área de
% Literatura Portuguesa, onde coordena o grupo de pesquisas Estudos
% Pessoanos. É autor de O renascimento do autor: autoria, heteronímia
% e fake memoirs (Hedra, 2019) e organizador de Fernando Pessoa & Cia.
% não heterônima (Mundaréu, 2019), entre outras publicações.