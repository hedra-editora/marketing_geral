% AUTOR_LIVRO_AUTOR.tex
% Preencher com o nome das cor ou composição RGB (ex: [r=0.862, g=0.118, b=0.118]) 
\usecolors[crayola]                % Paleta de cores pré-definida: wiki.contextgarden.net/Color#Pre-defined_colors

% Cores definidas pelo designer:
% MyGreen       r=0.251, g=0.678, b=0.290 % 40ad4a
% MyCyan        r=0.188, g=0.749, b=0.741 % 30bfbd
% MyRed         r=0.820, g=0.141, b=0.161 % d12429
% MyPink        r=0.980, g=0.780, b=0.761 % fac7c2
% MyGray        r=0.812, g=0.788, b=0.780 % cfc9c7
% MyOrange      r=0.980, g=0.671, b=0.290 % faab4a

% Configuração de cores
\definecolor[MyColor][OrchidPearl]      % ou ex: [r=0.862, g=0.118, b=0.118] % corresponde a RGB(220, 30, 30)
\definecolor[MyColorText][PigPink]  % ou ex: [r=0.862, g=0.118, b=0.118] % corresponde a RGB(167, 169, 172)

% Classe para diagramação dos posts
\environment{marketing.env}        

% Cabeço e rodapé: Informações (caso queira trocar alguma coisa)
        \def\MensagemSaibaMais  {SAIBA MAIS:}
        \def\MensagemSite           {HEDRA.COM.BR}
        \def\MensagemLink           {LINK NA BIO}

\starttext  %---------------------------------------------------------|

\def\MyBackground#1{
\defineoverlay
  [backgroundimage]
  [{\externalfigure[#1][height=\overlayheight]}]
}


\Mensagem{POETA FINGIDOR}

% Foto para background
\MyBackground{METABIBLIOTECA_PESSOA_EXTASE_3}

\startstandardmakeup[background=backgroundimage]
\startMyCampaign
\vfill\scale[factor=4]{\Seta\,FERNANDO PESSOA (1888--1935)}
\stopMyCampaign
\stopstandardmakeup

\page 
\Mensagem{POETA FINGIDOR}


\hyphenpenalty=10000
\exhyphenpenalty=10000

Nascido em Lisboa, {\bf FERNANDO PESSOA} é um dos maiores poetas da língua portuguesa e uma figura singular na literatura mundial. Sua vida {\bf ENIGMÁTICA} e sua genialidade são refletidas em uma {\bf OBRA VASTA E DIVERSIFICADA}.

\page

\hyphenpenalty=10000
\exhyphenpenalty=10000

Pessoa é conhecido por criar {\bf HETERÔNIMOS}, personas literárias com estilos e {\bf PERSPECTIVAS ÚNICAS}. Álvaro de Campos, Ricardo Reis e Alberto Caeiro são alguns dos mais famosos. 

 \page

\MyPicture{METABIBLIOTECA_PESSOA_EXTASE_1}

\hyphenpenalty=10000
\exhyphenpenalty=10000

Sua poesia aborda temas como amor e fragmentação e metafísica, com {\bf PRECISÃO NAS PALAVRAS} e intensidade emocional.

\page

%Durante sua vida publicou em livro apenas {\bf MENSAGEM} (1934). 

\MyCover{METABIBLIOTECA_PESSOA_EXTASE_THUMB}

\page %----------------------------------------------------------|

\Hedra

\stoptext %---------------------------------------------------------|


  % Em 1905, retorna definitivamente para sua cidade natal e ingressa na Faculdade de Letras da Universidade de Lisboa. Começa a publicar textos de crítica na revista \textit{A águia}, em 1912, e a colaborar em jornais e revistas, sendo a principal delas a  \textit{Orpheu}.

