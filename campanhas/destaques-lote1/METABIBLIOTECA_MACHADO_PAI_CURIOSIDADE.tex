% AUTOR_LIVRO_CURIOSIDADES.tex
% Preencher com o nome das cor ou composição RGB (ex: [r=0.862, g=0.118, b=0.118]) 
\usecolors[crayola] 			   % Paleta de cores pré-definida: wiki.contextgarden.net/Color#Pre-defined_colors

% Cores definidas pelo designer:
% MyGreen		r=0.251, g=0.678, b=0.290 % 40ad4a
% MyCyan		r=0.188, g=0.749, b=0.741 % 30bfbd
% MyRed			r=0.820, g=0.141, b=0.161 % d12429
% MyPink		r=0.980, g=0.780, b=0.761 % fac7c2
% MyGray		r=0.812, g=0.788, b=0.780 % cfc9c7
% MyOrange		r=0.980, g=0.671, b=0.290 % faab4a

% Configuração de cores
\definecolor[MyColor][Eggplant]      % ou ex: [r=0.862, g=0.118, b=0.118] % corresponde a RGB(220, 30, 30)
\definecolor[MyColorText][Canary]  % ou ex: [r=0.862, g=0.118, b=0.118] % corresponde a RGB(167, 169, 172)

% Classe para diagramação dos posts
\environment{marketing.env}		   

\starttext %---------------------------------------------------------|

\hyphenpenalty=10000
\exhyphenpenalty=10000

\Mensagem{EM CONTEXTO} %Sempre usar esse header

\startMyCampaign

\hyphenpenalty=10000
\exhyphenpenalty=10000

A {\bf CEGUEIRA TEMPORÁRIA} de
Machado de Assis %Aqui a manchete pode ser mais longa

\stopMyCampaign

\page %---------------------------------------------------------| 

\hyphenpenalty=10000
\exhyphenpenalty=10000

Após a publicação de {\bf IAIÁ GARCIA} em 1878, {\bf MACHADO DE ASSIS} enfrentou uma crise de saúde, que culminou em um quadro de cegueira temporária.

\page

Essa experiência, associada a uma pungente autoconsciência em relação aos limites das formas narrativas que praticou até então, motivou a elaboração do seu mais célebre romance, {\bf MEMÓRIAS PÓSTUMAS DE BRÁS CUBAS}.

\page

Impossibilitado de ver, {\bf MACHADO} teria ditado a sua esposa os primeiros movimentos do revolucionário romance. 

\page %---------------------------------------------------------|

\MyCover{METABIBLIOTECA_MACHADO_PAI_THUMB}

\page %---------------------------------------------------------|

\Hedra

\stoptext %---------------------------------------------------------|