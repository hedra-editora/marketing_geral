% METABIBLIOTECA_PESSOA_EXTASE_EFEMERIDE.tex
% Preencher com o nome das cor ou composição RGB (ex: [r=0.862, g=0.118, b=0.118]) 
\usecolors[crayola] 			   % Paleta de cores pré-definida: wiki.contextgarden.net/Color#Pre-defined_colors

% Cores definidas pelo designer:
% MyGreen		r=0.251, g=0.678, b=0.290 % 40ad4a
% MyCyan		r=0.188, g=0.749, b=0.741 % 30bfbd
% MyRed			r=0.820, g=0.141, b=0.161 % d12429
% MyPink		r=0.980, g=0.780, b=0.761 % fac7c2
% MyGray		r=0.812, g=0.788, b=0.780 % cfc9c7
% MyOrange		r=0.980, g=0.671, b=0.290 % faab4a

% Configuração de cores
\definecolor[MyColor][MyCyan]      % ou ex: [r=0.862, g=0.118, b=0.118] % corresponde a RGB(220, 30, 30)
\definecolor[MyColorText][black]  % ou ex: [r=0.862, g=0.118, b=0.118] % corresponde a RGB(167, 169, 172)

% Classe para diagramação dos posts
\environment{marketing.env}		   

\starttext %--------------------------------------------------------|

\hyphenpenalty=10000
\exhyphenpenalty=10000

\Mensagem{136 ANOS DE FERNANDO PESSOA} %Sempre usar esse header

\placefigure{}{\externalfigure[METABIBLIOTECA_PESSOA_EXTASE_3][height=\textwidth, factor=max]}


\vfill\scale[factor=6]{\Seta\,PRIMEIRO {\bf DRAMATURGO} DEPOIS POETA}

\page %---------------------------------------------------------| 

\hyphenpenalty=10000
\exhyphenpenalty=10000

 Poucos sabem, mas Fernando Pessoa, um dos maiores escritores da língua portuguesa e uma figura singular na literatura mundial, não comecou sua atividade literária pela poesia, mas pelo {\bf TEATRO}.

\page %---------------------------------------------------------|

Sua primeira peça, intitulada {\bf O MARINHEIRO}, publicada em 1915, marcou a sua estreia no mundo literário, revelando seu talento multifacetado. 

\page

O drama e a poesia de Pessoa tem algo em comum: {\bf A MULTIPLICIDADE DE VOZES POÉTICAS.} 
Além disso, a relação de Pessoa com o teatro está intrinsecamente ligada aos seus {\bf HETERÔNIMOS}: personagens literárias distintas, com biografias e perspectivas únicas, que o escritor concebeu.


\MyPicture{METABIBLIOTECA_PESSOA_EXTASE_1}

\hyphenpenalty=10000
\exhyphenpenalty=10000
 
 Entre os heterônimos mais conhecidos estão {\bf ALBERTO CAEIRO}, {\bf ÁLVARO DE CAMPOS} e {\bf RICARDO REIS}. 

\page


\MyCover{METABIBLIOTECA_PESSOA_EXTASE_THUMB}

\page %---------------------------------------------------------|

\Hedra

\stoptext %---------------------------------------------------------|


 
