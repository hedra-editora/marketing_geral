% FREUD_LITERATURA_EDICAO.tex
% Preencher com o nome das cor ou composição RGB (ex: [r=0.862, g=0.118, b=0.118]) 
\usecolors[crayola] 			   % Paleta de cores pré-definida: wiki.contextgarden.net/Color#Pre-defined_colors

% Cores definidas pelo designer:
% MyGreen		r=0.251, g=0.678, b=0.290 % 40ad4a
% MyCyan		r=0.188, g=0.749, b=0.741 % 30bfbd
% MyRed			r=0.820, g=0.141, b=0.161 % d12429
% MyPink		r=0.980, g=0.780, b=0.761 % fac7c2
% MyGray		r=0.812, g=0.788, b=0.780 % cfc9c7
% MyOrange		r=0.980, g=0.671, b=0.290 % faab4a

% Configuração de cores
\definecolor[MyColor][RedOrange]      % ou ex: [r=0.862, g=0.118, b=0.118] % corresponde a RGB(220, 30, 30)
\definecolor[MyColorText][black]     % ou ex: [r=0.86

% Classe para diagramação dos posts
\environment{marketing.env}		   

\starttext %---------------------------------------------------------|

\Mensagem{FREUD PAGA A DÍVIDA}

%\Mensagem{LITERATURA EM FOCO}

\startMyCampaign

\hyphenpenalty=10000
\exhyphenpenalty=10000

{\bf LITERATURA \\EM FOCO}

\stopMyCampaign

%\vfill\scale[lines=1.5]{\MyStar[MyColorText][none]}

\page %---------------------------------------------------------| 

\MyCover{FREUD_LITERATURA_THUMB}

\page %---------------------------------------------------------| 

\hyphenpenalty=10000
\exhyphenpenalty=10000


{\bf ESCRITOS SOBRE LITERATURA} reúne textos que, de certa forma, resgatam  o «débito» de {\bf FREUD} com a história literária.

\page

Foi pelo diálogo com as produções literárias que Freud pôde dar forma para as suas criações conceituais, como o {\bf «COMPLEXO DE ÉDIPO»}, que remete à estrutura narrativa de uma peça de {\bf SÓFOCLES}. 

\page

Nesse livro {\bf AS POSIÇÕES SE INVERTEM}: a psicanálise se volta para a literatura usando seus recursos para interpretá-la. Aqui, as obras literárias não estão mais a serviço de uma teoria psicanalítica, mas constituem {\bf OBJETO CENTRAL DA ANÁLISE}.

\page

Os textos que integram a obra tratam de autores como {\bf DOSTOIÉVSKI}, {\bf E. T. A HOFFMAN} e {\bf GOETHE}, relacionando {\bf SEUS ESCRITOS E SUAS BIOGRAFIAS} aos mecanismos da memória, à sensação do estranho familiar, e ao próprio «complexo de Édipo».

% \page %---------------------------------------------------------|

% A tradução é de {\bf SAULO KRIEGER}, formado em filosofia pela {\cap USP} e tradutor de 
% {\bf A VÊNUS DAS PELES} (Hedra, 2008); e o  posfácio de {\bf NOEMI MORITZ KON}, doutora de Psicologia Social do Instituto de Psicologia da {\cap USP} e  autora de {\bf FREUD E SEU DUPLO: REFLEXÕES ENTRE PSICANÁLISE E ARTE} (Edusp, 2014) e {\bf A VIAGEM: DA LITERATURA À PSICANÁLISE} (Cia das Letras, 2001).

\page %---------------------------------------------------------|

\Hedra

\stoptext %---------------------------------------------------------


%Tradução de {\bf Saulo Krieger}

% posfácio de Noemi Kon

% Saulo Kriegeré formado em filosofia pela Universidade de São Paulo (USP) e cursou psicanálise no Centro de Estudos Psicanalíticos (CEP).Tradutor e ensaísta, especializou-se na tradução de textos sobre psi-cologia e psicanálise, sendo colaborador assíduo da revista Mente ecérebro. Traduziu Cultura psicanalítica, de Ian Parker (Ideias e Letras, 2006), Após o fim da arte, de Arthur Danto (Edusp/Odysseus, 2006) e A Vênus das Peles, de Sacher-Masoch (Hedra, 2008).


% Noemi Moritz Koné psicanalista, membro do Departamento de Psicanálise do Instituto Sedes Sapientiae, mestre e doutora de Psicologia Social do Instituto de Psicologia da USP e autora deFreud e seuDuplo: Reflexões entre Psicanálise e Arte, A Viagem: da Literatura à Psicanálise e organizadora de 125 contos de Guy de Maupassant


% Psicanálise e literatura nunca estiveram distantes. Mas
% em nenhuma parte da vasta obra de Freud estão mais
% próximas do que em {\bf Escritos sobre literatura}, em que
% as ferramentas analíticas que a própria literatura ajudou
% a lapidar agora servem para reinterpretá-la. A personalidade 
% e a obra de Dostoiévski, a sensação do estranho
% familiar a partir do escritor “fantástico” E. T. A. Hoffman,
% os mecanismos de significação da memória sugeridos por
% Goethe e o próprio impulso criativo, que Freud relaciona
% à pulsão erótica, ganham novas luzes − além de sombras
% mais profundas.