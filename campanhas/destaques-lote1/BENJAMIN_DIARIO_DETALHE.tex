% Preencher com o nome das cor ou composição RGB (ex: [r=0.862, g=0.118, b=0.118]) 
\usecolors[crayola] 			   % Paleta de cores pré-definida: wiki.contextgarden.net/Color#Pre-defined_colors

% Cores definidas pelo designer:
% MyGreen		r=0.251, g=0.678, b=0.290 % 40ad4a
% MyCyan		r=0.188, g=0.749, b=0.741 % 30bfbd
% MyRed			r=0.820, g=0.141, b=0.161 % d12429
% MyPink		r=0.980, g=0.780, b=0.761 % fac7c2
% MyGray		r=0.812, g=0.788, b=0.780 % cfc9c7
% MyOrange		r=0.980, g=0.671, b=0.290 % faab4a

% Configuração de cores
\definecolor[MyColor][x=EB9B65]      % ou ex: [r=0.862, g=0.118, b=0.118] % corresponde a RGB(220, 30, 30)
\definecolor[MyColorText][black]  % ou ex: [r=0.862, g=0.118, b=0.118] % corresponde a RGB(167, 169, 172)

% Classe para diagramação dos posts
\environment{marketing.env}		   

\starttext

\Mensagem{POR DENTRO DA EDIÇÃO}

\MyCover{BENJAMIN_DIARIO_THUMB.pdf}

\page %---------------------------------------------------------|


\hyphenpenalty=10000
\exhyphenpenalty=10000

{\bf Diário pariesiense e outros escritos} reúne quinze textos do
crítico e filósofo alemão {\bf WALTER BENJAMIN} escritos entre os anos 1926 a 1936. 
Em uma organização inédita, a edição traz a primeira tradução em língua portuguesa de
seu pequeno diário redigido em Paris, além de textos como
“Três franceses”, “Imagem de Proust”, “Édipo ou o mito racional”.

\page

{\it «As traduções dos textos, em parte inéditos em língua
portuguesa, seguem o caráter experimental de perto, buscando a fidelidade como
critério, uma proximidade quase unilateral ao texto.»} {\blank[.4ex]}


{\hfill\tf --- Carla Milani Damião (org.)}

\page

\Hedra

\stoptext