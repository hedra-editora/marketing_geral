% AUTOR_LIVRO_EDICAO.tex
% Preencher com o nome das cor ou composição RGB (ex: [r=0.862, g=0.118, b=0.118]) 
\usecolors[crayola] 			   % Paleta de cores pré-definida: wiki.contextgarden.net/Color#Pre-defined_colors

% Cores definidas pelo designer:
% MyGreen		r=0.251, g=0.678, b=0.290 % 40ad4a
% MyCyan		r=0.188, g=0.749, b=0.741 % 30bfbd
% MyRed			r=0.820, g=0.141, b=0.161 % d12429
% MyPink		r=0.980, g=0.780, b=0.761 % fac7c2
% MyGray		r=0.812, g=0.788, b=0.780 % cfc9c7
% MyOrange		r=0.980, g=0.671, b=0.290 % faab4a

% Configuração de cores
\definecolor[MyColor][x=abd91e]      % ou ex: [r=0.862, g=0.118, b=0.118] % corresponde a RGB(220, 30, 30)
\definecolor[MyColorText][black]     % ou ex: [r=0.862, g=0.118, b=0.118] % corresponde a RGB(167, 169, 172)

% Classe para diagramação dos posts
\environment{marketing.env}		   

\starttext %---------------------------------------------------------|

\Mensagem{MÁRIO DE ANDRADE E A MÚSICA}

\startMyCampaign

\hyphenpenalty=10000
\exhyphenpenalty=10000

{\bf UMA ANTOLOGIA DE INESTIMÁVEL IMPORTÂNCIA}

\stopMyCampaign

%\vfill\scale[lines=1.5]{\MyStar[MyColorText][none]}

\page %---------------------------------------------------------| 

\MyCover{ANDRADE_ESTRANHA_THUMB}

\page %---------------------------------------------------------| 

\hyphenpenalty=10000
\exhyphenpenalty=10000

Mário de Andrade, {\bf UMA DAS PERSONALIDADES MAIS MARCANTES DO MODERNISMO BRASILEIRO}, além de poeta, romancista, historiador da arte e crítico, foi também músico de formação erudita. Esse amplo repertório o levou à redação de artigos e ensaios de fôlego a respeito de {\bf MÚSICA CLÁSSICA E POPULAR BRASILEIRA}.

\page %---------------------------------------------------------|

No conjunto de sua extensa obra, destacam-se os dez textos selecionados por Augusto Fischer neste volume intitulado {\bf A ESTRANHA FORÇA DA CANÇÃO}. Escritos entre 1930 e 1942, nestes artigos se observam as diferentes perspectivas do autor a respeito da canção popular.

\page
\hyphenpenalty=10000
\exhyphenpenalty=10000

«A fala dum povo é porventura, mais que a própria linguagem, a melhor característica, a {\bf MAIS ÍNTIMA REALIDADE} se não da sua maneira de pensar, pelo menos da sua maneira de expressão verbal.»


\page

\MyPicture{ANDRADE_ESTRANHA_1}

{\vfill\scale[factor=8]{\Seta\,{\bf Mário de Andrade}}}

\page %---------------------------------------------------------|

\Hedra

\stoptext %---------------------------------------------------------|