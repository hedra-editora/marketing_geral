% BAKUNIN_EFEMERIDE.tex
% Preencher com o nome das cor ou composição RGB (ex: [r=0.862, g=0.118, b=0.118]) 
\usecolors[crayola] 			   % Paleta de cores pré-definida: wiki.contextgarden.net/Color#Pre-defined_colors

% Cores definidas pelo designer:
% MyGreen		r=0.251, g=0.678, b=0.290 % 40ad4a
% MyCyan		r=0.188, g=0.749, b=0.741 % 30bfbd
% MyRed			r=0.820, g=0.141, b=0.161 % d12429
% MyPink		r=0.980, g=0.780, b=0.761 % fac7c2
% MyGray		r=0.812, g=0.788, b=0.780 % cfc9c7
% MyOrange		r=0.980, g=0.671, b=0.290 % faab4a

% Configuração de cores
\definecolor[MyColor][PermanentGeraniumLake]      % ou ex: [r=0.862, g=0.118, b=0.118] % corresponde a RGB(220, 30, 30)
\definecolor[MyColorText][white]     % ou ex: [r=0.862, g=0.118, b=0.118] % corresponde a RGB(167, 169, 172)

% Classe para diagramação dos posts
\environment{marketing.env}		   

\starttext %---------------------------------------------------------|

\hyphenpenalty=10000
\exhyphenpenalty=10000

\Mensagem{30 DE MAIO} %Sempre usar esse header

\MyPicture{BAKUNIN_EFEMERIDE_1}

\vfill\scale[factor=6]{\Seta\,210 ANOS DE {\bf MIKHAIL BAKUNIN}}

\page %---------------------------------------------------------| 

\hyphenpenalty=10000
\exhyphenpenalty=10000

Nascido em 30 de maio de 1814, oriundo de uma família nobre russa, {\bf BAKUNIN} é considerado o fundador do sindicalismo revolucionário e o expoente máximo do {\bf ANARQUISMO}.

\page %---------------------------------------------------------|

Participa diretamente da {\bf REVOLUÇÃO DE 1848}, em Paris. Em 1849, é feito prisioneiro
e {\bf CONDENADO À MORTE} por participar da insurreição de Dresden, mas logo é
deportado para a Rússia e enviado à Sibéria, de onde foge em 1861.

\page

 Em 1868, adere à Internacional, mas após a Comuna de Paris, 
é expulso da organização, o que provoca a {\bf CISÃO DO MOVIMENTO SOCIALISTA} em duas correntes: a
capitaneada por Marx, e a libertária ou anarquista, alicerçada nos princípios federativo e autogestionário.  

\page

«Contar a vida de Bakunin é contar a vida do socialismo e da
revolução na Europa durante mais de trinta anos, pois ele contribuiu
ou participou de todos os progressos da idéia e dos fatos revolucionários».\\
						\vfill\scale[factor=10]{\Seta\,Filippo Turati}


\page %---------------------------------------------------------|

\Hedra

\stoptext %---------------------------------------------------------|

