% AUTOR_LIVRO_EDICAO.tex
% Preencher com o nome das cor ou composição RGB (ex: [r=0.862, g=0.118, b=0.118]) 
\usecolors[crayola] 			   % Paleta de cores pré-definida: wiki.contextgarden.net/Color#Pre-defined_colors

% Cores definidas pelo designer:
% MyGreen		r=0.251, g=0.678, b=0.290 % 40ad4a
% MyCyan		r=0.188, g=0.749, b=0.741 % 30bfbd
% MyRed			r=0.820, g=0.141, b=0.161 % d12429
% MyPink		r=0.980, g=0.780, b=0.761 % fac7c2
% MyGray		r=0.812, g=0.788, b=0.780 % cfc9c7
% MyOrange		r=0.980, g=0.671, b=0.290 % faab4a

% Configuração de cores
\definecolor[MyColor][x=10ae7f]      % ou ex: [r=0.862, g=0.118, b=0.118] % corresponde a RGB(220, 30, 30)
\definecolor[MyColorText][black]     % ou ex: [r=0.862, g=0.118, b=0.118] % corresponde a RGB(167, 169, 172)

% Classe para diagramação dos posts
\environment{marketing.env}		   

\starttext %---------------------------------------------------------|

\Mensagem{POR DENTRO DA EDIÇÃO}

% \startMyCampaign

% \hyphenpenalty=10000
% \exhyphenpenalty=10000

% {\bf 
% NOSSO LIVRO É\ 
% LEGAL POR\ 
% MUITOS MOTIVOS, \
% E VAMOS TE\ 
% CONTAR QUAIS}

% \stopMyCampaign

% %\vfill\scale[lines=1.5]{\MyStar[MyColorText][none]}

% \page %---------------------------------------------------------| 

\MyCover{ACORDE_TATIT_THUMB}

\page %---------------------------------------------------------| 

\hyphenpenalty=10000
\exhyphenpenalty=10000

{\bf NO PRINCÍPIO ERA O MEIO} reúne textos, divididos em três blocos temáticos,
que desenvolvem a singularidade da canção como linguagem artística, tanto
no aspecto formal, quanto histórico e técnico. 

\page


O primeiro bloco define o sentido da conceituação. O segundo é formado por dois estudos de caso de
artistas vinculados a diferentes formas de composição, movimentos artísticos
e contextos históricos distintos: {\bf TOM JOBIM E ITAMAR ASSUMPÇÃO}. 

\page

O terceiro, por sua vez, diferente do tom ensaístico dos dois primeiros blocos, é formado
por textos com linguagem mais próxima à do jornalismo cultural.


\page %---------------------------------------------------------|

\hyphenpenalty=10000
\exhyphenpenalty=10000

«Não nos preocupemos com a canção. Ela tem a idade das culturas
humanas e certamente sobreviverá a todos nós. {\bf ONDE HOUVE LÍNGUA E
VIDA COMUNITÁRIA, HOUVE CANÇÃO}. Enquanto houver seres falantes,
haverá cancionistas convertendo suas falas em canto.»

{\vfill\scale[factor=5]{\Seta\,Trecho do livro {\bf No princípio era o meio}, de Luiz Tatit.}}

\page %---------------------------------------------------------|

\Hedra

\stoptext %---------------------------------------------------------|


10ae7f
f37c9d