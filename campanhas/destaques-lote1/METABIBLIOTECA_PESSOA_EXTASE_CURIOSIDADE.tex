% METABIBLIOTECA_PESSOA_EXTASE_CURIOSIDADE.tex
% Vamos falar sobre isso "curiosidades"
% > "EM CONTEXTO"

% Preencher com o nome das cor ou composição RGB (ex: [r=0.862, g=0.118, b=0.118]) 
\usecolors[crayola] 			   % Paleta de cores pré-definida: wiki.contextgarden.net/Color#Pre-defined_colors

% Cores definidas pelo designer:
% MyGreen		r=0.251, g=0.678, b=0.290 % 40ad4a
% MyCyan		r=0.188, g=0.749, b=0.741 % 30bfbd
% MyRed			r=0.820, g=0.141, b=0.161 % d12429
% MyPink		r=0.980, g=0.780, b=0.761 % fac7c2
% MyGray		r=0.812, g=0.788, b=0.780 % cfc9c7
% MyOrange		r=0.980, g=0.671, b=0.290 % faab4a

% Configuração de cores
\definecolor[MyColor][MyCyan]      % ou ex: [r=0.862, g=0.118, b=0.118] % corresponde a RGB(220, 30, 30)
\definecolor[MyColorText][RustyRed]  % ou ex: [r=0.862, g=0.118, b=0.118] % corresponde a RGB(167, 169, 172)

% Classe para diagramação dos posts
\environment{marketing.env}		   

\def\MyPicture#1{\vfill
                \starttikzpicture
                    \node[yshift=-5mm, inner sep=0] (image) at (0,0) {\externalfigure[#1][height=\textheight]};
                    \fill[MyColorText,opacity=0.3] (image.south west) rectangle (image.north east);
                \stoptikzpicture}  

\starttext %---------------------------------------------------------|

\hyphenpenalty=10000
\exhyphenpenalty=10000

\Mensagem{EM CONTEXTO}

\startMyCampaign
\hyphenpenalty=10000
\exhyphenpenalty=10000

A INFÂNCIA DE {\bf FERNANDO PESSOA}
NA ÁFRICA DO SUL

%\vfill\scale[lines=2]{\MyStar[MyColorText][none]} 					% Estrela pequena  

\stopMyCampaign

\page %---------------------------------------------------------| 

\hyphenpenalty=10000
\exhyphenpenalty=10000

Aos sete anos, após o falecimento do seu pai, {\bf FERNANDO PESSOA} muda-se com a mãe para {\bf DURBAN},
na África do Sul. 
\page

 Lá, frequentou escolas locais e teve seus primeiros contatos com a {\bf LÍNGUA INGLESA}, o que influenciaria profundamente sua escrita posterior.

\page
\MyPicture{METABIBLIOTECA_PESSOA_EXTASE_4}

\page

Em 1905, retorna definitivamente para sua cidade natal e ingressa na {\bf FACULDADE DE LETRAS
DA UNIVERSIDADE DE LISBOA.}

\page

Mesmo que breve, sua experiência em Durban contribuiu para sua {\bf FORMAÇÃO INTELECTUAL E CULTURAL}, permitindo-lhe desenvolver uma visão mais ampla do mundo e uma {\bf SENSIBILIDADE LINGUÍSTICA} que influenciaria sua escrita futura.

\page %---------------------------------------------------------|

\MyCover{METABIBLIOTECA_PESSOA_EXTASE_THUMB}

\page %---------------------------------------------------------|

\Hedra

\stoptext %---------------------------------------------------------|
	


% Embora sua estadia na África do Sul tenha sido relativamente curta em comparação com sua vida em Portugal, essa fase de sua vida deixou uma marca indelével em sua obra. A exposição à diversidade cultural e linguística da África do Sul, assim como as experiências de estar longe de sua terra natal, moldaram sua percepção do mundo e contribuíram para a riqueza de sua escrita, que transcende fronteiras geográficas e culturais.