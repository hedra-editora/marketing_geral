% AUTOR_LIVRO_EDICAO.tex
% Preencher com o nome das cor ou composição RGB (ex: [r=0.862, g=0.118, b=0.118]) 
\usecolors[crayola] 			   % Paleta de cores pré-definida: wiki.contextgarden.net/Color#Pre-defined_colors

% Cores definidas pelo designer:
% MyGreen		r=0.251, g=0.678, b=0.290 % 40ad4a
% MyCyan		r=0.188, g=0.749, b=0.741 % 30bfbd
% MyRed			r=0.820, g=0.141, b=0.161 % d12429
% MyPink		r=0.980, g=0.780, b=0.761 % fac7c2
% MyGray		r=0.812, g=0.788, b=0.780 % cfc9c7
% MyOrange		r=0.980, g=0.671, b=0.290 % faab4a

% Configuração de cores
\definecolor[MyColor][x=EB9B65]      % ou ex: [r=0.862, g=0.118, b=0.118] % corresponde a RGB(220, 30, 30)
\definecolor[MyColorText][black]     % ou ex: [r=0.862, g=0.118, b=0.118] % corresponde a RGB(167, 169, 172)

% Classe para diagramação dos posts
\environment{marketing.env}		   

\starttext %---------------------------------------------------------|

\Mensagem{POR DENTRO DA EDIÇÃO}

\startMyCampaign

\hyphenpenalty=10000
\exhyphenpenalty=10000

ENTRE FRANÇA E ALEMANHA
O {\bf DIÁRIO PARISIENSE} DE WALTER BENJAMIN

\stopMyCampaign

%\vfill\scale[lines=1.5]{\MyStar[MyColorText][none]}

\page %---------------------------------------------------------| 

\MyCover{BENJAMIN_DIARIO_THUMB.pdf}

\page %---------------------------------------------------------| 

\hyphenpenalty=10000
\exhyphenpenalty=10000

{\bf DIÁRIO PARIESIENSE E OUTROS ESCRITOS} reúne quinze textos do
crítico e filósofo alemão {\bf WALTER BENJAMIN} escritos entre os anos 1926 a 1936. 
Em uma organização inédita, a edição traz a primeira tradução em língua portuguesa de
seu pequeno diário redigido em Paris, além de textos como
«Três franceses», «Imagem de Proust», «Édipo ou o mito racional».

\page %---------------------------------------------------------|

\hyphenpenalty=10000
\exhyphenpenalty=10000

«Além da atenção dedicada aos escritores franceses, os
textos reunidos neste volume dão-nos notícias do trânsito entre Alemanha e França percorrido por Benjamin em vários sentidos: literário-crítico, filosófico, artístico,
político e biográfico.»

\vfill\hfill →

\page

Nesse trânsito, procuramos pelo “lugar” de Benjamin, como esse crítico literário exemplar, judeu-alemão e refugiado político, inserido no {\bf DEBATE LITERÁRIO FRANCÊS} no período entreguerras.»

{\vfill\scale[factor=5]{{\bf Carla Milani Damião}, na apresentação do livro {\it Diário}}\setupinterlinespace[line=1.5ex]\scale[factor=5]{{\it parisiense}, de Walter Benjamin.}}

\page %---------------------------------------------------------|

\Hedra

\stoptext %---------------------------------------------------------|

