% AUTOR_LIVRO_CLIPPING.tex
% Preencher com o nome das cor ou composição RGB (ex: [r=0.862, g=0.118, b=0.118]) 
\usecolors[crayola] 			   % Paleta de cores pré-definida: wiki.contextgarden.net/Color#Pre-defined_colors

% Cores definidas pelo designer:
% MyGreen		r=0.251, g=0.678, b=0.290 % 40ad4a
% MyCyan		r=0.188, g=0.749, b=0.741 % 30bfbd
% MyRed			r=0.820, g=0.141, b=0.161 % d12429
% MyPink		r=0.980, g=0.780, b=0.761 % fac7c2
% MyGray		r=0.812, g=0.788, b=0.780 % cfc9c7
% MyOrange		r=0.980, g=0.671, b=0.290 % faab4a

% Configuração de cores
\definecolor[MyColor][VividViolet]      % ou ex: [r=0.862, g=0.118, b=0.118] % corresponde a RGB(220, 30, 30)
\definecolor[MyColorText][white]     % ou ex: [r=0.862, g=0.118, b=0.118] % corresponde a RGB(167, 169, 172)

% Classe para diagramação dos posts
\environment{marketing.env}		   

% Comandos & Instruções %%%%%%%%%%%%%%%%%%%%%%%%%%%%%%%%%%%%%%%%%%%%%%%%%%%%%%%%%%%%%%%|

% Cabeço e rodapé: Informações (caso queira trocar alguma coisa)
 		\def\MensagemSaibaMais 	{SAIBA MAIS:}
 		\def\MensagemSite		{HEDRA.COM.BR}
 		\def\MensagemLink		{LINK NA BIO}

\starttext %---------------------------------------------------------|

\Mensagem{NA IMPRENSA}

\MyPhoto{clippinggama} %Usar este tamanho de imagem

\page %---------------------------------------------------------|

\hyphenpenalty=10000
\exhyphenpenalty=10000


No dia 9 de agosto, foi publicada no Brasil de Fato uma entrevista com Bruno Rodrigues de Lima, organizador do {\bf PROJETO LUIZ GAMA}, recém-premiado pelo Jabuti Acadêmico.

\page

«Quando o Prêmio Jabuti Acadêmico reconhece Luiz Gama como livro do ano de 2024, isso significa uma enormidade para o cenário literário jurídico brasileiro. Ou seja, nós estamos avançando, nós estamos colocando Luiz Gama --- e eu digo nós porque é uma tarefa coletiva, não é uma tarefa individual, nunca foi, nunca será --- no {\bf CÂNONE DA LITERATURA BRASILEIRA.}» 

\page %---------------------------------------------------------|

«O que eu sei é que é urgente colocar Luiz Gama na cesta básica literária, acadêmica, científica e histórica do Brasil. E é essa minha tarefa já desde garoto, que é ir nos arquivos da escravidão, recolher, material por material, escrito por escrito, papel por papel, os textos de quem eu considero o maior escritor de língua portuguesa nascido no Brasil, que é o baiano Luiz Gama.» 
{\vfill\scale[factor=5]{\Seta\,Trecho da entrevista com {\bf Bruno Rodrigues de Lima},}\setupinterlinespace[line=1.5ex]\scale[factor=5]{do Brasil de Fato, em 9 de agosto de 2024.}}

\page %---------------------------------------------------------|

\MyCover{GAMA_DIREITO_THUMB}

\page %---------------------------------------------------------|

\Hedra

\stoptext %---------------------------------------------------------|

