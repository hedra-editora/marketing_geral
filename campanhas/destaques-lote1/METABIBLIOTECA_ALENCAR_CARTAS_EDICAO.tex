% METABIBLIOTECA_ALENCAR_CARTAS_EDICAO.tex
% Preencher com o nome das cor ou composição RGB (ex: [r=0.862, g=0.118, b=0.118]) 
\usecolors[crayola] 			   % Paleta de cores pré-definida: wiki.contextgarden.net/Color#Pre-defined_colors

% Cores definidas pelo designer:
% MyGreen		r=0.251, g=0.678, b=0.290 % 40ad4a
% MyCyan		r=0.188, g=0.749, b=0.741 % 30bfbd
% MyRed			r=0.820, g=0.141, b=0.161 % d12429
% MyPink		r=0.980, g=0.780, b=0.761 % fac7c2
% MyGray		r=0.812, g=0.788, b=0.780 % cfc9c7
% MyOrange		r=0.980, g=0.671, b=0.290 % faab4a

% Configuração de cores
\definecolor[MyColor][MiddleGreenYellow]      % ou ex: [r=0.862, g=0.118, b=0.118] % corresponde a RGB(220, 30, 30)
\definecolor[MyColorText][Maroon]     % ou ex: [r=0.862, g=0.118, b=0.118] % corresponde a RGB(167, 169, 172)

% Classe para diagramação dos posts
\environment{marketing.env}		   

\starttext %---------------------------------------------------------|

\Mensagem{POR DENTRO DA EDIÇÃO}

\startMyCampaign

\hyphenpenalty=10000
\exhyphenpenalty=10000

{\bf 
REVOLUCIONÁRIO EM LETRAS 
CONSERVADOR EM POLÍTICA?}

\stopMyCampaign

%\vfill\scale[lines=1.5]{\MyStar[MyColorText][none]}

\page %---------------------------------------------------------| 

\MyCover{METABIBLIOTECA_ALENCAR_CARTAS_THUMB}

\page %---------------------------------------------------------| 

\hyphenpenalty=10000
\exhyphenpenalty=10000

Dirigindo-se a Dom Pedro {\cap II}, {\bf JOSÉ DE ALENCAR} articula uma defesa da escravidão brasileira.
Para o autor, o regime escravagista ia de encontro aos interesses da nação e ao {\bf PROGRESSO} da humanidade. 

\page 

 A leitura destas cartas completa o quebra-cabeça de um dos nossos autores mais fundamentais,
tornando talvez obsoleta a antítese que {\bf JOSÉ VERÍSSIMO} atribui a {\bf ALENCAR} --- “Revolucionário em letras, conservador em política”.

\page

 A presente edição procura fornecer um texto frequentemente escamoteado por seu teor {\bf CONTROVERSO}, mas precioso para o público interessado nos debates sobre {\bf RELAÇÕES RACIAIS}, bem como para os especialistas em {\bf JOSÉ DE ALENCAR}, Império do Brasil, escravidão e sistema representativo.

% \page



% \hyphenpenalty=10000
% \exhyphenpenalty=10000

% «No seio da barbaria, o homem, em luta
% contra a natureza, sente a necessidade
% de multiplicar suas forças. O único
% instrumento ao alcance é o próprio
% homem, seu semelhante; apropria-se
% dele ou pelo direito da geração ou pelo
% direito da conquista.

% \page

% Aí está o gérmen
% rude e informe da família, agregado
% dos fâmulos, reunião de servos. O mais
% antigo documento histórico, o Gênesis,
% nos mostra o homem filiando-se à família
% estranha pelo cativeiro.» 

% {\vfill\scale[factor=5]{{\bf José de Alencar},}\setupinterlinespace[line=1.5ex]\scale[factor=5]{\bf CARTAS A FAVOR DA ESCRAVIDÃO}\setupinterlinespace[line=1.5ex]\scale[factor=5]{de quebrar as linhas nos códigos.}}

\page

\Hedra

\stoptext






% As Novas cartas políticas são reeditadas
% aqui pela primeira vez desde o século XIX,
% após terem sido expurgadas das obras do
% autor. Trata-se de sete textos políticos,
% endereçados a D. Pedro II, que justificam
% uma instituição hoje universalmente
% condenada, publicação incontornável para
% a nossa historiografia política e literária,
% bem como para todos os interessados
% na história das relações raciais e na
% escravidão no Brasil e no mundo.
% 
% Sob o codinome Erasmo, em alusão a Erasmo de
% Roterdã, Alencar compõe praticamente um espelho de
% príncipe, na posição do letrado que prescreve ao gover-
% nante as normas de um bom governo. Dirigindo-se a Dom
% Pedro II, o autor procura demonstrar que abolir o cati-
% veiro ia de encontro aos interesses da nação e, mais do
% que isso, ao progresso da humanidade — aí incluídos os
% próprios escravos. O cativeiro seria, pois, o embrião civili-
% zacional, a forma pela qual o homem é retirado do estado
% bruto da natureza e aperfeiçoa-se.

% Mais do que isso, Alencar glorifica a densidade étnica
% e cultural que resulta do tráfico negreiro, na que talvez
% seja a única defesa da escravidão, em toda a América do
% XIX, baseada na mistura cultural dos povos. Os ecos do
% elogio à miscigenação ecoam em toda sua obra, vendo
% nela o esteio da nacionalidade brasileira em formação.
% A leitura destas Cartas, pois, por um lado completa o
% quebra-cabeça de um dos nossos autores mais fundamen-
% tais, tornando talvez obsoleta a antítese de José Veríssi-
% mo — “Revolucionário em letras, conservador em política”
% — já que as letras revolucionárias serviram perfeitamente
% à tribuna conservadora; e, por outro, lança luz na faceta
% oculta de José de Alencar, não para levar seu espírito aos
% tribunais, mas sobretudo para enriquecer o debate sobre
% a escravidão brasileira e suas muitas reminiscências em
% nossas relações raciais.
