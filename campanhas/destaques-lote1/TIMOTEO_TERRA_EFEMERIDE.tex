% TIMOTEO_TERRA_EFEMERIDE.tex
% Preencher com o nome das cor ou composição RGB (ex: [r=0.862, g=0.118, b=0.118]) 
\usecolors[crayola] 			   % Paleta de cores pré-definida: wiki.contextgarden.net/Color#Pre-defined_colors

% Cores definidas pelo designer:
% MyGreen		r=0.251, g=0.678, b=0.290 % 40ad4a
% MyCyan		r=0.188, g=0.749, b=0.741 % 30bfbd
% MyRed			r=0.820, g=0.141, b=0.161 % d12429
% MyPink		r=0.980, g=0.780, b=0.761 % fac7c2
% MyGray		r=0.812, g=0.788, b=0.780 % cfc9c7
% MyOrange		r=0.980, g=0.671, b=0.290 % faab4a

% Configuração de cores
\definecolor[MyColor][x=e06600]      % ou ex: [r=0.862, g=0.118, b=0.118] % corresponde a RGB(220, 30, 30)
\definecolor[MyColorText][black]     % ou ex: [r=0.862, g=0.118, b=0.118] % corresponde a RGB(167, 169, 172)

% Classe para diagramação dos posts
\environment{marketing.env}		   

\starttext %---------------------------------------------------------|

\hyphenpenalty=10000
\exhyphenpenalty=10000

\Mensagem{14 DE JUNHO} %Sempre usar esse header

\MyPhoto{TIMOTEO_TERRA_1}

\vfill\scale[factor=7]{\Seta\,8 ANOS DO {\bf MASSACRE DE CAARAPÓ}}

\page %---------------------------------------------------------| 

\hyphenpenalty=10000
\exhyphenpenalty=10000

O dia {\bf 14 DE JUNHO} é marcante para a luta dos {\bf KAIOWÁ} e {\bf GUARANI} por refletir a luta pela demarcação de suas terras e a violência que enfrentam.

\page 

Em maio de 2016, a Funai reconheceu o território indígena Dourados Amambaipeguá {\cap I}. Em junho, os indígenas intensificaram o processo de {\bf AUTODEMARCAÇÃO} de suas terras reconhecidas pelo órgão indigenista oficial do Estado brasileiro com a retomada do tekoha Toro Paso, localizado dentro dos limites da Fazenda Yvu. 

\page

Dois dias depois, 70 pessoas entraram na ocupação abrindo caminho para os fazendeiros que {\bf ATACARAM VIOLENTAMENTE} os indígenas. Em meio aos disparos dos invasores, {\bf CLODIODE DE SOUZA}, que socorria os feridos, foi atingido e morto. 

\page

As acusações indicam que a polícia não ajudou e, inclusive, participou do massacre. Durante o ataque, indígenas resistiram confiscando armas e queimando uma viatura policial. Leonardo de Souza, pai de Clodiode, foi {\bf PRESO} após ser considerado foragido. Os produtores rurais respondem em liberdade. No fim de 2023, foi decidido que iriam à {\bf JÚRI POPULAR}. Até hoje, seguem impunes.

\page

\MyCover{TIMOTEO_TERRA_THUMB}

\page %---------------------------------------------------------|

\Hedra

\stoptext %---------------------------------------------------------|




% O MPF/MS iniciou uma investigação, chamada de força-tarefa Avá Guarani, que resultou na prisão de cinco fazendeiros e na apreensão de munições de armas semelhantes aos projéteis encontrados pela comunidade, como demonstra reportagem do Conselho Indigenista Missionário (Cimi). Ainda segundo a reportagem, em outubro de 2016 “o MPF apresentou a denúncia à Justiça Federal em Dourados contra os cinco envolvidos na retirada forçada dos indígenas da Fazenda Yvu”. Contudo, menos de um mês depois os fazendeiros foram soltos e hoje são réus de um processo que abarca formação de milícia armada, homicídio qualificado, tentativa de homicídio qualificado, lesão corporal, dano qualificado e constrangimento ilegal. Os produtores rurais acusados de assassinarem Clodiode de Souza respondem em liberdade. O MPF foi procurado para entrevista e alegou que o processo contra os fazendeiros é sigiloso e que não poderiam dar muitas informações, mas nos disponibilizou um release sobre o caso.

% O MPF apresentou duas ações em relação a este episódio: a primeira descrita acima em relação aos produtores rurais e a segunda contra Leonardo de Souza que, de acordo com o MPF, teria praticado crimes após tomar conhecimento do assassinato de seu filho. No caso de Leonardo já houve condenação em primeira instância e há um recurso para ser julgado em segundo grau. A rapidez para a prisão de Leonardo e a lentidão no julgamento dos produtores rurais poderá levar o Estado brasileiro a ser responsabilizado pela Corte Interamericana de Direitos Humanos, da Organização dos Estados Americanos.

 
% Prisão para o indígena, liberdade para os fazendeiros

% Essa lentidão demonstra que há conflito de interesses, visto que foram descobertas conversas entre o delegado da Polícia Federal de Dourados da época, Denis Colares, e um dos fazendeiros acusados, o que levou o Ministério Público a denunciar alguns agentes da Polícia Federal de Dourados por falsidade ideológica. Segundo Flávio Machado, missionário do Cimi, há disputas e divergências entre a Justiça Estadual e a Federal. Ele relata que houve uma situação em “que estava havendo um conflito de competência entre as duas justiças e o indígena foi retirado da cadeia pela Justiça Federal. E a PM soube disso e foi até a cadeia impedir a saída do indígena e chegou lá os policiais federais estavam retirando o indígena e os policiais militares pegaram, literalmente, de um lado do indígena, os policiais federais do outro e o indígena ficou no meio sendo puxado”. Para Flávio, esse acontecimento marcou um símbolo da disputa entre os poderes Estadual e Federal no Mato Grosso do Sul.

% Enquanto os fazendeiros respondiam em liberdade pelas acusações, Leonardo de Souza, indígena Kaiowá, era considerado foragido, mesmo estando na aldeia. No dia 13 de dezembro de 2018, a Força Nacional de Segurança foi até a casa de Leonardo, na área Reserva Indígena Tey’i Kuê. com mandado de prisão. Durante o cumprimento, os policiais alegam que encontraram cerca de 300g de maconha na casa de Leonardo, que acabou sendo preso em flagrante por tráfico de drogas. Após a prisão, o caso passou a ser julgado sob o prisma de três diferentes processos, em diferentes circunstâncias. Um deles é de uma ação política dos povos indígenas da região que ocuparam, em 2016, a Secretaria Especial de Saúde Indígena (Sesai) e impediram a entrada e a saída de pessoas do local durante cerca de 10 horas. Por essa ação, Leonardo foi processado por sequestro e cárcere privado. Além do processo por tráfico internacional de drogas, devido ao fato de que o Mato Grosso do Sul faz fronteira com o Paraguai, e um terceiro processo por crime de tortura contra os policiais.

% Leonardo de Souza foi condenado em primeira instância por cárcere privado qualificado, roubo majorado pelo concurso de agentes, pela restrição da liberdade das vítimas e dano qualificado com pena de 16 anos e 04 meses de reclusão (que permanecerá preso) e 02 anos e 15 dias de reclusão (podendo responder em regime aberto ou semiaberto).

% Em dezembro de 2020, a Defensoria Pública da União entrou com pedido de Habeas Corpus no Supremo Tribunal Federal (STF) para que Leonardo fosse julgado em liberdade, assim como os fazendeiros. No entendimento do STF, baseado no argumento do ministro Alexandre de Moraes, Leonardo é considerado perigoso. Segundo Daniele de Souza Osório, defensora pública federal, uma das autoras do HC, essa diferença no tratamento entre os povos indígenas e os fazendeiros da região é devido ao racismo estrutural que perpassa também pelas esferas judiciais. “O número de presos provisórios no Brasil é altíssimo, os presídios estão lotados de pessoas pobres, negros, pardos e indígenas. No caso do Leonardo eu não tenho dúvida alguma que a escolha, tanto do sistema penal, da própria legislação e também do poder judiciário, é punir as pessoas mais enfraquecidas porque não tem justificativa para que acusados de um crime tão bárbaro quanto o praticado contra os indígenas – a gente tem que lembrar que foi um homicídio e 6 tentativas de homicídio – então, um crime contra a vida não seja considerado tão grave quanto uma reação de um pai”, afirma a defensora.

% Durante todo o processo é questionado o pedido de intérprete pelo Leonardo pois afirmavam que ele era um “índio integrado”. O argumento do ministro Alexandre de Moraes é baseado na ideia do integracionismo. Esta compreensão nega o direito indígena previsto na Constituição Federal, no artigo 231, que afirma que “são reconhecidos aos índios sua organização social, costumes, línguas, crenças e tradições”. Na votação do HC para que Leonardo pudesse responder às acusações em liberdade, Moraes afirma que “o fato de ele estar aculturado, já integrado à sociedade, também, a meu ver, não justificam tratamento diferenciado”. Ainda respondendo ao argumento da Defensoria Pública da União, o ministro Luís Roberto Barroso declarou que, “honestamente, não vejo teratologia, não vejo nenhum absurdo na prisão preventiva dessa pessoa, inclusive já condenada a penas relevantes por crime grave”. Segundo Daniele, “desde 1988, a Constituição Federal afasta qualquer ideia de integração dos indígenas. A Constituição preserva as diferenças culturais. E mesmo que nós não tivéssemos a Constituição assegurando isso, o Brasil é signatário de tratados internacionais, se comprometendo no plano internacional a preservar e manter a diversidade cultural”.

% Mas o tratamento no caso do Leonardo é apenas mais uma das ações de violência institucional. Daniele lembra, durante a entrevista, de algumas vezes em que foi abordada. Em suas palavras, “todas as vezes que Defensores Públicos se deslocam para aldeias coincidentemente há barreiras policiais próximas, há abordagem policial, sempre com questionamento sobre os deslocamentos, por exemplo”. Apesar de não ser uma prática comum, “as áreas de retomadas são extremamente cercadas por forças policiais”. No entanto, “não são forças policiais que ali estão para preservar a segurança das pessoas, são forças policiais que ali estão para preservar o patrimônio de proprietários rurais. E é isso que é muito significativo porque a presença dessas forças policiais não tem impedido ameaças aos indígenas, não tem impedido relato de disparo de arma durante a noite, os incêndios em casa de reza”.

% As disputas na região acontecem devido ao fato de que as terras são bastante produtivas. Segundo Flávio, do Cimi, “há 6 anos o poder executivo é aparelhado pelo agronegócio, o governador é um dos maiores fazendeiros do estado”. Isso acarreta na imobilidade da justiça quando se trata da defesa dos indígenas e também na impunidade quando se trata das acusações aos fazendeiros. Segundo a defensora, “os casos de homicídio de indígenas no Mato Grosso do Sul raramente têm a autoria descoberta”.

% Na busca por ajudas mais efetivas, os povos indígenas recorrem a denúncias internacionais que “tem gerado uma pressão sobre o estado brasileiro como nunca antes. Só que na nossa avaliação faltam ainda passos importantes como aspectos de sanções e aí sanções econômicas e bloqueios quando se trata de violação de direitos humanos”, disse Flávio.




 
% Aldeia em Caarapó (Foto: Rafael de Abreu)

 

% Leonardo de Souza tem diabetes, hipertensão e depressão diagnosticada após a morte de seu filho Clodiode de Souza. O laudo de corpo delito que foi produzido após a sua prisão demonstra que ele tem artrose no ombro esquerdo e no joelho direito, possui limitação nos movimentos e hérnia inguinal esquerda, classificado como portador de doença degenerativa crônica, portanto faz parte do grupo de risco da Covid-19.

% O pedido de Habeas Corpus impetrado pela Defensoria Pública da União no Supremo Tribunal Federal em 2 de abril de 2020 enfatizou que Leonardo fazia parte do grupo de risco da Covid-19 e mesmo assim o pedido foi negado. Leonardo de Souza pegou Covid-19 na prisão e embora tenha sobrevivido não se sabe se ficou com alguma sequela.

% Alguns meses depois, em setembro de 2020, Jesus de Souza, que também foi uma das vítimas do Massacre de Caarapó, filho de Leonardo, morreu de Covid-19. Leonardo de Souza teve sua família destruída pelos produtores rurais e pelo Estado brasileiro.

% O dia do massacre de Caarapó é um dia triste, mas também, para Elson “é um dia de luta e resistência”. O professor Elson Canteiro Gomes, da aldeia Kunumi Poty Vera, assim como os demais indígenas da região, não entende porque Leonardo, que não foi responsável por nenhuma morte, está preso; enquanto os fazendeiros acusados de homicídio estão em liberdade.

% “Por que o Estado brasileiro continua matando os povos indígenas Guarani e Kaiowá? Já destruiu toda sua família. Primeiro no massacre morreu seu filho Clodiode, agente de saúde, segundo morreu Jesus de Souza, e o pai vai para a cadeia. Por que o Estado brasileiro continua prendendo quem não tem culpa de nada? Sendo que os verdadeiros culpados do massacre estão livres, respondendo em processo de liberdade. Nós queremos justiça a favor do nosso povo, a favor do Clodiode, a favor do professor Jesus, que tenha justiça para os povos indígenas. E nós queremos culpar o verdadeiro inimigo do povo indígena que é o governo estadual e federal. Por que insiste ainda em prender o Leonardo? O que vale mais para o governo brasileiro, é a vida ou o dinheiro? A nossa luta, a nossa resistência, a nossa sobrevivência, depende daquelas pessoas que lutam por nós, lutam pela nossa vida e pelo nosso futuro. Nós queremos que seja reconhecido o nosso direito. Nós queremos justiça a favor do povo Guarani e Kaiowá. E queremos lembrar que o massacre de Caarapó foi um dia de luta e resistência”, questiona Elson