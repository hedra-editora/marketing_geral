% AUTOR_LIVRO_AUTOR.tex
% Preencher com o nome das cor ou composição RGB (ex: [r=0.862, g=0.118, b=0.118]) 
\usecolors[crayola]                % Paleta de cores pré-definida: wiki.contextgarden.net/Color#Pre-defined_colors

% Cores definidas pelo designer:
% MyGreen       r=0.251, g=0.678, b=0.290 % 40ad4a
% MyCyan        r=0.188, g=0.749, b=0.741 % 30bfbd
% MyRed         r=0.820, g=0.141, b=0.161 % d12429
% MyPink        r=0.980, g=0.780, b=0.761 % fac7c2
% MyGray        r=0.812, g=0.788, b=0.780 % cfc9c7
% MyOrange      r=0.980, g=0.671, b=0.290 % faab4a

% Configuração de cores
\definecolor[MyColor][MiddleGreenYellow]      % ou ex: [r=0.862, g=0.118, b=0.118] % corresponde a RGB(220, 30, 30)
\definecolor[MyColorText][Maroon]     % ou ex: [r=0.862, g=0.118, b=0.118] % corresponde a RGB(167, 169, 172)


% Classe para diagramação dos posts
\environment{marketing.env}        

% Cabeço e rodapé: Informações (caso queira trocar alguma coisa)
        \def\MensagemSaibaMais  {SAIBA MAIS:}
        \def\MensagemSite           {HEDRA.COM.BR}
        \def\MensagemLink           {LINK NA BIO}

\starttext  %---------------------------------------------------------|
\Mensagem{ÍCONE DO ROMANTISMO}

% Foto para background
\MyBackground{METABIBLIOTECA_ALENCAR_CARTAS_1}

\startstandardmakeup[background=backgroundimage]
\startMyCampaign
\vfill\scale[factor=4]{\Seta\,JOSÉ DE ALENCAR (1829--1877)}
\stopMyCampaign
\stopstandardmakeup

\page 
\Mensagem{ÍCONE DO ROMANTISMO}


\hyphenpenalty=10000
\exhyphenpenalty=10000

{\bf JOSÉ DE ALENCAR}, um dos maiores escritores brasileiros do século {\cap XIX}, atravessou diversos gêneros, do romance ao teatro, deixando um {\bf LEGADO ÚNICO} na literatura nacional.

\page

Figura central do romantismo brasileiro, entre seus romances figuram {\bf O GUARANI} e {\bf IRACEMA}, que demonstram sua habilidade em criar tramas envolventes e personagens inesquecíveis.
 \page

 Além de sua contribuição literária, {\bf ALENCAR} foi figura ativa na política e no jornalismo de sua época, assumindo {\bf POSIÇÕES BASTANTE CONSERVADORAS} --- como a defesa da escravidão.

\page %----------------------------------------------------------|

\MyCover{METABIBLIOTECA_ALENCAR_CARTAS_THUMB}

\page %----------------------------------------------------------|

\Hedra

\stoptext %---------------------------------------------------------|