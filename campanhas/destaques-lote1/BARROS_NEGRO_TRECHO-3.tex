% AUTOR_LIVRO_TRECHO.tex
% Preencher com o nome das cor ou composição RGB (ex: [r=0.862, g=0.118, b=0.118]) 
\usecolors[crayola] 			   % Paleta de cores pré-definida: wiki.contextgarden.net/Color#Pre-defined_colors

% Cores definidas pelo designer:
% MyGreen		r=0.251, g=0.678, b=0.290 % 40ad4a
% MyCyan		r=0.188, g=0.749, b=0.741 % 30bfbd
% MyRed			r=0.820, g=0.141, b=0.161 % d12429
% MyPink		r=0.980, g=0.780, b=0.761 % fac7c2
% MyGray		r=0.812, g=0.788, b=0.780 % cfc9c7
% MyOrange		r=0.980, g=0.671, b=0.290 % faab4a

% Configuração de cores
\definecolor[MyColor][black]      % ou ex: [r=0.862, g=0.118, b=0.118] % corresponde a RGB(220, 30, 30)
\definecolor[MyColorText][white]     % ou ex: [r=0.862, g=0.118, b=0.118] % corresponde a RGB(167, 169, 172)

% Classe para diagramação dos posts
\environment{marketing.env}		   
\def\startMyCampaign{\bgroup
            \FormularMI
            \switchtobodyfont[26pt]
            \setupinterlinespace[line=1.9ex]
            \setcharacterkerning[packed]}
\def\stopMyCampaign{\par\egroup}

\starttext %---------------------------------------------------------|

\hyphenation{tornase}
\Mensagem{DESTAQUE}

\startMyCampaign

\hyphenpenalty=10000
\exhyphenpenalty=10000
«O negro torna-se

 um corpo no qual realizou-se a mais absoluta visão expropriadora;  certamente, isto é o que fundamenta a visão de mundo, não só das vítimas dessa 

\vfill\hfill →

\page

redução ontológica, como do corpo social diferenciado por este significante identitário redutor. Em nenhum momento pode-se afirmar que essa redução de indivíduos ao corpo é uma

\vfill\hfill →
\page
anomalia pertencente ao passado, mas é, sim, a força motora da modernidade capitalista.»


\stopMyCampaign

{\vfill\scale[factor=6]{\Seta\,Trecho do livro {\bf Lugar de negro, lugar de}}\setupinterlinespace[line=1.5ex]\scale[factor=6]{{\bf branco?}, de Douglas Rodrigues Barros.}}

\page %---------------------------------------------------------| 

\MyCover{BARROS_NEGRO_THUMB}

\page %---------------------------------------------------------|

\Hedra

\stoptext %---------------------------------------------------------|
