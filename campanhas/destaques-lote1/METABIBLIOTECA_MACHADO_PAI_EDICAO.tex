% METABILBIOTECA_MACHADO_PAI_EDICAO.tex
% Preencher com o nome das cor ou composição RGB (ex: [r=0.862, g=0.118, b=0.118]) 
\usecolors[crayola] 			   % Paleta de cores pré-definida: wiki.contextgarden.net/Color#Pre-defined_colors

% Cores definidas pelo designer:
% MyGreen		r=0.251, g=0.678, b=0.290 % 40ad4a
% MyCyan		r=0.188, g=0.749, b=0.741 % 30bfbd
% MyRed			r=0.820, g=0.141, b=0.161 % d12429
% MyPink		r=0.980, g=0.780, b=0.761 % fac7c2
% MyGray		r=0.812, g=0.788, b=0.780 % cfc9c7
% MyOrange		r=0.980, g=0.671, b=0.290 % faab4a

% Configuração de cores
\definecolor[MyColor][Eggplant]      % ou ex: [r=0.862, g=0.118, b=0.118] % corresponde a RGB(220, 30, 30)
\definecolor[MyColorText][Canary]  % ou ex: [r=0.862, g=0.118, b=0.118] % corresponde a RGB(167, 169, 172)

% Classe para diagramação dos posts
\environment{marketing.env}		   

\starttext %---------------------------------------------------------|

\Mensagem{POR DENTRO DA EDIÇÃO}

\startMyCampaign

\hyphenpenalty=10000
\exhyphenpenalty=10000

{
\setupinterlinespace[2ex]
\kern-1.5mm 
{\bf MACHADO 

DE ASSIS}

E AS FISSURAS HISTÓRICAS 

DE UMA

MODERNIZAÇÃO TROPICAL

}

\stopMyCampaign

%\vfill\scale[lines=1.5]{\MyStar[MyColorText][none]}

\page %---------------------------------------------------------| 

\MyCover{METABIBLIOTECA_MACHADO_PAI_THUMB}

\page %---------------------------------------------------------| 

\hyphenpenalty=10000
\exhyphenpenalty=10000

{\bf PAI CONTRA MÃE E OUTROS CONTOS} é uma compilação de 33 narrativas breves do escritor carioca. Os contos abordam as {\bf CONTRADIÇÕES} de um Brasil que tenta se modernizar mas carrega arcaísmos herdados da colonização, como a escravidão e a política elitizada.

\page %---------------------------------------------------------|

 A coletânea, organizada por {\bf ALEXANDRE ROSA}, {\bf FLÁVIO RICARDO VASSOLER} e {\bf IEDA LEBENSZTAYN}, conta com títulos como {\it A cartomante}, {\it A causa secreta} e {\it O dicionário}.
\page

\hyphenpenalty=10000
\exhyphenpenalty=10000

«Ora, pegar escravos fugidios [\unknown] não seria nobre,
mas por ser instrumento da força com que se mantêm a lei e a
propriedade, trazia esta outra nobreza implícita das ações
reivindicadoras. [\unknown] a pobreza, a inaptidão para outros trabalhos, o acaso, e alguma vez o gosto de servir também, ainda que por
outra via, davam o impulso ao homem que se sentia bastante rijo para pôr
ordem à desordem.»


\vfill\scale[factor=fit]{{\bf MACHADO DE ASSIS} \sl Pai contra mãe}

\page %---------------------------------------------------------|

% \newcommand\copyrightorganizacao{Alexandre Rosa, Flávio Ricardo Vassoler, Ieda Lebensztayn}

% \textbf{Alexandre Rosa}, escritor e
%   cientista social formado pela \versal{FFLCH-USP}, é mestre em Literatura
%   Brasileira pelo Instituto de Estudos Brasileiros da \versal{USP}. Participou da
%   Coleção Ensaios Brasileiros Contemporâneos -- Volume Música (Funarte,
%   2017), com o ensaio \emph{Três Raps de São Paulo}, em parceria com
%   Guilherme Botelho e Walter Garcia.

% \textbf{Flávio Ricardo Vassoler} é doutor em Letras pela \versal{FFLCH-USP}, com estágio doutoral junto à
%   Northwestern University (\versal{EUA}). É autor das obras literárias \emph{Tiro
%   de Misericórdia} (nVersos, 2014) e \emph{O Evangelho segundo Talião}
%   (nVersos, 2013) e organizador do livro de ensaios \emph{Dostoiévski e
%   Bergman: o niilismo da modernidade} (Intermeios, 2012).

% \textbf{Ieda Lebensztayn} é
%   pesquisadora de pós"-doutorado na Biblioteca Brasiliana Mindlin,
%   \versal{BBM-USP/FFLCH-USP} (Processo \versal{CNP}q 166032/2015-8), com estudo a
%   respeito da recepção literária de Machado de Assis. Doutora em
%   Literatura Brasileira pela \versal{FFLCH-USP}, fez pós"-doutorado no \versal{IEB-USP}
%   sobre a correspondência de Graciliano Ramos. Autora de
%   \emph{Graciliano Ramos e a} Novidade\emph{: o astrônomo do inferno e
%   os meninos impossíveis} (Hedra, 2010). Organizou, com Thiago Mio
%   Salla, os livros \emph{Cangaços} e \emph{Conversas}, de Graciliano
%   Ramos, publicados em 2014 pela Record.
  
\Hedra

\stoptext %---------------------------------------------------------|