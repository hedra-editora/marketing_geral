% Preencher com o nome das cor ou composição RGB (ex: [r=0.862, g=0.118, b=0.118]) 
\usecolors[crayola] 			   % Paleta de cores pré-definida: wiki.contextgarden.net/Color#Pre-defined_colors

% Cores definidas pelo designer:
% MyGreen		r=0.251, g=0.678, b=0.290 % 40ad4a
% MyCyan		r=0.188, g=0.749, b=0.741 % 30bfbd
% MyRed			r=0.820, g=0.141, b=0.161 % d12429
% MyPink		r=0.980, g=0.780, b=0.761 % fac7c2
% MyGray		r=0.812, g=0.788, b=0.780 % cfc9c7
% MyOrange		r=0.980, g=0.671, b=0.290 % faab4a

% Configuração de cores
\definecolor[MyColor][x=f9d3b9]      % ou ex: [r=0.862, g=0.118, b=0.118] % corresponde a RGB(220, 30, 30)
\definecolor[MyColorText][black]  % ou ex: [r=0.862, g=0.118, b=0.118] % corresponde a RGB(167, 169, 172)

% Classe para diagramação dos posts
\environment{marketing.env}		   


% Comandos & Instruções %%%%%%%%%%%%%%%%%%%%%%%%%%%%%%%%%%%%%%%%%%%%%%%%%%%%%%%%%%%%%%%|

% Cabeço e rodabé: Informações (caso queira trocar alguma coisa)
% 		\def\MensagemSaibaMais{SAIBA MAIS:}
% 		\def\MensagemSite{HEDRA.COM.BR}
% 		\def\MensagemLink{LINK NA BIO}

% Pesos para os títulos:
%		\startMyCampaign...		 \stopMyCampaign
%		\stopMyCampaignSection...   \stopMyCampaignSection

% Aplicação de imagens: 
% 		\MyCover{capa.pdf}  	% Aplicação de capa de livro com sombra
%		\MyPicture{Imagem.png}  % Imagem com aplicação de filtro segundo cor MyColorText
%		\MyPhoto{}			    % Aplicação simples de imagem com tamamho \textwidth

% Aplicação de imagem com legenda:		
% 		\placefigure{Legenda}{\externalfigure[drop2-1.png][width=\textwidth]}

% Cabeço e rodabé: Opções
% 		\Mensagem{AGORA É QUE SÃO ELAS}
% 		\Hashtag{campanha de natal}
% 		\Mensagem{campanha de natal}

% Alteração de várias cores de background:
% \setupbackgrounds[page][background=color,backgroundcolor=MyGray]

% Estrela: 
% \vfill\scale[lines=2]{\MyStar[MyColorText][none]} 					% Estrela pequena  
% \startpositioning 											% Estrela grande
%  \position(-1,-.3){\scale[scale=980]{\MyStar[white][none]}}
% \stoppositioning

% Logos e selos: 				
% \Hedra
% \HedraAyllon	% Não está pronto
% \HedraAcorde	% Não está pronto
% \Ayllon		% Não está pronto
% \Acorde		% Não está pronto

% Atalhos: 						
% 		\Seta  % Seta para baixo

%%%%%%%%%%%%%%%%%%%%%%%%%%%%%%%%%%%%%%%%%%%%%%%%%%%%%%%%%%%%%%%%%%%%%%%%%%%%%%%%%%%%%%%|

\starttext
%\showframe  %Para mostrar somente as linhas.

\Mensagem{DESTAQUES}

\MyCover{BENJAMIN_CONTADOR_THUMB.pdf}



\page %---------------------------------------------------------|

{\MyPicture{BENJAMIN_CONTADOR_1.jpeg}}

\vfill
\scale[factor=fit]{Tradução do alemão de {\bf Patrícia Lavelle}}

\page 
\hyphenpenalty=10000
\exhyphenpenalty=10000

 «Contar histórias, na verdade, não
 é apenas uma arte, é muito mais uma
 dignidade, se é que não é, como no
 Oriente, um ofício. Contar termina em
 uma sabedoria, assim como por outro
 lado a sabedoria muitas vezes se revela
 numa narrativa. O contador de histórias
 é, portanto, alguém que sempre sabe dar
 conselhos. E, para recebê-los, é preciso
 que também se conte algo a ele.»  W.\,B.




\page %---------------------------------------------------------|

{\MyPicture{BENJAMIN_CONTADOR_4.jpeg}}

{\it Situado em
uma época marcada pelo empobrecimento ou
declínio da experiência, da memória coletiva
e do patrimônio cultural transmitido de
geração a geração, {\bf O contador de histórias}
aborda principalmente o esvanecimento dessa
tradicional figura em meio à mudança de modos
de produção, do artesanal ao industrial, bem
como o surgimento da comunicação de massa
na sociedade moderna.} \blank[1ex]

{\hfill\tf ---Carla Milani Damião}

\page



\Hedra

\stoptext

% Trecho
% =======
% Por que a arte de contar histórias está
% chegando ao fim? — eu já me fizera
% essa pergunta muitas vezes enquanto
% me entediava, sentado com outros
% convidados durante uma noite inteira
% em torno de uma mesa. Naquela tarde,
% porém, quando estava em pé no convés
% de passeio do “Bellver”, ao lado da
% cabine do timoneiro, e buscava com meu
% excelente binóculo todos os aspectos
% da imagem inigualável que Barcelona
% oferecia do alto do navio, acreditei ter
% encontrado a resposta para ela. O Sol
% descia sobre a cidade e parecia derretê-la. 
% Tudo que era vida havia se recolhido
% nas gradações cinza claro entre a copa
% das árvores, o cimento das construções
% e a rocha das montanhas distantes.
% (...) Contar histórias, na verdade, não
% é apenas uma arte, é muito mais uma
% dignidade, se é que não é, como no
% Oriente, um ofício. Contar termina em
% uma sabedoria, assim como por outro
% lado a sabedoria muitas vezes se revela
% numa narrativa. O contador de histórias
% é, portanto, alguém que sempre sabe dar
% conselhos. E, para recebê-los, é preciso
% que também se conte algo a ele.
%
% Walter Benjamin, “O lenço”

% Esta nova edição do famoso ensaio
% traduzido anteriormente como O narrador —
% por nomes como Erwin Theodor Rosenthal,
% Modesto Carone e Sérgio Paulo Rouanet —
% permite compreender de maneira aprofundada
% o contador de histórias, bem como o lugar que
% a narrativa oral ocupa entre os gêneros da
% epopeia, romance e short story. Situado em
% uma época marcada pelo empobrecimento ou
% declínio da experiência, da memória coletiva
% e do patrimônio cultural transmitido de
% geração a geração, O contador de histórias
% aborda principalmente o esvanecimento dessa
% tradicional figura em meio à mudança de modos
% de produção, do artesanal ao industrial, bem
% como o surgimento da comunicação de massa
% na sociedade moderna. (Carla Milani Damião)


% Walter Benjamin (1892-1940) foi um filósofo, crítico literário, tra-
% dutor (de Baudelaire, Proust e Balzac, entre outros) e também um
% ficcionista alemão. Estudou filosofia num ambiente dominado pelo
% neokantismo, em Berlim, Freiburg, Munique e Berna, onde defendeu
% tese de doutorado sobre os primeiros românticos alemães. Durante
% o seu exílio em Paris, nos anos trinta, foi ligado ao Instituto de Pes-
% quisa Social, embrião da chamada Escola de Frankfurt. Entre seus
% interlocutores e amigos, encontram-se personalidades marcantes do
% século xx como Theodor W. Adorno, Hannah Arendt, Bertolt Brecht
% e Gershom Scholem.
% 
% O contador de histórias e outros textos propõe uma nova tradu-
% ção anotada do clássico ensaio no qual Walter Benjamin esboça a
% figura do contador de histórias a partir de um comentário crítico
% do contista russo Nikolai Leskov; reúne também a pouco conhecida
% produção ficcional do próprio ensaísta, trazendo o conjunto de seus
% contos, alguns inéditos em português. O volume inclui ainda peças
% que Benjamin produziu para o rádio, deslocando a arte tradicional
% de contar histórias para a cena moderna, e textos híbridos, onde o
% crítico faz obra de ficção ou o contador de histórias filosofa.
% 
% Patrícia Lavelle é Professora do Departamento de Letras da puc-Rio,
% atuando no Programa de Pós-graduação em Literatura, Cultura e
% Contemporaneidade. É também Pesquisadora Associada à ehess-Pa-
% ris, onde defendeu doutorado em Filosofia e deu aulas. Sua tese foi
% publicada em livro: Religion et histoire. Sur le concept d’expérience
% chez Walter Benjamin. Paris: Cerf, col. Passages, 2008. Entre outros
% volumes coletivos, organizou Walter Benjamin. Paris: L’Herne, col.
% Cahiers de l’Herne, 2013.


% Coleção Walter Benjamin é um projeto acadêmico-editorial que
% envolve pesquisa, tradução e publicação de obras e textos seletos
% desse importante filósofo, crítico literário e historiador da cultura
% judeu-alemão, em volumes organizados por estudiosos versados em
% diferentes aspectos de sua obra, vida e pensamento. (Amon Pinho \& 
% Francisco De Ambrosis Pinheiro Machado)