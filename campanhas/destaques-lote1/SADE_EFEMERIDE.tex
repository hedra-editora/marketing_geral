% SADE_EFEMERIDE.tex
% Preencher com o nome das cor ou composição RGB (ex: [r=0.862, g=0.118, b=0.118]) 
\usecolors[crayola] 			   % Paleta de cores pré-definida: wiki.contextgarden.net/Color#Pre-defined_colors

% Cores definidas pelo designer:
% MyGreen		r=0.251, g=0.678, b=0.290 % 40ad4a
% MyCyan		r=0.188, g=0.749, b=0.741 % 30bfbd
% MyRed			r=0.820, g=0.141, b=0.161 % d12429
% MyPink		r=0.980, g=0.780, b=0.761 % fac7c2
% MyGray		r=0.812, g=0.788, b=0.780 % cfc9c7
% MyOrange		r=0.980, g=0.671, b=0.290 % faab4a

% Configuração de cores
\definecolor[MyColor][Carmine]      % ou ex: [r=0.862, g=0.118, b=0.118] % corresponde a RGB(220, 30, 30)
\definecolor[MyColorText][black]     % ou ex: [r=0.862, g=0.118, b=0.118] % corresponde a RGB(167, 169, 172)

% Classe para diagramação dos posts
\environment{marketing.env}		   

\starttext %---------------------------------------------------------|

\hyphenpenalty=10000
\exhyphenpenalty=10000

\Mensagem{2 DE JUNHO} %Sempre usar esse header

\MyPicture{SADE_EFEMERIDE_1}

\vfill\scale[factor=6]{\Seta\,284 ANOS DE {\bf MARQUÊS DE SADE}}

\page %---------------------------------------------------------| 

\hyphenpenalty=10000
\exhyphenpenalty=10000

Nobre, político revolucionário, filósofo e escritor francês, {\bf MARQUÊS DE SADE} foi conhecido principalmente pela sua sexualidade libertina e obras literárias eróticas.
Durante muito tempo foi chamado de «monstro» e sua obra considerada «maldita» e «pornográfica».

\page %---------------------------------------------------------|

Do seu nome derivou o termo «{\bf SADISMO}», devido à representação literária que fez de fantasias sexuais violentas

\page

A exploração ficcional de uma ampla gama de {\bf DESVIOS SEXUAIS} e uma vida de libertinagem escandalosa fizeram com que {\bf SADE} fosse encarcerado várias vezes em prisões e asilos, onde viveu por cerca de 30 anos.

\page
 
 Há um fascínio entorno da figura e obra de {\bf SADE}, a qual serviu de objeto de estudo para importantes intelectuais franceses, como Roland Barthes, Jacques Derrida e Michel Foucault, e foi adaptada por renomados diretores, como Pasolini.

\page

\Hedra

\stoptext %---------------------------------------------------------|


% Donatien Alphonse François de Sade, o Marquês de Sade (Paris, 2 de junho de 1740 – Saint-Maurice, 2 de dezembro de 1814) foi um nobre, político revolucionário, filósofo e escritor francês famoso por sua sexualidade libertina. Suas obras incluem romances, contos, peças de teatro, diálogos e tratados políticos. Durante sua vida, alguns deles foram publicados em seu próprio nome, enquanto outros, que Sade negou ter escrito, apareceram anonimamente. Ele é mais conhecido por suas obras eróticas, que combinavam discurso filosófico com pornografia, retratando fantasias sexuais com ênfase na violência, sofrimento, sexo anal (que ele chama de sodomia), crime e blasfêmia (contra o Cristianismo). Ele era um defensor da liberdade absoluta, sem restrições de moralidade, religião ou lei. As palavras sadismo e sádico são derivadas em referência às obras de ficção que ele escreveu, que retratavam vários atos de crueldade sexual. Enquanto Sade explorava mentalmente uma ampla gama de desvios sexuais, seu comportamento conhecido inclui "apenas o espancamento de uma empregada doméstica e uma orgia com várias prostitutas — comportamento que diverge significativamente da definição clínica de sadismo".[1][2] Sade era um defensor de bordéis públicos gratuitos fornecidos pelo Estado: a fim de evitar crimes na sociedade que são motivados pela luxúria e para reduzir o desejo de oprimir outros usando seu próprio poder, Sade recomendava bordéis públicos onde as pessoas poderiam satisfazer seus desejos.[3]

% Sem nenhuma acusação legal contra ele,[1] Sade foi encarcerado em várias prisões e um asilo de loucos por cerca de 32 anos de sua vida: 11 anos em Paris (10 dos quais foram passados na Bastilha), um mês na Conciergerie, dois anos em uma fortaleza, um ano no Convento Madelonnettes, três anos no Asilo Bicêtre, um ano na Prisão de Sainte-Pélagie e 12 anos no Asilo Charenton. Durante a Revolução Francesa, ele foi um delegado eleito na Convenção Nacional. Muitas de suas obras foram escritas na prisão.

% Continua a haver um fascínio por Sade entre os estudiosos e na cultura popular. Intelectuais franceses prolíficos como Roland Barthes, Jacques Derrida e Michel Foucault publicaram estudos sobre ele.[4] Por outro lado, o filósofo hedonista francês Michel Onfray atacou esse interesse, escrevendo que "é intelectualmente bizarro fazer de Sade um herói".[5] Também houve inúmeras adaptações cinematográficas de sua obra, a mais notável sendo Salò de Pasolini, uma adaptação do polêmico livro de Sade, 120 Dias de Sodoma.



% % Por muitos anos, os descendentes de Sade consideraram sua vida e trabalho um escândalo a ser reprimido. Isso não mudou até meados do século XX, quando o conde Xavier de Sade reivindicou o título de marquês, há muito caído em desuso, em seus cartões de visita[11] e se interessou pelos escritos de seu ancestral.


% A partir de 1763, Sade viveu principalmente em ou perto de Paris. Por causa de sua infâmia sexual, ele foi colocado sob vigilância da polícia, que fazia relatórios detalhados de suas atividades. Após várias curtas detenções, que incluíram um breve encarceramento no Château de Saumur (então uma prisão), ele foi exilado em seu château em Lacoste em 1768.[14]

% Nove anos depois, em 1772, Sade cometeu atos sexuais que incluíam sodomia com quatro prostitutas e seu criado, Latour.[15] Os dois homens foram condenados à morte à revelia por sodomia. Eles fugiram para a Itália, Sade levando a irmã de sua esposa com ele. Sade e Latour foram capturados e presos na Fortaleza de Miolans, na Saboia francesa, no final de 1772, mas escaparam quatro meses depois. 