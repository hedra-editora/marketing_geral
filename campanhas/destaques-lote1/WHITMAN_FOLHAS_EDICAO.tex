% AUTOR_LIVRO_EDICAO.tex
% Preencher com o nome das cor ou composição RGB (ex: [r=0.862, g=0.118, b=0.118]) 
\usecolors[crayola] 			   % Paleta de cores pré-definida: wiki.contextgarden.net/Color#Pre-defined_colors

% Cores definidas pelo designer:
% MyGreen		r=0.251, g=0.678, b=0.290 % 40ad4a
% MyCyan		r=0.188, g=0.749, b=0.741 % 30bfbd
% MyRed			r=0.820, g=0.141, b=0.161 % d12429
% MyPink		r=0.980, g=0.780, b=0.761 % fac7c2
% MyGray		r=0.812, g=0.788, b=0.780 % cfc9c7
% MyOrange		r=0.980, g=0.671, b=0.290 % faab4a

% Configuração de cores
\definecolor[MyColor][Amethyst]      % ou ex: [r=0.862, g=0.118, b=0.118] % corresponde a RGB(220, 30, 30)
\definecolor[MyColorText][white]     % ou ex: [r=0.862, g=0.118, b=0.118] % corresponde a RGB(167, 169, 172)

% Classe para diagramação dos posts
\environment{marketing.env}		   


\def\startMyCampaign{\bgroup
            \FormularMI
            \switchtobodyfont[26pt]
            \setupinterlinespace[line=1.9ex]
            \setcharacterkerning[packed]}
\def\stopMyCampaign{\par\egroup}

\starttext %---------------------------------------------------------|

\Mensagem{NOVAS FORMAS POÉTICAS}

\startMyCampaign

\hyphenpenalty=10000
\exhyphenpenalty=10000

{\bf 
UM TESTEMUNHO LITERÁRIO
DA MODERNIZAÇÃO
NORTE-AMERICANA}

\stopMyCampaign

%\vfill\scale[lines=1.5]{\MyStar[MyColorText][none]}

\page %---------------------------------------------------------| 

\MyCover{WHITMAN_FOLHAS_THUMB}

\page %---------------------------------------------------------| 

\hyphenpenalty=10000
\exhyphenpenalty=10000

Esta edição de {\bf FOLHAS DE RELVA}, livro fundador da poesia norte-americana moderna, é o resultado que {\bf WALT WHITMAN} chegou depois da publicação de {\bf SETE DIFERENTES VERSÕES} do livro, de 1855 a 1891.

\page

 Ridicularizada pela crítica por seu {\bf CARÁTER EXPERIMENTAL} e temáticas polêmicas, a obra aborda um período em que os {\cap EUA} aproximavam-se da {\bf MODERNIDADE}, ao mesmo tempo em que\\
  davam os {\bf PRIMEIROS PASSOS IMPERIALISTAS}. 

\page

Finalizada apenas uma semana antes da morte de {\bf WHITMAN}, a versão final do livro marca a conclusão do processo de {\bf PRODUÇÃO E REDEFINIÇÃO DO MATERIAL POÉTICO}, e explora, em toda sua amplitude, as contradições do processo de modernização norte-americano.


\page %---------------------------------------------------------|


\hyphenpenalty=10000
\exhyphenpenalty=10000

«A {\bf EDIÇÃO DO LEITO DE MORTE} marca a consolidação de um percurso tortuoso que condensa algumas das {\bf MAIS IMPORTANTES CONQUISTAS DA TÉCNICA LITERÁRIA MODERNA} e a busca de uma incerta {\bf POESIA NACIONAL} em um momento crítico da sociedade norte-americana.»


{\vfill\scale[factor=5]{{\bf Bruno Gambarotto}, na introdução de {\bf Folhas de relva}.}}

\page %---------------------------------------------------------|

\Hedra

\stoptext %---------------------------------------------------------|

