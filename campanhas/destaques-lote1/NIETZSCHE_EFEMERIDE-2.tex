\usecolors[crayola]

% Configuração de cores TickleMePink
\definecolor[MyColor][VividTangerine]      % ou ex: [r=0.862, g=0.118, b=0.118] % corresponde a RGB(220, 30, 30)
\definecolor[MyColorText][black]  % ou ex: [r=0.862, g=0.118, b=0.118] % corresponde a RGB(167, 169, 172)

% Classe para diagramação dos posts
\environment{marketing.env}        

% Cabeço e rodapé: Informações (caso queira trocar alguma coisa)
        \def\MensagemSaibaMais  {SAIBA MAIS:}
        \def\MensagemSite       {HEDRA.COM.BR}
        \def\MensagemLink       {LINK NA BIO}
      
\environment{extra.env}

\starttext  %---------------------------------------------------------|

\def\MyBackgroundMessage{25 DE AGOSTO}
\MyBackground{NIETZSCHE_BACKGROUND.jpeg}

\startMyCampaign
\hyphenpenalty=10000
\exhyphenpenalty=10000

\position(0,7.8){\scale[factor=3]{\Seta\,{\bf FRIEDRICH NIETZSCHE} (1844--1900)}}
\stopMyCampaign

\page %----------------------------------------------------------|

\Mensagem{25 DE AGOSTO}

\setupbackgrounds[page][background=color,backgroundcolor=MyColor]

Friedrich Nietzsche, filósofo e filólogo alemão, nasceu em uma família de pastores protestantes. Crítico mordaz da cultura ocidental, tornou-se professor de Letras Clássicas aos 25 anos, na Universidade da Basileia. Lá também se aproximou do compositor Richard Wagner. Serviu como enfermeiro voluntário na guerra franco-prussiana, mas contraiu difteria, que lhe prejudicou a saúde em definitivo --- até sua morte, em {\bf 25 DE AGOSTO DE 1900}. %Ao retornar a Basileia, intensificou a frequência à casa de Wagner.

\page %----------------------------------------------------------|

Em 1879, devido a constantes recaídas, deixou a universidade e passou a receber uma renda anual. A partir daí assumiu uma vida errante, dedicando-se exclusivamente à reflexão e à redação de suas obras. Em 1889, manifestaram-se os primeiros sintomas de problemas mentais, provavelmente decorrentes de sífilis.

\page %----------------------------------------------------------|

Dentre suas principais obras, se destacam «O nascimento da tragédia», «Assim falava Zaratustra», «Para além do bem e mal», «A genealogia da moral» e «O anticristo». Além de {\bf «SOBRE A UTILIDADE E A DESVANTAGEM DA HISTÓRIA PARA A VIDA»}, uma de suas considerações extemporâneas.

\page %----------------------------------------------------------|

\MyCover{NIETZSCHE_UTILIDADE_THUMB.png}

\page %----------------------------------------------------------|

\Hedra

\stoptext %---------------------------------------------------------|