% AUTOR_LIVRO_AUTOR.tex
% Preencher com o nome das cor ou composição RGB (ex: [r=0.862, g=0.118, b=0.118]) 
\usecolors[crayola] 			   % Paleta de cores pré-definida: wiki.contextgarden.net/Color#Pre-defined_colors

% Cores definidas pelo designer:
% MyGreen		r=0.251, g=0.678, b=0.290 % 40ad4a
% MyCyan		r=0.188, g=0.749, b=0.741 % 30bfbd
% MyRed			r=0.820, g=0.141, b=0.161 % d12429
% MyPink		r=0.980, g=0.780, b=0.761 % fac7c2
% MyGray		r=0.812, g=0.788, b=0.780 % cfc9c7
% MyOrange		r=0.980, g=0.671, b=0.290 % faab4a

% Configuração de cores
\definecolor[MyColor][BlueBell]	  %[r=0.862, g=0.118, b=0.118]      % ou ex: [r=0.862, g=0.118, b=0.118] % corresponde a RGB(220, 30, 30)
\definecolor[MyColorText][black] [r=0.655, g=0.663, b=0.675]      % ou ex: [r=0.862, g=0.118, b=0.118] % corresponde a RGB(167, 169, 172)

% Classe para diagramação dos posts
\environment{marketing.env}		   

% Cabeço e rodapé: Informações (caso queira trocar alguma coisa)
 		\def\MensagemSaibaMais  {SAIBA MAIS:}
 		\def\MensagemSite		{HEDRA.COM.BR}
 		\def\MensagemLink       {LINK NA BIO}


\def\startMyCampaign{\bgroup
            \FormularMI
            \switchtobodyfont[22pt]
            \setupinterlinespace[line=1.9ex]
            \setcharacterkerning[packed]}
\def\stopMyCampaign{\par\egroup}

\starttext %--------------------------------------------------------|

\Mensagem{DESTAQUE}

\hyphenpenalty=10000
\exhyphenpenalty=10000


\startMyCampaign

%\kern-2.5ex


%\vfill\scale[factor=6]{\Seta\,JOÃO BERNARDO}%

%\page %----------------------------------------------------------|%

\hyphenpenalty=10000
\exhyphenpenalty=10000

%«Após a segunda guerra mundial a forma mais perversa de fascismo tem sido a sua existência furtiva.»

%O resultado mais trágico foi a infiltração operada pelas ideias fascistas, tanto mais fácil quanto raramente se apresentam como tais. 

%A prova de que o fascismo não foi uma anomalia na evolução do capitalismo, que tivesse sido depois corrigida pelo curso da história, é que ele foi assimilado pelos regimes que o venceram e lhe sucederam. A difusão alcançada hoje nos meios universitários e jornalísticos pelo pós-modernismo e a hegemonia adquirida pela ecologia e pelos vários identitarismos, que reataram o fascismo e inauguraram um fascismo pós-fascista, como adiante analisarei com algum detalhe18 , não teria sido possível se a generalidade do público soubesse o que foi o fascismo. 

«O fascismo raramente é identificado quando surge na vida corrente, e por isso, ao mesmo tempo que a sociedade contemporânea exorciza como objecto de aversão uma imagem banalizada do fascismo, incorpora um fascismo pós-fascista.»


\stopMyCampaign

%Os grandes temas do fascismo, desarticulados do seu referencial, penetraram todo o espectro de opiniões e comportamentos da sociedade contemporânea, não só da direita e da esquerda mas mesmo daqueles, e são a maioria, que julgam não professar nenhumas ideias políticas. E certamente, nos seus termos estritos, essas ideias não são políticas, porque este é um fascismo sem nome. Mas não corresponderá ele, por isso mesmo, à essência do fascismo? 

\scale[factor=6]{\Seta\,{\bf Labirintos do fascismo: metamorfoses}}\setupinterlinespace[line=1.5ex]\scale[factor=6]{{\bf  do fascismo}, João Bernardo.}

\page

\MyCover{BERNARDO_LABIRINTOS_THUMB6.jpeg}

\page %----------------------------------------------------------|

\Hedra

\stoptext %---------------------------------------------------------|