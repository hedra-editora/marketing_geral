% DOSTO_DIARIO1873_CURIOSIDADE.tex
% Vamos falar sobre isso "curiosidades"
% > "EM CONTEXTO"

% Preencher com o nome das cor ou composição RGB (ex: [r=0.862, g=0.118, b=0.118]) 
\usecolors[crayola] 			   % Paleta de cores pré-definida: wiki.contextgarden.net/Color#Pre-defined_colors

% Cores definidas pelo designer:
% MyGreen		r=0.251, g=0.678, b=0.290 % 40ad4a
% MyCyan		r=0.188, g=0.749, b=0.741 % 30bfbd
% MyRed			r=0.820, g=0.141, b=0.161 % d12429
% MyPink		r=0.980, g=0.780, b=0.761 % fac7c2
% MyGray		r=0.812, g=0.788, b=0.780 % cfc9c7
% MyOrange		r=0.980, g=0.671, b=0.290 % faab4a

% Configuração de cores
\definecolor[MyColor][SteelTeal]      % ou ex: [r=0.862, g=0.118, b=0.118] % corresponde a RGB(220, 30, 30)
\definecolor[MyColorText][SunnyPearl]  % ou ex: [r=0.862, g=0.118, b=0.118] % corresponde a RGB(167, 169, 172)

% Classe para diagramação dos posts
\environment{marketing.env}		   

\starttext %---------------------------------------------------------|

\hyphenpenalty=10000
\exhyphenpenalty=10000

\Mensagem{EM CONTEXTO}

\startMyCampaign
\hyphenpenalty=10000
\exhyphenpenalty=10000

A PRISÃO DE
{\bf DOSTOIÉVSKI}

%\vfill\scale[lines=2]{\MyStar[MyColorText][none]} 					% Estrela pequena  

\stopMyCampaign

\page %---------------------------------------------------------|

{\MyPicture{DOSTO_DIARIO1873_2}

\page %---------------------------------------------------------| 


Em 1849, {\bf Dostoiévski} foi condenado à morte por seu envolvimento com um grupo literário radical que criticava o regime czarista. No entanto, minutos antes de sua execução, a sentença foi comutada para trabalhos forçados na Sibéria.

\page 

Assim, o escritor foi enviado para uma prisão na Sibéria, onde passou cerca de quatro anos cumprindo sua pena. Enquanto estava preso, {\bf Dostoiévski} foi exposto a condições extremamente hostis e à crueldade do sistema penal da época. 

\page

Essa experiência teve um impacto profundo em sua vida e em sua produção literária posterior, figurando em muitas de suas obras sob a forma de temas recorrentes, como culpa, redenção, sofrimento, dilemas morais e as complexidades da natureza humana.

\page %---------------------------------------------------------|

\MyCover{DOSTO_DIARIO1873_THUMB}

\page

{\bf Diário de um escritor (1873)} é o primeiro dos {\it Diários} de Dostoiévski, os quais reunem mais de mil páginas de ensaios, crônicas e contos produzidos pelo autor entre 1873 e 1881, originalmente para sua coluna jornalística de mesmo nome. Com esta obra, o leitor terá a chance de acompanhar o próprio processo criativo do autor, que constrói uma teoria estética ao mesmo tempo que a aplica, como observa Irineu Franco Perpetuo na apresentação desta edição.  

\page

\Hedra

\stoptext %---------------------------------------------------------|
			