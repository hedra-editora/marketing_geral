% Preencher com o nome das cor ou composição RGB (ex: [r=0.862, g=0.118, b=0.118]) 
\usecolors[crayola] 			   % Paleta de cores pré-definida: wiki.contextgarden.net/Color#Pre-defined_colors

% Cores definidas pelo designer:
% MyGreen		r=0.251, g=0.678, b=0.290 % 40ad4a
% MyCyan		r=0.188, g=0.749, b=0.741 % 30bfbd
% MyRed			r=0.820, g=0.141, b=0.161 % d12429
% MyPink		r=0.980, g=0.780, b=0.761 % fac7c2
% MyGray		r=0.812, g=0.788, b=0.780 % cfc9c7
% MyOrange		r=0.980, g=0.671, b=0.290 % faab4a

% Configuração de cores
\definecolor[MyColor][Almond]      % ou ex: [r=0.862, g=0.118, b=0.118] % corresponde a RGB(220, 30, 30)
\definecolor[MyColorText][black]  % ou ex: [r=0.862, g=0.118, b=0.118] % corresponde a RGB(167, 169, 172)

% Classe para diagramação dos posts
\environment{marketing.env}		   


% Comandos & Instruções %%%%%%%%%%%%%%%%%%%%%%%%%%%%%%%%%%%%%%%%%%%%%%%%%%%%%%%%%%%%%%%|

% Cabeço e rodabé: Informações (caso queira trocar alguma coisa)
% 		\def\MensagemSaibaMais{SAIBA MAIS:}
% 		\def\MensagemSite{HEDRA.COM.BR}
% 		\def\MensagemLink{LINK NA BIO}

% Pesos para os títulos:
%		\startMyCampaign...		 \stopMyCampaign
%		\stopMyCampaignSection...   \stopMyCampaignSection

% Aplicação de imagens: 
% 		\MyCover{capa.pdf}  	% Aplicação de capa de livro com sombra
%		\MyPicture{Imagem.png}  % Imagem com aplicação de filtro segundo cor MyColorText
%		\MyPhoto{}			    % Aplicação simples de imagem com tamamho \textwidth

% Aplicação de imagem com legenda:		
% 		\placefigure{Legenda}{\externalfigure[drop2-1.png][width=\textwidth]}

% Cabeço e rodabé: Opções
% 		\Mensagem{AGORA É QUE SÃO ELAS}
% 		\Hashtag{campanha de natal}
% 		\Mensagem{campanha de natal}

% Alteração de várias cores de background:
% \setupbackgrounds[page][background=color,backgroundcolor=MyGray]

% Estrela: 
% \vfill\scale[lines=2]{\MyStar[MyColorText][none]} 					% Estrela pequena  
% \startpositioning 											% Estrela grande
%  \position(-1,-.3){\scale[scale=980]{\MyStar[white][none]}}
% \stoppositioning

% Logos e selos: 				
% \Hedra
% \HedraAyllon	% Não está pronto
% \HedraAcorde	% Não está pronto
% \Ayllon		% Não está pronto
% \Acorde		% Não está pronto

% Atalhos: 						
% 		\Seta  % Seta para baixo

%%%%%%%%%%%%%%%%%%%%%%%%%%%%%%%%%%%%%%%%%%%%%%%%%%%%%%%%%%%%%%%%%%%%%%%%%%%%%%%%%%%%%%%|

\starttext
%\showframe  %Para mostrar somente as linhas.

\Mensagem{DESTAQUES}

\MyCover{BASBAUM_ARTE_THUMB.pdf}



\page %---------------------------------------------------------|

\MyPicture{BASBAUM_ARTE_2.jpeg}

\page 


O livro, organizado por Ricardo Basbaum, reúne quarenta ensaios de artistas,
pesquisadores, críticos e curadores de arte contemporâneos sobre arte
brasileira das duas primeiras décadas do século XXI. A obra está dividido em
seis seções: Panorama, Artevismo, Crítica/Curadoria, Pertencimento, Brasis e
Territórios em disputa.


% A partir desse amplo panorama, que aborda o ativismo nas artes, o papel da
% crítica e da curadoria no circuito artístico, as artes dos diferentes Brasis e
% sua relação com os projetos políticos em questão, percebe-se a função da arte
% em revisitar a história do país em um processo de autodescoberta de uma nação
% múltipla e díspar. No momento de ruptura democrática e civilizacional que pelo
% qual o país atravessa, a arte mostra-se atenta e ativa em sua função de
% desbravar o impensado.


\page

%\startpositioning
%                \position(0,3mm){
%                \clip[width=1cm, height=2cm, hoffset=3mm, voffset=5mm]{
%                \externalfigure[BASBAUM_ARTE_3-crop.pdf][height=\textwidth]}}	
%\stoppositioning


\startpositioning
                \position(-2cm,-5mm){
                \clip[height=7cm]{
                \externalfigure[BASBAUM_ARTE_3-crop.pdf][height=1.8\textheight]
                }}	
\stoppositioning

\page

\startpositioning
                \position(-2cm,3mm){
                \clip[height=7cm,voffset=7cm]{
                \externalfigure[BASBAUM_ARTE_3-crop.pdf][height=1.8\textheight]
                }}	
\stoppositioning

\page

	
\startpositioning
                \position(-1.3cm,-4mm){
                \clip[height=7cm]{
                \externalfigure[BASBAUM_ARTE_4-crop.pdf][height=1.8\textheight]
                }}	
\stoppositioning
	

\page 

\Hedra

\stoptext


% Para texto no post
% ==================
%
% Arte contemporânea brasileira: texturas, dicções, ficções, estratégias reúne 39 ensaios de
% artistas, críticos de arte e historiadores da arte contemporâneos. Os artigos se concentram nas
% produções artísticas realizadas nas últimas três décadas do século xx, de 1970 a 1999. Apesar
% das diferenças entre os textos — que seguem o caráter heterogêneo da arte — seus autores
% propõem um recorte que configure uma intervenção sobre o objeto analisado. Mais que
% comentários distantes ou desinteressados, trazem à tona uma estreita cumplicidade entre os
% escritores e seus objetos, revelando o perfil de uma ativa inserção e participação nas dinâmicas
% que sugerem. Para isso, o livro é dividido em duas partes. Na primeira, estão reunidos textos
% que relacionam um conjunto de artistas cuja atuação se revelou ou tem se revelado decisiva,
% nas últimas quatro décadas do século xx, para a demarcação dos rumos da arte brasileira
% contemporânea. Na segunda parte, estão os escritos que procuram alguma caracterização do
% campo contemporâneo, de modo a abrir possibilidades de atuação — nos níveis nacional e
% internacional — para a produção local.
% 
% 
% Ricardo Basbaum vive e trabalha no Rio de Janeiro. Artista, escritor, crítico e curador, traba-
% lha em torno das relações sociais e interpessoais, desenvolvendo uma abordagem comunicativa
% para impulsionar a circulação de ações e formas. Com diagramas, desenhos, textos e insta-
% lações cria dispositivos interativos nos quais a experiência pessoal e individual dos atores
% e observadores participantes desempenha papel relevante. Autor de Além da pureza visual
% (Zouk, 2006), Flying Letters Manifestos (com Alex Hamburger, Par(ent)esis, 2013), Manual do
% Artista-etc (Azougue, 2013) e Diagrams: 1994-ongoing (Errant Bodies Press, 2016). Professor
% do Departamento de Arte da uff.