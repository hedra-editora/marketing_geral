% AUTOR_LIVRO_EDICAO.tex
% Preencher com o nome das cor ou composição RGB (ex: [r=0.862, g=0.118, b=0.118]) 
\usecolors[crayola] 			   % Paleta de cores pré-definida: wiki.contextgarden.net/Color#Pre-defined_colors

% Cores definidas pelo designer:
% MyGreen		r=0.251, g=0.678, b=0.290 % 40ad4a
% MyCyan		r=0.188, g=0.749, b=0.741 % 30bfbd
% MyRed			r=0.820, g=0.141, b=0.161 % d12429
% MyPink		r=0.980, g=0.780, b=0.761 % fac7c2
% MyGray		r=0.812, g=0.788, b=0.780 % cfc9c7
% MyOrange		r=0.980, g=0.671, b=0.290 % faab4a

% Configuração de cores
\definecolor[MyColor][OrangeYellow]      % ou ex: [r=0.862, g=0.118, b=0.118] % corresponde a RGB(220, 30, 30)
\definecolor[MyColorText][black]     % ou ex: [r=0.862, g=0.118, b=0.118] % corresponde a RGB(167, 169, 172)

% Classe para diagramação dos posts
\environment{marketing.env}		   

\starttext %---------------------------------------------------------|

\Mensagem{POR DENTRO DA EDIÇÃO}

\startMyCampaign

\hyphenpenalty=10000
\exhyphenpenalty=10000

CONHECENDO {\bf O DESERTOR} DE MANUEL INÁCIO DA SILVA ALVARENGA

\stopMyCampaign

%\vfill\scale[lines=1.5]{\MyStar[MyColorText][none]}

\page %---------------------------------------------------------| 

\MyCover{ALVARENGA_DESERTOR_THUMB}

\page %---------------------------------------------------------| 

\hyphenpenalty=10000
\exhyphenpenalty=10000

Manuel Inácio da Silva Alvarenga foi um poeta luso-brasileiro do Brasil colonial associado ao movimento da «Arcádia Lusitana». Ao lado de Basílio da Gama, Alvarenga Peixoto e Gonzaga compõe a geração que Antonio Candido uniu sob o título {\bf «APOGEU DA FORMA»}.

\page

Em {\bf O DESERTOR}, os elementos estruturais do poema narrativo são\\ singularizados pelas façanhas de um grupo de estudantes que decide largar a faculdade de Coimbra para cultivar a indolência.

\page

A celebração em tom épico desse acontecimento sem menor importância guarda espaço, ainda, para {\bf TRAÇOS SATÍRICOS}, identificados sobretudo na celebração da reforma pombalina e na manifesta confiança nos poderes da ciência moderna.
\page
 
 Assim, a obra de Silva Alvarenga, transitando entre a poesia satírica e o poema heroi-cômico, constitui-se como um {\bf MARCO FUNDAMENTAL DA FORMAÇÃO DA LITERATURA BRASILEIRA}.


\page %---------------------------------------------------------|

\Hedra

\stoptext %---------------------------------------------------------|






















