% AUTOR_LIVRO_TRECHO.tex
% Preencher com o nome das cor ou composição RGB (ex: [r=0.862, g=0.118, b=0.118]) 
\usecolors[crayola] 			   % Paleta de cores pré-definida: wiki.contextgarden.net/Color#Pre-defined_colors

% Cores definidas pelo designer:
% MyGreen		r=0.251, g=0.678, b=0.290 % 40ad4a
% MyCyan		r=0.188, g=0.749, b=0.741 % 30bfbd
% MyRed			r=0.820, g=0.141, b=0.161 % d12429
% MyPink		r=0.980, g=0.780, b=0.761 % fac7c2
% MyGray		r=0.812, g=0.788, b=0.780 % cfc9c7
% MyOrange		r=0.980, g=0.671, b=0.290 % faab4a

% Configuração de cores
\definecolor[MyColor][black]      % ou ex: [r=0.862, g=0.118, b=0.118] % corresponde a RGB(220, 30, 30)
\definecolor[MyColorText][white]     % ou ex: [r=0.862, g=0.118, b=0.118] % corresponde a RGB(167, 169, 172)

% Classe para diagramação dos posts
\environment{marketing.env}		   
\def\startMyCampaign{\bgroup
            \FormularMI
            \switchtobodyfont[28pt]
            \setupinterlinespace[line=1.9ex]
            \setcharacterkerning[packed]}
\def\stopMyCampaign{\par\egroup}

\starttext %---------------------------------------------------------|

\hyphenation{tornase}
\Mensagem{DESTAQUE}

\startMyCampaign

\hyphenpenalty=10000
\exhyphenpenalty=10000
«É preciso dizer que o campo do saber contém em si um elemento de poder e, portanto, de disputa.
 Essa disputa, para nós --- críticos e militantes
\vfill\hfill →

\page

 antirracistas ---, não deve ser a da tentativa de tornar a teoria popular, senão a de fazer com que o popular se torne teórico.»


\stopMyCampaign

{\vfill\scale[factor=6]{\Seta\,Trecho do livro {\bf Lugar de negro, lugar de}}\setupinterlinespace[line=1.5ex]\scale[factor=6]{{\bf branco?}, de Douglas Rodrigues Barros.}}

\page %---------------------------------------------------------| 

\MyCover{BARROS_NEGRO_THUMB}

\page %---------------------------------------------------------|

\Hedra

\stoptext %---------------------------------------------------------|
