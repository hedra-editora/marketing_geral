% AUTOR_LIVRO_AUTOR.tex
% Preencher com o nome das cor ou composição RGB (ex: [r=0.862, g=0.118, b=0.118]) 
\usecolors[crayola] 			   % Paleta de cores pré-definida: wiki.contextgarden.net/Color#Pre-defined_colors

% Cores definidas pelo designer:
% MyGreen		r=0.251, g=0.678, b=0.290 % 40ad4a
% MyCyan		r=0.188, g=0.749, b=0.741 % 30bfbd
% MyRed			r=0.820, g=0.141, b=0.161 % d12429
% MyPink		r=0.980, g=0.780, b=0.761 % fac7c2
% MyGray		r=0.812, g=0.788, b=0.780 % cfc9c7
% MyOrange		r=0.980, g=0.671, b=0.290 % faab4a

% Configuração de cores
\definecolor[MyColor][PewterBlue]      % ou ex: [r=0.862, g=0.118, b=0.118] % corresponde a RGB(220, 30, 30)
\definecolor[MyColorText][TigersEye]  % ou ex: [r=0.862, g=0.118, b=0.118] % corresponde a RGB(167, 169, 172)


% Classe para diagramação dos posts
\environment{marketing.env}		   

% Cabeço e rodapé: Informações (caso queira trocar alguma coisa)
 		\def\MensagemSaibaMais  {SAIBA MAIS:}
 		\def\MensagemSite		{HEDRA.COM.BR}
 		\def\MensagemLink       {LINK NA BIO}

\starttext %--------------------------------------------------------|

\Mensagem{A POETA DA GRÉCIA ARCAICA}

\hyphenpenalty=10000
\exhyphenpenalty=10000

%\startMyCampaign

\MyPicture{SAFO_HINO_1}

%\stopMyCampaign

\vfill\scale[factor=6]{\Seta\,SAFO DE LESBOS ({\it c.} 630 a.\,C.--580 a.\,C)}

\page %----------------------------------------------------------|

\hyphenpenalty=10000
\exhyphenpenalty=10000

{\bf SAFO DE LESBOS} nasceu, segundo a tradição, de uma família aristocrática em Êresos, na costa ocidental da {\bf ILHA DE LESBOS}. 

\page

 \hyphenpenalty=10000
 \exhyphenpenalty=10000

Desde seu tempo, {\bf SAFO} figura entre os expoentes da poesia grega {\bf MÉLICA}, sendo o único nome feminino no conjunto de poetas da {\bf GRÉCIA ARCAICA}.

\page

Muitos outros dados sobre sua vida podem ser colhidos nos testemunhos antigos. Vistos de perto, porém, eles se mostram demasiado frágeis, {\bf CONTRADITÓRIOS}, anedóticos, configurando-se antes como peças de uma {\bf BIOGRAFIA FICCIONALIZANTE}, sempre em reconstrução, baseada no que nos restou da {\bf OBRA SÁFICA}.
\page %----------------------------------------------------------|

\MyCover{SAFO_HINO_THUMB}

\page %----------------------------------------------------------|

\Hedra

\stoptext %---------------------------------------------------------|

