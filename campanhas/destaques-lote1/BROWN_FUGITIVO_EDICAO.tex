% BROWN_FUGITIVO_EDICAO.tex
% Preencher com o nome das cor ou composição RGB (ex: [r=0.862, g=0.118, b=0.118]) 
\usecolors[crayola] 			   % Paleta de cores pré-definida: wiki.contextgarden.net/Color#Pre-defined_colors

% Cores definidas pelo designer:
% MyGreen		r=0.251, g=0.678, b=0.290 % 40ad4a
% MyCyan		r=0.188, g=0.749, b=0.741 % 30bfbd
% MyRed			r=0.820, g=0.141, b=0.161 % d12429
% MyPink		r=0.980, g=0.780, b=0.761 % fac7c2
% MyGray		r=0.812, g=0.788, b=0.780 % cfc9c7
% MyOrange		r=0.980, g=0.671, b=0.290 % faab4a

% Configuração de cores
\definecolor[MyColor][x=562613]      % ou ex: [r=0.862, g=0.118, b=0.118] % corresponde a RGB(220, 30, 30)
\definecolor[MyColorText][white]     % ou ex: [r=0.862, g=0.118, b=0.118] % corresponde a RGB(167, 169, 172)

% Classe para diagramação dos posts
\environment{marketing.env}		   

\starttext %---------------------------------------------------------|

\Mensagem{NARRAÇÃO E ESCRAVIDÃO}

\startMyCampaign

\hyphenpenalty=10000
\exhyphenpenalty=10000

{\bf 
UM TESTEMUNHO EM PRIMEIRA PESSOA DA ESCRAVIDÃO}

\stopMyCampaign

%\vfill\scale[lines=1.5]{\MyStar[MyColorText][none]}

\page %---------------------------------------------------------| 

\MyCover{NARRATIVAS_BROWN_THUMB.pdf}

\page %---------------------------------------------------------| 

\hyphenpenalty=10000
\exhyphenpenalty=10000

Publicada no bastião abolicionista de Boston pela Sociedade Antiescravista de Massachusetts em julho de 1847, a {\bf NARRATIVA DE WILLIAM WELLS BROWN, ESCRAVO FUGITIVO} é um apelo exaltado à abolição da escravidão nos Estados Unidos e, por consequência, em todo o mundo. 

\page %---------------------------------------------------------|

Redigido pelo ex-escravizado {\bf WILLIAM BROWN}, o livro comoveu os leitores e incitou o debate na época da sua primeira edição, vendendo 8000 exemplares em dois anos e sendo reeditado nove vezes em quatro décadas.

\page

A narrativa representa um testemunho em primeira pessoa da escravidão americana e coloca os leitores cara a cara com o ambiente de {\bf VIOLÊNCIA SOCIAL} que impactava radicalmente a personalidade, a família e o desenvolvimento moral entre os escravizados. 

\page

\hyphenpenalty=10000
\exhyphenpenalty=10000

«Poucas pessoas tiveram maior oportunidade para conhecer a escravidão em todos os seus aspectos mais terríveis do que William W. Brown. Ele esteve por trás dos panos. Visitou suas câmaras secretas. {\bf OS FERROS DELA PENETRARAM A SUA ALMA}.»

\scale[factor=fit]{\Seta\,{\bf J. C. Hathaway}, Prefácio a primeira edição}

\page %---------------------------------------------------------|

\Hedra

\stoptext %---------------------------------------------------------|



% Antes de seremexpostos à venda, os escravos foram vestidos e levadospara o pátio. Alguns foram colocados a dançar, alguns apular, alguns a cantar e alguns a jogar cartas. O objetivoera fazer com que parecessem alegres e felizes. Meu deverera garantir que eles estariam nessas situações antes dachegada dos compradores, e muitas vezes os pus a dançarenquanto seus rostos ainda estavam úmidos de lágrimas

