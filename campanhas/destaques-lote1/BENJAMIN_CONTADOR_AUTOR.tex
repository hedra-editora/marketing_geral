% AUTOR_LIVRO_AUTOR.tex
% Preencher com o nome das cor ou composição RGB (ex: [r=0.862, g=0.118, b=0.118]) 
\usecolors[crayola] 			   % Paleta de cores pré-definida: wiki.contextgarden.net/Color#Pre-defined_colors

% Cores definidas pelo designer:
% MyGreen		r=0.251, g=0.678, b=0.290 % 40ad4a
% MyCyan		r=0.188, g=0.749, b=0.741 % 30bfbd
% MyRed			r=0.820, g=0.141, b=0.161 % d12429
% MyPink		r=0.980, g=0.780, b=0.761 % fac7c2
% MyGray		r=0.812, g=0.788, b=0.780 % cfc9c7
% MyOrange		r=0.980, g=0.671, b=0.290 % faab4a

% Configuração de cores

\definecolor[MyColor][x=f9d3b9]      % ou ex: [r=0.862, g=0.118, b=0.118] % corresponde a RGB(220, 30, 30)
\definecolor[MyColorText][Manatee]  % ou ex: [r=0.862, g=0.118, b=0.118] % corresponde a RGB(167, 169, 172)

% Classe para diagramação dos posts
\environment{marketing.env}		   

% Cabeço e rodapé: Informações (caso queira trocar alguma coisa)
 		\def\MensagemSaibaMais  {SAIBA MAIS:}
 		\def\MensagemSite		{HEDRA.COM.BR}
 		\def\MensagemLink       {LINK NA BIO}

\starttext %--------------------------------------------------------|

\Mensagem{ENTRE LITERATURA E FILOSOFIA}

\hyphenpenalty=10000
\exhyphenpenalty=10000

%\startMyCampaign

\MyPicture{BENJAMIN_CONTADOR_1}

%\stopMyCampaign

\vfill\scale[factor=6]{\Seta\,WALTER BENJAMIN (1892--1940)}

\page %----------------------------------------------------------|

\hyphenpenalty=10000
\exhyphenpenalty=10000


Nascido em Berlim, em uma família judaica, {\bf WALTER BENJAMIN} foi filósofo, crítico literário, e também um
ficcionista alemão. Durante o seu {\bf EXÍLIO} em Paris, nos anos trinta, foi ligado ao Instituto de Pesquisa Social, embrião da chamada {\bf ESCOLA DE FRANKFURT}.


\page

 {\bf BENJAMIN} escreveu extensivamente sobre literatura, cinema, teatro, história e política e é conhecido por sua {\bf ABORDAGEM ÚNICA E INTERDISCIPLINAR}. Prolífico tradutor da literatura francesa, traduziu para a língua alemã obras de {\bf BAUDELAIRE}, {\bf PROUST} e {\bf BALZAC}, entre outros.

\page

\MyPicture{BENJAMIN_CONTADOR_2}


 Entre seus
interlocutores e amigos, encontram-se {\bf THEODOR W. ADORNO}, {\bf HANNAH ARENDT}, {\bf BERTOLT BRECHT}
e {\bf GERSHOM SCHOLEM}.

\page %----------------------------------------------------------|

\MyCover{BENJAMIN_CONTADOR_THUMB}

\page %----------------------------------------------------------|

\Hedra

\stoptext %---------------------------------------------------------|


% Walter Benjamin (1892-1940) foi um filósofo, crítico literário, tra-
% dutor (de Baudelaire, Proust e Balzac, entre outros) e também um
% ficcionista alemão. Estudou filosofia num ambiente dominado pelo
% neokantismo, em Berlim, Freiburg, Munique e Berna, onde defendeu
% tese de doutorado sobre os primeiros românticos alemães. Durante
% o seu exílio em Paris, nos anos trinta, foi ligado ao Instituto de Pes-
% quisa Social, embrião da chamada Escola de Frankfurt. Entre seus
% interlocutores e amigos, encontram-se personalidades marcantes do
% século xx como Theodor W. Adorno, Hannah Arendt, Bertolt Brecht
% e Gershom Scholem.
% 
% O contador de histórias e outros textos propõe uma nova tradu-
% ção anotada do clássico ensaio no qual Walter Benjamin esboça a
% figura do contador de histórias a partir de um comentário crítico
% do contista russo Nikolai Leskov; reúne também a pouco conhecida
% produção ficcional do próprio ensaísta, trazendo o conjunto de seus
% contos, alguns inéditos em português. O volume inclui ainda peças
% que Benjamin produziu para o rádio, deslocando a arte tradicional
% de contar histórias para a cena moderna, e textos híbridos, onde o
% crítico faz obra de ficção ou o contador de histórias filosofa.
% 
% Patrícia Lavelle é Professora do Departamento de Letras da puc-Rio,
% atuando no Programa de Pós-graduação em Literatura, Cultura e
% Contemporaneidade. É também Pesquisadora Associada à ehess-Pa-
% ris, onde defendeu doutorado em Filosofia e deu aulas. Sua tese foi
% publicada em livro: Religion et histoire. Sur le concept d’expérience
% chez Walter Benjamin. Paris: Cerf, col. Passages, 2008. Entre outros
% volumes coletivos, organizou Walter Benjamin. Paris: L’Herne, col.
% Cahiers de l’Herne, 2013.


% Coleção Walter Benjamin é um projeto acadêmico-editorial que
% envolve pesquisa, tradução e publicação de obras e textos seletos
% desse importante filósofo, crítico literário e historiador da cultura
% judeu-alemão, em volumes organizados por estudiosos versados em
% diferentes aspectos de sua obra, vida e pensamento. (Amon Pinho \& 
% Francisco De Ambrosis Pinheiro Machado)