% AUTOR_LIVRO_AUTOR.tex
% Preencher com o nome das cor ou composição RGB (ex: [r=0.862, g=0.118, b=0.118]) 
\usecolors[crayola]                % Paleta de cores pré-definida: wiki.contextgarden.net/Color#Pre-defined_colors

% Cores definidas pelo designer:
% MyGreen        r=0.251, g=0.678, b=0.290 % 40ad4a
% MyCyan        r=0.188, g=0.749, b=0.741 % 30bfbd
% MyRed            r=0.820, g=0.141, b=0.161 % d12429
% MyPink        r=0.980, g=0.780, b=0.761 % fac7c2
% MyGray        r=0.812, g=0.788, b=0.780 % cfc9c7
% MyOrange        r=0.980, g=0.671, b=0.290 % faab4a

% Configuração de cores
\definecolor[MyColor][x=414b47]      % ou ex: [r=0.862, g=0.118, b=0.118] % corresponde a RGB(220, 30, 30)
\definecolor[MyColorText][white]  % ou ex: [r=0.862, g=0.118, b=0.118] % corresponde a RGB(167, 169, 172)

% Classe para diagramação dos posts
\environment{marketing.env}           

% Cabeço e rodapé: Informações (caso queira trocar alguma coisa)
         \def\MensagemSaibaMais  {SAIBA MAIS:}
         \def\MensagemSite        {HEDRA.COM.BR}
         \def\MensagemLink       {LINK NA BIO}

\starttext %--------------------------------------------------------|

\Mensagem{O HOMEM QUE AMAVA AS BALEIAS}

\hyphenpenalty=10000
\exhyphenpenalty=10000

%\startMyCampaign

\MyPhoto{melville}

%\stopMyCampaign

\vfill\scale[factor=6]{\Seta\,HERMAN MELVILLE (1819--1891)}

\page %----------------------------------------------------------|

\hyphenpenalty=10000
\exhyphenpenalty=10000

Herman Melville, autor de {\bf MOBY DICK} (1851), um dos romances mais importantes da literatura ocidental, não explorou os mares apenas literariamente, mas também como {\bf MARINHEIRO}. 

\page

Sua experiência em baleeiros durante a juventude aparece em {\bf MOBY DICK} e em {\bf TAIPI}, seu romance de estreia.
Esse último centra-se, na verdade, no que se passou depois que o escritor desertou o navio e foi feito de prisioneiro por um povo nativo na Polinésia francesa.

\page

As obras de Melville, portanto, reelaboram episódios biográficos e associam-nos à relatos de viagem da época, criando um todo que mistura {\bf FATO E FICÇÃO}. Sua biografia está entrelaçadas em cada página que escreveu, e através de suas obras podemos captar fragmentos de sua vida.

\page %----------------------------------------------------------|

\Hedra

\stoptext %---------------------------------------------------------|


