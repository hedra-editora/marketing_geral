% AUTOR_LIVRO_AUTOR.tex
% Preencher com o nome das cor ou composição RGB (ex: [r=0.862, g=0.118, b=0.118]) 
\usecolors[crayola]                % Paleta de cores pré-definida: wiki.contextgarden.net/Color#Pre-defined_colors

% Cores definidas pelo designer:
% MyGreen        r=0.251, g=0.678, b=0.290 % 40ad4a
% MyCyan        r=0.188, g=0.749, b=0.741 % 30bfbd
% MyRed            r=0.820, g=0.141, b=0.161 % d12429
% MyPink        r=0.980, g=0.780, b=0.761 % fac7c2
% MyGray        r=0.812, g=0.788, b=0.780 % cfc9c7
% MyOrange        r=0.980, g=0.671, b=0.290 % faab4a

% Configuração de cores
\definecolor[MyColor][x=414b47]      % ou ex: [r=0.862, g=0.118, b=0.118] % corresponde a RGB(220, 30, 30)
\definecolor[MyColorText][white]  % ou ex: [r=0.862, g=0.118, b=0.118] % corresponde a RGB(167, 169, 172)

% Classe para diagramação dos posts
\environment{marketing.env}           

% Cabeço e rodapé: Informações (caso queira trocar alguma coisa)
         \def\MensagemSaibaMais  {SAIBA MAIS:}
         \def\MensagemSite        {HEDRA.COM.BR}
         \def\MensagemLink       {LINK NA BIO}

\starttext %--------------------------------------------------------|

\Mensagem{O HOMEM QUE AMAVA BALEIAS}

\hyphenpenalty=10000
\exhyphenpenalty=10000

%\startMyCampaign

\MyPhoto{melville}

%\stopMyCampaign

\vfill\scale[factor=6]{\Seta\,HERMAN MELVILLE (1819--1891)}

\page %----------------------------------------------------------|

\hyphenpenalty=10000
\exhyphenpenalty=10000

Pulvinar ante, a ultricies magna {\bf TRECHO EM DESTAQUE, MAS PODE HAVER MAIS DE UM}, 
mas sempre em negrito e caixa alta. Lorem ipsum dolor sit amet, consectetur adipiscing elit. Praesent sit amet pulvinar ante, a ultricies
magna. Etiam placerat quis tellus sed ultrices. Duis aliquet sed quam non
tincidunt. Donec sit amet tempor urna. Quisque auctor justo enim. Curabitur
vel est consectetur, sodales orci a, eleifend lacus. 

\page %----------------------------------------------------------|

\Hedra

\stoptext %---------------------------------------------------------|
Hoje comemoramos o aniversário do escritor, poeta e ensaísta norte-americano {\bf HERMAN MELVILLE}. Autor de um dos livros mais emblemáticos de literatura de aventura, o célebre {\bf MOBY DICK} (1851), um dos romances mais importantes da literatura ocidental. 

\page %----------------------------------------------------------|

\hyphenpenalty=10000
\exhyphenpenalty=10000

A escrita de Melville baseia-se sobretudo em suas vivências de {\bf MARINHEIRO}, as quais lhe proporcionaram um distanciamento que certamente contribuiu para a minuciosa análise que realiza das {\bf CONTRADIÇÕES DA SOCIEDADE NORTE-AMERICANA}. 

\page

% Apesar de hoje {\bf MOBY DICK} ser sua obra mais conhecida, quando vivo Melville era conhecido como o autor de {\bf TAIPI}, seu romance de estreia e responsável por consagrá-lo como um dos mais conhecidos autores dos {\cap EUA}. 

\hyphenpenalty=10000
\exhyphenpenalty=10000

 Sua morte, bem como o centenário de seu nascimento, comemorado em 1919, foram de extrema importância para renovar o interesse pela figura de Melville e reavivar os estudos acadêmicos voltados para sua obra, a qual tinha finalmente ascendido à categoria dos {\bf CLÁSSICOS}.

\page

% \MyCover{MELVILLE_TAIPI_THUMB}

\page %----------------------------------------------------------|

\Hedra

\stoptext %---------------------------------------------------------|


Herman Melville, autor de clássicos imortais como "Moby Dick", não apenas explorou os mares em suas aventuras literárias, mas também navegou pelas profundezas de sua própria alma. Suas experiências como marinheiro, sua busca incessante por sentido e suas batalhas pessoais estão entrelaçadas em cada página que escreveu. Através de personagens complexos e cenários vívidos, Melville revela fragmentos de sua vida e de suas reflexões mais íntimas. Sua obra é um convite para nos lançarmos em uma jornada introspectiva, onde cada leitura é uma nova viagem rumo ao autoconhecimento. 🌊✒️ #HermanMelville #Literatura #Autobiografia #MobyDick #ViagemInterior