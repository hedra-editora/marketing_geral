% AUTOR_LIVRO_EFEMERIDE.tex
% Preencher com o nome das cor ou composição RGB (ex: [r=0.862, g=0.118, b=0.118]) 
\usecolors[crayola] 			   % Paleta de cores pré-definida: wiki.contextgarden.net/Color#Pre-defined_colors

% Cores definidas pelo designer:
% MyGreen		r=0.251, g=0.678, b=0.290 % 40ad4a
% MyCyan		r=0.188, g=0.749, b=0.741 % 30bfbd
% MyRed			r=0.820, g=0.141, b=0.161 % d12429
% MyPink		r=0.980, g=0.780, b=0.761 % fac7c2
% MyGray		r=0.812, g=0.788, b=0.780 % cfc9c7
% MyOrange		r=0.980, g=0.671, b=0.290 % faab4a

% Configuração de cores
\definecolor[MyColor][Sapphire]      % ou ex: [r=0.862, g=0.118, b=0.118] % corresponde a RGB(220, 30, 30)
\definecolor[MyColorText][white]     % ou ex: [r=0.862, g=0.118, b=0.118] % corresponde a RGB(167, 169, 172)

% Classe para diagramação dos posts
\environment{marketing.env}		   

\starttext %---------------------------------------------------------|

\hyphenpenalty=10000
\exhyphenpenalty=10000

\Mensagem{10 DE JUNHO} %Sempre usar esse header

\MyPicture{CAMOES_EFEMERIDE_1}

\vfill\scale[factor=6]{\Seta\,444 ANOS SEM {\bf CAMÕES}}

\page %---------------------------------------------------------| 

\hyphenpenalty=10000
\exhyphenpenalty=10000


{\bf LUÍS VAZ DE CAMÕES} foi um poeta português, tido como uma das maiores figuras da literatura lusófona e da tradição ocidental. 

\page %---------------------------------------------------------|

Quase nada sabemos da vida de Camões. Mas, segundo a tradição, sua vida teria sido bastante turbulenta. Ele teria sido expulso da corte de D. João {\cap III} e enfrentado {\bf PRISÕES, TORTURAS} e {\bf EXÍLIOS}.

\page

Humanista por excelência, o poeta jamais se submeteu às exigências das regras morais e religiosas e se empenhou em tecer, em suas obras, uma {\bf CRÍTICA FEROZ} à Corte real --- “corrompida, estúpida e decadente” --- e ao fanatismo religioso.

\page

% A peça teatral {\bf AUTO DE EL-REI SELEUCO}, por exemplo, zomba abertamente do matrimônio como instituição sagrada, além de aludir ao amor incestuoso de D. João {\cap III} pela sua madrasta, D. Maria, o que teria levado ao falecimento prematuro (e possível assassinato) de D. Manuel.

Exilado da pátria, {\bf CAMÕES} conheceu de perto as barbaridades das guerras, os abusos dos poderosos e dos fanáticos. Esteve várias vezes nos cárceres, tendo por pouco escapado da Santa Inquisição.

\page

As suas experiências como combatente ao lado das tropas portuguesas no Oriente renderam-lhe a perda do seu olho direito, fato pelo qual foi apelidado de «{\bf O DIABO ZAROLHO}», e a inspiração para sua obra-prima, {\bf OS LUSÍADAS}, um poema épico nacional.

\page

 % Em 1579, Lisboa foi assolada pela peste bubônica. Contaminado, Camões foi recolhido a um abrigo para indigentes. Ali, só e esquecido, morreu juntamente com o antigo Portugal, cujas tropas de Felipe II da Espanha se preparavam para invadir.

\Hedra

\stoptext %---------------------------------------------------------|


% Logo após a sua morte a sua obra lírica foi reunida na coletânea Rimas, tendo deixado também três obras de teatro cómico. Enquanto viveu queixou-se várias vezes de alegadas injustiças que sofrera, e da escassa atenção que a sua obra recebia, mas pouco depois de falecer a sua poesia começou a ser reconhecida como valiosa e de alto padrão estético por vários nomes importantes da literatura europeia, ganhando prestígio sempre crescente entre o público e os conhecedores e influenciando gerações de poetas em vários países. Camões foi um renovador da língua portuguesa e fixou-lhe um duradouro cânone; tornou-se um dos mais fortes símbolos de identidade da sua pátria e é uma referência para toda a comunidade lusófona internacional. Hoje a sua fama está solidamente estabelecida e é considerado um dos grandes vultos literários da tradição ocidental, sendo traduzido para várias línguas e tornando-se objeto de uma vasta quantidade de estudos críticos. 

% % Camões viveu no século XVI, tempo enriquecido pelo renascimento das artes e do livre pensar. Mas também gerador das lutas religiosas que, incialmente, contestavam a supremacia e a intolerância do Catolicismo Romano, buscando modernizá-lo e torná-lo mais próximo dos homens.

% As Reformas iniciais de Lutero, que, entretanto, desembocaram em guerras pelo poder, geraram o Protestantismo associado ao poder político, que se tornou pelo menos tão despótico quanto o próprio Catolicismo, o qual reagiu com as medidas destruidoras da liberdade da Contrarreforma.

% O fanatismo religioso dos Savonarollas e Torquemadas, por um lado, e dos Calvinos e Knoxs, pelo outro, competiam entre si na destruição da liberdade do pensar.

% E as guerras de conquistas, as torturas, os assassinatos e as fogueiras arderam à exaustão desde o final daquele século e no que se seguiu.

% Renascentista na Corte católica

% Em Coimbra, já nos bancos escolares, lá pelos anos de 1530, Luís Vaz de Camões sofreu as hostilidades dos fanáticos, sendo acusado de paganismo. Principiou a universidade mas, expulso, foi obrigado a deixá-la.



% Homem de coragem e fiel aos amigos, estando um dia numa procissão de Corpus Christi, um soldado agrediu a um seu amigo e Camões feriu-o com a espada no pescoço. Preso, foi torturado nas masmorras reais e permaneceu um ano atrás das grades, conseguindo indulto por se comprometer a seguir como combatente nas naus coloniais para o Oriente.

% No Oriente, tão pouco Camões deixou de passar por atropelos com as autoridades eclesiásticas e temporais. Mas foi graças a esta passagem que ele imaginou “Os Lusíadas”, descrevendo maravilhosamente todos os lugares pelos quais navegara a esquadra de Vasco da Gama e o épico papel por ela representado.


% Participou, entre 1551 e 1554, de uma série de aventuras guerreiras, como contra o Rei Chambé e contra os turcos no Mar Vermelho, retornando a Goa, possessão portuguesa na Índia, em 1555.

% Em Goa, atacou publicamente a corrupção e desmandos dos governantes e dos padres. Novamente preso, foi deportado para as Ilhas Moluscas.

% Acontece que o castigo foi considerado excessivo para o poeta; houve intervenção e se obteve para ele, em 1558, o cargo de procurador-mor de defuntos em Macau.

% Dizem que, para enganar seus perseguidores, pois era péssimo funcionário, esquecia os mortos e dedicava seu tempo a escrever e esconder “Os Luzíadas” em cavernas e grutas. Mas certo dia, declarado dilapidador da fazenda pública, foi preso e colocado na nau “Prata e Seda” para ser levado e julgado em Goa.

% Viveu, então, algum tempo dentre os budistas e com eles ampliou seu ecumenismo. Finalmente chegou a Goa, e ficou prisioneiro entre 1560 e 1562. Soube na prisão da morte prematura de um amor de sua juventude, Catarina de Ataíde, a Natércia, pessoa nobre que os pais haviam proibido que com ele se casasse.

% Camões jamais mendigou ou ao menos pediu apoio financeiro para a nobreza. Pobre, recolheu-se ao convívio popular modestamente, e jamais aceitou um convite que fosse para frequentar os salões de uma Corte, por ele definida como “corrompida, estúpida e decadente”.


% Morria um autor que seria imortalizado pelas obras, que nos legaram a postura corajosa de quem jamais se curvou perante os poderosos e os fanáticos que se creem representantes de Deus na Terra.

% “Os Lusíadas”

% “Os Lusíadas” refletem a decadência do feudalismo e é o principal arauto literário de um Novo Mundo que despertava.

% Camões, trezentos anos antes da Revolução Francesa, questiona os fidalgos, os “filhos de algo”, suas posições de superioridade e os preconceitos sociais e raciais.

% A obra-prima de Camões é grito épico de independência moral. Nela nada se encontra de estreito, de restrito; como todo o homem autêntico de espírito renascentista, ele sempre mais e mais se aproxima do paganismo da natureza que do espírito cristão castrador.

% Jamais sua lírica se contamina com preconceitos, combate-os com a fé no humanismo. Como exemplo, a maior recompensa que vê para Vasco da Gama e sua gente é a Ilha dos Amores, florida e perfumada, com suas ninfas em sua estonteante nudez, com o coração e corpo abertos à espera dos heróis lusitanos.

% No Canto VII de “Os Lusíadas”, o poeta tece um franco elogio às mulheres liberais de Malabar, nas costas das Índias:

% O que hoje nos chama atenção é o fato de o censor da Inquisição ter-se manifestado favorável à impressão do poema. Acontece que todo o poema é rigorosamente herético.

% Por exemplo, Vasco da Gama embora visasse dilatar o império e a fé cristã, só se viu auxiliado pelos deuses e deusas do velho mundo pagão. É verdade também que Baco tentou que suas naus naufragassem, mas Vênus protegeu-as, pois prometera a Zeus torná-los, os Lusos, tão famosos quanto os heróis da Antiguidade.
% Camões: autor do mais importante poema em Língua Portuguesa | Foto: Reprodução

% Podemos imaginar o nível de pressão sobre o censor exercida por D. Sebastião, o Rei, o qual tinha em muito menor grau o fanatismo religioso do falecido avô, o Rei João III. E a maior obra jamais escrita em língua portuguesa lhe fora dedicada.

% Ao lado da essência épica, em todo o longo poema, Camões assenta na questão econômica e não em crendices religiosas a base da vida social.

% Se “Os Lusíadas” têm o caráter patriótico também possui um caráter universal, pois “toda terra é pátria para o forte”.

% Camões figura dentre os grandes épicos do mundo moderno, lídimos continuadores de Homero, Virgílio e Dante.

