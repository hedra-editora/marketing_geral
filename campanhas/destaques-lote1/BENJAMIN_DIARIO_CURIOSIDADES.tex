% Preencher com o nome das cor ou composição RGB (ex: [r=0.862, g=0.118, b=0.118]) 
\usecolors[crayola] 			   % Paleta de cores pré-definida: wiki.contextgarden.net/Color#Pre-defined_colors

% Cores definidas pelo designer:
% MyGreen		r=0.251, g=0.678, b=0.290 % 40ad4a
% MyCyan		r=0.188, g=0.749, b=0.741 % 30bfbd
% MyRed			r=0.820, g=0.141, b=0.161 % d12429
% MyPink		r=0.980, g=0.780, b=0.761 % fac7c2
% MyGray		r=0.812, g=0.788, b=0.780 % cfc9c7
% MyOrange		r=0.980, g=0.671, b=0.290 % faab4a

% Configuração de cores
\definecolor[MyColor][x=c0e016]      % ou ex: [r=0.862, g=0.118, b=0.118] % corresponde a RGB(220, 30, 30)
\definecolor[MyColorText][black]  % ou ex: [r=0.862, g=0.118, b=0.118] % corresponde a RGB(167, 169, 172)

% Classe para diagramação dos posts
\environment{marketing.env}		   

\starttext
%\showframe  %Para mostrar somente as linhas.

\hyphenpenalty=10000
\exhyphenpenalty=10000

\Mensagem{EM CONTEXTO}

\startMyCampaign
\hyphenpenalty=10000
\exhyphenpenalty=10000

A HISTÓRIA 

POR 
TRÁS 

DO SUICÍDIO DE  
{\bf WALTER BENJAMIN}

%\vfill\scale[lines=2]{\MyStar[MyColorText][none]} 					% Estrela pequena  

\stopMyCampaign

\page

{\MyPicture{BENJAMIN_CONTADOR_4.jpeg}}

\page Em 1940, de posse de um visto transitório para os Estados Unidos,
 {\bf Benjamin} deixou Paris e dirigiu-se para a fronteira franco-espanhola.
 Chegou em Lourdes no dia 24 de setembro e, após uma complicada viagem de
 trem até a aldeia de Banyuls-sur-Mer, perto de Portbou, Walter Benjamin
 iniciou a subida dos Pirinéus a pé.

\page

No dia 26 de setembro, após horas de uma árdua caminhada, {\bf Benjamin} e um
pequeno grupo de refugiados que viajavam com ele, incluindo a fotógrafa Henny
Gurland e seu filho, finalmente chegaram a Portbou, do lado espanhol.
Entretanto, ao tentar passar pela aduana, foi informado de que a
política espanhola havia mudado de repente e que eles seriam deportados de
volta à França na manhã seguinte.

\page

Temendo ser entregue para os nazistas, {\bf Benjamin} tomou uma decisão
trágica. Na noite de 26 de setembro de 1940, em um quarto do Hotel de
Francia, em Portbou, Benjamin cometeu suicídio por overdose de morfina.

\page

Os outros que viajavam com ele tiveram permissão de passagem no dia seguinte e
chegaram em segurança à Lisboa em 30 de setembro de 1940. 


\page

A sua tentativa de fuga e suicídio são retratados na série
{\bf Transatlântico} (2023), que relata a história de um jornalista americano, 
entre os anos de 1940--41, que ajuda mais
de dois mil refugiados que correm o risco de vida a fugirem para os {\cap  eua}.


\page

\MyCover{BENJAMIN_DIARIO_THUMB.pdf}

\page

{\bf Diário pariense e outros escritos} reúne textos de {\bf Walter Benjamin}
produzidos entre os anos de 1926 e 1936. Durante os últimos anos do período
nazista, Benjamin passou pela Espanha, Dinamarca, e finalmente, França, onde
exilou-se definitivamente.


\page %---------------------------------------------------------|



\Hedra

\stoptext