% AUTOR_LIVRO_EFEMERIDE.tex
% Preencher com o nome das cor ou composição RGB (ex: [r=0.862, g=0.118, b=0.118]) 
\usecolors[crayola] 			   % Paleta de cores pré-definida: wiki.contextgarden.net/Color#Pre-defined_colors

% Cores definidas pelo designer:
% MyGreen		r=0.251, g=0.678, b=0.290 % 40ad4a
% MyCyan		r=0.188, g=0.749, b=0.741 % 30bfbd
% MyRed			r=0.820, g=0.141, b=0.161 % d12429
% MyPink		r=0.980, g=0.780, b=0.761 % fac7c2
% MyGray		r=0.812, g=0.788, b=0.780 % cfc9c7
% MyOrange		r=0.980, g=0.671, b=0.290 % faab4a

% Configuração de cores
\definecolor[MyColor][x=f2a80e]      % ou ex: [r=0.862, g=0.118, b=0.118] % corresponde a RGB(220, 30, 30)
\definecolor[MyColorText][black]     % ou ex: [r=0.862, g=0.118, b=0.118] % corresponde a RGB(167, 169, 172)

% Classe para diagramação dos posts
\environment{marketing.env}		   

\starttext %---------------------------------------------------------|

\hyphenpenalty=10000
\exhyphenpenalty=10000

\Mensagem{07 DE JULHO} %Sempre usar esse header

\MyPicture{AZEVEDO_LOBO_1}

\vfill\scale[factor=6]{\Seta\,169 ANOS DE {\bf ARTUR AZEVEDO}}

\page %---------------------------------------------------------| 

\hyphenpenalty=10000
\exhyphenpenalty=10000

{\bf ARTUR AZEVEDO} foi um dos primeiros dramaturgos e maiores comediógrafos brasileiros. Nascido em São Luís, sua vocação literária se manifestou precocemente. Aos 15 anos, escreveu a peça «Amor por Anexins», que alcançou grande êxito na capital maranhense e, depois, em todo o Brasil.

\page

Sua produção dramática abarca mais de duzentas peças, entre originais, traduções e adaptações. As mais conhecidas são as comédias e burletas, dentre as quais se destacam {\bf A CAPITAL FEDERAL} (1897) e {\bf O MAMBEMBE} (1904).

\page

Diferentemente dos literatos de sua época, Artur Azevedo escrevia para o {\bf PÚBLICO POPULAR}. Algumas de suas peças, de um {\bf ABOLICIONISMO} ardente, sofreram a censura imperial, sendo publicados no período republicano com o título de «O escravocrata».

\page

Co-fundador da {\bf ACADEMIA BRASILEIRA DE LETRAS}, o escritor maranhense faleceu no Rio de Janeiro em outubro de 1908. Meses depois, foi inaugurado o {\bf TEATRO MUNICIPAL DO RIO DE JANEIRO}, cuja criação ele havia defendido durante cerca de três décadas.

\page


% {\bf A PELE DO LOBO E OUTRAS PEÇAS} inclui cinco textos curtos de Artur
% Azevedo, cuja temática gira em torno de {\bf COSTUMES NACIONAIS}. 

% {\bf AMOR POR ANEXINS} (1870) foi a primeira peça do autor. {\bf A PELE DO LOBO} (1875) faz uma sátira divertida ao sistema de policiamento do Império. {\bf O ORÁCULO} (1907) é um texto que dialoga com a tradição da comédia. {\bf COMO EU ME DIVERTI!} (1893) e {\bf O CORDAO}
% (1908) tratam do carnaval e são exemplos importantes da conturbada posição de Artur Azevedo entre os escritores de seu tempo, por ser um autor eminentemente popular.

\page %---------------------------------------------------------|

\MyCover{AZEVEDO_LOBO_THUMB}

\page %---------------------------------------------------------|

\Hedra

\stoptext %---------------------------------------------------------|
