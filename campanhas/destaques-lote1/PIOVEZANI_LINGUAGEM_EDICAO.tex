% PIOVEZANI_LINGUAGEM_EDICAO.tex
% Preencher com o nome das cor ou composição RGB (ex: [r=0.862, g=0.118, b=0.118]) 
\usecolors[crayola] 			   % Paleta de cores pré-definida: wiki.contextgarden.net/Color#Pre-defined_colors

% Cores definidas pelo designer:
% MyGreen		r=0.251, g=0.678, b=0.290 % 40ad4a
% MyCyan		r=0.188, g=0.749, b=0.741 % 30bfbd
% MyRed			r=0.820, g=0.141, b=0.161 % d12429
% MyPink		r=0.980, g=0.780, b=0.761 % fac7c2
% MyGray		r=0.812, g=0.788, b=0.780 % cfc9c7
% MyOrange		r=0.980, g=0.671, b=0.290 % faab4a

% Configuração de cores
\definecolor[MyColor][x=6d6d6e]      % ou ex: [r=0.862, g=0.118, b=0.118] % corresponde a RGB(220, 30, 30)
\definecolor[MyColorText][white]     % ou ex: [r=0.862, g=0.118, b=0.118] % corresponde a RGB(167, 169, 172)

% Classe para diagramação dos posts
\environment{marketing.env}		   

\starttext %---------------------------------------------------------|

\Mensagem{A LINGUAGEM FASCISTA}

\startMyCampaign

\hyphenpenalty=10000
\exhyphenpenalty=10000

{\bf O QUE BENITO MUSSOLINI E JAIR BOLSONARO
TINHAM EM COMUM?}

\stopMyCampaign

%\vfill\scale[lines=1.5]{\MyStar[MyColorText][none]}

\page %---------------------------------------------------------| 

\MyCover{PIOVEZANI_LINGUAGEM_THUMB.pdf}

\page %---------------------------------------------------------| 

\hyphenpenalty=10000
\exhyphenpenalty=10000

A partir de uma perspectiva histórica e da exposição do uso da linguagem pelo regime nazista, {\bf A LINGUAGEM FASCISTA} traça um paralelo entre dois casos emblemáticos da linguagem fascista: os discursos de Benito Mussolini e de Jair Bolsonaro.

\page 

\hyphenpenalty=10000
\exhyphenpenalty=10000

A comparação entre seus 

desempenhos oratórios expõe ao leitor as propriedades dessa linguagem, seus recursos e seu funcionamento, mas também sua conservação e suas transformações ao passar da Itália do século {\cap xx} ao Brasil do século {\cap xxi}. 

\page %---------------------------------------------------------|

\hyphenpenalty=10000
\exhyphenpenalty=10000

«A frequente referência ao povo não significava de modo algum que o nazismo tivesse um real interesse em ouvir a sua voz.
{\bf FALAR ÀS MASSAS PARA MAIS BEM CALAR O POVO:} esta não seria a primeira nem a última vez que assistiríamos a esse perverso expediente.»

\vfill\scale[factor=6]{{\bf A linguagem fascista}, Carlos Piovezani e}\setupinterlinespace[line=1.5ex]\scale[factor=6]{Emilio Gentile.}

\page %---------------------------------------------------------|

\Hedra

\stoptext %---------------------------------------------------------|
