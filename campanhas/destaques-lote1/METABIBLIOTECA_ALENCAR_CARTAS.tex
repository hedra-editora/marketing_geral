% Preencher com o nome das cor ou composição RGB (ex: [r=0.862, g=0.118, b=0.118]) 
\usecolors[crayola] 			   % Paleta de cores pré-definida: wiki.contextgarden.net/Color#Pre-defined_colors

% Cores definidas pelo designer:
% MyGreen		r=0.251, g=0.678, b=0.290 % 40ad4a
% MyCyan		r=0.188, g=0.749, b=0.741 % 30bfbd
% MyRed			r=0.820, g=0.141, b=0.161 % d12429
% MyPink		r=0.980, g=0.780, b=0.761 % fac7c2
% MyGray		r=0.812, g=0.788, b=0.780 % cfc9c7
% MyOrange		r=0.980, g=0.671, b=0.290 % faab4a


% Configuração de cores
\definecolor[MyColor][Almond]      % ou ex: [r=0.862, g=0.118, b=0.118] % corresponde a RGB(220, 30, 30)
\definecolor[MyColorText][black]  % ou ex: [r=0.862, g=0.118, b=0.118] % corresponde a RGB(167, 169, 172)

% Classe para diagramação dos posts
\environment{marketing.env}		   


% Comandos & Instruções %%%%%%%%%%%%%%%%%%%%%%%%%%%%%%%%%%%%%%%%%%%%%%%%%%%%%%%%%%%%%%%|

% Cabeço e rodabé: Informações (caso queira trocar alguma coisa)
% 		\def\MensagemSaibaMais{SAIBA MAIS:}
% 		\def\MensagemSite{HEDRA.COM.BR}
% 		\def\MensagemLink{LINK NA BIO}

% Pesos para os títulos:
%		\startMyCampaign...		 \stopMyCampaign
%		\stopMyCampaignSection...   \stopMyCampaignSection

% Aplicação de imagens: 
% 		\MyCover{capa.pdf}  	% Aplicação de capa de livro com sombra
%		\MyPicture{Imagem.png}  % Imagem com aplicação de filtro segundo cor MyColorText
%		\MyPhoto{}			    % Aplicação simples de imagem com tamamho \textwidth

% Aplicação de imagem com legenda:		
% 		\placefigure{Legenda}{\externalfigure[drop2-1.png][width=\textwidth]}

% Cabeço e rodabé: Opções
% 		\Mensagem{AGORA É QUE SÃO ELAS}
% 		\Hashtag{campanha de natal}
% 		\Mensagem{campanha de natal}

% Alteração de várias cores de background:
% \setupbackgrounds[page][background=color,backgroundcolor=MyGray]

% Estrela: 
% \vfill\scale[lines=2]{\MyStar[MyColorText][none]} 					% Estrela pequena  
% \startpositioning 											% Estrela grande
%  \position(-1,-.3){\scale[scale=980]{\MyStar[white][none]}}
% \stoppositioning

% Logos e selos: 				
% \Hedra
% \HedraAyllon	% Não está pronto
% \HedraAcorde	% Não está pronto
% \Ayllon		% Não está pronto
% \Acorde		% Não está pronto

% Atalhos: 						
% 		\Seta  % Seta para baixo

%%%%%%%%%%%%%%%%%%%%%%%%%%%%%%%%%%%%%%%%%%%%%%%%%%%%%%%%%%%%%%%%%%%%%%%%%%%%%%%%%%%%%%%|

\starttext
%\showframe  %Para mostrar somente as linhas.

\Mensagem{DESTAQUES}

\MyCover{METABIBLIOTECA_ALENCAR_CARTAS_THUMB.pdf}

\page %---------------------------------------------------------|

\startpositioning
                \position(-3mm,0mm){
                \clip[height=70mm,voffset=6mm]{
                \externalfigure[METABIBLIOTECA_ALENCAR_CARTAS_1.jpeg][width=\textwidth]
                }}	
\stoppositioning



\page 

\hyphenpenalty=10000
\exhyphenpenalty=10000

Por que publicar um livro tão polêmico como {\bf Cartas a favor da escravidão}?
É preciso, sobretudo, voltar às origens do pensamento mais conservador, 
capaz de justificar as barbáries da escravidão. 

\MyPicture{METABIBLIOTECA_ALENCAR_CARTAS_2.jpeg}

\page 

Dirigindo-se a Dom
Pedro II, José de Alencar procura demonstrar que abolir o cativeiro 
ia de encontro aos interesses da nação e, mais do
que isso, ao progresso da humanidade — aí incluídos os
próprios escravos. 

\page

O cativeiro seria, pois, o embrião civilizacional, a forma pela qual o homem é retirado do estado
bruto da natureza e aperfeiçoa-se.


\vfill\scale[factor=fit]{\tfxx Organização e notas de {\bf Tâmis Parron}.}


\page


\MyPicture{METABIBLIOTECA_ALENCAR_CARTAS_3.jpeg}


\page 

«No seio da barbaria, o homem, em luta
contra a natureza, sente a necessidade
de multiplicar suas forças. O único
instrumento ao alcance é o próprio
homem, seu semelhante; apropria-se
dele ou pelo direito da geração ou pelo
direito da conquista. 

\page

Aí está o gérmen
rude e informe da família, agregado
dos fâmulos, reunião de servos. O mais
antigo documento histórico, o Gênesis,
nos mostra o homem filiando-se à família
estranha pelo cativeiro.» 


\page

\Hedra

\stoptext





% As Novas cartas políticas são reeditadas
% aqui pela primeira vez desde o século XIX,
% após terem sido expurgadas das obras do
% autor. Trata-se de sete textos políticos,
% endereçados a D. Pedro II, que justificam
% uma instituição hoje universalmente
% condenada, publicação incontornável para
% a nossa historiografia política e literária,
% bem como para todos os interessados
% na história das relações raciais e na
% escravidão no Brasil e no mundo.
% 
% Sob o codinome Erasmo, em alusão a Erasmo de
% Roterdã, Alencar compõe praticamente um espelho de
% príncipe, na posição do letrado que prescreve ao gover-
% nante as normas de um bom governo. Dirigindo-se a Dom
% Pedro II, o autor procura demonstrar que abolir o cati-
% veiro ia de encontro aos interesses da nação e, mais do
% que isso, ao progresso da humanidade — aí incluídos os
% próprios escravos. O cativeiro seria, pois, o embrião civili-
% zacional, a forma pela qual o homem é retirado do estado
% bruto da natureza e aperfeiçoa-se.

% Mais do que isso, Alencar glorifica a densidade étnica
% e cultural que resulta do tráfico negreiro, na que talvez
% seja a única defesa da escravidão, em toda a América do
% XIX, baseada na mistura cultural dos povos. Os ecos do
% elogio à miscigenação ecoam em toda sua obra, vendo
% nela o esteio da nacionalidade brasileira em formação.
% A leitura destas Cartas, pois, por um lado completa o
% quebra-cabeça de um dos nossos autores mais fundamen-
% tais, tornando talvez obsoleta a antítese de José Veríssi-
% mo — “Revolucionário em letras, conservador em política”
% — já que as letras revolucionárias serviram perfeitamente
% à tribuna conservadora; e, por outro, lança luz na faceta
% oculta de José de Alencar, não para levar seu espírito aos
% tribunais, mas sobretudo para enriquecer o debate sobre
% a escravidão brasileira e suas muitas reminiscências em
% nossas relações raciais.