% Preencher com o nome das cor ou composição RGB (ex: [r=0.862, g=0.118, b=0.118]) 
\usecolors[crayola] 			   % Paleta de cores pré-definida: wiki.contextgarden.net/Color#Pre-defined_colors

% Cores definidas pelo designer:
% MyGreen		r=0.251, g=0.678, b=0.290 % 40ad4a
% MyCyan		r=0.188, g=0.749, b=0.741 % 30bfbd
% MyRed			r=0.820, g=0.141, b=0.161 % d12429
% MyPink		r=0.980, g=0.780, b=0.761 % fac7c2
% MyGray		r=0.812, g=0.788, b=0.780 % cfc9c7
% MyOrange		r=0.980, g=0.671, b=0.290 % faab4a

% Configuração de cores
\definecolor[MyColor][BananaMania]      % ou ex: [r=0.862, g=0.118, b=0.118] % corresponde a RGB(220, 30, 30)
\definecolor[MyColorText][BurntUmber]  % ou ex: [r=0.862, g=0.118, b=0.118] % corresponde a RGB(167, 169, 172)


% Classe para diagramação dos posts
\environment{marketing.env}		   

% Comandos & Instruções %%%%%%%%%%%%%%%%%%%%%%%%%%%%%%%%%%%%%%%%%%%%%%%%%%%%%%%%%%%%%%%|

% Cabeço e rodapé: Informações (caso queira trocar alguma coisa)
 		\def\MensagemSaibaMais{SAIBA MAIS:}
 		\def\MensagemSite{HEDRA.COM.BR}
 		\def\MensagemLink{LINK NA BIO}

% Pesos para os títulos:
%		\startMyCampaign...		 \stopMyCampaign
%		\stopMyCampaignSection...   \stopMyCampaignSection

% Aplicação de imagens: 
% 		\MyCover{capa.pdf}  	% Aplicação de capa de livro com sombra
%		\MyPicture{Imagem.png}  % Imagem com aplicação de filtro segundo cor MyColorText
%		\MyPhoto{}			    % Aplicação simples de imagem com tamamho \textwidth

% Aplicação de imagem com legenda:		
% 		\placefigure{Legenda}{\externalfigure[drop2-1.png][width=\textwidth]}

% Cabeço e rodabé: Opções
% 		\Mensagem{AGORA É QUE SÃO ELAS}
% 		\Hashtag{campanha de natal}
% 		\Mensagem{campanha de natal}

% Alteração de várias cores de background:
% \setupbackgrounds[page][background=color,backgroundcolor=MyGray]

% Estrela: 
% \vfill\scale[lines=2]{\MyStar[MyColorText][none]} 			% Estrela pequena  
% \startpositioning 											% Estrela grande
%  \position(-1,-.3){\scale[scale=980]{\MyStar[white][none]}}
% \stoppositioning

% Logos e selos: 				
% \Hedra
% \HedraAyllon	% Não está pronto
% \HedraAcorde	% Não está pronto
% \Ayllon		% Não está pronto
% \Acorde		% Não está pronto

% Atalhos: 						
% 		\Seta  % Seta para baixo

% Espaçamentos:
% \setupinterlinespace[line=1.9ex]		% para regular o entelinha (colocar \par ao fim do período)
% \hyphenpenalty=10000   			    % evitar quebras

%%%%%%%%%%%%%%%%%%%%%%%%%%%%%%%%%%%%%%%%%%%%%%%%%%%%%%%%%%%%%%%%%%%%%%%%%%%%%%%%%%%%%%%|

\starttext
%\showframe  %Para mostrar somente as linhas.

\Mensagem{EM CONTEXTO}

\startMyCampaign

\hyphenpenalty=10000
\exhyphenpenalty=10000
O NASCIMENTO

DE AFRODITE
SEGUNDO
{\bf HESÍODO}
\stopMyCampaign

\page %---------------------------------------------------------|

\hyphenpenalty=10000
\exhyphenpenalty=10000


Como a grande maioria dos mitos gregos, a história do nascimento de Afrodite tem mais de uma versão. As duas variações que relatam sua origem vem justamente dos grandes responsáveis por formar a base da mitologia grega --- {\bf Homero} e {\bf Hesíodo}.

\page

Se para {\bf Homero}, Afrodite é resultado do casamento de Zeus e da oceâniade Dione, para {\bf Hesíodo} a deusa do amor teria sua origem atrelada à revolta de Crono. Da combinação do sêmen de Urano, quem teve sua genitália decepada por seu filho, e a espuma do mar, teria surgido a mais bela das deusas.

\page

«[\unknown] primeiro da numinosa Citera achegou-se, /
e então de lá atingiu o oceânico Chipre. /
E saiu a respeitada, bela deusa, e grama em volta /
crescia sob os pés esbeltos: a ela Afrodite /
espumogênita e Citereia bela-coroa /
chamam deuses e varões, porque na espuma /
foi criada; Citereia, pois, alcançou Citera; /
cipriogênita, pois nasceu em Chipre cercado-de-mar; /
e ama sorriso, pois da genitália surgiu.»


\blank[medium]
\scale[factor=fit]{Tradução do grego de {\bf Christian Werner}}

\page

Nessa versão, ela viaja em direção ao lar dos deuses, impelida pelo vento oeste Zéfiro, em cima de uma concha nacarada que lhe serve de veículo. A deusa faz sua chegada triunfal no Olimpo, deixando todos aberrebatados diante de tamanho espetáculo. Essa cena foi revisitada pelo pintor italiano Sandro Botticelli, em seu {\bf O nascimento de Vênus}


\page

\placefigure{{\cap Botticelli}, Sandro. {\it O nascimento de Vênus}. 1485--1486.}{\externalfigure[HESIODO_TEOGONIA_THUMB2][width=\textwidth]}

\page %---------------------------------------------------------|

\MyCover{HESIODO_TEOGONIA_THUMB.pdf}

\page

{\bf Teogonia} é um poema de 1022 versos hexâmetros
datílicos que descreve a origem e a genealogia dos
deuses. Muito do que sabemos sobre os antigos mitos gregos
é graças a esse poema que, pela narração em primeira 
pessoa do próprio poeta, sistematiza e organiza
as histórias da criação do mundo e do nascimento
dos deuses, com ênfase especial em Zeus e em suas
façanhas até chegar ao poder.

\page

\Hedra

\stoptext
	