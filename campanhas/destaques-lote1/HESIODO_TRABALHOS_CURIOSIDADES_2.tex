% HESIODO_TRABALHOS_CURIOSIDADES_2.tex
% Vamos falar sobre isso "curiosidades"
% > "EM CONTEXTO"

% Preencher com o nome das cor ou composição RGB (ex: [r=0.862, g=0.118, b=0.118]) 
\usecolors[crayola] 			   % Paleta de cores pré-definida: wiki.contextgarden.net/Color#Pre-defined_colors

% Cores definidas pelo designer:
% MyGreen		r=0.251, g=0.678, b=0.290 % 40ad4a
% MyCyan		r=0.188, g=0.749, b=0.741 % 30bfbd
% MyRed			r=0.820, g=0.141, b=0.161 % d12429
% MyPink		r=0.980, g=0.780, b=0.761 % fac7c2
% MyGray		r=0.812, g=0.788, b=0.780 % cfc9c7
% MyOrange		r=0.980, g=0.671, b=0.290 % faab4a

% Configuração de cores
\definecolor[MyColor][Almond]      % ou ex: [r=0.862, g=0.118, b=0.118] % corresponde a RGB(220, 30, 30)
\definecolor[MyColorText][r=0.07058823529411765, g=0.6235294117647059, b=0.4196078431372549]  % ou ex: [r=0.862, g=0.118, b=0.118] % corresponde a RGB(167, 169, 172)

% Classe para diagramação dos posts
\environment{marketing.env}		   

\starttext %---------------------------------------------------------|

\hyphenpenalty=10000
\exhyphenpenalty=10000

\Mensagem{EM CONTEXTO}

\startMyCampaign
\hyphenpenalty=10000
\exhyphenpenalty=10000

FORÇA, FOGO E LIBERDADE
O {\bf MITO} 
DE PROMETEU
SEGUNDO 
{\bf HESÍODO}

%\vfill\scale[lines=2]{\MyStar[MyColorText][none]} 					% Estrela pequena  

\stopMyCampaign

\page %---------------------------------------------------------|

A história de Prometeu é contada por {\bf Hesíodo} em seu {\bf Trabalhos e dias}. 

Prometeu era um Titã conhecido por sua inteligência e astúcia. Ele desafiou Zeus, o pai dos deuses, ao enganá-lo e roubar o fogo divino para entregá-lo aos humanos, auxiliando assim na evolução da civilização.

\page 

Para punir Prometeu por este ato de desobediência, Zeus ordenou que Hefesto criasse Pandora, a primeira mulher, e a enviou aos humanos junto ao seu pito que continha os males do mundo. Ao abrir o vaso, Pandora faria com que estes males fossem liberados.

\page

Como vingança adicional, Zeus acorrentou Prometeu a uma rocha e mandou diariamente uma águia para devorar seu fígado, que durante a noite se regenerava, resultando em um destino eternamente doloroso para o Titã. Essa cena foi pintada por Peter Paul Rubens, uma das figuras mais importantes do estilo barroco.

\page


\placefigure{{\cap RUBENS}, Peter Paul. {\it Prometeu acorrentado} (detalhe). 1611--18.}{\externalfigure[HESIODO_TRABALHOS_THUMB2][width=\textwidth]}

\page %---------------------------------------------------------|

\MyCover{HESIODO_TRABALHOS_THUMB.pdf}

\page

{\bf Trabalhos e dias} é um poema épico de 828 versos
em que são contados alguns dos mitos gregos mais
conhecidos, como o de Prometeu e o de
Pandora. Com a ajuda das Musas, o poeta narra em primeira pessoa e se dirige 
a seu irmão Perses, na tentativa de ensinar a ele
verdades divinas a respeito das práticas humanas.

\page

\Hedra


\stoptext

%Trabalhos e dias é o poema grego no qual se
% mencionam a caixa de Pandora — na verdade uma
% ânfora —, as linhagens, raças ou idades do homem
% e uma poética representação das estações do
% ano e das atividades agrícolas associadas a elas.
% Nesse livro, porém, não é de quase-super-homens
% como Aquiles e Odisseu que se fala, mas de
% outros tipos de heróis: o poeta que de tudo sabe;
% o bom rei, que zela pela justiça em sua comu-
% nidade; e o agricultor bem-sucedido que, para
% produzir riqueza por meio de sua propriedade ou
% fazenda, deve não só trabalhar arduamente, mas
% atentar a uma série enorme de regras climáticas,
% morais e religiosas, aquilo que nós chamamos de
% acaso também espreita.
% 
% Além de Trabalhos e dias, somente chegaram
% inteiros até nós os poemas Teogonia e Escudo de
% Héracles, entre aqueles atribuídos na Antiguidade
% ao grego Hesíodo, poeta que teria vivido por volta
% dos séculos VIII e VII a.C., ou seja, mais ou menos
% na mesma época que Homero, tido pelos antigos
% como o autor da Ilíada e da Odisseia.
% 
% É possível enumerar vários aspectos relacio-
% nados à cultura grega do mesmo período — como
% a introdução e a expansão do uso da escrita — que
% fazem muitos pesquisadores duvidar que tenha
% havido um poeta histórico chamado Hesíodo e que
% ele tenha composto por escrito os poemas asso-
% ciados a seu nome. Mas, para entender um poema
% como Trabalhos e dias, o próprio texto ainda é
% nossa principal ferramenta, de sorte que muitas
% das questões a ele pertinentes precisarão con-
% tinuar sem uma resposta categórica, como, por
% exemplo, quando