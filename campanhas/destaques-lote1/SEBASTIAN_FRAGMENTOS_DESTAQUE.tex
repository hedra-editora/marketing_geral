% Preencher com o nome das cor ou composição RGB (ex: [r=0.862, g=0.118, b=0.118]) 
\usecolors[crayola] 			   % Paleta de cores pré-definida: wiki.contextgarden.net/Color#Pre-defined_colors

% Cores definidas pelo designer:
% MyGreen		r=0.251, g=0.678, b=0.290 % 40ad4a
% MyCyan		r=0.188, g=0.749, b=0.741 % 30bfbd
% MyRed			r=0.820, g=0.141, b=0.161 % d12429
% MyPink		r=0.980, g=0.780, b=0.761 % fac7c2
% MyGray		r=0.812, g=0.788, b=0.780 % cfc9c7
% MyOrange		r=0.980, g=0.671, b=0.290 % faab4a

% Configuração de cores
\definecolor[MyColor][PigPink]      % ou ex: [r=0.862, g=0.118, b=0.118] % corresponde a RGB(220, 30, 30)
\definecolor[MyColorText][black]     % ou ex: [r=0.862, g=0.118, b=0.118] % corresponde a RGB(167, 169, 172)

% Classe para diagramação dos posts
\environment{marketing.env}		   

\starttext %---------------------------------------------------------|

\Mensagem{MIHAIL SEBASTIAN}

\startMyCampaign

\hyphenpenalty=10000
\exhyphenpenalty=10000
«Dentre todas as avarezas essa é a que eu mais detesto: poupar migalhas de vida para mais tarde, para a velhice, para os momentos de tédio eventual\ldots{}» 

\vfill\hfill →

\page

\hyphenpenalty=10000
\exhyphenpenalty=10000

«\ldots{}Não consigo fazer esse tipo de economia. Mordo uma maçã na medida da minha fome: uma vez saciado, jogo-a fora.»

\stopMyCampaign

{\vfill\scale[factor=6]{\Seta\,Trecho de {\bf Fragmentos de um diário}}\setupinterlinespace[line=1.5ex]\scale[factor=6]{{\bf encontrado}, de Mihail Sebastian.}} 

\page %---------------------------------------------------------| 

\MyCover{./SEBASTIAN_FRAGMENTOS_THUMB.jpg}

\page %---------------------------------------------------------|

\Hedra

\stoptext %---------------------------------------------------------|