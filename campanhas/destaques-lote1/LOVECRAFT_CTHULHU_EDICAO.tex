% LOVECRAFT_CTHULHU_EDICAO.tex
% Preencher com o nome das cor ou composição RGB (ex: [r=0.862, g=0.118, b=0.118]) 
\usecolors[crayola] 			   % Paleta de cores pré-definida: wiki.contextgarden.net/Color#Pre-defined_colors

% Cores definidas pelo designer:
% MyGreen		r=0.251, g=0.678, b=0.290 % 40ad4a
% MyCyan		r=0.188, g=0.749, b=0.741 % 30bfbd
% MyRed			r=0.820, g=0.141, b=0.161 % d12429
% MyPink		r=0.980, g=0.780, b=0.761 % fac7c2
% MyGray		r=0.812, g=0.788, b=0.780 % cfc9c7
% MyOrange		r=0.980, g=0.671, b=0.290 % faab4a

% Configuração de cores
\definecolor[MyColor][x=c43322]      % ou ex: [r=0.862, g=0.118, b=0.118] % corresponde a RGB(220, 30, 30)
\definecolor[MyColorText][black]     % ou ex: [r=0.862, g=0.118, b=0.118] % corresponde a RGB(167, 169, 172)

% Classe para diagramação dos posts
\environment{marketing.env}		   

\starttext %---------------------------------------------------------|

\Mensagem{POR DENTRO DA EDIÇÃO}

\startMyCampaign

\hyphenpenalty=10000
\exhyphenpenalty=10000

{\bf O HORROR CÓSMICO DE H.P. LOVECRAFT}

\stopMyCampaign

%\vfill\scale[lines=1.5]{\MyStar[MyColorText][none]}

\page %---------------------------------------------------------| 

\MyCover{LOVECRAFT_ CTHULHU_THUMB.jpg}

\page %---------------------------------------------------------| 

\hyphenpenalty=10000
\exhyphenpenalty=10000

{\bf O CHAMADO DE CTHULHU} apresenta a figura mais popular de Lovecraft, centro da série sobre os Grandes Antigos, as gigantescas e incompreensíveis criaturas anteriores a esta Terra. É a cristalização, numa imagem, de um tipo específico de terror chamado {\bf CÓSMICO}: mas um cósmico íntimo e literário.

\page

 Em seu Cthulhu, um monstro que dorme no fundo do mar --- descomunal e de dimensões inqualificáveis ---, o autor procedeu a uma metamorfose do próprio Kraken, monstro marinho da mitologia escandinava, para encontrar um código de seus próprios horrores: mas que funcionou bem, porque o verdadeiro mergulho no medo de um é o mergulho no {\bf MEDO DE TODOS}. 


 \page

 Um dos grandes clássicos de horror do século {\cap XX}, {\bf O CHAMADO DE CTHULHU} permite um desconcertante passeio pelo universo macabro de um dos grandes mestres do horror.

\page %---------------------------------------------------------|

\hyphenpenalty=10000
\exhyphenpenalty=10000

«Algum dia a recomposição desse conhecimento dissociado abrirá visões tão aterradoras
da realidade, e da nossa temível posição nela, que ficaremos loucos com
a revelação ou fugiremos da luz mortal para a paz e a segurança de uma
nova idade das trevas.»

{\vfill\scale[factor=5]{\Seta\,Trecho do livro {\bf O chamado de Cthulhu}, de H.P.}\setupinterlinespace[line=1.5ex]\scale[factor=5]{Lovecraft.}}

\page %---------------------------------------------------------|

\Hedra

\stoptext %---------------------------------------------------------|