% AUTOR_LIVRO_AUTOR.tex
% Preencher com o nome das cor ou composição RGB (ex: [r=0.862, g=0.118, b=0.118]) 
\usecolors[crayola]                % Paleta de cores pré-definida: wiki.contextgarden.net/Color#Pre-defined_colors

% Cores definidas pelo designer:
% MyGreen       r=0.251, g=0.678, b=0.290 % 40ad4a
% MyCyan        r=0.188, g=0.749, b=0.741 % 30bfbd
% MyRed         r=0.820, g=0.141, b=0.161 % d12429
% MyPink        r=0.980, g=0.780, b=0.761 % fac7c2
% MyGray        r=0.812, g=0.788, b=0.780 % cfc9c7
% MyOrange      r=0.980, g=0.671, b=0.290 % faab4a

% Configuração de cores
\definecolor[MyColor][ShinyShamrock]      % ou ex: [r=0.862, g=0.118, b=0.118] % corresponde a RGB(220, 30, 30)
\definecolor[MyColorText][SpringGreen]     % ou ex: [r=0.862, g=0.118, b=0.118] % corresponde a RGB(167, 169, 172)

% Classe para diagramação dos posts
\environment{marketing.env}        


\def\MyBackground#1{
\defineoverlay
  [backgroundimage]
  [{\externalfigure[#1][height=\overlayheight]}]
}

% Cabeço e rodapé: Informações (caso queira trocar alguma coisa)
        \def\MensagemSaibaMais  {SAIBA MAIS:}
        \def\MensagemSite           {HEDRA.COM.BR}
        \def\MensagemLink           {LINK NA BIO}

\starttext  %---------------------------------------------------------|
\Mensagem{FIGURA INCONTORNÁVEL}

% Foto para background
\MyBackground{FREUD_LITERATURA_2}

\startstandardmakeup[background=backgroundimage]
\startMyCampaign
\vfill\scale[factor=4]{\Seta\,SIGMUND FREUD (1856--1939)}
\stopMyCampaign
\stopstandardmakeup

\page 
\Mensagem{FIGURA INCONTORNÁVEL}


\hyphenpenalty=10000
\exhyphenpenalty=10000

Criador da {\bf PSICANÁLISE} e um dos mais influentes pensadores modernos, {\bf FREUD} nasceu no então Império Austro-Húngaro, cuja capital, Viena, era um dos principais centros de {\bf INOVAÇÃO ARTÍSTICA E CIENTÍFICA}.

\page

Proveninente de uma família de comerciantes judeus, {\bf FREUD} formou-se em medicina na Universidade de Viena, especializando-se em {\bf NEUROLOGIA}. 

\page

Como ainda não existiam as {\bf DROGAS NEUROPSIQUIÁTRICAS}, eram poucos os recursos disponíveis para o tratamento dos pacientes. Em busca por {\bf CONHECIMENTO E MÉTODOS}, Freud percorreu a Europa e suas clínicas.

\page

 Durante esse tempo, {\bf FIRMOU O SEU VERDADEIRO OBJETO DE PESQUISA}, que não eram mais os nervos ou cérebro, mas a {\bf MENTE}. Assim, Freud deixou de lado a neurologia e criou a {\bf PSICANÁLISE} --- um novo, vasto e revolucionário campo de pesquisa e de especulação teórica.


\page

\MyPicture{FREUD_LITERATURA_1}

 Em 1930, Freud recebeu o {\bf PRÊMIO GOETHE}, um dos maiores da literatura mundial. 


\page

 A sua obra, que inclui filosofia, arte e cultura, teve um {\bf IMPACTO IMENSURÁVEL NA CULTURA OCIDENTAL}, de modo que Freud consolidou-se como uma {\bf FIGURA INCONTORNÁVEL DO PENSAMENTO CONTEMPORÂNEO.}
\page

\MyCover{FREUD_LITERATURA_THUMB}

\page %----------------------------------------------------------|

\Hedra

\stoptext %---------------------------------------------------------|


% Sigmund Schlomo Freud (Freiberg in Mähren, 1856 — Londres,1939), o criador da psicanálise e um dos mais influentes pensadores modernos, nasceu no então Império Austro-Húngaro, cuja capital,Viena, à época um dos principais centros de inovação artística ecientífica, rivalizava com Paris. O jovem Freud, cumprindo o papelde ascensão social da família de comerciantes judeus, formou-seem medicina na Universidade de Viena (1881), especializando-se em neurologia. Mas a “ciência dos nervos” (nevrologia), como amedicina em geral, não tinha então recursos reais a oferecer aospacientes: não existiam as drogas neuropsiquiátricas. Em sua buscapor conhecimento e capacidade de ação, Freud percorreu a Europa
% em um “périplo de clínicas”, cujo melhor resultado foi o contatocom a de Charcot, em Paris, que usava a hipnose para abordar ahisteria. Freud adota a técnica, mas vê nela grandes limitações,incluindo a diferente capacidade de indução dos pacientes. Porémo “grande salto” fora dado: não eram mais os nervos, o cérebro,o objeto de pesquisa, mas a mente. Freud, por fim, abandonaria aneurologia (estudo dos nervos) e criaria a psicanálise (análise damente). Um novo, vasto e revolucionário campo de pesquisa e de es-peculação teórica, além de um novo e igualmente vasto campo parao (auto)conhecimento humano (a partir, entre outros, deA interpre-tação dos sonhos[1899] e de seus estudos sobre a sexualidade), a obrade Freud impactaria extensa e profundamente a cultura ocidental noséculo, da psiquiatria às artes plásticas, do cinema à literatura, dafilosofia à sociologia, do comportamento ao próprio senso comum.O homem freudiano, assim como, antes, o homem copernicano e ohomem galileano, jamais seria o mesmo. E esse homem, em muitosaspectos, é o próprio homem contemporâneo. Sigmund Freud, cujaobra se estenderia para além da psicanálise para incluir filosofia,arte e cultura, recebeu em 1930 o Prêmio Goethe, um dos maioresda literatura mundial.

