% AUTOR_LIVRO_EDICAO.tex
% Preencher com o nome das cor ou composição RGB (ex: [r=0.862, g=0.118, b=0.118]) 
\usecolors[crayola] 			   % Paleta de cores pré-definida: wiki.contextgarden.net/Color#Pre-defined_colors

% Cores definidas pelo designer:
% MyGreen		r=0.251, g=0.678, b=0.290 % 40ad4a
% MyCyan		r=0.188, g=0.749, b=0.741 % 30bfbd
% MyRed			r=0.820, g=0.141, b=0.161 % d12429
% MyPink		r=0.980, g=0.780, b=0.761 % fac7c2
% MyGray		r=0.812, g=0.788, b=0.780 % cfc9c7
% MyOrange		r=0.980, g=0.671, b=0.290 % faab4a

% Configuração de cores
\definecolor[MyColor][x=c0e016]      % ou ex: [r=0.862, g=0.118, b=0.118] % corresponde a RGB(220, 30, 30)
\definecolor[MyColorText][black]     % ou ex: [r=0.862, g=0.118, b=0.118] % corresponde a RGB(167, 169, 172)

% Classe para diagramação dos posts
\environment{marketing.env}		   

\starttext %---------------------------------------------------------|

\Mensagem{POR DENTRO DA EDIÇÃO}

\startMyCampaign

\hyphenpenalty=10000
\exhyphenpenalty=10000

{\bf MÁRIO DE ANDRADE E  IDENTIDADE NACIONAL ATRAVÉS DA CANÇÃO}

\stopMyCampaign


\page
\MyCover{ANDRADE_ESTRANHA_THUMB}

\page %---------------------------------------------------------| 

\hyphenpenalty=10000
\exhyphenpenalty=10000

{\bf A ESTRANHA FORÇA DA CANÇÃO} reúne dez textos de Mário de Andrade, escritos entre 1930 e 1942, sobre a canção popular brasileira.

\page

 A obra apresenta os pensamentos do autor acerca da música e do modo como esta se relaciona com  a {\bf CONSTRUÇÃO DA IDENTIDADE NACIONAL}. Neste sentido, propõe-se uma abordagem crítica para compreender o papel central da música popular brasileira na formação da {\bf IDEIA DE NAÇÃO}.

\page %---------------------------------------------------------|

Ao longo das suas investigações, Mário de Andrade explora as maneiras por meio das quais a canção {\bf TRADUZ OS VALORES E TENSÕES} de uma sociedade nacional em formação. 

\page

Em textos como «A pronúncia cantada...», o autor explora a interação da música popular brasileira com a língua portuguesa, realizando uma {\bf ANÁLISE LINGUÍSTICA DO CANTAR BRASILEIRO} e investigando as peculiaridades fonéticas da fala e como elas se manifestam nas canções.


\page

Em «Dicionário musical brasileiro», por sua vez, o Mário compila alguns dos principais termos e vocábulos da música brasileira, apresentando as {\bf CARACTERÍSTICAS HISTÓRICAS E CULTURAIS} que definem a sonoridade do país.

\page

Todos os textos se complementam para articular uma pesquisa profunda sobre a música brasileira, essa forma de {\bf EXPRESSÃO COLETIVA} que conecta o indivíduo ao sentimento de pertencimento e à construção da identidade nacional.

\page

\Hedra

\stoptext %---------------------------------------------------------|




