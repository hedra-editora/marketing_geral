% AUTOR_LIVRO_EDICAO.tex
% Preencher com o nome das cor ou composição RGB (ex: [r=0.862, g=0.118, b=0.118]) 
\usecolors[crayola] 			   % Paleta de cores pré-definida: wiki.contextgarden.net/Color#Pre-defined_colors

% Cores definidas pelo designer:
% MyGreen		r=0.251, g=0.678, b=0.290 % 40ad4a
% MyCyan		r=0.188, g=0.749, b=0.741 % 30bfbd
% MyRed			r=0.820, g=0.141, b=0.161 % d12429
% MyPink		r=0.980, g=0.780, b=0.761 % fac7c2
% MyGray		r=0.812, g=0.788, b=0.780 % cfc9c7
% MyOrange		r=0.980, g=0.671, b=0.290 % faab4a

% Configuração de cores
\definecolor[MyColor][x=c0e016]      % ou ex: [r=0.862, g=0.118, b=0.118] % corresponde a RGB(220, 30, 30)
\definecolor[MyColorText][black]     % ou ex: [r=0.862, g=0.118, b=0.118] % corresponde a RGB(167, 169, 172)

% Classe para diagramação dos posts
\environment{marketing.env}		   

\starttext %---------------------------------------------------------|

\Mensagem{POR DENTRO DA EDIÇÃO}

\startMyCampaign

\hyphenpenalty=10000
\exhyphenpenalty=10000

{\bf MÁRIO DE ANDRADE E  IDENTIDADE NACIONAL ATRAVÉS DA CANÇÃO}

\stopMyCampaign


\page
\MyCover{ANDRADE_ESTRANHA_THUMB}

\page %---------------------------------------------------------| 

\hyphenpenalty=10000
\exhyphenpenalty=10000

{\bf A ESTRANHA FORÇA DA CANÇÃO} reúne dez textos de Mário de Andrade, escritos entre 1930 e 1942, sobre a canção popular brasileira.

\page

 A obra apresenta as variações do pensamento do autor acerca da música ao longo do tempo, oferecendo uma análise da relação entre a música e a {\bf CONSTRUÇÃO DA IDENTIDADE NACIONAL}. Neste sentido, propõe-se uma abordagem crítica para compreender o papel central da música popular brasileira na formação da {\bf IDEIA DE NAÇÃO}.

\page %---------------------------------------------------------|

Ao longo das suas investigações, Mário de Andrade explorando as maneiras por meio das quais a canção traduz os valores e tensões de uma sociedade nacional em formação e transformação. 

\page

Em textos como «Ensaio sobre a música popular brasileira», o autor oferece um panorama geral sobre a música, questionando seu papel como uma forma de expressão coletiva que conecta o indivíduo ao sentimento de pertencimento e à construção da identidade do «ser brasileiro».

\page %---------------------------------------------------------|

Em «Gravação nacional», Mário de Andrade analisa criticamente a indústria fonográfica e seu impacto na popularização dos ritmos e compositores brasileiros. O autor examina como a gravação de discos influenciou o mercado musical, destacando os ritmos e autores que ganharam destaque. Além disso, há reflexões sobre as transformações trazidas pela indústria para a música popular em termos da moldagem e homogeneização dos estilos, distanciando-os de suas raízes culturais. 


\page
Em «A pronúncia cantada...», é explorada a interação da música popular brasileira com a língua portuguesa. Neste sentido, Mário de Andrade realiza uma análise linguística do cantar brasileiro, investigando as peculiaridades fonéticas da fala e como elas se traduzem nas canções, criando uma sonoridade única e característica da música brasileira. 

\page

Dicionário musical brasileiro, por sua vez, compila alguns dos principais termos e vocábulos da música brasileira, apresentando as características históricas e culturais que definem a sonoridade do país. Assim, Mário de Andrade revela o significado e a evolução de termos essenciais para compreender a música popular brasileira, oferecendo uma visão aprofundada sobre os elementos que formam seus ritmos e estilos.

\Hedra

\stoptext %---------------------------------------------------------|





