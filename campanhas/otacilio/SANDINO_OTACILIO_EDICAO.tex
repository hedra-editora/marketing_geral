% AUTOR_LIVRO_EDICAO.tex
% Preencher com o nome das cor ou composição RGB (ex: [r=0.862, g=0.118, b=0.118]) 
\usecolors[crayola] 			   % Paleta de cores pré-definida: wiki.contextgarden.net/Color#Pre-defined_colors

% Cores definidas pelo designer:
% MyGreen		r=0.251, g=0.678, b=0.290 % 40ad4a
% MyCyan		r=0.188, g=0.749, b=0.741 % 30bfbd
% MyRed			r=0.820, g=0.141, b=0.161 % d12429
% MyPink		r=0.980, g=0.780, b=0.761 % fac7c2
% MyGray		r=0.812, g=0.788, b=0.780 % cfc9c7
% MyOrange		r=0.980, g=0.671, b=0.290 % faab4a

% Configuração de cores
\definecolor[MyColor][x=4cb7c4]      % ou ex: [r=0.862, g=0.118, b=0.118] % corresponde a RGB(220, 30, 30)
\definecolor[MyColorText][black]     % ou ex: [r=0.862, g=0.118, b=0.118] % corresponde a RGB(167, 169, 172)

% Classe para diagramação dos posts
\environment{marketing.env}		   

\starttext %---------------------------------------------------------|

\Mensagem{POR DENTRO DA EDIÇÃO}

% \startMyCampaign

% \hyphenpenalty=10000
% \exhyphenpenalty=10000

% {\bf 
% NOSSO LIVRO É\ 
% LEGAL POR\ 
% MUITOS MOTIVOS, \
% E VAMOS TE\ 
% CONTAR QUAIS}

% \stopMyCampaign

% %\vfill\scale[lines=1.5]{\MyStar[MyColorText][none]}

% \page %---------------------------------------------------------| 

\MyCover{OTACILIO_CAPA}

\page %---------------------------------------------------------| 

\hyphenpenalty=10000
\exhyphenpenalty=10000

{\bf OTACÍLIO BATISTA} foi um renomado cantador e repentista brasileiro. Sua lírica se estabeleceu em João Pessoa, Paraíba, onde saiu do anonimato e se consagrou em seu dom artístico. 

\page

A {\bf POESIA DE REPENTE} foi seu veículo de expressão e Batista se destacou como um dos grandes nomes deste gênero.
Contribuiu significativamente para a literatura de cordel, tornando-se uma figura emblemática neste gênero.

\page

A biografia {\bf OTACÍLIO BATISTA, UMA HISTÓRIA DO REPENTE BRASILEIRO} foi escrita por Sandino Patriota, seu neto, e lançada em 2023, ano que marcou o centenário do nascimento do artista.


\page %---------------------------------------------------------|



\hyphenpenalty=10000
\exhyphenpenalty=10000

«Otacílio é o poeta de um interregno, de uma transição, de uma viagem cheia de desencontros,
idas e voltas e também de conflitos. Do campo para a cidade; da tradição oral do repente
ao registro da poesia escrita, Otacílio viveu e participou dessas
rupturas, muitas vezes contra a própria vontade, mas sempre em
uma posição de destaque.»


{\vfill\scale[factor=5]{\Seta\,Trecho do livro {\bf Otacílio Batista, uma história do}}\setupinterlinespace[line=1.5ex]\scale[factor=5]{{\bf repente brasileiro}, de Sandino Patriota.}}


\page %---------------------------------------------------------|

\Hedra

\stoptext %---------------------------------------------------------|







