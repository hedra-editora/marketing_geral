% Preencher com o nome das cor ou composição RGB (ex: [r=0.862, g=0.118, b=0.118]) 
\usecolors[crayola] 			   % Paleta de cores pré-definida: wiki.contextgarden.net/Color#Pre-defined_colors

% Cores definidas pelo designer:
% MyGreen		r=0.251, g=0.678, b=0.290 % 40ad4a
% MyCyan		r=0.188, g=0.749, b=0.741 % 30bfbd
% MyRed			r=0.820, g=0.141, b=0.161 % d12429
% MyPink		r=0.980, g=0.780, b=0.761 % fac7c2
% MyGray		r=0.812, g=0.788, b=0.780 % cfc9c7
% MyOrange		r=0.980, g=0.671, b=0.290 % faab4a

% Configuração de cores
\definecolor[MyColor][x=fff870]      % ou ex: [r=0.862, g=0.118, b=0.118] % corresponde a RGB(220, 30, 30)
\definecolor[MyColorText][black]     % ou ex: [r=0.862, g=0.118, b=0.118] % corresponde a RGB(167, 169, 172)

% Classe para diagramação dos posts
\environment{marketing.env}		   

\starttext %---------------------------------------------------------|

\Mensagem{A TRANSFORMAÇÃO DE KAFKA}

\startMyCampaign
\tfx

\hyphenpenalty=10000
\exhyphenpenalty=10000

100 ANOS DE KAFKA\\
AUTOR DE\\ {\bf A TRANSFORMAÇÃO}\\
E NÃO\\ {\bf A METAMORFOSE}?

\stopMyCampaign

%\vfill\scale[lines=1.5]{\MyStar[MyColorText][none]}

\page %---------------------------------------------------------| 

\MyCover{./KAFKA_TRANSFORMACAO_THUMB.pdf}

\page %---------------------------------------------------------| 

\hyphenpenalty=10000
\exhyphenpenalty=10000

Tradicionalemente traduzido para o português como {\it A metamorfose}, {\it Die Verwandlung} também pode ser considerado como {\bf A TRANSFORMAÇÃO}. O termo é mais amplo e versátil, pois pode referir-se a mudanças físicas, psicológicas ou situacionais, sem necessariamente carregar a mesma conotação natural e completa de {\it metamorfose}.

\page %---------------------------------------------------------| 

\hyphenpenalty=10000
\exhyphenpenalty=10000

{\it Metamorfose} carrega uma conotação mais biológica e específica, associada a mudanças naturais e inevitáveis, como a metamorfose de uma lagarta em borboleta. Metaformose pode induz a pensar em um personagem trágico e mais triste, evocando uma mudança biológica inevitável e dolorosa. Por outro lado, {\it transformação} sugere uma conotação mais psicológica, o que abre a narrativa para o lúdico, o tom jocoso e humorístico.

\page %---------------------------------------------------------| 

\placefigure{}{\externalfigure[KAFKA_TRANSFORMAÇÃO_2.jpeg][width=\textwidth]}


\page %---------------------------------------------------------| 

\hyphenpenalty=10000
\exhyphenpenalty=10000

O escritor argentino Jorge Luis Borges também criticava o título
consagrado nas traduções, afirmando que a língua alemã possui a
palavra {\it Metamorphose}, e Kafka a adotaria se sua intenção fosse de
fato privilegiar a mutação biológica, o que não é o caso. E também o
crítico Otto Maria Carpeaux, em um texto de 1941, marco inaugural da
recepção crítica de Kafka no Brasil, referiu-se à história de Gregor
Samsa como\ldots{} {\bf A TRANSFORMAÇÃO}!

\page %---------------------------------------------------------|

\Hedra

\stoptext %---------------------------------------------------------|