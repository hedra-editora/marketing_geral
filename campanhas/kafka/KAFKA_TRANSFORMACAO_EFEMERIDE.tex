% AUTOR_LIVRO_EFEMERIDE.tex
% Preencher com o nome das cor ou composição RGB (ex: [r=0.862, g=0.118, b=0.118]) 
\usecolors[crayola] 			   % Paleta de cores pré-definida: wiki.contextgarden.net/Color#Pre-defined_colors

% Cores definidas pelo designer:
% MyGreen		r=0.251, g=0.678, b=0.290 % 40ad4a
% MyCyan		r=0.188, g=0.749, b=0.741 % 30bfbd
% MyRed			r=0.820, g=0.141, b=0.161 % d12429
% MyPink		r=0.980, g=0.780, b=0.761 % fac7c2
% MyGray		r=0.812, g=0.788, b=0.780 % cfc9c7
% MyOrange		r=0.980, g=0.671, b=0.290 % faab4a

% Configuração de cores
\definecolor[MyColor][Melon]      % ou ex: [r=0.862, g=0.118, b=0.118] % corresponde a RGB(220, 30, 30)
\definecolor[MyColorText][black]     % ou ex: [r=0.862, g=0.118, b=0.118] % corresponde a RGB(167, 169, 172)

% Classe para diagramação dos posts
\environment{marketing.env}		   

\starttext %---------------------------------------------------------|

\hyphenpenalty=10000
\exhyphenpenalty=10000

\Mensagem{O MAL DO HOMEM MODERNO} %Sempre usar esse header

\MyPicture{KAFKA_TRANSFORMAÇÃO_1}

\vfill\scale[factor=6]{\Seta\,100 ANOS SEM {\bf FRANZ KAFKA}}

\page %---------------------------------------------------------| 

\hyphenpenalty=10000
\exhyphenpenalty=10000

 Esse ano comemoramos {\bf 100 ANOS DA MORTE DE FRANZ KAFKA}. Por ter nascido em uma família judia em Praga, em 1883, momento no qual a cidade pertencia ao Império Austro-Húngaro, cresceu sob a influência de três culturas diversas – a judia, a tcheca e a alemã.
 
% A sua complexa herança cultural e o fato de ter escrito toda a sua obra em alemão levanta a questão sobre a sua afiliação literária – se à tradição germânica, tcheca ou judaica –, a qual permanece motivo de debates sem uma conclusão definitiva.

\page %---------------------------------------------------------|

A sua complexa {\bf HERANÇA CULTURAL}, assim como sua carreira como {\bf FUNCIONÁRIO PÚBLICO}, são elementos biográficos que transparecem em sua obra.

\page

A sua experiência profissional, por exemplo, aparece na forma de mundo burocrático e metódico, que o escritor vai criticar e ironizar. Como formula {\bf WALTER BENJAMIN}: «O mundo das chancelarias e dos arquivos, das salas mofadas, escuras e decadentes, é o mundo de Kafka». 

\page

 O constante confronto entre seus personagens e o poder das instituições demonstra a impotência do ser humano, sobretudo do homem do século {\cap XX}. A atualidade dos temas abordados por Kafka, que discorrem sobre a {\bf CONDIÇÃO DO HOMEM MODERNO}, contribui para sua consagração como {\bf UM DOS ESCRITORES MAIS INFLUENTES E LIDOS DA LITERATURA MODERNA}.

\page
\MyCover{KAFKA_TRANSFORMACAO_THUMB}

\page %---------------------------------------------------------|

\Hedra

\stoptext %---------------------------------------------------------|


