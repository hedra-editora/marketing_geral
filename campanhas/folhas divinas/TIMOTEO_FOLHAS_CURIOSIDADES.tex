% AUTOR_LIVRO_CURIOSIDADES.tex
% Preencher com o nome das cor ou composição RGB (ex: [r=0.862, g=0.118, b=0.118]) 
\usecolors[crayola] 			   % Paleta de cores pré-definida: wiki.contextgarden.net/Color#Pre-defined_colors

% Cores definidas pelo designer:
% MyGreen		r=0.251, g=0.678, b=0.290 % 40ad4a
% MyCyan		r=0.188, g=0.749, b=0.741 % 30bfbd
% MyRed			r=0.820, g=0.141, b=0.161 % d12429
% MyPink		r=0.980, g=0.780, b=0.761 % fac7c2
% MyGray		r=0.812, g=0.788, b=0.780 % cfc9c7
% MyOrange		r=0.980, g=0.671, b=0.290 % faab4a

% Configuração de cores
\definecolor[MyColor][MangoTango]      % ou ex: [r=0.862, g=0.118, b=0.118] % corresponde a RGB(220, 30, 30)
\definecolor[MyColorText][black]     % ou ex: [r=0.862, g=0.118, b=0.118] % corresponde a RGB(167, 169, 172)

% Classe para diagramação dos posts
\environment{marketing.env}		   

\starttext %---------------------------------------------------------|

\hyphenpenalty=10000
\exhyphenpenalty=10000

\Mensagem{EM CONTEXTO} %Sempre usar esse header

\startMyCampaign

\hyphenpenalty=10000
\exhyphenpenalty=10000

VOCÊ CONHECE A ORIGEM INDÍGENA DA {\bf ERVA-MATE}?
\stopMyCampaign

\page %---------------------------------------------------------| 


\MyPhoto{FIGURA-10}

\vfill{\scale[factor=4]{Ilustração do livro {\bf A folha divina}, de Timóteo Verá Tupã Popygua}}
\page
\hyphenpenalty=10000
\exhyphenpenalty=10000

A {\bf ERVA-MATE}, planta utilizada no preparo do chimarrão, bebida tradicionalmente consumida no sul do continente sul-americano, foi difundida entre os não indígenas a partir da {\bf COLONIZAÇÃO ESPANHOLA} da América.

\page %---------------------------------------------------------|

A planta popularmente conhecida como erva-mate é chamada pelos povos Guarani de {\bf KA'A MIRI'I}, «folhas divinas», e desempenha um papel central na {\bf FILOSOFIA E COSMOGONIA} desses povos.

\page

A {\it Ka'a Miri'i} é utilizada para cura dos {\it nhe'ẽ}, «espíritos», e dos corpos doentes, assim como para
nomeação das crianças Nhande'iva'e que nascem nas aldeias, estando intimamente ligada à compreensão indígena sobre {\bf O MUNDO E SUA ORIGEM}.

\page

Desde o século {\cap XVII}, durante o violento processo colonial, os espanhóis passaram a destituir a erva-mate da {\bf CARGA COSMOLÓGICA}, reduzindo-a ao papel de bebida estimulante.

\page

Ironicamente, a erva-mate passou a ser consumida pelos \\{\bf TRABALHADORES INDÍGENAS ESCRAVIZADOS} nas minas de prata de Potosí, já que auxiliava a suportar o trabalho durante mais horas. Assim, a origem do consumo da erva-mate foi escamoteada conforme ela foi se tornando mercadoria e integrando o {\bf SISTEMA-MUNDO CAPITALISTA}.

\page

\MyCover{TIMOTEO_FOLHA_THUMB}

\page %---------------------------------------------------------|

\Hedra

\stoptext %---------------------------------------------------------|



