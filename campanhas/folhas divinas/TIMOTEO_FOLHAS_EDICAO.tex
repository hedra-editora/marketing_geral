% AUTOR_LIVRO_EDICAO.tex
% Preencher com o nome das cor ou composição RGB (ex: [r=0.862, g=0.118, b=0.118]) 
\usecolors[crayola] 			   % Paleta de cores pré-definida: wiki.contextgarden.net/Color#Pre-defined_colors

% Cores definidas pelo designer:
% MyGreen		r=0.251, g=0.678, b=0.290 % 40ad4a
% MyCyan		r=0.188, g=0.749, b=0.741 % 30bfbd
% MyRed			r=0.820, g=0.141, b=0.161 % d12429
% MyPink		r=0.980, g=0.780, b=0.761 % fac7c2
% MyGray		r=0.812, g=0.788, b=0.780 % cfc9c7
% MyOrange		r=0.980, g=0.671, b=0.290 % faab4a

% Configuração de cores
\definecolor[MyColor][MangoTango]      % ou ex: [r=0.862, g=0.118, b=0.118] % corresponde a RGB(220, 30, 30)
\definecolor[MyColorText][black]     % ou ex: [r=0.862, g=0.118, b=0.118] % corresponde a RGB(167, 169, 172)

% Classe para diagramação dos posts
\environment{marketing.env}		   

\starttext %---------------------------------------------------------|

\Mensagem{POR DENTRO DA EDIÇÃO}

\startMyCampaign

\hyphenpenalty=10000
\exhyphenpenalty=10000

{\bf COMO SURGIU O LIVRO «A FOLHA DIVINA»?}

\stopMyCampaign

%\vfill\scale[lines=1.5]{\MyStar[MyColorText][none]}

\page %---------------------------------------------------------| 

\MyCover{TIMOTEO_FOLHA_THUMB}

\page %---------------------------------------------------------| 

\hyphenpenalty=10000
\exhyphenpenalty=10000

{\bf A FOLHA DIVINA} é resultado de um trabalho de anos de pesquisa de Timóteo Verá Tupã Popygua com seus {\it xeramõi} e {\it xejaryi}, «avôs e avós».

\page %---------------------------------------------------------|

\hyphenpenalty=10000
\exhyphenpenalty=10000

Além disso, foram cruciais os estudos de {\bf FREG J.\,STOKES} nos arquivos históricos da América do Sul e Europa, e as muitas conversas entre Timóteo, Freg e Anita Ekman ao longo das caminhadas pelo Paraguai, Argentina e as {\it tekoas}, «aldeias», no 
Brasil.

\page

«Nas últimas duas décadas, tive a sorte de ter caminhado ao lado
de Timóteo Verá Tupã Popygua e aprendido — através de suas palavras —, entre 
uma roda de chimarrão e outra nas {\it tekoas}, “aldeias”,
a pensar e a sentir a história deste especial território da América
do Sul, onde floresce a {\it ka’a ete’i}: a {\it Ka’aguy porã}, a Mata Atlântica.»

{\vfill\scale[factor=5]{{\bf Anita Ekman}, na apresentação de «A folha divina».}}

\page %---------------------------------------------------------|

\Hedra

\stoptext %---------------------------------------------------------|


