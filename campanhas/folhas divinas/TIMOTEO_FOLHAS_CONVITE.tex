\usecolors[crayola]          % Paleta de cores pré-definida: wiki.contextgarden.net/Color#Pre-defined_colors
\definecolor[MyColor][MacaroniandCheese]      % ou ex: [r=0.862, g=0.118, b=0.118] % corresponde a RGB(220, 30, 30)
\definecolor[MyColorText][black]  % ou ex: [r=0.862, g=0.118, b=0.118] % corresponde a RGB(167, 169, 172)

\environment{marketing.env}      

% Cabeço e rodapé: Informações (caso queira trocar alguma coisa)
    \def\MensagemSaibaMais  {LIVRARIA DA TRAVESSA}
    \def\MensagemSite   {RUA DOS PINHEIROS, 513}
    \def\MensagemLink       {14 DE NOVEMBRO ÀS 19H}

\definepapersize    [post]        [width=93.1mm, height=135.3mm] %10.23cm
\definepapersize    [14x21]       [width=140mm, height=210mm]
\setuppapersize     [post]        %[14x21]

\setuplayout[marking=empty,
          location=middle,
          width=55mm,
          height=33mm,
          leftmargin=33mm,
          rightmargin=2mm,
          backspace=35mm,
          header=2mm,
          headerdistance=0mm,
          footer=1mm,
          footerdistance=0mm,
          topspace=0mm,
          leftedge=0mm,
          leftedgedistance=0mm,
          leftmargindistance=3mm,
          rightmargindistance=0mm,
          rightedgedistance=0mm,
          rightedge=0mm,
          topdistance=0mm,
          bottomdistance=0mm,
    %      top=21mm,
    %      bottom=21mm,
]

\definelayout[MyLayout][
          height=103.3mm,
%         backspace=5mm,
%         header=6.6mm,
          footer=9mm,
%         topspace=0mm,
%         topdistance=0mm,
%         headerdistance=3mm,
          footerdistance=1mm,
%         bottomdistance=0mm,
]

\definelayout[MyLayoutWithHeader][
          height=11cm,
%          backspace=5mm,
          height=165.4mm,
          header=8.1mm,
          footer=9mm,
%          topspace=0mm,
%          topdistance=0mm,
          headerdistance=2mm,
          footerdistance=0mm,
%          bottomdistance=0mm,
]

\def\Mensagem#1{
\def\MySlogan{\uppercasing{#1}}
\definelayer[Mensagem]
\setuplayer[Mensagem][width=\paperwidth, height=.8pt, color=MyColorText]
\setlayer[Mensagem]{
    \position(\dimexpr\leftmarginwidth-31mm,2.5mm){\FormularMImedium\bold  \switchtobodyfont[10pt] \color[color=MyColorText]{\bf \MySlogan}}
    \position(1mm,8mm){\blackrule[width=\paperwidth, height=.6pt, color=MyColorText]}
    \position(1mm,\dimexpr\paperheight-\footerheight-8mm){\blackrule[width=\paperwidth, height=.6pt, color=MyColorText]}
    \position(\dimexpr\leftmarginwidth-30mm,\dimexpr\paperheight-7mm){\switchtobodyfont[7pt]\FormularMImedium\bold \uppercasing{\MensagemSaibaMais}} 
    \position(\dimexpr\leftmarginwidth+2mm,\dimexpr\paperheight-7mm){\switchtobodyfont[7pt]\FormularMImedium\bold \uppercasing{↙\MensagemSite}}
    \position(\dimexpr\leftmarginwidth+2mm,\dimexpr\paperheight-4mm){\switchtobodyfont[7pt]\FormularMImedium\bold \uppercasing{↙\MensagemLink}}
    }        
\setuplayout[MyLayoutWithHeader]\setupbackgrounds[header][background=Mensagem]
}

\define\Hedra{
\setcharacterkerning[reset]
\Times
\switchtobodyfont[50pt] 
\hfill \switchtobodyfont[50pt] hedra \hfill \mbox{}  }

\starttext %--------------------------------------------------------|
%\showframe

\Mensagem{CONVITE DE LANÇAMENTO}
\tfx

\starttikzpicture[remember picture,overlay]
\node at (-18mm,-19mm) {\externalfigure[RAGUSA_LIRA_THUMB.pdf][width=3cm]};
\node at (42mm,-104mm) {\setcharacterkerning[reset]\Times\MyColorText\switchtobodyfont[15pt] hedra};
\node at (16mm,-104.6mm) {\externalfigure[LOGO_LIVRARIA.png][width=3.5cm]};
\stoptikzpicture
    
\inmargin{\tfxx
\setupinterlinespace[1ex]\mbox{}\blank[4cm]
}
    
\setupinterlinespace[3ex]

%\blank[4.7em]

\kern-3ex {\switchtobodyfont[18pt]\bf LIRA GREGA}

\blank[0.5em]

\tfxx\setupinterlinespace[2.5ex]
Giuliana Ragusa\\

\blank[3ex]

\setupinterlinespace[2.5ex]
A Editora Hedra e a Livraria Ponta de Lança convidam para o lançamento do livro «A folha divina», que acontecerá dia {\bf 15 DE NOVEMBRO, ÀS 19H}. O lançamento contará com a presença do autor, Timóteo Verá Tupã Popygua, e da organizadora, Anita Ekman. O livro narra a história da origem da erva-mate, explora o conhecimento guarani sobre o «bem viver», e destaca a importância da preservação da Mata Atlântica. Além disso, «A folha divina» aborda a apropriação capitalista da erva-mate, explorada há séculos pelos «juruá», o que contribui para a devastação das matas nativas.

\blank[3ex]

\starttikzpicture[remember picture,overlay]
\node at (41mm,-7mm) {\scale[factor=40]{\MyPortrait{RAGUSA_LIRA_FOTO_PB}}};
\stoptikzpicture

\stoptext %---------------------------------------------------------|

