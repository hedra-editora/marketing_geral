% Preencher com o nome das cor ou composição RGB (ex: [r=0.862, g=0.118, b=0.118]) 
\usecolors[crayola] 			   % Paleta de cores pré-definida: wiki.contextgarden.net/Color#Pre-defined_colors

% Cores definidas pelo designer:
% MyGreen		r=0.251, g=0.678, b=0.290 % 40ad4a
% MyCyan		r=0.188, g=0.749, b=0.741 % 30bfbd
% MyRed			r=0.820, g=0.141, b=0.161 % d12429
% MyPink		r=0.980, g=0.780, b=0.761 % fac7c2
% MyGray		r=0.812, g=0.788, b=0.780 % cfc9c7
% MyOrange		r=0.980, g=0.671, b=0.290 % faab4a

% Configuração de cores
\definecolor[MyColor][x=4c0055]      % ou ex: [r=0.862, g=0.118, b=0.118] % corresponde a RGB(220, 30, 30)
\definecolor[MyColorText][white]     % ou ex: [r=0.862, g=0.118, b=0.118] % corresponde a RGB(167, 169, 172)

% Classe para diagramação dos posts
\environment{marketing.env}

\starttext %---------------------------------------------------------|

\Mensagem{17 DE MAIO}

\startMyCampaign

\hyphenpenalty=10000
\exhyphenpenalty=10000
\MyPicture{lgbt.jpg}
{\tfx POR QUE 17 DE MAIO É O DIA INTERNACIONAL \scale[factor=fit]{\bf CONTRA A HOMOFOBIA?}}
\stopMyCampaign

\page


\page %---------------------------------------------------------| 

\hyphenpenalty=10000
\exhyphenpenalty=10000
É um dia para relembrar a decisão da Organização Mundial da Saúde (OMS) que, em 1990, retirou a homossexualidade da Classificação Estatística Internacional de Doenças e Problemas Relacionados com a Saúde (CID).

\page %---------------------------------------------------------| 

\hyphenpenalty=10000
\exhyphenpenalty=10000

Um dos precursores dessa luta foi o psicanalista e psiquiatra norte- -americano {\bf ROBERT STOLLER} (1924-1991). Já nos anos 1960, ele introduziu a noção de gênero na psicanálise. Ao criar essa ideia de {\bf IDENTIDADE DE GÊNERO}, Stoller ajudou a despatologizar as sexualidades dissidentes.

\page %---------------------------------------------------------|

\MyCover{./STOLLER_PERVERSAO_4CAPA.pdf}

\page %---------------------------------------------------------|

\MyCover{./STOLLER_PERVERSAO_CAPA.pdf}

\page %---------------------------------------------------------|

\Hedra

\stoptext %---------------------------------------------------------|