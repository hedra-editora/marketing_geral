% AUTOR_LIVRO_TRECHO.tex
% Preencher com o nome das cor ou composição RGB (ex: [r=0.862, g=0.118, b=0.118]) 
\usecolors[crayola] 			   % Paleta de cores pré-definida: wiki.contextgarden.net/Color#Pre-defined_colors

% Cores definidas pelo designer:
% MyGreen		r=0.251, g=0.678, b=0.290 % 40ad4a
% MyCyan		r=0.188, g=0.749, b=0.741 % 30bfbd
% MyRed			r=0.820, g=0.141, b=0.161 % d12429
% MyPink		r=0.980, g=0.780, b=0.761 % fac7c2
% MyGray		r=0.812, g=0.788, b=0.780 % cfc9c7
% MyOrange		r=0.980, g=0.671, b=0.290 % faab4a

% Configuração de cores
\definecolor[MyColor][x=e59fc3]      % ou ex: [r=0.862, g=0.118, b=0.118] % corresponde a RGB(220, 30, 30)
\definecolor[MyColorText][black]     % ou ex: [r=0.862, g=0.118, b=0.118] % corresponde a RGB(167, 169, 172)

% Classe para diagramação dos posts
\environment{marketing.env}		   

\starttext %---------------------------------------------------------|

\Mensagem{DESTAQUE}

\startMyCampaign

\hyphenpenalty=10000
\exhyphenpenalty=10000
«--- Quando nos veremos de novo?

--- Talvez nunca mais\ldots{} talvez hoje mesmo\ldots{}

--- Você ainda vai ficar muito tempo por aqui?


\vfill\hfill →

--- Não tenho como lhe dizer.

--- Justo você, que sabe tudo?

--- De que adianta, se não posso lhe dizer tudo o que sei\ldots!»

\stopMyCampaign

{\vfill\scale[factor=6]{\Seta\,Trecho do livro {\bf Para serem lidas à noite}, de}\setupinterlinespace[line=1.5ex]\scale[factor=6]{Ion Minulescu.}}

\page %---------------------------------------------------------| 

\MyCover{MINULESCU_NOITE_THUMB}

\page %---------------------------------------------------------|

\Hedra

\stoptext %---------------------------------------------------------|