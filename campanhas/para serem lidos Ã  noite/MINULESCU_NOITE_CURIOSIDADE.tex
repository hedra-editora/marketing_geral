% AUTOR_LIVRO_CURIOSIDADES.tex
% Preencher com o nome das cor ou composição RGB (ex: [r=0.862, g=0.118, b=0.118]) 
\usecolors[crayola] 			   % Paleta de cores pré-definida: wiki.contextgarden.net/Color#Pre-defined_colors

% Cores definidas pelo designer:
% MyGreen		r=0.251, g=0.678, b=0.290 % 40ad4a
% MyCyan		r=0.188, g=0.749, b=0.741 % 30bfbd
% MyRed			r=0.820, g=0.141, b=0.161 % d12429
% MyPink		r=0.980, g=0.780, b=0.761 % fac7c2
% MyGray		r=0.812, g=0.788, b=0.780 % cfc9c7
% MyOrange		r=0.980, g=0.671, b=0.290 % faab4a

% Configuração de cores
\definecolor[MyColor][x=e59fc3]      % ou ex: [r=0.862, g=0.118, b=0.118] % corresponde a RGB(220, 30, 30)
\definecolor[MyColorText][black]     % ou ex: [r=0.862, g=0.118, b=0.118] % corresponde a RGB(167, 169, 172)

% Classe para diagramação dos posts
\environment{marketing.env}		   

\starttext %---------------------------------------------------------|

\hyphenpenalty=10000
\exhyphenpenalty=10000

\Mensagem{ROMÊNIA EM FOCO} %Sempre usar esse header

\startMyCampaign

\hyphenpenalty=10000
\exhyphenpenalty=10000

PATRONO DA ARTE MODERNA

A COLEÇÃO DE {\bf ION MINULESCU}
\stopMyCampaign

\page %---------------------------------------------------------| 

\hyphenpenalty=10000
\exhyphenpenalty=10000

Durante as suas várias viagens pela Romênia e pelo exterior, Minulescu criou o hábito de coletar livros raros e objetos de arte.

\page

De esculturas barrocas e ícones ortodoxos à gravuras japonesas, a coleção do escritor tornou-se bastante ampla e heterogênea. A sua parte mais significativa corresponde, contudo, à {\bf ARTE MODERNA}.

\page

Apesar de já ser um escritor reconhecido nos anos 1920, Minulescu foi um grande {\bf APOIADOR DA NOVA GERAÇÃO DE ARTISTAS} de vanguarda romenos através dos salões oficiais que organizava e colecionando suas obras.

\page

Sua coleção contém obras de artistas romenos conhecidos do período entreguerras, como Nina Arbore, Alexandru Ciucurencu, Cecilia Cuțescu-Storck, Oscar Han e Max Herman Maxy. Um dos itens mais celebrados da coleção de Minulescu é um retrato seu pintado por {\bf VICTOR BRAUNER} em 1924.

\MyPicture{pintura}


\page

Em 1934, Minulescu e sua esposa, a poeta Claudia Millian, mudaram-se para um edifício modernista, onde a coleção, que àquela altura continha cerca de {\bf TREZENTAS OBRAS DE ARTE}, passou a ser exposta. Atualmente, a Casa Memorial Ion Minulescu e Claudia Millian pertence ao Museu Nacional da Literatura Romena. Os interiores como eram durante a vida de Minulescu, mantendo assim a integridade da coleção e sua visão artística.


\page %---------------------------------------------------------|

\MyCover{MINULESCU_NOITE_THUMB}

\page %---------------------------------------------------------|

\Hedra

\stoptext %---------------------------------------------------------|





