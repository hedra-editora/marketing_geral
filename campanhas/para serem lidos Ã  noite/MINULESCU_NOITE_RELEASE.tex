\setuppapersize[A4]
\usecolors[crayola]
\setupbackgrounds[paper][background=color,backgroundcolor=Almond]
	
	\definefontfeature
		[default]
		[default]
		[expansion=quality,protrusion=quality,onum=yes]
	\setupalign[fullhz,hanging]
	\definefontfamily [mainface] [sf] [Formular]
	\setupbodyfont[mainface,11pt]

% Indenting [4.4 cont-enp.p.65]
			\setupindenting[yes, 3ex]  % none small medium big next first dimension
			\indenting[next]           % never not no yes always first next
			
			% [cont-ent.p.76]
			\setupspacing[broad]  %broad packed
			% O tamanho do espaço entre o ponto final e o começo de uma sentença. 


\startsetups[grid][mypenalties]
    \setdefaultpenalties
    \setpenalties\widowpenalties{2}{10000}
    \setpenalties\clubpenalties {2}{10000}
\stopsetups

\setuppagenumbering
  [location={}]            % Estilo dos números de páginat

\setuphead[subject]
[style=bfb]		

\setuplayout[
          location=middle,
          %
          leftedge=0mm,
          leftedgedistance=0mm,
          leftmargin=20mm,
          leftmargindistance=0mm,
          width=100mm,
          rightmargindistance=0mm,
          rightmargin=20mm,
          rightedgedistance=0mm,
          rightedge=0mm,
          backspace=20mm,
          %
          top=21mm,
          topdistance=0mm,
          header=0mm,
          headerdistance=0mm,
          height=250mm,
          footerdistance=0mm,
          footer=0mm,
          bottomdistance=0mm,
          bottom=21mm,
          topspace=21mm,
        setups=mypenalties,
]

\setupalign[right]

\starttext
{\bfb Um dos maiores nomes do modernismo romeno do século {\cap XX} e XXX}

\blank[big]


\noindent{\it Em} Para serem lidos à noite {\it Ion Minulescu XXXXXXX.}

\blank[1cm]

\inoutermargin[width=60mm,hoffset=1cm,style=tfx,,voffset=3.5cm]{
\externalfigure[MINULESCU_NOITE_THUMB][width=50mm]
}


\inoutermargin[width=70mm,hoffset=1cm,voffset=4.5cm,style=tfx]
{\noindent{\bf Título} {\em Para serem lidos à noite}\\
{\bf Autor} Ion Minulescu\\
{\bf Tradução} Fernando Klabin\\
{\bf Editora} Hedra\\
{\bf ISBN} 978-85-7715-935-2\\
{\bf Pág.} 116\\
{\bf Pré-venda} XXXX\\
{\bf Preço} XXXXX
}


 \noindent{} Dentro de sua ampla trajetória ficcional, o romeno Minulescu, uma das figuras literárias do século {\cap XX} mais populares do país, se volta para a literatura fantástica e sobrenatural em {\em Para serem lidos à noite}, articulando real-irreal, lógico-ilógico, sagrado-profano. O título e advertência alinham-se e prenunciam uma pletora de mistérios sem fim.

O teor fantástico dos quatro contos reunidos no volume corrobora a influência do simbolismo sobre o autor, sendo ele admirador de mestres do gênero, como Oscar Wilde e Edgar Allan Poe. 

Na fronteira entre realidade e imaginário, Minulescu articula uma série de mistérios e “jogos de mostras e máscaras”, a serem vislumbrados pelo leitor notívago. 


% \blank[big]

% \page
\subject{Trechos do livro}

  \startitemize
    \item
    {\bf Capítulo {\em Bate-papo com o coisa-ruim}}

    \startitemize
    \item
    --- A imaginação dos poetas, na maior parte das vezes, ultrapassa a realidade e estrangula o verossímil. Ainda bem que a maioria das pessoas que frequenta a Igreja não lê poesia, e aqueles que lêem e acreditam na conversa fiada dos poetas não vão à Igreja.

    \stopitemize
  
  \item
    {\bf Capítulo {\em O homem do coração de ouro}}
\startitemize
\item --- Claro que conheço!\unknown Mas onde está o anel?\unknown Por que você arrancou a pedra?\unknown\\
--- Não fui eu quem arrancou.\\
--- Então quem foi?\\
--- Ele!\unknown\\
--- Ele quem?\unknown\\
--- O homem do coração de ouro!\\
--- Admirável título para uma novela fantástica!, exclamei.\\

    \stopitemize

\startitemize
\item
--- Você teria a bondade de me dizer quantos anos tem?\\
--- Trezentos e onze anos, e cento e noventa e oito dias,
considerando, claro, os trinta dias dos anos bissextos.\\
--- E por que é que você está há tanto tempo por aqui?\\
--- Não posso morrer até estar completo, como todos os
mortais.\\
--- Falta-lhe algo?\\
--- Sim\unknown\\
--- Algum órgão importante?\\
--- O mais importante de todos\unknown O coração! [\unknown]

  \stopitemize

\stoptext