% AUTOR_LIVRO_AUTOR.tex
% Preencher com o nome das cor ou composição RGB (ex: [r=0.862, g=0.118, b=0.118]) 
\usecolors[crayola] 			   % Paleta de cores pré-definida: wiki.contextgarden.net/Color#Pre-defined_colors

% Cores definidas pelo designer:
% MyGreen		r=0.251, g=0.678, b=0.290 % 40ad4a
% MyCyan		r=0.188, g=0.749, b=0.741 % 30bfbd
% MyRed			r=0.820, g=0.141, b=0.161 % d12429
% MyPink		r=0.980, g=0.780, b=0.761 % fac7c2
% MyGray		r=0.812, g=0.788, b=0.780 % cfc9c7
% MyOrange		r=0.980, g=0.671, b=0.290 % faab4a

% Configuração de cores
\definecolor[MyColor][x=e59fc3]      % ou ex: [r=0.862, g=0.118, b=0.118] % corresponde a RGB(220, 30, 30)
\definecolor[MyColorText][black]     % ou ex: [r=0.862, g=0.118, b=0.118] % corresponde a RGB(167, 169, 172)

% Classe para diagramação dos posts
\environment{marketing.env}		   

% Cabeço e rodapé: Informações (caso queira trocar alguma coisa)
 		\def\MensagemSaibaMais  {SAIBA MAIS:}
 		\def\MensagemSite		{HEDRA.COM.BR}
 		\def\MensagemLink       {LINK NA BIO}

\starttext %--------------------------------------------------------|

\Mensagem{ROMÊNIA EM FOCO}

\hyphenpenalty=10000
\exhyphenpenalty=10000

%\startMyCampaign

\MyPicture{minulescu}

%\stopMyCampaign

\vfill\scale[factor=6]{\Seta\,ION MINULESCU (1881--1944)}

\page %----------------------------------------------------------|

\hyphenpenalty=10000
\exhyphenpenalty=10000

Um dos maiores nomes do modernismo romeno romenos do século {\cap XX}, {\bf ION MINULESCU} transitou entre a poesia, prosa, crítica literária e dramaturgia.


\page

Durante a juventude estudou Direito em Paris, onde entrou em contato com o {\bf SIMBOLISMO FRANCÊS}, que influenciou profundamente a sua escrita. 

\page

Por mais que Minulescu tenha se tornado conhecido na Romênia sobretudo por sua contribuição ao movimento simbolista, foram suas incursões pelo {\bf FANTÁSTICO} que consolidaram-no como uma das figuras mais populares da literatura romena do século {\cap XX}.


\page

Esse desvio para a literatura fantástica e sobrenatural fica evidente em {\bf PARA SEREM LIDOS À NOITE} e suas narrativas que articulam real-irreal, lógico-ilógico, sagrado-profano.

\page %----------------------------------------------------------|

\MyCover{MINULESCU_NOITE_THUMB}

\page %----------------------------------------------------------|

\Hedra

\stoptext %---------------------------------------------------------|