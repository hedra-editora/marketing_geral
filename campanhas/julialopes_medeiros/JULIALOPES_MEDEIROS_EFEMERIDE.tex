% AUTOR_LIVRO_EFEMERIDE.tex
% Preencher com o nome das cor ou composição RGB (ex: [r=0.862, g=0.118, b=0.118]) 
\usecolors[crayola] 			   % Paleta de cores pré-definida: wiki.contextgarden.net/Color#Pre-defined_colors

% Cores definidas pelo designer:
% MyGreen		r=0.251, g=0.678, b=0.290 % 40ad4a
% MyCyan		r=0.188, g=0.749, b=0.741 % 30bfbd
% MyRed			r=0.820, g=0.141, b=0.161 % d12429
% MyPink		r=0.980, g=0.780, b=0.761 % fac7c2
% MyGray		r=0.812, g=0.788, b=0.780 % cfc9c7
% MyOrange		r=0.980, g=0.671, b=0.290 % faab4a

% Configuração de cores
\definecolor[MyColor][x=e87fb1]      % ou ex: [r=0.862, g=0.118, b=0.118] % corresponde a RGB(220, 30, 30)
\definecolor[MyColorText][black]     % ou ex: [r=0.862, g=0.118, b=0.118] % corresponde a RGB(167, 169, 172)

% Classe para diagramação dos posts
\environment{marketing.env}		   

\starttext %---------------------------------------------------------|

\hyphenpenalty=10000
\exhyphenpenalty=10000

\Mensagem{30 DE MAIO} %Sempre usar esse header

\MyPicture{THUMB_JULIA_LOPES}

\vfill\scale[factor=6]{\Seta\,90 ANOS SEM {\bf JÚLIA LOPES DE ALMEIDA}}

\page %---------------------------------------------------------| 

\hyphenpenalty=10000
\exhyphenpenalty=10000

90 anos atrás, o Brasil perdia {\bf JÚLIA LOPES DE ALMEIDA}, escritora amplamente reconhecida pelo público, consagrada pela publicação de romances, contos e crônicas. Neste mês de junho, a Hedra inicia a publicação das {\bf OBRAS COMPLETAS} da autora 
com o romance {\bf A FAMÍLIA MEDEIROS}, publicado em folhetim em 1891, em livro em 
1892 e reeditado e revisto pela autora em 1919.

\page %---------------------------------------------------------|

{\bf A FAMÍLIA MEDEIROS} surpreende pela atualidade: nesse romance, duas gerações de 
uma família tradicional de fazendeiros da região de Campinas (SP) entram em conflito. 
Os mais jovens defendem a abolição e a emancipação das mulheres; já o patriarca dos 
Medeiros é um brutal escravista. 

\page %---------------------------------------------------------|

\MyCover{JULIA_MEDEIROS_THUMB.pdf}

\page %---------------------------------------------------------|

\Hedra

\stoptext %---------------------------------------------------------|]
	
