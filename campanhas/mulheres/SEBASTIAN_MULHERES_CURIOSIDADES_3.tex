% AUTOR_LIVRO_CURIOSIDADES.tex
% Preencher com o nome das cor ou composição RGB (ex: [r=0.862, g=0.118, b=0.118]) 
\usecolors[crayola] 			   % Paleta de cores pré-definida: wiki.contextgarden.net/Color#Pre-defined_colors

% Cores definidas pelo designer:
% MyGreen		r=0.251, g=0.678, b=0.290 % 40ad4a
% MyCyan		r=0.188, g=0.749, b=0.741 % 30bfbd
% MyRed			r=0.820, g=0.141, b=0.161 % d12429
% MyPink		r=0.980, g=0.780, b=0.761 % fac7c2
% MyGray		r=0.812, g=0.788, b=0.780 % cfc9c7
% MyOrange		r=0.980, g=0.671, b=0.290 % faab4a

% Configuração de cores

\definecolor[MyColor][MiddleGreenYellow]      % ou ex: [r=0.862, g=0.118, b=0.118] % corresponde a RGB(220, 30, 30)
\definecolor[MyColorText][black]     % ou ex: [r=0.862, g=0.118, b=0.118] % corresponde a RGB(167, 169, 172)

% Classe para diagramação dos posts
\environment{marketing.env}		   

\starttext %---------------------------------------------------------|

\hyphenpenalty=10000
\exhyphenpenalty=10000

\Mensagem{LITERATURA ROMENA EM FOCO} %Sempre usar esse header

\startMyCampaign

\hyphenpenalty=10000
\exhyphenpenalty=10000

PARALELOS LITERÁRIOS\\
 {\bf SEBASTIAN E PROUST} 
%Aqui a manchete pode ser mais longa

\stopMyCampaign

\page %---------------------------------------------------------| 

\hyphenpenalty=10000
\exhyphenpenalty=10000

\MyPhoto{SEBASTIAN_MULHERES_1}

\page

{\bf MULHERES} é um romance carregado de sensibilidade. A atenção que Sebastian dispensa às minúcias da vida social revelam {\bf UMA VIDA INTERIOR AGITADA}, que se esconde por trás das aparências e em ângulos mortos do cotidiano.

% Atos triviais parecem atrelados a uma verdade profunda do sujeito, que o romancista é capaz de captar e realçar, ao menos momentaneamente. Como se, ao se ater às camadas externas de nossa socialização, outras camadas de interioridade se revelassem, transpassando o {\bf VÉU DO VISÍVEL}, do prosaico, da vida cotidiana.

\page %---------------------------------------------------------|

A intimidade dissecada aproxima {\bf MULHERES} de uma certa {\it estética proustiana} em que predomina a busca incessante pelo {\bf ÍNTIMO DO SUJEITO}: suas fantasias,
desejos, afetos. 

\page

É conhecida a admiração de {\bf SEBASTIAN} pelo escritor francês. Ele chega até a publicar, em 1938, o primeiro ensaio a respeito das correspondências de {\bf PROUST}.

\page

«Há na presença de Proust junto aos outros uma {\bf INTENSA SEDE DE INTIMIDADE}, que
se torna enfim irrealizável, mas que pelo menos é satisfeita por
pequenos acordos momentâneos, passageiros, anestésicos»

{\vfill\scale[factor=5]{{\bf A correspondência de Marcel Proust}, Mihail Sebastian.}}

\page

\MyCover{SEBASTIAN_MULHERES_THUMB}

\page %---------------------------------------------------------|

\Hedra

\stoptext %---------------------------------------------------------|


