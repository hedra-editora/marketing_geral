% AUTOR_LIVRO_CURIOSIDADES.tex
% Preencher com o nome das cor ou composição RGB (ex: [r=0.862, g=0.118, b=0.118]) 
\usecolors[crayola] 			   % Paleta de cores pré-definida: wiki.contextgarden.net/Color#Pre-defined_colors

% Cores definidas pelo designer:
% MyGreen		r=0.251, g=0.678, b=0.290 % 40ad4a
% MyCyan		r=0.188, g=0.749, b=0.741 % 30bfbd
% MyRed			r=0.820, g=0.141, b=0.161 % d12429
% MyPink		r=0.980, g=0.780, b=0.761 % fac7c2
% MyGray		r=0.812, g=0.788, b=0.780 % cfc9c7
% MyOrange		r=0.980, g=0.671, b=0.290 % faab4a

% Configuração de cores

\definecolor[MyColor][MiddleGreenYellow]      % ou ex: [r=0.862, g=0.118, b=0.118] % corresponde a RGB(220, 30, 30)
\definecolor[MyColorText][black]     % ou ex: [r=0.862, g=0.118, b=0.118] % corresponde a RGB(167, 169, 172)

% Classe para diagramação dos posts
\environment{marketing.env}		   

\starttext %---------------------------------------------------------|

\hyphenpenalty=10000
\exhyphenpenalty=10000

\Mensagem{LITERATURA ROMENA EM FOCO} %Sempre usar esse header

\startMyCampaign

\hyphenpenalty=10000
\exhyphenpenalty=10000

 INTERCÂMBIO CULTURAL\\
{\bf ROMÊNIA E FRANÇA}

\stopMyCampaign

\page %---------------------------------------------------------| 

\hyphenpenalty=10000
\exhyphenpenalty=10000

Durante os séculos {\cap XIX} e {\cap XX}, a cultura francesa exerceu uma grande influência sobre a elite intelectual romena. Paris era vista como um {\bf CENTRO CULTURAL} e muitos romenos foram estudar na França, trazendo de volta ideias e estilos literários.

\page

A instabilidade política após a Segunda Guerra Mundial, assim como conservadorismo que encontravam em solo romeno, contribuiu para para que muitos escritores e intelectuais romenos se exilassem na França --- visto como um {\bf ESPAÇO DE LIBERDADE E EXPERIMENTAÇÃO VANGUARDISTA}.

\page

{\bf EMIL CIORAN}, {\bf EUGÈNE IONESCO}, e {\bf MIRCEA ELIADE}, viveram na França e escreveram boa parte de suas obras em francês, mantendo-se fortemente arraigados à cultura francófona. 

\page

{\bf MIHAIL SEBASTIAN}, por sua vez, viveu na França entre 1929 e 1931, e muito provavelmente circulou no ambiente artístico e boêmio, que 
é percorrido pelo personagem principal de seu romance {\bf MULHERES}.

\page

\starttikzpicture[remember picture,overlay]
    \node at (4.45,-3.5) {\externalfigure[sebastian3][width=9cm]};
\stoptikzpicture


\starttikzpicture[remember picture,overlay]
\node at (4.3,-5.6) {\tfxx Cartão de estudante de Mihail Sebastian, Universidade de Paris};
\stoptikzpicture

\page

\MyCover{SEBASTIAN_MULHERES_THUMB}

\page %---------------------------------------------------------|

\Hedra

\stoptext %---------------------------------------------------------|



FRANÇA E ROMENIA

país muito francofono 
muitos romenos viveram e conviveram entre si na frança (Emil Ciorian, Ionesco, Eliade) -- romeno para ter sucesso tem de sair da romênia; todos que tiveram sucesso conquistaram a fama fora, normalmente na frança (facilidade linguistica, por ser uma língua similar)

frança como uma especie de modelo para a romenia, que se considera uma especie de irmã menor.
artistas romenos, em um contexto de um pais muito tradicional e conversador, enxergam a frança como um espaço de liberdade,de experimentação vanguardista.



Muitos escritores e intelectuais romenos do século {\cap XX} tiveram uma ligação estreita com a França, seja através de migração, estudos ou influência cultural.


ntelectual afinado com as vanguardas europeias, com a música clássica
e, em particular, com a literatura francesa, despontou precocemente --- junto com outros autores como Emil Cioran,
Mircea Eliade e o franco"-romeno Eugène Ionesco.


mantém fortemente arraigado em uma cultura francófona.
endo morado na França entre 1929 e 1931, é provável
que Sebastian tenha circulado nesse ambiente artístico e boêmio que
será percorrido pelo personagem principal de \textit{Mulheres}, Ştefan Valeriu. É lá, por exemplo,


    Emil Cioran:
        Contexto: Cioran foi um filósofo e ensaísta romeno que escreveu principalmente em francês após se mudar para Paris em 1937.
        Relação com a França: Ele escolheu escrever em francês e se estabeleceu em Paris, onde passou a maior parte de sua vida. A obra de Cioran é marcada pelo pessimismo e pela crítica existencial, e ele é frequentemente associado ao existencialismo francês.

    Eugène Ionesco:
        Contexto: Ionesco, nascido na Romênia, é um dos mais proeminentes dramaturgos do Teatro do Absurdo.
        Relação com a França: Ele se mudou para a França em 1938 e se naturalizou francês. Suas peças, como "A Cantora Careca" e "Rinoceronte", foram escritas em francês e tiveram grande impacto no teatro francês e internacional. Ionesco é considerado uma figura central no Teatro do Absurdo francês.

    Mircea Eliade:
        Contexto: Eliade foi um historiador das religiões, escritor e filósofo romeno, conhecido por seus estudos sobre mitologia e religião.
        Relação com a França: Embora tenha vivido em vários países, incluindo a Romênia e os Estados Unidos, Eliade também passou um tempo significativo na França, onde manteve contatos intelectuais importantes. Ele estudou na Sorbonne e suas obras foram amplamente traduzidas e discutidas no meio acadêmico francês.

    Mihail Sebastian:
        Contexto: Sebastian foi um romancista, dramaturgo e jornalista romeno de origem judaica, conhecido por seus diários e sua obra "O Acidente".
        Relação com a França: Embora Sebastian tenha passado a maior parte de sua vida na Romênia e não tenha vivido na França como os outros mencionados, ele foi fortemente influenciado pela literatura e pelo pensamento francês. Suas obras refletem essa influência, e ele admirava profundamente a cultura francesa, como evidenciado em seus escritos e traduções.

Contexto Cultural e Histórico

    