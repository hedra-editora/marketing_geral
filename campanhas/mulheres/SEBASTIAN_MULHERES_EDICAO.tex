% AUTOR_LIVRO_EDICAO.tex
% Preencher com o nome das cor ou composição RGB (ex: [r=0.862, g=0.118, b=0.118]) 
\usecolors[crayola] 			   % Paleta de cores pré-definida: wiki.contextgarden.net/Color#Pre-defined_colors

% Cores definidas pelo designer:
% MyGreen		r=0.251, g=0.678, b=0.290 % 40ad4a
% MyCyan		r=0.188, g=0.749, b=0.741 % 30bfbd
% MyRed			r=0.820, g=0.141, b=0.161 % d12429
% MyPink		r=0.980, g=0.780, b=0.761 % fac7c2
% MyGray		r=0.812, g=0.788, b=0.780 % cfc9c7
% MyOrange		r=0.980, g=0.671, b=0.290 % faab4a

% Configuração de cores
\definecolor[MyColor][MiddleGreenYellow]      % ou ex: [r=0.862, g=0.118, b=0.118] % corresponde a RGB(220, 30, 30)
\definecolor[MyColorText][black]     % ou ex: [r=0.862, g=0.118, b=0.118] % corresponde a RGB(167, 169, 172)

% Classe para diagramação dos posts
\environment{marketing.env}		   

\starttext %---------------------------------------------------------|

\Mensagem{LITERATURA ROMENA EM FOCO}

\startMyCampaign

\hyphenpenalty=10000
\exhyphenpenalty=10000


{\bf OS AMORES DE ŞTEFAN}

\stopMyCampaign

%\vfill\scale[lines=1.5]{\MyStar[MyColorText][none]}

\page %---------------------------------------------------------| 

«Cada capítulo do romance leva o nome de uma ou mais mulheres que passaram pela vida do protagonista. Ele retoma seus amores do passado, descreve e analisa meticulosamente a natureza de cada uma de suas amantes, além de seus próprios sentimentos. Assim, por trás a galeria de retratos femininos que se constitui ao longo do romance, é o próprio homem, Stefan Valeriu, que se revela como objeto de análise do leitor.»

{\vfill\scale[factor=5]{{\bf Nome de quem escreveu a análise}, qualificação de}\setupinterlinespace[line=1.5ex]\scale[factor=5]{XPTO professora na Universidade de Nova York. Lembre}\setupinterlinespace[line=1.5ex]\scale[factor=5]{de quebrar as linhas nos códigos.}}

\page %---------------------------------------------------------|

\Hedra

\stoptext %---------------------------------------------------------|