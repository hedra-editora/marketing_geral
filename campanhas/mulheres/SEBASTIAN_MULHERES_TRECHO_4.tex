% AUTOR_LIVRO_TRECHO.tex
% Preencher com o nome das cor ou composição RGB (ex: [r=0.862, g=0.118, b=0.118]) 
\usecolors[crayola] 			   % Paleta de cores pré-definida: wiki.contextgarden.net/Color#Pre-defined_colors

% Cores definidas pelo designer:
% MyGreen		r=0.251, g=0.678, b=0.290 % 40ad4a
% MyCyan		r=0.188, g=0.749, b=0.741 % 30bfbd
% MyRed			r=0.820, g=0.141, b=0.161 % d12429
% MyPink		r=0.980, g=0.780, b=0.761 % fac7c2
% MyGray		r=0.812, g=0.788, b=0.780 % cfc9c7
% MyOrange		r=0.980, g=0.671, b=0.290 % faab4a

% Configuração de cores
\definecolor[MyColor][MiddleGreenYellow]      % ou ex: [r=0.862, g=0.118, b=0.118] % corresponde a RGB(220, 30, 30)
\definecolor[MyColorText][black]     % ou ex: [r=0.862, g=0.118, b=0.118] % corresponde a RGB(167, 169, 172)

% Classe para diagramação dos posts
\environment{marketing.env}		   

\starttext %---------------------------------------------------------|

\Mensagem{DESTAQUE}

\startMyCampaign

\hyphenpenalty=10000
\exhyphenpenalty=10000
«Paira um silêncio imenso... Não. Não paira um silêncio imenso. Isso parece citação de livro. O que paira é uma imensa bagunça,  
\vfill\hfill →


\page
uma imensa balbúrdia zoológica, grilos que se arrastam, gafanhotos que se agitam, escaravelhos que se chocam no ar, 

\vfill\hfill →

\page
ferindo sonoramente os élitros e depois caindo com um barulho denso, de chumbo.
\vfill\hfill →



\page
 Em meio a tudo isso, sua respiração, a de Ştefan Valeriu, é um mero detalhe, sinal irrisório de vida.»

\stopMyCampaign


{\vfill\scale[factor=6]{\Seta\,Trecho de {\bf Mulheres}, de Mihail Sebastian.}

\page %---------------------------------------------------------| 

\MyCover{SEBASTIAN_MULHERES_THUMB}

\page %---------------------------------------------------------|

\Hedra

\stoptext %---------------------------------------------------------|