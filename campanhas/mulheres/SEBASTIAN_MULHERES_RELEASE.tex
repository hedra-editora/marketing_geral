\setuppapersize[A4]
\usecolors[crayola]
\setupbackgrounds[paper][background=color,backgroundcolor=Almond]
	
	\definefontfeature
		[default]
		[default]
		[expansion=quality,protrusion=quality,onum=yes]
	\setupalign[fullhz,hanging]
	\definefontfamily [mainface] [sf] [Formular]
	\setupbodyfont[mainface,11pt]

% Indenting [4.4 cont-enp.p.65]
			\setupindenting[yes, 3ex]  % none small medium big next first dimension
			\indenting[next]           % never not no yes always first next
			
			% [cont-ent.p.76]
			\setupspacing[broad]  %broad packed
			% O tamanho do espaço entre o ponto final e o começo de uma sentença. 


\startsetups[grid][mypenalties]
    \setdefaultpenalties
    \setpenalties\widowpenalties{2}{10000}
    \setpenalties\clubpenalties {2}{10000}
\stopsetups

\setuppagenumbering
  [location={}]            % Estilo dos números de páginat

\setuphead[subject]
[style=bfb]		

\setuplayout[
          location=middle,
          %
          leftedge=0mm,
          leftedgedistance=0mm,
          leftmargin=20mm,
          leftmargindistance=0mm,
          width=100mm,
          rightmargindistance=0mm,
          rightmargin=20mm,
          rightedgedistance=0mm,
          rightedge=0mm,
          backspace=20mm,
          %
          top=21mm,
          topdistance=0mm,
          header=0mm,
          headerdistance=0mm,
          height=250mm,
          footerdistance=0mm,
          footer=0mm,
          bottomdistance=0mm,
          bottom=21mm,
          topspace=21mm,
        setups=mypenalties,
]

\setupalign[right]

\starttext
{\bfb Uma crônica sentimental peculiar}

\blank[big]

\noindent{\it Ao contrariar o ideal romântico e burguês do amor, Sebastian explora a falta, o vazio, as contradições, tensões e desencontros que caracterizam o sentimento amoroso e, mais amplamente, as relações humanas.}

\blank[1cm]

\inoutermargin[width=60mm,hoffset=1cm,style=tfx,,voffset=3.5cm]{
\externalfigure[SEBASTIAN_MULHERES_THUMB][width=50mm]
}

\inoutermargin[width=70mm,hoffset=1cm,voffset=4.5cm,style=tfx]
{\noindent{\bf Título} Mulheres\\
{\bf Autor} Mihail Sebastian\\
{\bf Tradução} Fernando Klabin\\
{\bf Editora} Hedra\\
{\bf ISBN} 978-85-7715-934-5\\
{\bf Páginas} 152\\
%{\bf Pré-venda} XXXX\\
{\bf Preço} 49,00
}

\noindent{}{\em Renée, Marthe, Odette, Emilie, Maria, Arabela}. São estes os nomes das mulheres que simbolizam retratos de um lugar e de uma época --- a Europa entre guerras --- que o romancista e dramaturgo Mihail Sebastian, pseudônimo de Iosif Mendel Hechter, conheceu bem. Publicado pela primeira vez no Brasil, {\em Mulheres} (1933) é o livro mais sensível de um dos maiores autores da literatura romena moderna. O romance, formado por quatro histórias cheias de nuances e sutilezas, foi a segunda obra do autor, precedida apenas pela novela {\em Fragmentos de um diário encontrado} (1932), publicada pela Ayllon em 2020 e até então inédita em português.

{\em Mulheres} é uma crônica sentimental peculiar. Ou, melhor, um álbum de fotografias da vida de Ştefan Valeriu, apresentado em ordem cronológica. Por mais que as quatro partes do livro sejam vagamente conectadas, detalhes e memórias flutuam entre elas. Cada parte focaliza em um diferente estágio de sua vida e nas mulheres que fazem parte dele. Habilmente, Sebastian muda a perspectiva e a voz nos movimentos de sua composição. Mas esta não é uma coleção de conquistas ou um {\em cherchez la femme} bem-desenvolvido: as mulheres citadas por Ştefan o {\em fazem e definem} já que, em última análise, ele talvez não tenha contornos próprios definidos (ou revelados).

Os capítulos, que levam sempre nomes de uma ou mais mulheres, não necessariamente focam em relacionamentos amorosos. Em «Emilie», moça que dá nome a um dos capítulos, Ştefan tem um caso com Mado, {\em affair} que é apenas plano de fundo da história focada na {\em figura-título}, por quem ele não desenvolve interesse romântico. Já Maria, personagem que fala em primeira pessoa através de uma carta escrita a Ştefan, trata sobre seu próprio relacionamento complexo com Andrei (que também é amigo de Ştefan). %cortaria esse parágrafo

Outro dado importante são os recortes espaciais dentro do texto. Sebastian por vezes cria, como se diz no cinema, um {\em espaço off}. Ou seja, são retratados {\em apenas alguns} episódios de uma vida. E isso fica explícito, por exemplo, na provocação cruel de um interlúdio emocionante perto da conclusão de «Emilie»: uma história paralela permanece não contada, exceto por um breve resumo. Este, aliás, é um dos artifícios que fazem desta obra uma composição de pinceladas vívidas. {\em Mulheres} é construído a partir de algumas revelações, mas sempre deixa muito por dizer. E por isso é uma obra de ficção impressionante. 

Sebastian é particularmente bom em encontrar um equilíbrio entre sentimentalismo, romance e paixão. Mas sua narrativa traduz um certo desencantamento dos personagens. Sobretudo de Ştefan, hermético ao {\em transbordamento amoroso}, que carrega uma dor íntima difícil de definir. 
 Essa dor é sugerida por alusões, que se sobrepõem à «preguiça» --- que pode ser entendida como melancolia, ou talvez excesso de sensibilidade --- constantemente assumida tanto para si quanto para os outros. 
 %-não entendi bem essa frase, cortaria a partir de essa dor -julia

Ao contrariar o ideal romântico e burguês do amor como uma experiência que {\em transborda e transcende}, Sebastian explora a falta, o vazio, as contradições, tensões e desencontros que caracterizam o sentimento amoroso e, mais amplamente, as relações humanas. Trata-se de uma incompreensão comum, universal, na medida em que se situa fora de um tempo, que pertence a {\em todo o tempo}, o que dá justamente a dimensão atemporal da escrita do autor romeno. Ao perscrutar a intimidade de seus personagens, seus jogos de sedução e segredos de alcova, Sebastian propõe um exercício similar a seu leitor, que se vê confrontado com seus desejos mais incômodos e inconfessáveis.

\subject{Sobre o autor}

A vida de Mihail Sebastian não foi propriamente marcada por mulheres, mas pelos acontecimentos da primeira metade do século {\cap xx}: a onda de antissemitismo que varreu a Europa. Sebastian foi um dos mais importantes romancistas e dramaturgos romenos do século passado. Intelectual alinhado às vanguardas europeias, à música clássica e, em particular, à literatura francesa, despontou precocemente --- junto a outros autores como Emil Cioran, Mircea Eliade e o franco-romeno Eugène Ionesco. Durante seu auge, entre 1920 a 1930, a Romênia passou por um período de grande agitação política e cultural. Ao mesmo tempo que dialogava com outras potências europeias no campo da literatura, sobretudo a França, a juventude intelectual romena também apoiava a ascensão do fascismo --- representado, à época, pela Guarda de Ferro (em romeno, {\em Garda de Fier}), organização que levou ao poder o ditador Ion Antonescu em 1940 e fomentou o {\em pogrom} de Bucareste no ano seguinte.

A despeito de sua notoriedade no meio artístico local, o jovem escritor judeu se viu pouco a pouco excluído do cânone literário, perseguido e ostracizado por um antissemitismo latente. Esse movimento foi detalhadamente relatado em seu \emph{Diário}, escrito entre 1935 e 1944 mas publicado na Romênia somente em 1996, após o fim da censura imposta pelo regime comunista entre 1945 e 1989. Tendo sobrevivido a «tristes anos de humilhação e fracasso» sob o regime romeno pró-Hitler e à Segunda Guerra Mundial, conforme descreve em seu diário, Sebastian morre tragicamente em 1945, aos 38 anos, atropelado por um caminhão militar soviético. Professor recém-contratado à época, ele se dirigia à Universidade de Bucareste, onde daria sua primeira aula sobre Balzac.

Segundo Marian Ochoa de Eribe, tradutora da edição espanhola de {\em Mulheres}, «Sebastian é um exemplo da tragédia dos judeus da Europa Central. O autor foi afastado do seu trabalho na Fundação Real, teve que estrear peças sob um pseudônimo e sobreviveu o melhor que pôde à humilhante pobreza cotidiana e à ameaça permanente de ser enviado para um campo de trabalhos forçados {[}\ldots{]} No entanto, soube manter uma mente lúcida e capacidade de distinção moral que colocam o seu vibrante testemunho de vida ao nível de Primo Levi ou Victor Klemperer».

Embora integrem sua obra globalmente, os acontecimentos históricos que permearam a trajetória do autor não são mencionados em {\em Mulheres}, seu romance de estreia publicado em 1932 pela editora Nationale Ciornei, de Bucareste. Tampouco é possível obter um retrato fiel da realidade pela leitura do livro --- da geografia ou da sociologia de uma Romênia profunda, tradicional e culturalmente marcada. Sebastian tem o cuidado de contornar espaços, temáticas ou categorias que remetam a uma origem ou uma identidade, o que não nos impede de identificar sutilezas narrativas sobre a época e a sociedade no olhar do escritor, que se mantém fortemente arraigado em uma cultura francófona.

\subject{Sobre o tradutor}

Fernando Klabin nasceu em São Paulo e formou-se em Ciência Política pela Universidade de Bucareste, onde foi agraciado com a Ordem do Mérito Cultural da Romênia no grau de Oficial, em 2016. Além de tradutor, exerce atividades ocasionais como fotógrafo, escritor, ator e artista plástico.

\subject{Sobre a introdutora}

Mirella Botaro é graduada em Letras pela Universidade de São Paulo, mestra em Literatura Comparada pela École Normale Supérieure e doutora em Estudos Brasileiros, Portugueses e Africanos de Língua Portuguesa pela Universidade de Sorbonne. É vencedora do prêmio {\cap gis} Études Africaines en France. Atualmente é professora de Estudos Brasileiros na Universidade Sorbonne Nouvelle e traduz obras literárias do francês para o português.

\subject{Trechos do livro}

\startitemize
\item Creio que o amor fosse para ela uma dificuldade mais física do que
moral. Se eu não temesse uma expressão equívoca, diria que o amor se
tornava, no seu caso, uma questão de equilíbrio. Aquilo que deve ter lhe
parecido impossível no amor deve ter sido a mudança do eixo de rotação.
Ser um animal vertical e de repente passar para uma posição horizontal
--- eis o que deve ter torturado seus sonhos sensuais, se é que os teve.
Creio que o mistério do amor, para ela, se concentrava por completo
nessa queda, que a vida, em sua inteireza, se alicerçava nesse fato, o
que ultrapassava as suas forças.

Pediria perdão ao leitor por esses detalhes desavergonhados, mas, para
ser sincero, pouco me importa o leitor, e muito me importa Emilie
Vignon. Conto a vida dela em primeiro lugar porque eu quero chegar a
compreender alguma coisa da alma dessa moça, a quem no passado eu talvez
não tenha dado a devida atenção.

\item Paira um silêncio imenso\ldots{} Não. Não paira um silêncio imenso. Isso
parece citação de livro. O que paira é uma imensa bagunça, uma imensa
balbúrdia zoológica, grilos que se arrastam, gafanhotos que se agitam,
escaravelhos que se chocam no ar, ferindo sonoramente os élitros e
depois caindo com um barulho denso, de chumbo. Em meio a tudo isso, sua
respiração, a de Ştefan Valeriu, é um mero detalhe, sinal irrisório de
vida, irrisório e capital como o do esquilo que pula, ou o do gafanhoto
que se detém na ponta da sua bota, achando ser uma pedra. É benfazejo
saber-se aqui, um animal, uma criatura viva, um quadrúpede sem
importância, que dorme e respira numa área de dois metros quadrados,
debaixo de um sol que pertence a todos.

\item Deixei-a falar por muito tempo. Hoje não me lembro mais de tudo o que me
contou. Banalidades, acontecimentos triviais, reflexões, perguntas,
recordações --- tudo narrado de maneira indiferente, com o mesmo tom de
voz, comedida e desprovida de brilho nos olhos, o que demonstrava quão
pouca importância tinham todas aquelas coisas, e eu ouvia tudo aquilo,
por ela, e ela falava, provavelmente por cansaço.

Saímos tarde dali, eram quase duas da manhã. As estações de metrô já
haviam fechado fazia muito tempo e não havia nenhum táxi à vista.

Propus que fôssemos para minha casa.

--- Impossível.

--- Não, é possível, sim. Só para dormir, não para outra coisa. É mais
simples, e fica mais perto.

Refletiu um pouco; via-se com clareza que não era uma questão de pudor,
mas de comodidade. No final das contas, deve ter concluído que de fato
seria mais simples.

--- Tá bom.

\item  No verão, viajamos para Talloires, onde nos hospedamos numa pousada muito barata, mas de ambiente refinado (um deleite para o gosto burguês de Arabela), e lá desempenhamos, despretensiosamente, o papel de «jovens recém-casados felizes», na companhia de gente decente e fofoqueira. Arabela cintilava de orgulho em meio às amigas da pousada, todas elas esposas criteriosas, e como lhe caía bem dessa vez a aura de «mulher casada», ela que durante tantos anos perambulara por um universo duvidoso e agitado. Sentia-me realmente contente por ter concedido àquela mulher a única volúpia para a qual provavelmente fora predestinada: a ilusão do amor legítimo. 
%E me alegrava ao ver como Arabela aos poucos perdia a sombra de dúvida --- ou talvez de pânico --- que algumas vezes cobriu, no passado, o seu sorriso.

\item --- Vou embora amanhã e me pergunto se não estaria indo tarde demais. Um instante tarde demais.

--- Isso significa?

--- Isso significa que sua passagem pelo terraço de manhã, de camisa branca, de pescoço descoberto, com seu nome estrangeiro, que ninguém na pousada consegue pronunciar direito, com essa sua juventude decidida e confusa, com sua vida desconhecida, com os jornais estrangeiros que você recebe de lugares remotos, com as cartas que lhe chegam em envelopes com selos estranhos, com suas crispações rabugentas, com suas alegrias explosivas, com sua paixão pela leitura de livros e por rolar na grama, é uma imagem agradável.

Ştefan pega na sua mão para beijá-la, mas a encontra tão tranquila, tão admirada com seu aperto emocionado, tão segura de si, que, sem poder mais soltá-la, com medo de que o gesto seja demasiado brutal e, também, sem poder mantê-la presa na sua, ele sugere que parta.
\stopitemize

\stoptext