
TINDER
Cada capítulo do romance leva o nome de uma ou mais mulheres que passaram
pela vida do protagonista. Ele retoma seus amores do passado,
descreve e analisa meticulosamente a natureza de cada uma de suas
amantes, além de seus próprios sentimentos.
Assim, por trás da galeria de retratos femininos que se constitui ao
longo do romance, é o próprio homem, Ştefan Valeriu, que se revela como
objeto de análise ao lei


HOMEM QUE FALA SOBRE MULHERES

O ponto de vista é sempre o de Ştefan. E isso permite ao leitor o acesso direto ao modo de
apreender e interpretar a realidade que o cerca, além de seus desejos
e contradições íntimas. É a sua voz que ouvimos ecoar quando
o narrador afirma, sobre uma de suas amantes:


CONTRA O IDEAL DE AMOR BURGUES E MACHADO

Ao contrariar o ideal romântico e burguês do amor como uma experiência
que \textit{transborda e transcende}, Sebastian explora a falta, o vazio, as
contradições, tensões e desencontros que caracterizam igualmente o
sentimento amoroso e, mais amplamente, as relações humanas. Que sentido
poderíamos dar à ruptura de Ştefan e Arabela, que de tão repentina é
vivida como um ato absolutamente trivial, após anos de vida comum?
Trata-se de uma incompreensão comum, universal de certa forma, a medida
que se situa fora de um tempo, pertencendo, portanto, a todo o tempo, o
que dá justamente a dimensão atemporal da escrita do autor romeno. Ao
perscrutar a intimidade de seus personagens, seus jogos de sedução e
segredos de alcova, Sebastian propõe um exercício similar a seu
leitor, que se vê confrontado com seus desejos mais incômodos e
inconfessáveis.

{Comentário mordaz que nos dá a medida da distância entre
Ştefan e a moral burguesa da época e nos remete, em certa medida, a uma
certa \textit{ironia machadiana} familiar aos destinatários brasileiros do
romance.}

O diálogo franco e direto que Mihail Sebastian estabelece com um
hipotético leitor também o aproxima de Machado de Assis, escritor que em
toda a sua produção tampouco deixa de interpelar seu leitor, incluindo"-o
no jogo narrativo como se fosse um personagem. Assim o faz
Ştefan, o herói de \emph{Mulheres}, ao descrever e comentar, não sem
sarcasmo, o encontro insólito entre Irimia, um compatriota radicado em
Paris, e Emilie Vignou, mulher que ``mesmo feia, {[}\ldots{]} tinha por
vezes um ar de resignação que {[}o{]} atraía'':\footnote{Ver página \pageref{resignacao}.}}

\begin{quote}
Pediria perdão ao leitor por estes detalhes desavergonhados, mas, para
ser sincero, pouco me importa o leitor, e muito me importa Emilie
Vignon. Conto a vida dela em primeiro lugar porque eu quero chegar a
compreender alguma coisa da alma dessa moça, a quem no passado eu talvez
não tenha dado a devida atenção.\footnote{Ver página \pageref{atenção}.}
\end{quote}



PROUST

\textls[10]{\emph{Mulheres} é um romance carregado de sensibilidade. A atenção particular que Sebastian
dispensa às minúcias da vida social revelam, de fato, uma vida interior
agitada, mas que se esconde por trás das aparências e em
ângulos mortos do cotidiano. Assim, atos triviais como comer ou
cumprimentar alguém parecem atrelados a uma verdade profunda do sujeito,
que o romancista é capaz de captar e realçar, ao menos momentaneamente.
Como se, ao se ater às camadas externas de nossa socialização, outras
camadas de interioridade se revelassem, transpassando o \textit{véu do visível},
do prosaico, da vida cotidiana.}

\textls[5]{Não seria exagero afirmar que a intimidade dissecada
o aproxima de uma certa \textit{estética proustiana} em que predomina,
precisamente, a busca incessante pelo íntimo do sujeito: suas fantasias,
desejos, afetos. É conhecida e notória a admiração de
Sebastian por Marcel Proust, que publica o último tomo de \emph{Em
busca do tempo perdido} em 1927, quando o escritor romeno tinha apenas
20 anos. Seu profundo conhecimento da obra do autor francês o leva a mergulhar
na intimidade do próprio Proust, uma vez que Sebastian publica, em
1938, o primeiro ensaio a respeito de suas correspondências. ``Há na
presença de Proust junto aos outros uma intensa sede de intimidade, que
se torna enfim irrealizável, mas que pelo menos é satisfeita por
pequenos acordos momentâneos, passageiros, anestésicos'', afirma o
romancista romeno no referido ensaio.}


