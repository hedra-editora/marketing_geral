
TINDER
Cada capítulo do romance leva o nome de uma ou mais mulheres que passaram
pela vida do protagonista. Ele retoma seus amores do passado,
descreve e analisa meticulosamente a natureza de cada uma de suas
amantes, além de seus próprios sentimentos.
Assim, por trás da galeria de retratos femininos que se constitui ao
longo do romance, é o próprio homem, Ştefan Valeriu, que se revela como
objeto de análise ao lei


HOMEM QUE FALA SOBRE MULHERES

O ponto de vista é sempre o de Ştefan. E isso permite ao leitor o acesso direto ao modo de
apreender e interpretar a realidade que o cerca, além de seus desejos
e contradições íntimas. É a sua voz que ouvimos ecoar quando
o narrador afirma, sobre uma de suas amantes:

há algo de machista, afinal é um livro escrito na década de 30, num país cristão ortodoxo, hiper tradicional.
no capítulo de arabella isso fica mais explícito.

-- capitulo de maria como uma rendenção? dá espaço para voz feminina



CONTRA O IDEAL DE AMOR BURGUES


Ao contrariar o ideal romântico e burguês do amor como uma experiência
que \textit{transborda e transcende}, Sebastian explora a falta, o vazio, as
contradições, tensões e desencontros que caracterizam igualmente o
sentimento amoroso e, mais amplamente, as relações humanas. Que sentido
poderíamos dar à ruptura de Ştefan e Arabela, que de tão repentina é
vivida como um ato absolutamente trivial, após anos de vida comum?
Trata-se de uma incompreensão comum, universal de certa forma, a medida
que se situa fora de um tempo, pertencendo, portanto, a todo o tempo, o
que dá justamente a dimensão atemporal da escrita do autor romeno. Ao
perscrutar a intimidade de seus personagens, seus jogos de sedução e
segredos de alcova, Sebastian propõe um exercício similar a seu
leitor, que se vê confrontado com seus desejos mais incômodos e
inconfessáveis.


BLECHER E SEBASTIAN

menção ao blecher no diário do sebastian