% SEBASTIAN_DIARIO_EFEMERIDE.tex
% Preencher com o nome das cor ou composição RGB (ex: [r=0.862, g=0.118, b=0.118]) 
\usecolors[crayola] 			   % Paleta de cores pré-definida: wiki.contextgarden.net/Color#Pre-defined_colors

% Cores definidas pelo designer:
% MyGreen		r=0.251, g=0.678, b=0.290 % 40ad4a
% MyCyan		r=0.188, g=0.749, b=0.741 % 30bfbd
% MyRed			r=0.820, g=0.141, b=0.161 % d12429
% MyPink		r=0.980, g=0.780, b=0.761 % fac7c2
% MyGray		r=0.812, g=0.788, b=0.780 % cfc9c7
% MyOrange		r=0.980, g=0.671, b=0.290 % faab4a

% Configuração de cores
\definecolor[MyColor][SunnyPearl]      % ou ex: [r=0.862, g=0.118, b=0.118] % corresponde a RGB(220, 30, 30)
\definecolor[MyColorText][black]     % ou ex: [r=0.862, g=0.118, b=0.118] % corresponde a RGB(167, 169, 172)

% Classe para diagramação dos posts
\environment{marketing.env}		   

\starttext %---------------------------------------------------------|

\hyphenpenalty=10000
\exhyphenpenalty=10000

\Mensagem{LITERATURA ROMENA EM FOCO} %Sempre usar esse header

\MyPicture{SEBASTIAN_DIARIO_1}

\vfill\scale[factor=6]{\Seta\,{\bf MIHAIL SEBASTIAN} (1907--1945)}

\page %---------------------------------------------------------| 

\hyphenpenalty=10000
\exhyphenpenalty=10000

{\bf MIHAIL SEBASTIAN}, pseudônimo de Iosif Mendel Hechter, foi um dos mais importantes escritores romenos --- além de dramaturgo, jornalista e ensaísta. Fez parte do mesmo círculo literário de Emil Cioran, Eugène Ionesco e Mircea Eliade e, assim como eles, Sebastian incorporou o caráter rebelde das {\bf VANGUARDAS EUROPEIAS DE 1920 E 1930}. 

\page

Mas não alcançou o mesmo reconhecimento de seus \\contemporâneos. Com a ascensão do {\bf NAZISMO}, o escritor judeu, que antes era tido como um jovem promissor do cenário literário romeno, passou a ser excluído e execrado.

\page

 A sua vida foi {\bf INTERROMPIDA SUBITAMENTE} em 29/05/1945 --- pouco antes da Segunda Guerra Mundial chegar ao fim --- quando, a caminho de ministrar uma palestra sobre Balzac, o escritor foi atropelado por um caminhão do exército.

\page

\MyCover{SEBASTIAN_DIARIO_THUMB}

\page %---------------------------------------------------------|

\Hedra

\stoptext %---------------------------------------------------------|


