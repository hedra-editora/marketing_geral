% AUTOR_LIVRO_AUTOR.tex
% Preencher com o nome das cor ou composição RGB (ex: [r=0.862, g=0.118, b=0.118]) 
\usecolors[crayola] 			   % Paleta de cores pré-definida: wiki.contextgarden.net/Color#Pre-defined_colors

% Cores definidas pelo designer:
% MyGreen		r=0.251, g=0.678, b=0.290 % 40ad4a
% MyCyan		r=0.188, g=0.749, b=0.741 % 30bfbd
% MyRed			r=0.820, g=0.141, b=0.161 % d12429
% MyPink		r=0.980, g=0.780, b=0.761 % fac7c2
% MyGray		r=0.812, g=0.788, b=0.780 % cfc9c7
% MyOrange		r=0.980, g=0.671, b=0.290 % faab4a

% Configuração de cores
\definecolor[MyColor][MiddleGreenYellow]      % ou ex: [r=0.862, g=0.118, b=0.118] % corresponde a RGB(220, 30, 30)
\definecolor[MyColorText][black]     % ou ex: [r=0.862, g=0.118, b=0.118] % corresponde a RGB(167, 169, 172)

% Classe para diagramação dos posts
\environment{marketing.env}		   

% Cabeço e rodapé: Informações (caso queira trocar alguma coisa)
 		\def\MensagemSaibaMais  {SAIBA MAIS:}
 		\def\MensagemSite		{HEDRA.COM.BR}
 		\def\MensagemLink       {LINK NA BIO}

\starttext %--------------------------------------------------------|

\Mensagem{LITERATURA ROMENA EM FOCO}

\hyphenpenalty=10000
\exhyphenpenalty=10000

%\startMyCampaign

\MyPicture{sebastian4}

%\stopMyCampaign

\vfill\scale[factor=6]{\Seta\,MIHAIL SEBASTIAN (1907--1945)}

\page %----------------------------------------------------------|

\hyphenpenalty=10000
\exhyphenpenalty=10000

{\bf MIHAIL SEBASTIAN}, nascido Iosif Hechter, foi um dos principais representantes da literatura romena do século XX. Sua vida e obra foram profundamente marcadas pelos eventos da primeira metade do século, especialmente pelo {\bf ANTISSEMITISMO} que permeou a Europa.
\page %----------------------------------------------------------|

Nascido em uma família judia de Craiova, sua formação intelectual foi influenciada por {\bf MOVIMENTOS VANGUARDISTAS EUROPEUS} e pela literatura francesa. Durante seu tempo em Bucareste, destacou-se como romancista e dramaturgo, estabelecendo diálogos significativos com a cultura europeia.

\page

Sebastian também enfrentou a realidade sombria do regime pró-Hitler da Romênia. Apesar de seu talento reconhecido, foi ostracizado e forçado a publicar sob pseudônimos. Seu diário, escrito entre 1935 e 1944 e publicado apenas em 1996, revela a luta contra a humilhação e a pobreza em meio ao crescente antissemitismo.

\page

Obras como {\bf MULHERES}, seu romance de estreia, refletem não apenas suas experiências pessoais, mas também um contexto social complexo. A {\bf MORTE TRÁGICA} de Sebastian, atropelado por um caminhão militar soviético em 1945, selou a história de um autor que, apesar das adversidades, deixou um legado literário profundo e impactante.


\page

\MyCover{SEBASTIAN_MULHERES_THUMB}

\page %----------------------------------------------------------|

\Hedra

\stoptext %---------------------------------------------------------|