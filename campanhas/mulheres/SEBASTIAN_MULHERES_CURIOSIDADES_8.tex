% AUTOR_LIVRO_CURIOSIDADES.tex
% Preencher com o nome das cor ou composição RGB (ex: [r=0.862, g=0.118, b=0.118]) 
\usecolors[crayola] 			   % Paleta de cores pré-definida: wiki.contextgarden.net/Color#Pre-defined_colors

% Cores definidas pelo designer:
% MyGreen		r=0.251, g=0.678, b=0.290 % 40ad4a
% MyCyan		r=0.188, g=0.749, b=0.741 % 30bfbd
% MyRed			r=0.820, g=0.141, b=0.161 % d12429
% MyPink		r=0.980, g=0.780, b=0.761 % fac7c2
% MyGray		r=0.812, g=0.788, b=0.780 % cfc9c7
% MyOrange		r=0.980, g=0.671, b=0.290 % faab4a

% Configuração de cores
\definecolor[MyColor][MiddleGreenYellow]      % ou ex: [r=0.862, g=0.118, b=0.118] % corresponde a RGB(220, 30, 30)
\definecolor[MyColorText][black]     % ou ex: [r=0.862, g=0.118, b=0.118] % corresponde a RGB(167, 169, 172)

% Classe para diagramação dos posts
\environment{marketing.env}		   

\starttext %---------------------------------------------------------|

\hyphenpenalty=10000
\exhyphenpenalty=10000

\Mensagem{EM CONTEXTO} %Sempre usar esse header

\startMyCampaign

\hyphenpenalty=10000
\exhyphenpenalty=10000

A TRÁGICA MORTE DE {\bf MIHAIL SEBASTIAN}

\stopMyCampaign

\page %---------------------------------------------------------| 

\hyphenpenalty=10000
\exhyphenpenalty=10000

A morte de Mihail Sebastian, em 1945, aos 38 anos, foi uma tragédia que teve implicações significativas para a cultura romena. O autor foi {\bf ATROPELADO POR UM CAMINHÃO MILITAR SOVIÉTICO} em Bucareste, enquanto se dirigia à Universidade para ministrar sua primeira aula sobre Balzac.

\page %---------------------------------------------------------|

No contexto do crescente \\antissemitismo e ascensão do fascismo na Romênia, Sebastian já havia enfrentado anos de {\bf PERSEGUIÇÃO} e ostracismo. Seu diário, escrito entre 1935 e 1944, registra suas experiências de humilhação sob o regime pró-Hitler e os horrores da Segunda Guerra Mundial, refletindo a angustiante realidade da vida judaica na época.

\page

A abrupta interrupção de sua carreira literária não apenas privou o mundo de uma voz crítica e inovadora, mas também simbolizou a perda irreparável de intelectuais e artistas em um período marcado pela repressão e pela violência. A trajetória de Sebastian continua a ser um {\bf TESTEMUNHO DA RESILIÊNCIA CULTURAL} em face de adversidades históricas.

\page

\MyCover{SEBASTIAN_MULHERES_THUMB}

\page %---------------------------------------------------------|

\Hedra

\stoptext %---------------------------------------------------------|




