% AUTOR_LIVRO_CURIOSIDADES.tex
% Preencher com o nome das cor ou composição RGB (ex: [r=0.862, g=0.118, b=0.118]) 
\usecolors[crayola] 			   % Paleta de cores pré-definida: wiki.contextgarden.net/Color#Pre-defined_colors

% Cores definidas pelo designer:
% MyGreen		r=0.251, g=0.678, b=0.290 % 40ad4a
% MyCyan		r=0.188, g=0.749, b=0.741 % 30bfbd
% MyRed			r=0.820, g=0.141, b=0.161 % d12429
% MyPink		r=0.980, g=0.780, b=0.761 % fac7c2
% MyGray		r=0.812, g=0.788, b=0.780 % cfc9c7
% MyOrange		r=0.980, g=0.671, b=0.290 % faab4a

% Configuração de cores

\definecolor[MyColor][MiddleGreenYellow]      % ou ex: [r=0.862, g=0.118, b=0.118] % corresponde a RGB(220, 30, 30)
\definecolor[MyColorText][black]     % ou ex: [r=0.862, g=0.118, b=0.118] % corresponde a RGB(167, 169, 172)

% Classe para diagramação dos posts
\environment{marketing.env}		   

\starttext %---------------------------------------------------------|

\hyphenpenalty=10000
\exhyphenpenalty=10000

\Mensagem{LITERATURA ROMENA EM FOCO} %Sempre usar esse header

\startMyCampaign

\hyphenpenalty=10000
\exhyphenpenalty=10000

PARALELOS LITERÁRIOS:  
{\bf MACHADO DE ASSIS
E MIHAIL
SEBASTIAN}

%Aqui a manchete pode ser mais longa

\stopMyCampaign

\page %---------------------------------------------------------| 

\hyphenpenalty=10000
\exhyphenpenalty=10000

No romance {\bf MULHERES}, o protagonista tece uma série de comentários mordazes que nos dão a medida da distância entre Stefan e a moral burguesa da época e nos remete, em certa medida, à {\bf IRONIA MACHADIANA}.

\page

Além do registro irônico, o diálogo franco e direto que Mihail Sebastian estabelece com um
hipotético leitor também o aproxima de {\bf MACHADO DE ASSIS}, escritor que em
toda a sua produção interpela seu leitor, incluindo-o
no {\bf JOGO NARRATIVO} como se fosse um personagem. 

\page


«Pediria perdão ao leitor por estes detalhes desavergonhados, mas, para
ser sincero, {\bf POUCO ME IMPORTA O LEITOR}, e muito me importa Emilie
Vignon.»


\page %---------------------------------------------------------|

\MyCover{SEBASTIAN_MULHERES_THUMB}

\page %---------------------------------------------------------|

\Hedra

\stoptext %---------------------------------------------------------|