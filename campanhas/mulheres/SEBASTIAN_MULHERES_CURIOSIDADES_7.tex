% AUTOR_LIVRO_CURIOSIDADES.tex
% Preencher com o nome das cor ou composição RGB (ex: [r=0.862, g=0.118, b=0.118]) 
\usecolors[crayola] 			   % Paleta de cores pré-definida: wiki.contextgarden.net/Color#Pre-defined_colors

% Cores definidas pelo designer:
% MyGreen		r=0.251, g=0.678, b=0.290 % 40ad4a
% MyCyan		r=0.188, g=0.749, b=0.741 % 30bfbd
% MyRed			r=0.820, g=0.141, b=0.161 % d12429
% MyPink		r=0.980, g=0.780, b=0.761 % fac7c2
% MyGray		r=0.812, g=0.788, b=0.780 % cfc9c7
% MyOrange		r=0.980, g=0.671, b=0.290 % faab4a

% Configuração de cores
\definecolor[MyColor][MiddleGreenYellow]      % ou ex: [r=0.862, g=0.118, b=0.118] % corresponde a RGB(220, 30, 30)
\definecolor[MyColorText][black]     % ou ex: [r=0.862, g=0.118, b=0.118] % corresponde a RGB(167, 169, 172)

% Classe para diagramação dos posts
\environment{marketing.env}		   

\starttext %---------------------------------------------------------|

\hyphenpenalty=10000
\exhyphenpenalty=10000

\Mensagem{EM CONTEXTO} %Sempre usar esse header

\startMyCampaign

\hyphenpenalty=10000
\exhyphenpenalty=10000

{\bf MIHAIL \\SEBASTIAN} E 
O ANTISSEMITISMO

\stopMyCampaign

\page %---------------------------------------------------------| 

\hyphenpenalty=10000
\exhyphenpenalty=10000

Na década de 1930, a Romênia experimentou um aumento significativo do antissemitismo, exacerbado por {\bf IDEOLOGIAS NACIONALISTAS} e pela ascensão de movimentos fascistas, como a Guarda de Ferro. 


\page

Nesse contexto, {\bf MIHAIL SEBASTIAN}, um proeminente romancista e dramaturgo judeu, tornou-se um reflexo das tensões sociais e políticas da época.

\page %---------------------------------------------------------|


O antissemitismo atravessou a vida do autor, incluindo discriminação no âmbito profissional e cultural. Sebastian, apesar de sua notoriedade literária, enfrentou a {\bf EXCLUSÃO DO CÂNONE LITERÁRIO ROMENO} e foi compelido a publicar suas obras sob pseudônimos. 

\page

Essa marginalização comprometeu sua carreira e evidenciou a precariedade da vida judaica na Romênia no contexto da ascensão do fascismo, marcada pela violência constante.

\page

Seu diário, redigido entre 1935 e 1944, serve como um importante {\bf TESTEMUNHO DAS EXPERIÊNCIAS VIVIDAS SOB UM REGIME ANTIJUDAICO}. Através de sua prosa, Sebastian documenta as angústias de um intelectual que, embora profundamente enraizado na cultura romena, se vê cada vez mais alienado em sua própria terra.

\page

\MyCover{SEBASTIAN_MULHERES_THUMB}

\page %---------------------------------------------------------|

\Hedra

\stoptext %---------------------------------------------------------|