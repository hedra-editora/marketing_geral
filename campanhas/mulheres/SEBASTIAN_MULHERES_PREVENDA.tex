% Preencher com o nome das cor ou composição RGB (ex: [r=0.862, g=0.118, b=0.118]) 
\usecolors[crayola] 			   % Paleta de cores pré-definida: wiki.contextgarden.net/Color#Pre-defined_colors

% Cores definidas pelo designer:
% MyGreen		r=0.251, g=0.678, b=0.290 % 40ad4a
% MyCyan		r=0.188, g=0.749, b=0.741 % 30bfbd
% MyRed			r=0.820, g=0.141, b=0.161 % d12429
% MyPink		r=0.980, g=0.780, b=0.761 % fac7c2
% MyGray		r=0.812, g=0.788, b=0.780 % cfc9c7
% MyOrange		r=0.980, g=0.671, b=0.290 % faab4a

% Configuração de cores
\definecolor[MyColor][YellowGreen]      % ou ex: [r=0.862, g=0.118, b=0.118] % corresponde a RGB(220, 30, 30)
\definecolor[MyColorText][Black]  % ou ex: [r=0.862, g=0.118, b=0.118] % corresponde a RGB(167, 169, 172)

% Classe para diagramação dos posts
\environment{marketing.env}		   

% Comandos & Instruções %%%%%%%%%%%%%%%%%%%%%%%%%%%%%%%%%%%%%%%%%%%%%%%%%%%%%%%%%%%%%%%|

% Cabeço e rodapé: Informações (caso queira trocar alguma coisa)
 		\def\MensagemSaibaMais 	{SAIBA MAIS:}
 		\def\MensagemSite		{HEDRA.COM.BR}
 		\def\MensagemLink		{LINK NA BIO}

\starttext %---------------------------------------------------------|

\Mensagem{PRÉ-VENDA}

\MyCover{./SEBASTIAN_MULHERES_THUMB.pdf}

\vfill\scale[factor=6]{\Seta\,{\bf 30\% DE DESCONTO}} % Data entra somente após aprovação da Mayara

\page %---------------------------------------------------------|

\hyphenpenalty=10000
\exhyphenpenalty=10000

Na contramão do ideal romântico e burguês do amor, «Mulheres» explora temas como a falta, o vazio, as contradições, as tensões e os {\bf DESENCONTROS QUE CARACTERIZAM O SENTIMENTO AMOROSO} e, mais amplamente, as relações humanas. 

\vfill\hfill
\scale[factor=5]{→}

\page %---------------------------------------------------------|

\hyphenpenalty=10000
\exhyphenpenalty=10000

E é uma crônica sentimental peculiar: ou, melhor, um álbum de fotografias da vida de Ştefan Valeriu. Esta {\bf NÃO É UMA COLEÇÃO DE CONQUISTAS} ou um «cherchez la femme» bem desenvolvido. É mais complexo do que isso. As mulheres citadas por Ştefan o «fazem e definem» já que, em última análise, ele não tem contornos próprios definidos ou revelados. 

\vfill
\scale[factor=5]{\Seta\,Tradução de Fernando Klabin, {\bf direta do romeno}.}

\page %---------------------------------------------------------|

\hyphenpenalty=10000
\exhyphenpenalty=10000

«Mulheres» é o romance de estreia de Mihail Sebastian (1907--1945), um dos maiores intelectuais romenos do século {\cap xx}, afinado com o {\bf CARÁTER REBELDE DAS VANGUARDAS ARTÍSTICAS EUROPEIAS} de 1920 e 1930. A despeito da notoriedade que construiu no meio artístico, não teve o reconhecimento de seus contemporâneos por ser judeu, e passou a ser excluído desse círculo.

\vfill
\scale[factor=5]{\Seta\,De R\$\,45,00 {\bf por R\$\,31,50}, 152 páginas, 1ª edição.}

\page %---------------------------------------------------------|

\Hedra

\stoptext %---------------------------------------------------------|