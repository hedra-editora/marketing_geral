\setuppapersize[A4]
\usecolors[crayola]
\setupbackgrounds[paper][background=color,backgroundcolor=Almond]
    \mainlanguage[pt]  

    \definefontfeature
        [default]
        [default]
        [expansion=quality,protrusion=quality,onum=yes]
    \setupalign[fullhz,hanging]
    \definefontfamily [mainface] [sf] [Formular]
    \setupbodyfont[mainface,11pt]

% Indenting [4.4 cont-enp.p.65]
            \setupindenting[yes, 3ex]  % none small medium big next first dimension
            \indenting[next]           % never not no yes always first next
            
            % [cont-ent.p.76]
            \setupspacing[broad]  %broad packed
            % O tamanho do espaço entre o ponto final e o começo de uma sentença. 


\startsetups[grid][mypenalties]
    \setdefaultpenalties
    \setpenalties\widowpenalties{2}{10000}
    \setpenalties\clubpenalties {2}{10000}
\stopsetups

\setuppagenumbering
  [location={}]            % Estilo dos números de páginat

\setuphead[subject]
[style=bfb]     

\setuplayout[
          location=middle,
          %
          leftedge=0mm,
          leftedgedistance=0mm,
          leftmargin=20mm,
          leftmargindistance=0mm,
          width=100mm,
          rightmargindistance=0mm,
          rightmargin=20mm,
          rightedgedistance=0mm,
          rightedge=0mm,
          backspace=20mm,
          %
          top=21mm,
          topdistance=0mm,
          header=0mm,
          headerdistance=0mm,
          height=250mm,
          footerdistance=0mm,
          footer=0mm,
          bottomdistance=0mm,
          bottom=21mm,
          topspace=21mm,
        setups=mypenalties,
]

\setupalign[right]

\starttext
{\bfd A folha divina\\}
{\tfb Timóteo Verá Tupã Popyguá}

\blank[big]

\hyphenpenalty=10000
\exhyphenpenalty=10000

\noindent {\it Transmitindo uma sabedoria ancestral,} A folha divina, {\it conta a história da} ka’a miri’i,  {“erva-mate”, e sua importância dentro a visão de mundo dos Guarani Mbya.}


% A toca iluminada {\it apresenta as experiências de Blecher em sanatórios durante a década de 1930, onde a vivência na espécie de um mundo-bolha pode se sobrepor à realidade, por vezes estranha e cheio de rotinas. E é nesse lugar que sua vida interior passa a assumir um papel cada vez mais ampliado}

\blank[1cm]

\inoutermargin[width=60mm,hoffset=1cm,style=tfx,,voffset=2cm]{
\externalfigure[TIMOTEO_FOLHA_THUMB][width=50mm]
}


\inoutermargin[width=70mm,hoffset=1.1cm,voffset=2.7cm,style=tfx]
{\noindent{\bf Título} {\em A folha divina}\\
{\bf Autor} Timóteo Verá Tupã Popyguá\\
{\bf Organização e apresentação} Anita Ekman\\
{\bf Posfácio} Freg J. Stokes\\
{\bf Ilustrações} Nhamandu Mirim Nilmar da Silva Vilharve\\
{\bf Editora} Hedra\\
{\bf ISBN} 978-85-7715-964-2\\
{\bf Páginas} 80\\
{\bf Preço} 59,90
}

\inoutermargin[width=70mm,hoffset=1.1cm,voffset=8.5cm,style=tfx]
{{\bf Sobre o autor} {\it Timóteo Verá Tupã Popyguá} é uma figura proeminente na luta pelos direitos do povo Guarani Mbya. Cacique da Tekoa da reserva indígena Takuari, tem se dedicado a preservar e disseminar a rica sabedoria ancestral de sua comunidade. Timóteo é também filósofo e autor do livro {\it A terra uma só} (Hedra, 2022), no qual explora a relação íntima entre os seres humanos e a natureza.}

\inoutermargin[width=70mm,hoffset=1.1cm,voffset=14cm,style=tfx]
{{\bf Sobre a organizadora} {\it Anita Ekman} artista visual, performer, curadora e pesquisadora de arte pré-colonial e história da floresta tropical, além de co-curadora de exposições voltadas às artes e história das mulheres indígenas com Sandra Benites. Tem elaborado projetos para o Peabody Museum de Harvard e Instituto Goethe, este último em colaboração com líderes indígenas como Carlos Papa e Cristine Takua, além de Freg J. Stokes.}

% \inoutermargin[width=70mm,hoffset=1.1cm,voffset=16.6cm,style=tfx]
% {{\bf Sobre o apresentador} Leonardo Francisco Soares é professor do Instituto de Letras e Linguística da Universidade Federal de Uberlândia ({\cap ILEEL/UFU}) e do programa de pós-graduação em Estudos Literários do {\cap ILEEL/UFU}. Publicou, dentre outros, um texto na coletânea {\it Guerra e literatura: ensaios em emergência} (Alameda, 2022)}


\inoutermargin[width=70mm,hoffset=-10cm,voffset=18.5cm,style=tfx]
{\definefontfamily [Times] [rm] [Times New Roman]
                   [tf=file:TimesLTStd-Roman.otf]

\setcharacterkerning[reset] \switchtobodyfont[Times,50pt] hedra \hfill \mbox{}
}


\inoutermargin[width=70mm,hoffset=-5.5cm,voffset=18.7cm,style=tfx]
{\definefontfamily [Times] [rm] [Times New Roman]
                   [tf=file:TimesLTStd-Roman.otf]

\setcharacterkerning[reset] \switchtobodyfont[Times,50pt] comercial@edlab.press \hfill \mbox{}
}


\inoutermargin[width=70mm,hoffset=-5.5cm,voffset=19.2cm,style=tfx]
{\definefontfamily [Times] [rm] [Times New Roman]
                   [tf=file:TimesLTStd-Roman.otf]

\setcharacterkerning[reset] \switchtobodyfont[Times,50pt] www.hedra.com.br \hfill \mbox{}
}

\startalignment[middle]
{\tfc \it Tradição e resistência: a recuperação da origem indígena da erva-mate}
\stopalignment


\blank[.5cm]

\noindent {\it A folha divina}, de Timóteo Verá Tupã Popyguá, é uma obra que mergulha na rica cosmogonia indígena, explorando a conexão espiritual e cultural que esse povo mantém com as folhas divinas, em especial a {\it ka’a miri’i}. Conhecida como “erva-mate”, essa planta desempenha um papel central nas mitologias, tradições e identidades guarani.

Os mitos ancestrais, práticas e sabedorias registradas pelo autor articulam uma reflexão profunda sobre as folhas divinas e seu uso tradicional, que se insere em um contexto de fortalecimento espiritual e rituais comunitários.

A narrativa também denuncia os impactos da exploração predatória do meio ambiente, destacando a importância não apenas das folhas divinas, mas de toda Mata Atlântica na preservação da alegria, do equilíbrio espiritual e da harmonia com a natureza.

Diante da apropriação europeia da erva-mate, ocorrida no contexto da colonização espanhola, o livro se dedica a reconstituir a origem indígena da planta e a profundidade da filosofia guarani.
A partir dessa leitura, passamos a enxergar a erva-mate enquanto resultado de um profundo conflito entre culturas, no qual a
versão dos originários detentores da tradição relativa ao uso dessas
folhas divinas, e das florestas e terras onde elas milenarmente foram
cultivadas, segue sendo silenciada. 

\stoptext

