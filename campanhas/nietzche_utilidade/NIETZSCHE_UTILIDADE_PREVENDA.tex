% Preencher com o nome das cor ou composição RGB (ex: [r=0.862, g=0.118, b=0.118]) 
\usecolors[crayola] 			   % Paleta de cores pré-definida: wiki.contextgarden.net/Color#Pre-defined_colors

% Cores definidas pelo designer:
% MyGreen		r=0.251, g=0.678, b=0.290 % 40ad4a
% MyCyan		r=0.188, g=0.749, b=0.741 % 30bfbd
% MyRed			r=0.820, g=0.141, b=0.161 % d12429
% MyPink		r=0.980, g=0.780, b=0.761 % fac7c2
% MyGray		r=0.812, g=0.788, b=0.780 % cfc9c7
% MyOrange		r=0.980, g=0.671, b=0.290 % faab4a

% Configuração de cores
\definecolor[MyColor][x=3f9cd0]      % ou ex: [r=0.862, g=0.118, b=0.118] % corresponde a RGB(220, 30, 30)
\definecolor[MyColorText][Black]  % ou ex: [r=0.862, g=0.118, b=0.118] % corresponde a RGB(167, 169, 172)

% Classe para diagramação dos posts
\environment{marketing.env}		   

% Comandos & Instruções %%%%%%%%%%%%%%%%%%%%%%%%%%%%%%%%%%%%%%%%%%%%%%%%%%%%%%%%%%%%%%%|

% Cabeço e rodapé: Informações (caso queira trocar alguma coisa)
 		\def\MensagemSaibaMais 	{SAIBA MAIS:}
 		\def\MensagemSite		{HEDRA.COM.BR}
 		\def\MensagemLink		{LINK NA BIO}

\starttext %---------------------------------------------------------|

\Mensagem{PRÉ-VENDA}

\MyCover{NIETZSCHE_UTILIDADE_THUMB.pdf}

\vfill\scale[factor=6]{\Seta\,{\bf 30\% DE DESCONTO}} % Data entra somente após aprovação da Mayara

\page %---------------------------------------------------------|

\hyphenpenalty=10000
\exhyphenpenalty=10000

Obra fundamental para compreender a filosofia da história e a filosofia da cultura em Nietzsche, {\bf SOBRE A UTILIDADE E A DESVANTAGEM DA HISTÓRIA PARA A VIDA}, publicada em 1874, discute os princípios, limites e objetivos do saber histórico.

\page %---------------------------------------------------------|

\hyphenpenalty=10000
\exhyphenpenalty=10000

Introdução e tradução de {\bf ANDRÉ LUIS MOTA ITAPARICA}, doutor em filosofia pela Universidade de São
Paulo (USP) e professor da Universidade Federal do Recôncavo da Bahia (UFRB).

\vfill
\scale[factor=5]{\Seta\,De R\$\,XXXXXX {\bf por R\$\,XXXXXX}, 122 páginas, 2ª edição.}

\page %---------------------------------------------------------|

\Hedra

\stoptext %---------------------------------------------------------|