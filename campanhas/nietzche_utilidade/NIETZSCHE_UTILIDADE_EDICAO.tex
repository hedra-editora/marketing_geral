% Preencher com o nome das cor ou composição RGB (ex: [r=0.862, g=0.118, b=0.118]) 
\usecolors[crayola] 			   % Paleta de cores pré-definida: wiki.contextgarden.net/Color#Pre-defined_colors

% Cores definidas pelo designer:
% MyGreen		r=0.251, g=0.678, b=0.290 % 40ad4a
% MyCyan		r=0.188, g=0.749, b=0.741 % 30bfbd
% MyRed			r=0.820, g=0.141, b=0.161 % d12429
% MyPink		r=0.980, g=0.780, b=0.761 % fac7c2
% MyGray		r=0.812, g=0.788, b=0.780 % cfc9c7
% MyOrange		r=0.980, g=0.671, b=0.290 % faab4a

% Configuração de cores
\definecolor[MyColor][x=3f9cd0]      % ou ex: [r=0.862, g=0.118, b=0.118] % corresponde a RGB(220, 30, 30)
\definecolor[MyColorText][black]     % ou ex: [r=0.862, g=0.118, b=0.118] % corresponde a RGB(167, 169, 172)

% Classe para diagramação dos posts
\environment{marketing.env}		   

\starttext %---------------------------------------------------------|

\Mensagem{TRÊS CONCEITOS}

\startMyCampaign

\hyphenpenalty=10000
\exhyphenpenalty=10000

{\bf 
O SENTIDO HISTÓRICO, A-HISTÓRICO E SUPRA-
-HISTÓRICO EM NIETZSCHE}

\stopMyCampaign

%\vfill\scale[lines=1.5]{\MyStar[MyColorText][none]}

\page %---------------------------------------------------------| 

\MyCover{./NIETZSCHE_UTILIDADE_THUMB.pdf}

\page %---------------------------------------------------------| 

\hyphenpenalty=10000
\exhyphenpenalty=10000

\startnarrower[5*right]
Apesar de o homem ser
essencialmente um ser histórico, pois {\bf O PASSADO E A MEMÓRIA} fazem parte de sua experiência no mundo, para Nietzsche é preciso também integrar uma visão a-histórica à experiência humana.
\stopnarrower

\starttikzpicture[remember picture,overlay]
\node at (7.33,4)
{\externalfigure
              [./nietzsche_munch1.jpg]
              [width=.4\textwidth]};
\stoptikzpicture

\page %---------------------------------------------------------| 

\hyphenpenalty=10000
\exhyphenpenalty=10000

\startnarrower[5*left]
Escreve o filósofo: «O animal vive de forma {\bf A-HISTÓRICA}: pois ele se absorve no presente; não sabe dissimular; não sabe ser outra coisa senão sincero».
\stopnarrower

\starttikzpicture[remember picture,overlay]
\node at (1.58,2)
{\externalfigure
              [./nietzsche_munch2.jpg]
              [width=.4\textwidth]};
\stoptikzpicture

\vfill
\startnarrower[5*left]
\setupinterlinespace[line=1.4ex]
{\itxx\Seta\,Nietzsche pintado por Edvard Munch, 1906}
\stopnarrower

\page %---------------------------------------------------------|

\hyphenpenalty=10000
\exhyphenpenalty=10000

Além do histórico e do a-histórico, há ainda uma terceira forma de se relacionar com o passado, o {\bf PONTO DE VISTA SUPRA-HISTÓRICO}: uma visão daquilo que é eterno. Ele é alcançado quando a pesquisa histórica, ao identificar o acaso, a insensatez e a injustiça que alimentam os processos históricos, conduz a uma perturbadora indiferença para com o passado e à superação do ponto de vista histórico.

\vfill
\scale[factor=5]{\Seta\,De R\$\,XXXXXX {\bf por R\$\,XXXXXX}, 122 páginas, 2ª edição.}

\page %---------------------------------------------------------|

\Hedra

\stoptext %---------------------------------------------------------|