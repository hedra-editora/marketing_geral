% Preencher com o nome das cor ou composição RGB (ex: [r=0.862, g=0.118, b=0.118]) 
\usecolors[crayola] 			   % Paleta de cores pré-definida: wiki.contextgarden.net/Color#Pre-defined_colors

% Cores definidas pelo designer:
% MyGreen		r=0.251, g=0.678, b=0.290 % 40ad4a
% MyCyan		r=0.188, g=0.749, b=0.741 % 30bfbd
% MyRed			r=0.820, g=0.141, b=0.161 % d12429
% MyPink		r=0.980, g=0.780, b=0.761 % fac7c2
% MyGray		r=0.812, g=0.788, b=0.780 % cfc9c7
% MyOrange		r=0.980, g=0.671, b=0.290 % faab4a

% Configuração de cores
\definecolor[MyColor][x=3f9cd0]      % ou ex: [r=0.862, g=0.118, b=0.118] % corresponde a RGB(220, 30, 30)
\definecolor[MyColorText][black]     % ou ex: [r=0.862, g=0.118, b=0.118] % corresponde a RGB(167, 169, 172)

% Classe para diagramação dos posts
\environment{marketing.env}

\starttext %---------------------------------------------------------|

\Mensagem{FILOSOFANDO}

\startMyCampaign

\hyphenpenalty=10000
\exhyphenpenalty=10000

CONHEÇA AS {\bf CONSIDERAÇÕES EXTEMPORÂNEAS} DE NIETZSCHE
\stopMyCampaign

\page %---------------------------------------------------------| 

\MyCover{./NIETZSCHE_UTILIDADE_THUMB.pdf}

\page %---------------------------------------------------------| 

\hyphenpenalty=10000
\exhyphenpenalty=10000

Escritas entre 1873 e 1876, as {\bf CONSIDERAÇÕES\\ EXTEMPORÂNEAS} são um conjunto de quatro ensaios nos quais Nietzsche reflete sobre diferentes aspectos da cultura moderna.

\page %---------------------------------------------------------|

\placefigure{}{\externalfigure[manuscrito.jpeg][width=.6\textwidth]}

\vfill\scale[factor=fit]{\Seta\,Manuscrito para o primeiro capítulo da segunda extemporânea}

\page %---------------------------------------------------------| 

Como escreveu em {\bf SOBRE A UTILIDADE E A DESVANTAGEM DA HISTÓRIA PARA A VIDA}, segunda das extemporâneas: “Permito-me confessar, até pela minha profissão de filólogo clássico: não saberia que sentido teria a filologia clássica em nossos dias senão o de intervir extemporaneamente — isto é, contra a época, sobre a época e a favor de uma época futura”.

\page %---------------------------------------------------------|

\Hedra

\stoptext %---------------------------------------------------------|