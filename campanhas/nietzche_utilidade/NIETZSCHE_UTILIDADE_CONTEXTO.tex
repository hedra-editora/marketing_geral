% AUTOR_LIVRO_CURIOSIDADE.tex
% Vamos falar sobre isso "curiosidades"
% > "EM CONTEXTO"

% Preencher com o nome das cor ou composição RGB (ex: [r=0.862, g=0.118, b=0.118]) 
\usecolors[crayola] 			   % Paleta de cores pré-definida: wiki.contextgarden.net/Color#Pre-defined_colors

% Cores definidas pelo designer:
% MyGreen		r=0.251, g=0.678, b=0.290 % 40ad4a
% MyCyan		r=0.188, g=0.749, b=0.741 % 30bfbd
% MyRed			r=0.820, g=0.141, b=0.161 % d12429
% MyPink		r=0.980, g=0.780, b=0.761 % fac7c2
% MyGray		r=0.812, g=0.788, b=0.780 % cfc9c7
% MyOrange		r=0.980, g=0.671, b=0.290 % faab4a

% Configuração de cores
\definecolor[MyColor][x=3f9cd0]      % ou ex: [r=0.862, g=0.118, b=0.118] % corresponde a RGB(220, 30, 30)
\definecolor[MyColorText][black]     % ou ex: [r=0.862, g=0.118, b=0.118] % corresponde a RGB(167, 169, 172)

% Classe para diagramação dos posts
\environment{marketing.env}		   

\starttext %---------------------------------------------------------|

\hyphenpenalty=10000
\exhyphenpenalty=10000

\Mensagem{EM CONTEXTO} %Sempre usar esse header

\startMyCampaign

\hyphenpenalty=10000
\exhyphenpenalty=10000

{\bf NIETZSCHE} DISCUTE A DOENÇA DE SEU SÉCULO: O {\bf SENTIDO HISTÓRICO}\\ DA VIDA

\starttikzpicture[remember picture,overlay]
\node at (8.8,4)
{\externalfigure
              [./nietzsche.jpg]
              [width=.55\textwidth]};
\stoptikzpicture

\stopMyCampaign

\page %---------------------------------------------------------| 

\hyphenpenalty=10000
\exhyphenpenalty=10000

\startnarrower[3*left]
Em diálogo com o hegelianismo e o positivismo, Nietzsche realiza, no livro {\bf SOBRE A UTILIDADE E A DESVANTAGEM DA HISTÓRIA PARA A VIDA}, um estudo sobre o historicismo que identifica nessas escolas, antes de tudo, a exacerbação de uma faculdade propriamente humana: o sentido histórico.
\stopnarrower

\starttikzpicture[remember picture,overlay]
\node at (-.5,4.95)
{\externalfigure
              [./nietzsche.jpg]
              [width=.55\textwidth]};
\stoptikzpicture

\page %---------------------------------------------------------| 

\hyphenpenalty=10000
\exhyphenpenalty=10000

Dada a impossibilidade de nos desvencilharmos da história, pois o homem é um ser histórico, cabe saber até que ponto ela {\bf AUXILIA OU PREJUDICA A VIDA} --- vista aqui não como um conceito biológico, mas como a experiência da vida humana, que só pode ser pensada no interior de uma cultura.

\page %---------------------------------------------------------| 

\hyphenpenalty=10000
\exhyphenpenalty=10000

\startnarrower[4*left]
Sua crítica não se reduz à disciplina histórica, nem às correntes filológicas de então, mas às próprias concepções de ciência e de conhecimento que permeiam essa prática e, mais que isso, às consequências que essa prática pode ter em toda uma cultura.
\stopnarrower

\starttikzpicture[remember picture,overlay]
\node at (1,6)
{\externalfigure
              [./herodoto.png]
              [width=.45\textwidth]};
\stoptikzpicture

\startnarrower[6*right]
\setupinterlinespace[line=1.4ex]
\setupbodyfont[9pt]\tfxx
{\it O historiador grego Heródoto,\\\itxx considerado o “pai da história”}
\stopnarrower

\page %---------------------------------------------------------|

\MyCover{./NIETZSCHE_UTILIDADE_THUMB.pdf}

\page %---------------------------------------------------------|

\Hedra

\stoptext %---------------------------------------------------------|