% AUTOR_LIVRO_EDICAO.tex
% Preencher com o nome das cor ou composição RGB (ex: [r=0.862, g=0.118, b=0.118]) 
\usecolors[crayola] 			   % Paleta de cores pré-definida: wiki.contextgarden.net/Color#Pre-defined_colors

% Cores definidas pelo designer:
% MyGreen		r=0.251, g=0.678, b=0.290 % 40ad4a
% MyCyan		r=0.188, g=0.749, b=0.741 % 30bfbd
% MyRed			r=0.820, g=0.141, b=0.161 % d12429
% MyPink		r=0.980, g=0.780, b=0.761 % fac7c2
% MyGray		r=0.812, g=0.788, b=0.780 % cfc9c7
% MyOrange		r=0.980, g=0.671, b=0.290 % faab4a

% Configuração de cores
\definecolor[MyColor][x=3f9cd0]      % ou ex: [r=0.862, g=0.118, b=0.118] % corresponde a RGB(220, 30, 30)
\definecolor[MyColorText][black]     % ou ex: [r=0.862, g=0.118, b=0.118] % corresponde a RGB(167, 169, 172)

% Classe para diagramação dos posts
\environment{marketing.env}		   

\starttext %---------------------------------------------------------|

\Mensagem{POR DENTRO DA EDIÇÃO}

\startMyCampaign

\hyphenpenalty=10000
\exhyphenpenalty=10000

{\bf 
O «FLOP» DE NIETZSCHE}
QUE SE TORNOU UMA DAS SUAS 
MAIORES OBRAS

\stopMyCampaign

%\vfill\scale[lines=1.5]{\MyStar[MyColorText][none]}

\page %---------------------------------------------------------| 

\MyCover{NIETZSCHE_UTILIDADE_THUMB}

\page %---------------------------------------------------------| 

\hyphenpenalty=10000
\exhyphenpenalty=10000

Publicada em 1874, a segunda das quatro considerações extemporâneas, {\bf SOBRE A UTILIDADE E A 

DESVANTAGEM DA HISTÓRIA PARA A VIDA}, foi a que teve a menor repercussão quando de seu lançamento. 

\page


Contudo, posteriormente, esse escrito passaria a ter um reconhecimento por parte da fortuna crítica só comparável, entre as primeiras obras de Nietzsche, ao {\bf NASCIMENTO DA TRAGÉDIA}.

\page %---------------------------------------------------------|

\hyphenpenalty=10000
\exhyphenpenalty=10000

Nessa consideração extemporânea Nietzsche pensa sobre a história, não apenas como disciplina acadêmica, mas como {\bf FUNDAMENTO 

ANTROPOLÓGICO E FORMA DE VIDA}. A maior parte do livro trata das várias consequências do excesso de sentido histórico para uma cultura.

\page

«A segunda extemporânea traz à luz o que há de perigoso, corrosivo e
envenenador da vida na nossa forma de prática científica [\ldots] Nesse tratado o “sentido histórico”, do qual nosso século se orgulha, foi pela primeira vez reconhecido como doença, como {\bf SIGNO TÍPICO DA DECADÊNCIA}.»

{\vfill\scale[factor=5]{{\bf André Itaparica}, na introdução do livro «Sobre a}\setupinterlinespace[line=1.5ex]\scale[factor=5]{utilidade e a desvantagem da história para a vida»,}\setupinterlinespace[line=1.5ex]\scale[factor=5]{de Friedrich Nietzsche.}}

\page %---------------------------------------------------------|

\Hedra

\stoptext %---------------------------------------------------------|