% AUTOR_LIVRO_AUTOR.tex
% Preencher com o nome das cor ou composição RGB (ex: [r=0.862, g=0.118, b=0.118]) 
\usecolors[crayola] 			   % Paleta de cores pré-definida: wiki.contextgarden.net/Color#Pre-defined_colors

% Cores definidas pelo designer:
% MyGreen		r=0.251, g=0.678, b=0.290 % 40ad4a
% MyCyan		r=0.188, g=0.749, b=0.741 % 30bfbd
% MyRed			r=0.820, g=0.141, b=0.161 % d12429
% MyPink		r=0.980, g=0.780, b=0.761 % fac7c2
% MyGray		r=0.812, g=0.788, b=0.780 % cfc9c7
% MyOrange		r=0.980, g=0.671, b=0.290 % faab4a

% Configuração de cores
\definecolor[MyColor][Magenta]      % ou ex: [r=0.862, g=0.118, b=0.118] % corresponde a RGB(220, 30, 30)
\definecolor[MyColorText][black]     % ou ex: [r=0.862, g=0.118, b=0.118] % corresponde a RGB(167, 169, 172)

% Classe para diagramação dos posts
\environment{marketing.env}		   

% Cabeço e rodapé: Informações (caso queira trocar alguma coisa)
 		\def\MensagemSaibaMais  {SAIBA MAIS:}
 		\def\MensagemSite		{HEDRA.COM.BR}
 		\def\MensagemLink       {LINK NA BIO}

\starttext %--------------------------------------------------------|

\Mensagem{DA MAIOR IMPORTÂNCIA}

\hyphenpenalty=10000
\exhyphenpenalty=10000

%\startMyCampaign

\MyPicture{julia lopes3}

%\stopMyCampaign

\vfill\scale[factor=6]{\Seta\,JÚLIA LOPES DE ALMEIDA (1862--1934)}

\page %----------------------------------------------------------|

\hyphenpenalty=10000
\exhyphenpenalty=10000

{\bf JÚLIA LOPES DE ALMEIDA} foi uma das mais importantes literatas do século {\cap XIX} e início do {\cap XX}. Escritora prolífica, conquistou o público com romances, contos, novelas, peças teatrais, crônicas e até livros didáticos.

\page %----------------------------------------------------------|

Com uma escrita que transita entre a crítica social e a beleza literária, Júlia Lopes de Almeida retratou com profundidade o Rio de Janeiro da Belle Époque. Em suas obras, representa {\bf PROCESSO DE MODERNIZAÇÃO} do Brasil no final do século {\cap XIX}, período marcado pela abolição da escravidão em 1888 e passagem da monarquia para a república. 

\page

Embora central para a literatura brasileira, a obra de Júlia Lopes de Almeida sofreu um apagamento significativo ao longo do tempo, especialmente após sua morte. Sua produção, marcada por uma crítica social profunda e uma abordagem inovadora das questões de gênero, foi gradualmente {\bf MARGINALIZADA NO CÂNONE LITERÁRIO}. 

\page

No entanto, nos últimos anos, o reconhecimento da importância de Júlia Lopes de Almeida tem sido revitalizado, especialmente no contexto de estudos de gênero e na reavaliação crítica da literatura brasileira, o que tem permitido a reavaliação e revalorização de sua trajetória e contribuição para a construção do discurso literário nacional.

\page

\MyCover{THUMB_LIVRO.pdf}

\page %----------------------------------------------------------|

\Hedra

\stoptext %---------------------------------------------------------|
 
