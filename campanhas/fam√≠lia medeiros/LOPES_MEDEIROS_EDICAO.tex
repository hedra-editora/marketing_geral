% AUTOR_LIVRO_EDICAO.tex
% Preencher com o nome das cor ou composição RGB (ex: [r=0.862, g=0.118, b=0.118]) 
\usecolors[crayola] 			   % Paleta de cores pré-definida: wiki.contextgarden.net/Color#Pre-defined_colors

% Cores definidas pelo designer:
% MyGreen		r=0.251, g=0.678, b=0.290 % 40ad4a
% MyCyan		r=0.188, g=0.749, b=0.741 % 30bfbd
% MyRed			r=0.820, g=0.141, b=0.161 % d12429
% MyPink		r=0.980, g=0.780, b=0.761 % fac7c2
% MyGray		r=0.812, g=0.788, b=0.780 % cfc9c7
% MyOrange		r=0.980, g=0.671, b=0.290 % faab4a

% Configuração de cores
\definecolor[MyColor][Magenta]      % ou ex: [r=0.862, g=0.118, b=0.118] % corresponde a RGB(220, 30, 30)
\definecolor[MyColorText][black]     % ou ex: [r=0.862, g=0.118, b=0.118] % corresponde a RGB(167, 169, 172)

% Classe para diagramação dos posts
\environment{marketing.env}		   

\starttext %---------------------------------------------------------|

\Mensagem{POR DENTRO DA EDIÇÃO}

\startMyCampaign

\hyphenpenalty=10000
\exhyphenpenalty=10000

INAUGURANDO UMA COLEÇÃO
{\bf A FAMÍLIA MEDEIROS}

\stopMyCampaign

%\vfill\scale[lines=1.5]{\MyStar[MyColorText][none]}

\page %---------------------------------------------------------| 

\MyCover{JULIA_MEDEIROS_THUMB}

\page %---------------------------------------------------------| 

\hyphenpenalty=10000
\exhyphenpenalty=10000

Escrito entre os anos de 1886 e de 1888, e veiculado pela primeira vez em formato de folhetim em 1891 no jornal Gazeta de Notícias, {\bf A FAMÍLIA MEDEIROS} é o primeiro romance escrito e o segundo publicado por Júlia Lopes de Almeida.

\page %---------------------------------------------------------|


\hyphenpenalty=10000
\exhyphenpenalty=10000

Após obter uma {\bf RECEPÇÃO ENTUSIASMADA} por parte da crítica e do público leitor, a obra foi editada em formato de livro e publicada. Enquanto a autora ainda era viva, o romance recebeu mais duas edições: a primeira, em 1894, e a segunda, cujo texto foi revisto pela própria Júlia, foi publicada em 1919.

\page

\hyphenpenalty=10000
\exhyphenpenalty=10000

Visando celebrar os 160 anos de nascimento da romancista e iniciando um movimento inédito de {\bf REEDIÇÃO DAS OBRAS COMPLETAS DE JÚLIA LOPES DE ALMEIDA}, a Editora Hedra inaugura a sua coleção justamente com o lançamento de {\bf A FAMÍLIA MEDEIROS}.

\page

\Hedra

\stoptext %---------------------------------------------------------|

