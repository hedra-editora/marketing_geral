\setuppapersize[A4]
\usecolors[crayola]
\setupbackgrounds[paper][background=color,backgroundcolor=Almond]
\mainlanguage[pt]	

	\definefontfeature
		[default]
		[default]
		[expansion=quality,protrusion=quality,onum=yes]
	\setupalign[fullhz,hanging]
	\definefontfamily [mainface] [sf] [Formular]
	\setupbodyfont[mainface,11pt]

% Indenting [4.4 cont-enp.p.65]
			\setupindenting[yes, 3ex]  % none small medium big next first dimension
			\indenting[next]           % never not no yes always first next
			
			% [cont-ent.p.76]
			\setupspacing[broad]  %broad packed
			% O tamanho do espaço entre o ponto final e o começo de uma sentença. 


\startsetups[grid][mypenalties]
    \setdefaultpenalties
    \setpenalties\widowpenalties{2}{10000}
    \setpenalties\clubpenalties {2}{10000}
\stopsetups

\setuppagenumbering
  [location={}]            % Estilo dos números de páginat

\setuphead[subject]
[style=bfb]		

\setuplayout[
          location=middle,
          %
          leftedge=0mm,
          leftedgedistance=0mm,
          leftmargin=20mm,
          leftmargindistance=0mm,
          width=100mm,
          rightmargindistance=0mm,
          rightmargin=20mm,
          rightedgedistance=0mm,
          rightedge=0mm,
          backspace=20mm,
          %
          top=21mm,
          topdistance=0mm,
          header=0mm,
          headerdistance=0mm,
          height=250mm,
          footerdistance=0mm,
          footer=0mm,
          bottomdistance=0mm,
          bottom=21mm,
          topspace=21mm,
        setups=mypenalties,
]

\setupalign[right]

\starttext
{\bfc A família Medeiros\\} 
{\tfb Júlia Lopes de Almeida}

\blank[big]

\hyphenpenalty=10000
\exhyphenpenalty=10000


\noindent {\it Ambientado em Campinas no século {\cap XIX}, o romance expõe o conflito geracional entre o conservador Comendador Medeiros, seu filho, Otávio, e sobrinha, Eva, defensores da abolição.}

\blank[1cm]

\inoutermargin[width=60mm,hoffset=1cm,style=tfx,,voffset=1cm]{
\externalfigure[THUMB_TEMPORARIO.pdf][width=50mm]
}


\inoutermargin[width=70mm,hoffset=1.1cm,voffset=1.7cm,style=tfx]
{\noindent{\bf Título} {\it A família Medeiros}\\
{\bf Autor} Júlia Lopes de Almeida\\
{\bf Organização} Anna Faedrich e Rafael Balseiro Zin\\
% {\bf Aparatos} Alfredo Sousa, Norma Telles e Rafael Balseiro Zin\\
{\bf Editora} Hedra\\
{\bf ISBN} 978-85-7715-721-1\\
{\bf Pág.}  XX\\
{\bf Preço} XX 
}

\inoutermargin[width=70mm,hoffset=1.1cm,voffset=6.5cm,style=tfx]
{{\bf Sobre a autora}  {\it Júlia Lopes de Almeida}, nascida no Rio de Janeiro em 1862, destacou-se como um fenômeno literário, escrevendo romances, contos, peças teatrais e crônicas que capturaram a Belle Époque carioca. Participou ativamente do meio literário e foi uma das idealizadoras da Academia Brasileira de Letras, da qual foi excluída por ser mulher. Defensora da emancipação feminina, criticou a educação restrita às mulheres e incentivou a independência financeira, deixando um legado que foi injustamente esquecido ao longo do tempo.}

\inoutermargin[width=70mm,hoffset=1.1cm,voffset=14cm,style=tfx]
{{\bf Sobre os organizadores} 

{\it Anna Faedrich} é doutora em Letras ({\cap PUCRS}) e professora na Universidade Federal Fluminense ({\cap UFF}). É autora de {\it Teorias da autoficção} ({\cap E}d{\cap UERJ}, 2022) e {\it Escritoras silenciadas} (Macabéa/\,Fundação Biblioteca Nacional, 2022).
 

 {\it Rafael Balseiro Zin} é sociólogo e doutor em ciências sociais {\cap puc-sp}. Investiga a trajetória intelectual das escritoras abolicionistas no Brasil, especialmente Maria Firmina dos Reis e Júlia Lopes de Almeida.}

% \inoutermargin[width=70mm,hoffset=1.1cm,voffset=16.6cm,style=tfx]
% {{\bf Sobre o apresentador} Leonardo Francisco Soares é professor do Instituto de Letras e Linguística da Universidade Federal de Uberlândia ({\cap ILEEL/UFU}) e do programa de pós-graduação em Estudos Literários do {\cap ILEEL/UFU}. Publicou, dentre outros, um texto na coletânea {\it Guerra e literatura: ensaios em emergência} (Alameda, 2022)}


\inoutermargin[width=70mm,hoffset=-10cm,voffset=19.7cm,style=tfx]
{\definefontfamily [Times] [rm] [Times New Roman]
                   [tf=file:TimesLTStd-Roman.otf]

\setcharacterkerning[reset] \switchtobodyfont[Times,50pt] hedra \hfill \mbox{}
}


\noindent{\tfb {\it Brasil em transição: resistência e progresso em} A família Medeiros.}

\blank[.5cm]

\hyphenpenalty=10000
\exhyphenpenalty=10000

{\it A família Medeiros} (1892), segundo romance publicado por Júlia Lopes de Almeida, retrata os conflitos entre as gerações dessa família. Enquanto o Comendador, cafeicultor brutal, resiste à emancipação dos escravizados e à implementação do trabalho assalariado, Eva, sua sobrinha, e Otávio, seu filho, defendem abertamente os ideais abolicionistas e republicanos. 


Cada uma das duas gerações administra uma propriedade rural: a Fazenda Genoveva, conduzida pela mão forte do Comendador e seus  asseclas, insiste na brutalidade da exploração da mão de obra  escravizada, que, por sua vez, resiste articulando uma revolta,  um dos pontos altos do enredo. Trata-se do oposto do que ocorre  na fazenda Mangueiral, sob a responsabilidade de Eva, cujas atividades  são conduzidas com respeito à dignidade humana por meio da partilha dos lucros.

O registro desse ambiente social e político conturbado, no estado de São Paulo dos últimos anos do século {\cap XIX}, faz de {\it A
família Medeiros} uma obra fundamental para a compreensão do Brasil  contemporâneo. Além do retrato de um momento crucial da nossa história  --- os momentos finais da crise do Segundo Reinado, a abolição da escravidão e a Proclamação da República ---, o livro surpreende pela atualidade  de passagens em que o ambiente familiar, cindido pelo debate político,
se radicaliza, refletindo duas chagas abertas da sociedade brasileira que ainda estão por resolver, apesar dos avanços recentes: o racismo e a emancipação das mulheres.        

%FALTOU FALAR DA DIMENSÃO LINGUISTICA... E ACHO QUE ESSE TEXTO FICOU MUITO PRESO NO ENREDO DO ROMANCE.

\setuplayout[
          location=middle,
          %
          leftedge=0mm,
          leftedgedistance=0mm,
          leftmargin=20mm,
          leftmargindistance=0mm,
          width=160mm,
          rightmargindistance=0mm,
          rightmargin=20mm,
          rightedgedistance=0mm,
          rightedge=0mm,
          backspace=20mm,
          %
          top=21mm,
          topdistance=0mm,
          header=0mm,
          headerdistance=0mm,
          height=250mm,
          footerdistance=0mm,
          footer=0mm,
          bottomdistance=0mm,
          bottom=21mm,
          topspace=21mm,
        setups=mypenalties,
]

% \subject{Sobre a autora}

%  {\it Júlia Lopes de Almeida}, nascida no Rio de Janeiro em 1862, destacou-se como um fenômeno literário, escrevendo romances, contos, peças teatrais e crônicas que capturaram a Belle Époque carioca. Participou ativamente do meio literário e foi uma das idealizadoras da Academia Brasileira de Letras, da qual foi excluída por ser mulher. Defensora da emancipação feminina, criticou a educação restrita às mulheres e incentivou a independência financeira, deixando um legado que foi injustamente esquecido ao longo do tempo.


\stoptext

RELEASE DO ROGERIO

Com a {\it A família Medeiros}, a Editora Hedra inicia a 
publicação das Obras Completas de Júlia Lopes de Almeida, em 18 volumes. 

O romance veio a público, pela primeira vez, em 1891, 
em folhetim, na {\it Gazeta de Notícias}, e em livro no ano seguinte. 
A edição de referência para este volume é a última, de 1919, ano em 
que a obra foi reeditada e a autora teve a chance de revisá-la, 
conferindo-lhe forma definitiva, depois de quase trinta anos da 
primeira edição. 

Ambientado na região de Campinas, no estado de São Paulo, o livro retrata os
costumes e conflitos de duas gerações da família do Comendador
Medeiros, um cafeicultor brutal que resiste à iminente libertação dos
escravizados. Sua sobrinha Eva e seu filho Otávio, por sua vez, 
defendem abertamente os ideais abolicionistas e republicanos. 

Cada uma das duas gerações administra uma propriedade rural: a 
Fazenda Genoveva, conduzida pela mão forte do Comendador e seus 
asseclas, insiste na brutalidade da exploração da mão de obra 
escravizada, que, por sua vez, resiste articulando uma revolta, 
um dos pontos altos do enredo. Trata-se do oposto do que ocorre 
na fazenda Mangueiral, sob a responsabilidade de Eva, cujas atividades 
são conduzidas com respeito à dignidade humana por meio da partilha 
dos lucros. 


O registro desse ambiente social e político conturbado, no estado de São Paulo dos últimos anos do século {\cap XIX}, faz de {\it A
família Medeiros} uma obra fundamental para a compreensão do Brasil  contemporâneo. Além do retrato de um momento crucial da nossa história  --- os momentos finais da crise do Segundo Reinado, a abolição da escravidão e a Proclamação da República ---, o livro surpreende pela atualidade  de passagens em que o ambiente familiar, cindido pelo debate político,
se radicaliza, refletindo duas chagas abertas da sociedade brasileira que ainda estão por resolver, apesar dos avanços recentes: o racismo e a emancipação das mulheres.    