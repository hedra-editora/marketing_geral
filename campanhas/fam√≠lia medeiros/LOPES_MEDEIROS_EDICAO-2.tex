% AUTOR_LIVRO_EDICAO.tex
% Preencher com o nome das cor ou composição RGB (ex: [r=0.862, g=0.118, b=0.118]) 
\usecolors[crayola] 			   % Paleta de cores pré-definida: wiki.contextgarden.net/Color#Pre-defined_colors

% Cores definidas pelo designer:
% MyGreen		r=0.251, g=0.678, b=0.290 % 40ad4a
% MyCyan		r=0.188, g=0.749, b=0.741 % 30bfbd
% MyRed			r=0.820, g=0.141, b=0.161 % d12429
% MyPink		r=0.980, g=0.780, b=0.761 % fac7c2
% MyGray		r=0.812, g=0.788, b=0.780 % cfc9c7
% MyOrange		r=0.980, g=0.671, b=0.290 % faab4a

% Configuração de cores
\definecolor[MyColor][Magenta]      % ou ex: [r=0.862, g=0.118, b=0.118] % corresponde a RGB(220, 30, 30)
\definecolor[MyColorText][black]     % ou ex: [r=0.862, g=0.118, b=0.118] % corresponde a RGB(167, 169, 172)

% Classe para diagramação dos posts
\environment{marketing.env}		   

\starttext %---------------------------------------------------------|

\Mensagem{POR DENTRO DA EDIÇÃO}

\startMyCampaign

\hyphenpenalty=10000
\exhyphenpenalty=10000

O TEMA DA ESCRAVIDÃO EM
{\bf A FAMÍLIA MEDEIROS}

\stopMyCampaign

%\vfill\scale[lines=1.5]{\MyStar[MyColorText][none]}

\page %---------------------------------------------------------| 

\MyCover{THUMB_LIVRO.pdf}

\page %---------------------------------------------------------| 

\hyphenpenalty=10000
\exhyphenpenalty=10000

 Manifestando no romance os seus {\bf IDEAIS ANTIESCRAVISTAS}, Júlia Lopes de Almeida reconstrói os horrores do sistema, além de apresentar em detalhes o tratamento e os castigos físicos infligidos aos escravizados. 

\page %---------------------------------------------------------|

A escritora chega a render homenagem a {\bf LUIZ GAMA}, {\bf ANTÔNIO BENTO} e {\bf JOSÉ DO PATROCÍNIO}, líderes do movimento abolicionista que eram abominados pelos proprietários de escravos, mas vistos pela romancista como homens gloriosos.

\page

Segundo a filha da escritora, Margarida Lopes de Almeida, o livro é «acompanhado pela sombra negra dos escravos açoitados e pelos clarões vislumbres de uma {\bf LIBERDADE QUE SE APROXIMAVA}».

\page %---------------------------------------------------------|

\Hedra

\stoptext %---------------------------------------------------------|

