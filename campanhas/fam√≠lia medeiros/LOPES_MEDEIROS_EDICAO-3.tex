% AUTOR_LIVRO_EDICAO.tex
% Preencher com o nome das cor ou composição RGB (ex: [r=0.862, g=0.118, b=0.118]) 
\usecolors[crayola] 			   % Paleta de cores pré-definida: wiki.contextgarden.net/Color#Pre-defined_colors

% Cores definidas pelo designer:
% MyGreen		r=0.251, g=0.678, b=0.290 % 40ad4a
% MyCyan		r=0.188, g=0.749, b=0.741 % 30bfbd
% MyRed			r=0.820, g=0.141, b=0.161 % d12429
% MyPink		r=0.980, g=0.780, b=0.761 % fac7c2
% MyGray		r=0.812, g=0.788, b=0.780 % cfc9c7
% MyOrange		r=0.980, g=0.671, b=0.290 % faab4a

% Configuração de cores
\definecolor[MyColor][Magenta]      % ou ex: [r=0.862, g=0.118, b=0.118] % corresponde a RGB(220, 30, 30)
\definecolor[MyColorText][black]     % ou ex: [r=0.862, g=0.118, b=0.118] % corresponde a RGB(167, 169, 172)

% Classe para diagramação dos posts
\environment{marketing.env}		   

\starttext %---------------------------------------------------------|

\Mensagem{POR DENTRO DA EDIÇÃO}

\startMyCampaign

\hyphenpenalty=10000
\exhyphenpenalty=10000

{\bf A FAMÍLIA MEDEIROS}

TRÊS TEXTOS,

 UMA HISTÓRIA

\stopMyCampaign

%\vfill\scale[lines=1.5]{\MyStar[MyColorText][none]}

\page %---------------------------------------------------------| 

\MyCover{JULIA_MEDEIROS_THUMB}

\page %---------------------------------------------------------| 

\hyphenpenalty=10000
\exhyphenpenalty=10000

A edição da Hedra de «A família Medeiros» conta com {\bf TRÊS TEXTOS INTRODUTÓRIOS} que analisam a vida e a obra de Júlia Lopes de Almeida: um escrito em 1919, outro em 2009 e um terceiro escrito especialmente para nossa edição. 

\page %---------------------------------------------------------|

O prefácio de Alfredo Sousa, da edição de 1919, é essencial como {\bf DOCUMENTO HISTÓRICO}. Nele, encontramos uma intrigante nota biográfica sobre Júlia, que reflete a visão que seus contemporâneos têm de sua obra.

\page
O texto de Norma Telles, presente na edição de 2009, estabelece uma ligação vital entre o passado e o presente. Sua análise marca o {\bf RENASCIMENTO DO INTERESSE} pela obra de Júlia, 
mostrando como leitores e pesquisadores de todo o Brasil estão se debruçando sobre suas contribuições e redescobrindo seu valor literário.

\page

O prefácio de Rafael Balseiro Zin, por sua vez, representa uma nova geração de acadêmicos que se dedica à {\bf REDESCOBERTA DE VOZES ESQUECIDAS}. Com um foco na obra de Júlia Lopes e Maria Firmina dos Reis, Rafael é um dos expoentes da nova geração dedicada à valorização de autoras que foram silenciadas ao longo da história da literatura brasileira.

\page %---------------------------------------------------------|

Assim, nossa edição oferece um panorama que demonstra como Júlia e sua obra foram interpretadas ao longo dos anos. 
Cada texto oferece uma perspectiva única e revela a importância da {\bf RECUPERAÇÃO DE SUA TRAJETÓRIA LITERÁRIA}.

\page
\Hedra

\stoptext %---------------------------------------------------------|


