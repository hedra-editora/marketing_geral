\setuppapersize[A4]
\usecolors[crayola]
\setupbackgrounds[paper][background=color,backgroundcolor=Almond]
	
	\definefontfeature
		[default]
		[default]
		[expansion=quality,protrusion=quality,onum=yes]
	\setupalign[fullhz,hanging]
	\definefontfamily [mainface] [sf] [Formular]
	\setupbodyfont[mainface,11pt]

% Indenting [4.4 cont-enp.p.65]
			\setupindenting[yes, 3ex]  % none small medium big next first dimension
			\indenting[next]           % never not no yes always first next
			
			% [cont-ent.p.76]
			\setupspacing[broad]  %broad packed
			% O tamanho do espaço entre o ponto final e o começo de uma sentença. 


\startsetups[grid][mypenalties]
    \setdefaultpenalties
    \setpenalties\widowpenalties{2}{10000}
    \setpenalties\clubpenalties {2}{10000}
\stopsetups

\setuppagenumbering
  [location={}]            % Estilo dos números de páginat

\setuphead[subject]
[style=bfb]		

\setuplayout[
          location=middle,
          %
          leftedge=0mm,
          leftedgedistance=0mm,
          leftmargin=20mm,
          leftmargindistance=0mm,
          width=100mm,
          rightmargindistance=0mm,
          rightmargin=20mm,
          rightedgedistance=0mm,
          rightedge=0mm,
          backspace=20mm,
          %
          top=21mm,
          topdistance=0mm,
          header=0mm,
          headerdistance=0mm,
          height=250mm,
          footerdistance=0mm,
          footer=0mm,
          bottomdistance=0mm,
          bottom=21mm,
          topspace=21mm,
        setups=mypenalties,
]

\setupalign[right]

\starttext
{\bfb A guinada sobrenatural de um dos maiores autores do modernismo romeno}

\blank[big]


\noindent Para serem lidas à noite  {\it reúne quatro contos sobrenaturais que exploram a interseção entre o real e o irreal. Com humor e ironia, a obra convida o leitor a mergulhar em mistérios profundos, revelando as complexidades da condição humana em uma atmosfera noturna envolvente.}

\blank[1cm]

\inoutermargin[width=60mm,hoffset=1cm,style=tfx,,voffset=3.5cm]{
\externalfigure[MINULESCU_NOITE_THUMB][width=50mm]
}


\inoutermargin[width=70mm,hoffset=1cm,voffset=4.5cm,style=tfx]
{\noindent{\bf Título} {\em Para serem lidas à noite}\\
{\bf Autor} Ion Minulescu\\
{\bf Tradução} Fernando Klabin\\
{\bf Apresentação} Leonardo Francisco Soares\\
{\bf Editora} Hedra\\
{\bf ISBN} 978-85-7715-935-2\\
{\bf Pág.} 110\\
{\bf Preço} 36,55
}


Dentro de sua ampla trajetória ficcional, o romeno Ion Minulescu, uma das figuras literárias mais populares do país, se volta para a literatura fantástica em {\em Para serem lidas à noite}, articulando real-irreal, lógico-ilógico, sagrado-profano.

Desde sua provocante nota de abertura – “leia-as de noite, ou então, não as leia nunca” –, a obra convida o leitor a adentrar um universo enigmático e encantador. Composto por quatro contos, o livro é um convite à exploração das nuances da imaginação, combinando mistério, humor e ironia,  o leitor a explorar 

Os contos reunidos funcionam como labirintos nos quais os limites entre o real e o irreal desvanecem. Seja recorrendo a célebres tópicos como o pacto com o diabo, seja concebendo intrigantes relatos como o de uma gravata comprada na cidade romena de Braîla, a obra propõe uma viagem pelos meandros do espírito. 

Um dos aspectos mais fascinantes dessas quatro narrativas de Minulescu é a maneira como se articulam em um jogo de espelhos. Cada enigma narrativo é apresentado dentro de outro, como uma caixa de Pandora que se abre para revelar novos segredos a cada página virada. Estrutura essa que não só enriquece a experiência de leitura, mas também permite uma reflexão mais profunda sobre a interconexão das vozes das personagens e o mundo que as cerca. 

% O autor utiliza esse recurso para mostrar como as histórias pessoais se entrelaçam com o tecido da sociedade, abordando questões universais que ressoam com a experiência humana.

Outro traço distintivo do estilo do autor é a convivência do cômico e do sombrio.
Através do humor sutil com que os personagens enfrentam as adversidades e da pungente ironia do narrador, o escritor propõe uma reflexão crítica sobre a condição humana e transparece engenhosamente a fragilidade da fronteira que separa o ridículo e o trágico.

% Além disso, a obra é marcada por uma fina ironia que permeia o discurso. O humor sutil com que os personagens enfrentam suas adversidades e dilemas existenciais não apenas atenua a tensão, como também provoca uma reflexão crítica sobre a condição humana. Esse equilíbrio entre o cômico e o sombrio é uma característica distintiva do estilo do autor, que habilmente nos lembra do caráter tênue da linha que separa o ridículo e o trágico.

Nesses contos de prosa envolvente e estrutura narrativa inovadora, a ambientação noturna sugerida na nota de abertura não é meramente uma recomendação. Trata-se de uma personagem à parte que proporciona um cenário ideal para os mistérios, para os “jogos de mostras e máscaras” articulados por Minulescu e a serem vislumbrados pelo leitor notívago.

% Nesses contos, a ambientação noturna sugerida na nota de abertura não é meramente uma recomendação; ela é quase uma personagem à parte, que proporciona um cenário ideal para os mistérios, criando uma atmosfera de expectativa e suspense.
% "Para serem lidas à noite" é, portanto, uma obra que não se limita a entreter; ela provoca questionamentos e instiga a curiosidade. O leitor é convidado a mergulhar em um mundo no qual o sobrenatural e o cotidiano se encontram, no qual cada conto revela mais do que aparenta. Em resumo, esta reunião de contos é uma celebração do mistério e da complexidade da vida, um convite à reflexão sobre os limites da imaginação e da experiência humana. Com uma prosa envolvente e uma estrutura narrativa inovadora.


\subject{Sobre o autor}

Ion Minulescu, nascido em 1881 em Bucareste, destaca-se como uma das figuras mais fascinantes da literatura romena do século {\cap XX}. Escritor, poeta, crítico literário e dramaturgo, sua trajetória é marcada por uma riqueza criativa que transita entre o simbolismo e o fantástico, influenciando gerações de leitores e escritores.

Sua formação em direito na vibrante Paris proporcionou a Minulescu um contato profundo com as correntes literárias da época, em especial o simbolismo francês. Ao longo de sua carreira, ele conseguiu traduzir essa influência em uma linguagem poética e sofisticada, que revela uma sensibilidade única na representação dos sentimentos e da realidade.

% Entre suas obras mais notáveis, destaca-se {\it Para serem lidas à noite}, um mergulho em temas de sonho e introspecção, evocando atmosferas que capturam a essência do fantástico. O livro se configura como uma verdadeira ode à noite e ao mistério, onde Minulescu convida o leitor a refletir sobre a vida e as emoções em um espaço atemporal. Essa obra é um testemunho da sua habilidade em conjugar lirismo e profundidade psicológica, transformando a leitura em uma experiência única e envolvente.

Embora seja amplamente reconhecido por sua contribuição ao movimento simbolista, Minulescu destaca-se também na esfera da literatura fantástica, como verifica-se em seu célebre {\it Para serem lidas à noite}. Suas incursões nesse gênero revelam um talento excepcional para criar atmosferas oníricas e enredos que desafiam a lógica e mesclam o cotidiano com o extraordinário, conferindo às suas narrativas uma aura quase mágica.

% A atuação de Minulescu não se limita apenas à sua escrita; sua atuação como jornalista e editor também foi fundamental para o desenvolvimento cultural na Romênia. O autor não hesitou em se posicionar criticamente em relação às questões sociais e políticas de seu tempo, mostrando que a literatura pode e deve dialogar com a realidade. 

Falecido em 1944, Minulescu permanece uma figura indispensável Da literatura romena, com uma obra que transcende fronteiras e continua a ressoar profundamente nos leitores contemporâneos. 

% \blank[big]

\subject{Sobre o tradutor}

Fernando Klabin nasceu em São Paulo e formou-se em Ciência Política pela Universidade de Bucareste, onde foi agraciado com a Ordem do Mérito Cultural da Romênia no grau de Oficial, em 2016. Além de tradutor exerce atividades ocasionais como fotógrafo, escritor, ator e artista plástico.

\subject{Sobre o apresentador}

Leonardo Francisco Soares é professor associado do Instituto de Letras e Linguística da Universidade Federal de Uberlândia ({\cap ILEEL/UFU}) e professor permanente do programa de pós-graduação em Estudos Literários do {\cap ILEEL/UFU}. Publicou, dentre outros, um texto na coletânea {\it Guerra e literatura: ensaios em emergência} (Alameda, 2022)



\subject{Trechos do livro}

  \startitemize
    \item
    {\bf Capítulo {\em Bate-papo com o coisa-ruim}}

    % \startitemize
    % \item
    %  A imaginação dos poetas, na maior parte das vezes, ultrapassa a realidade e estrangula o verossímil. Ainda bem que a maioria das pessoas que frequenta a Igreja não lê poesia, e aqueles que lêem e acreditam na conversa fiada dos poetas não vão à Igreja.

    % \stopitemize

    \startitemize
    \item
      Jamais esquecerei aquele momento de terror, acentuado pela vergonha de não poder manifestá-lo diante da pessoa que o produzira em nós dois.
Amarelo como a cera, de olhos arregalados atrás de nós, Oreste não conseguiu segurar a emoção diante daquela constatação fantástica. Com a voz embargada pela síncope suprema em que sua alma parecia deixar o corpo, ele sussurrou tão baixo que mal se fez ouvir:
  
     --- Onde está sua sombra, Seu Damian? Você não faz sombra sobre a terra?

    \stopitemize

  \item
    {\bf Capítulo {\em O homem do coração de ouro}}
\startitemize
\item --- Onde está o anel?\unknown Por que você arrancou a pedra?\unknown\\
--- Não fui eu quem arrancou.\\
--- Então quem foi?\\
--- Ele!\unknown\\
--- Ele quem?\unknown\\
--- O homem do coração de ouro!\\
--- Admirável título para uma novela fantástica!, exclamei.\\

    \stopitemize

\startitemize
\item
--- Você teria a bondade de me dizer quantos anos tem?\\
--- Trezentos e onze anos, e cento e noventa e oito dias,
considerando, claro, os trinta dias dos anos bissextos.\\
--- E por que é que você está há tanto tempo por aqui?\\
--- Não posso morrer até estar completo, como todos os
mortais.\\
--- Falta-lhe algo?\\
--- Sim\unknown\\
--- Algum órgão importante?\\
--- O mais importante de todos\unknown O coração! [\unknown]

  \stopitemize

\stoptext