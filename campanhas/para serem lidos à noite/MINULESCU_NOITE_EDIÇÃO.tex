% AUTOR_LIVRO_EDICAO.tex
% Preencher com o nome das cor ou composição RGB (ex: [r=0.862, g=0.118, b=0.118]) 
\usecolors[crayola] 			   % Paleta de cores pré-definida: wiki.contextgarden.net/Color#Pre-defined_colors

% Cores definidas pelo designer:
% MyGreen		r=0.251, g=0.678, b=0.290 % 40ad4a
% MyCyan		r=0.188, g=0.749, b=0.741 % 30bfbd
% MyRed			r=0.820, g=0.141, b=0.161 % d12429
% MyPink		r=0.980, g=0.780, b=0.761 % fac7c2
% MyGray		r=0.812, g=0.788, b=0.780 % cfc9c7
% MyOrange		r=0.980, g=0.671, b=0.290 % faab4a

% Configuração de cores
\definecolor[MyColor][x=e59fc3]      % ou ex: [r=0.862, g=0.118, b=0.118] % corresponde a RGB(220, 30, 30)
\definecolor[MyColorText][black]     % ou ex: [r=0.862, g=0.118, b=0.118] % corresponde a RGB(167, 169, 172)

% Classe para diagramação dos posts
\environment{marketing.env}		   

\starttext %---------------------------------------------------------|

\Mensagem{ROMÊNIA EM FOCO}

\startMyCampaign

\hyphenpenalty=10000
\exhyphenpenalty=10000

{\bf LEIA-AS DE NOITE, OU ENTÃO NÃO AS LEIA JAMAIS.}

\stopMyCampaign

%\vfill\scale[lines=1.5]{\MyStar[MyColorText][none]}

\page %---------------------------------------------------------| 

\MyCover{MINULESCU_NOITE_THUMB}

\page %---------------------------------------------------------| 

\hyphenpenalty=10000
\exhyphenpenalty=10000

{\bf PARA SEREM LIDAS À NOITE} integra a ampla trajetória ficcional de Ion Minulescu, um dos maiores nomes do modernismo romeno.

\page

Reunião de quatro contos, a obra representa uma guinada para o {\bf FANTÁSTICO E SOBRENATURAL}.
O teor fantástico das narrativas corrobora a influência do simbolismo sobre o autor, admirador de mestres do gênero, como {\bf OSCAR WILDE} e {\bf EDGAR ALLAN POE}. 

\page

Narrados com humor e fina ironia, os contos se passam na fronteira entre {\bf REALIDADE E IMAGINÁRIO}, em que se articulam uma série de mistérios e «jogos de mostras e máscaras», a serem vislumbrados pelo leitor notívago.

\page
 Trata-se de um convite irresistível para adentrar {\bf MISTÉRIOS INFINDÁVEIS} e explorar as nuances mais sombrias e encantadoras da imaginação — mas também da condição humana. 

% \page


%  Desde o célebre tópico do pacto com o diabo até o fantasmagórico caso de uma gravata comprada na cidade romena de Braîla, os textos deste volume são pautados por binômios como {\it real-irreal}, {\it lógico-ilógico}, {\it  sagrado-profano}. A partir de uma estrutura que se multiplica em jogo de espelhos, os enigmas narrativos são sempre apresentados um dentro do outro, em {\it spin off}, como uma caixa de Pandora.

\page %---------------------------------------------------------|

\Hedra

\stoptext %---------------------------------------------------------|