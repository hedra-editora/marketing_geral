% AUTOR_LIVRO_RADAR.tex
% Preencher com o nome das cor ou composição RGB (ex: [r=0.862, g=0.118, b=0.118]) 
\usecolors[crayola] 			   % Paleta de cores pré-definida: wiki.contextgarden.net/Color#Pre-defined_colors

% Cores definidas pelo designer:
% MyGreen		r=0.251, g=0.678, b=0.290 % 40ad4a
% MyCyan		r=0.188, g=0.749, b=0.741 % 30bfbd
% MyRed			r=0.820, g=0.141, b=0.161 % d12429
% MyPink		r=0.980, g=0.780, b=0.761 % fac7c2
% MyGray		r=0.812, g=0.788, b=0.780 % cfc9c7
% MyOrange		r=0.980, g=0.671, b=0.290 % faab4a

% Configuração de cores
\definecolor[MyColor][white]      % ou ex: [r=0.862, g=0.118, b=0.118] % corresponde a RGB(220, 30, 30)
\definecolor[MyColorText][black]  % ou ex: [r=0.862, g=0.118, b=0.118] % corresponde a RGB(167, 169, 172)

% Classe para diagramação dos posts
\environment{marketing.env}		   

% Comandos & Instruções %%%%%%%%%%%%%%%%%%%%%%%%%%%%%%%%%%%%%%%%%%%%%%%%%%%%%%%%%%%%%%%|

% Cabeço e rodapé: Informações (caso queira trocar alguma coisa)
 		\def\MensagemSaibaMais 	{SAIBA MAIS:}
 		\def\MensagemSite		{HEDRA.COM.BR}
 		\def\MensagemLink		{LINK NA BIO}

\starttext %---------------------------------------------------------|

\Mensagem{NO RADAR}

\MyCover{THUMB_LIVRO.pdf}

\page %---------------------------------------------------------|

\hyphenpenalty=10000
\exhyphenpenalty=10000

Os Caxinauá, também conhecidos como {\bf HUNI KUIN}, são um grupo étnico indígena que habita a região amazônica, principalmente no estado do Acre, no Brasil. Sua cultura é rica em tradições, rituais xamânicos e artesanato, sendo reconhecidos por suas belas pinturas corporais e arte plumária. Os {\bf CAXINAUÁ} prezam a conexão com a natureza e a espiritualidade, realizando cerimônias sagradas e cultivando uma relação harmoniosa com o meio ambiente. Sua língua, o Pano, é parte essencial de sua identidade e resistência cultural. Atualmente, os Caxinauá enfrentam desafios como a preservação de suas terras, a manutenção de suas tradições frente às pressões da sociedade moderna e a busca pela valorização de sua cultura ancestral. É fundamental reconhecer e respeitar a história, os conhecimentos e a vivência dos Caxinauá, contribuindo para a promoção da diversidade cultural e para a garantia de seus direitos enquanto povos originários da Amazônia.

\page %---------------------------------------------------------|

\hyphenpenalty=10000
\exhyphenpenalty=10000

Aqui entram os nomes em destaque do livro, gente conhecida ou importante que valha mencionar. O livro é organizado por {\bf JOSÉZINHO DAS COUVE} e {\bf MARIANA XAUBLAU}. A tradução é de {\bf GEOVÁ DAS CRUZES}, e a apresentação é de {\bf VIRGULINO MORETTI}.


\page %---------------------------------------------------------|

\Hedra

\stoptext %---------------------------------------------------------|