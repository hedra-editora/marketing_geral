% Preencher com o nome das cor ou composição RGB (ex: [r=0.862, g=0.118, b=0.118]) 
\usecolors[crayola] 			   % Paleta de cores pré-definida: wiki.contextgarden.net/Color#Pre-defined_colors

% Cores definidas pelo designer:
% MyGreen		r=0.251, g=0.678, b=0.290 % 40ad4a
% MyCyan		r=0.188, g=0.749, b=0.741 % 30bfbd
% MyRed			r=0.820, g=0.141, b=0.161 % d12429
% MyPink		r=0.980, g=0.780, b=0.761 % fac7c2
% MyGray		r=0.812, g=0.788, b=0.780 % cfc9c7
% MyOrange		r=0.980, g=0.671, b=0.290 % faab4a

% Configuração de cores
\definecolor[MyColor][Almond]      % ou ex: [r=0.862, g=0.118, b=0.118] % corresponde a RGB(220, 30, 30)
\definecolor[MyColorText][Salmon]  % ou ex: [r=0.862, g=0.118, b=0.118] % corresponde a RGB(167, 169, 172)

% Classe para diagramação dos posts
\environment{marketing.env}		   

% Comandos & Instruções %%%%%%%%%%%%%%%%%%%%%%%%%%%%%%%%%%%%%%%%%%%%%%%%%%%%%%%%%%%%%%%|

% Cabeço e rodapé: Informações (caso queira trocar alguma coisa)
 		\def\MensagemSaibaMais{SAIBA MAIS:}
 		\def\MensagemSite{HEDRA.COM.BR}
 		\def\MensagemLink{LINK NA BIO}

% Pesos para os títulos:
%		\startMyCampaign...		 \stopMyCampaign
%		\stopMyCampaignSection...   \stopMyCampaignSection

% Aplicação de imagens: 
% 		\MyCover{capa.pdf}  	% Aplicação de capa de livro com sombra
%		\MyPicture{Imagem.png}  % Imagem com aplicação de filtro segundo cor MyColorText
%		\MyPhoto{}			    % Aplicação simples de imagem com tamamho \textwidth

% Aplicação de imagem com legenda:		
% 		\placefigure{Legenda}{\externalfigure[drop2-1.png][width=\textwidth]}

% Cabeço e rodabé: Opções
% 		\Mensagem{AGORA É QUE SÃO ELAS}
% 		\Hashtag{campanha de natal}
% 		\Mensagem{campanha de natal}

% Alteração de várias cores de background:
% \setupbackgrounds[page][background=color,backgroundcolor=MyGray]

% Estrela: 
% \vfill\scale[lines=2]{\MyStar[MyColorText][none]} 			% Estrela pequena  
% \startpositioning 											% Estrela grande
%  \position(-1,-.3){\scale[scale=980]{\MyStar[white][none]}}
% \stoppositioning

% Logos e selos: 				
% \Hedra
% \HedraAyllon	% Não está pronto
% \HedraAcorde	% Não está pronto
% \Ayllon		% Não está pronto
% \Acorde		% Não está pronto

% Atalhos: 						
% 		\Seta  % Seta para baixo

% Espaçamentos:
% \setupinterlinespace[line=1.9ex]		% para regular o entelinha (colocar \par ao fim do período)
% \hyphenpenalty=10000   			    % evitar quebras

%%%%%%%%%%%%%%%%%%%%%%%%%%%%%%%%%%%%%%%%%%%%%%%%%%%%%%%%%%%%%%%%%%%%%%%%%%%%%%%%%%%%%%%|

\starttext
%\showframe  %Para mostrar somente as linhas.

\startMyCampaign
Você sabia que\\
{\bf O GIRASSOL}\\
é amarelo? E que\\ 
é uma das maiores\\ 
flores brasileiras

\stopMyCampaign

\page %---------------------------------------------------------|

\hyphenpenalty=10000
\exhyphenpenalty=10000

{\bf GIRASSOL}, {\bf ROSA} e {\bf CAMÉLIA} são flores especiais que aparecem no verão. Lorem ipsum dolor sit amet, consectetur adipiscing elit. Praesent sit amet pulvinar ante, a ultricies magna. Etiam placerat quis tellus sed ultrices.

\MyPicture{NIETZSCHE_GREGOS_DETALHE-1.png}

\page %---------------------------------------------------------|

Praesent sit amet pulvinar ante, a ultricies magna. Etiam placerat quis tellus sed ultrices. Duis aliquet sed quam non tincidunt. Donec sit amet tempor urna. Quisque auctor justo enim. Curabitur vel est consectetur, sodales orci a, eleifend lacus. Morbi quis convallis risus. Sed eget scelerisque urna. Fusce et placerat orci, nec commodo ligula. Donec lobortis urna quis lorem euismod, eget bibendum sapien tincidunt. Nunc mollis ipsum eu sapien placerat, at suscipit nunc hendrerit.

\page %---------------------------------------------------------|

\Hedra

\stoptext