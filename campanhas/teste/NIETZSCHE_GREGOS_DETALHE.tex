% Preencher com o nome das cor ou composição RGB (ex: [r=0.862, g=0.118, b=0.118]) 
\usecolors[crayola] 			   % Paleta de cores pré-definida: wiki.contextgarden.net/Color#Pre-defined_colors

% Cores definidas pelo designer:
% MyGreen		r=0.251, g=0.678, b=0.290 % 40ad4a
% MyCyan		r=0.188, g=0.749, b=0.741 % 30bfbd
% MyRed			r=0.820, g=0.141, b=0.161 % d12429
% MyPink		r=0.980, g=0.780, b=0.761 % fac7c2
% MyGray		r=0.812, g=0.788, b=0.780 % cfc9c7
% MyOrange		r=0.980, g=0.671, b=0.290 % faab4a

% Configuração de cores
\definecolor[MyColor][x=E34F75]      % ou ex: [r=0.862, g=0.118, b=0.118] % corresponde a RGB(220, 30, 30)
\definecolor[MyColorText][x=635558]  % ou ex: [r=0.862, g=0.118, b=0.118] % corresponde a RGB(167, 169, 172)

% Classe para diagramação dos posts
\environment{marketing.env}		   


% Comandos & Instruções %%%%%%%%%%%%%%%%%%%%%%%%%%%%%%%%%%%%%%%%%%%%%%%%%%%%%%%%%%%%%%%|

% Cabeço e rodabé: Informações (caso queira trocar alguma coisa)
% 		\def\MensagemSaibaMais{SAIBA MAIS:}
% 		\def\MensagemSite{HEDRA.COM.BR}
% 		\def\MensagemLink{LINK NA BIO}

% Pesos para os títulos:
%		\startMyCampaign...		 \stopMyCampaign
%		\stopMyCampaignSection...   \stopMyCampaignSection

% Aplicação de imagens: 
% 		\MyCover{capa.pdf}  	% Aplicação de capa de livro com sombra
%		\MyPicture{Imagem.png}  % Imagem com aplicação de filtro segundo cor MyColorText
%		\MyPhoto{}			    % Aplicação simples de imagem com tamamho \textwidth

% Aplicação de imagem com legenda:		
% 		\placefigure{Legenda}{\externalfigure[drop2-1.png][width=\textwidth]}

% Cabeço e rodabé: Opções
% 		\Mensagem{AGORA É QUE SÃO ELAS}
% 		\Hashtag{campanha de natal}
% 		\Mensagem{campanha de natal}

% Alteração de várias cores de background:
% \setupbackgrounds[page][background=color,backgroundcolor=MyGray]

% Estrela: 
% \vfill\scale[lines=2]{\MyStar[MyColorText][none]} 			% Estrela pequena  
% \startpositioning 											% Estrela grande
%  \position(-1,-.3){\scale[scale=980]{\MyStar[white][none]}}
% \stoppositioning

% Logos e selos: 				
% \Hedra
% \HedraAyllon	% Não está pronto
% \HedraAcorde	% Não está pronto
% \Ayllon		% Não está pronto
% \Acorde		% Não está pronto

% Atalhos: 						
% 		\Seta  % Seta para baixo

% Espaçamentos:
% \setupinterlinespace[line=1.9ex]		% para regular o entelinha (colocar \par ao fim do período)
% \hyphenpenalty=10000   			    % evitar quebras

%%%%%%%%%%%%%%%%%%%%%%%%%%%%%%%%%%%%%%%%%%%%%%%%%%%%%%%%%%%%%%%%%%%%%%%%%%%%%%%%%%%%%%%|

\starttext
\hyphenpenalty=10000
\exhyphenpenalty=10000

\Mensagem{VAMOS FALAR SOBRE ISSO}

\startMyCampaign
%DIVULGAÇÃO
%\Seta LANÇAMENTO
{\bf NIETZSCHE TRAZ À TONA O QUE «NENHUM CONHECIMENTO PODERÁ NOS ROUBAR: O GRANDE HOMEM».}

\stopMyCampaign

\page %---------------------------------------------------------|

\MyPicture{NIETZSCHE_GREGOS_DETALHE-1.png}

A partir do jogo heraclitiano, cumpre indagar: o que aprender com {\it A filosofia na era trágica dos gregos}, de Friederich Nietzsche?

\page %---------------------------------------------------------|

Nesta obra póstuma e inacabada, o filósofo alemão permite-se reproduzir «grandes homens» com a ponta de sua pena. E, para isso, cita os primeiros filósofos, que se prestam a modelos-vivos: Tales de Mileto, Anaximandro de Mileto, Heráclito de Éfeso, Parmênides de Eleia, Zenão de Eleia, Anaxágoras de Clazômenas.

\page %---------------------------------------------------------|

Ao enfatizar não só as antigas hipóteses de interpretação do homem e do universo (mas também suas vidas singulares), o filósofo alemão não pretende cultuar personalidades ou erigir ídolos. E tampouco poderia ser diferente. Afinal de contas: «Outros povos possuem santos, enquanto que os gregos, por sua vez, têm sábios».

\page %---------------------------------------------------------|

«Eu conto a história de tais filósofos de um modo simplificado: espero destacar apenas o ponto de cada sistema que é um pedaço de {\it personalidade} e pertence àquele aspecto incontestável e indiscutível, a ser preservado pela história.»

\vfill
\scale[factor=fit]{\tfxx Citação que faz parte da introdução de {\bf Fernando de Moraes Barros}.}

\page %---------------------------------------------------------|

\MyCover{NIETZSCHE_GREGOS_DETALHE_THUMB.pdf}

\page %---------------------------------------------------------|

\Hedra

\stoptext