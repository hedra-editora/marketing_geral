% AUTOR_LIVRO_TRECHO.tex
% Preencher com o nome das cor ou composição RGB (ex: [r=0.862, g=0.118, b=0.118]) 
\usecolors[crayola] 			   % Paleta de cores pré-definida: wiki.contextgarden.net/Color#Pre-defined_colors

% Cores definidas pelo designer:
% MyGreen		r=0.251, g=0.678, b=0.290 % 40ad4a
% MyCyan		r=0.188, g=0.749, b=0.741 % 30bfbd
% MyRed			r=0.820, g=0.141, b=0.161 % d12429
% MyPink		r=0.980, g=0.780, b=0.761 % fac7c2
% MyGray		r=0.812, g=0.788, b=0.780 % cfc9c7
% MyOrange		r=0.980, g=0.671, b=0.290 % faab4a

% Configuração de cores
\definecolor[MyColor][x=e7cd16]      % ou ex: [r=0.862, g=0.118, b=0.118] % corresponde a RGB(220, 30, 30)
\definecolor[MyColorText][black]     % ou ex: [r=0.862, g=0.118, b=0.118] % corresponde a RGB(167, 169, 172)

% Classe para diagramação dos posts
\environment{marketing.env}		   

\def\startMyCampaign{\bgroup
            \FormularMI
            \switchtobodyfont[26.5pt]
            \setupinterlinespace[line=1.9ex]
            \setcharacterkerning[packed]}
\def\stopMyCampaign{\par\egroup}


\starttext %---------------------------------------------------------|

\Mensagem{DESTAQUE}

\startMyCampaign

\hyphenpenalty=10000
\exhyphenpenalty=10000
«Quais as\\ possibilidades de um teatro antifascista ou anticapitalista quando
 não há condições de contato com a base social, num tempo de reorganização da esquerda?
\vfill\hfill →

\page

A escolha inicial do tusp foi concentrar-se no campo que lhe dizia respeito, o de um teatro universitário.»


\stopMyCampaign

{\vfill\scale[factor=6]{\Seta\,Trecho do livro {\bf Teatro anticapitalista: o}}\setupinterlinespace[line=1.5ex]\scale[factor=6]{{\bf caso {\cap TUSP} (1966--1969)}, de Maria Lívia Goes e}\setupinterlinespace[line=1.5ex]\scale[factor=6]{Sérgio de Carvalho.}}

\page %---------------------------------------------------------| 

\MyCover{CARVALHO_TUSP_THUMB}

\page %---------------------------------------------------------|

\Hedra

\starttikzpicture[remember picture,overlay]
\node at (1.5,-2) {\externalfigure[logo_proac.png][width=3cm]};
\node at (4.5,-1.92) {\externalfigure[logo_cult.png][width=1.2cm]};
\node at (7.5,-2.25) {\externalfigure[logo_gov.png][width=3cm]};

\stoptikzpicture



\stoptext %---------------------------------------------------------|
