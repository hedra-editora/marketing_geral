% AUTOR_LIVRO_CURIOSIDADES.tex
% Preencher com o nome das cor ou composição RGB (ex: [r=0.862, g=0.118, b=0.118]) 
\usecolors[crayola] 			   % Paleta de cores pré-definida: wiki.contextgarden.net/Color#Pre-defined_colors

% Cores definidas pelo designer:
% MyGreen		r=0.251, g=0.678, b=0.290 % 40ad4a
% MyCyan		r=0.188, g=0.749, b=0.741 % 30bfbd
% MyRed			r=0.820, g=0.141, b=0.161 % d12429
% MyPink		r=0.980, g=0.780, b=0.761 % fac7c2
% MyGray		r=0.812, g=0.788, b=0.780 % cfc9c7
% MyOrange		r=0.980, g=0.671, b=0.290 % faab4a

% Configuração de cores
\definecolor[MyColor][x=e7cd16]      % ou ex: [r=0.862, g=0.118, b=0.118] % corresponde a RGB(220, 30, 30)
\definecolor[MyColorText][black]     % ou ex: [r=0.862, g=0.118, b=0.118] % corresponde a RGB(167, 169, 172)

% Classe para diagramação dos posts
\environment{marketing.env}		   

\starttext %---------------------------------------------------------|

\hyphenpenalty=10000
\exhyphenpenalty=10000

\Mensagem{EM CONTEXTO} %Sempre usar esse header

\startMyCampaign

\hyphenpenalty=10000
\exhyphenpenalty=10000

{\bf OS FUZIS DE DONA TEREZA CARRAR}

A MAIOR 

REALIZAÇÃO DA BREVE HISTÓRIA DO {\bf TUSP}

\stopMyCampaign

\page %---------------------------------------------------------| 

\MyPicture{IMAGEM_FUZIS}
{\scale[factor=5]{Páginas 10 e 11 do programa da peça {\it Os fuzis},} \setupinterlinespace[line=1.5ex]\scale[factor=5]{encenada pelo {\cap TUSP} em 1968.}}



\page
\hyphenpenalty=10000
\exhyphenpenalty=10000
{\bf OS FUZIS DA SENHORA CARRAR}, peça de Bertolt Brecht, é centrada no dilema de Tereza Carrar, uma mulher que enfrenta a tensão entre permanecer neutra ou participar de uma resistência armada.

\page %---------------------------------------------------------|


 O {\cap TUSP} (Teatro dos Universitários de São Paulo) desempenhou um papel crucial na adaptação dessa peça no Brasil. A montagem de 1968, dirigida por Flávio Império, foi um marco não apenas pela qualidade estética, mas também pela {\bf SINTONIA COM OS DEBATES POLÍTICOS E SOCIAIS DO PERÍODO}.

 \page

 Incorporando elementos simbólicos como armas sob os assentos da plateia e utilizando músicas, projeções e coreografias, a encenação envolveu o público em um apelo direto à {\bf RESISTÊNCIA CONTRA A DITADURA}. Essa montagem é considerada «a maior realização da breve história do {\cap TUSP}», consolidando a peça como uma referência na trajetória do grupo.

\page

\MyPicture{IMAGEM_FUZIS-2}
{\scale[factor=5]{Páginas 6 e 7 do programa da peça {\it Os fuzis},} \setupinterlinespace[line=1.5ex]\scale[factor=5]{encenada pelo {\cap TUSP} em 1968.}}

\page

A encenação do {\cap TUSP} representou um chamado à participação política e resistência, tornando-se {\bf UM MARCO NO TEATRO POLITIZADO} da época. A montagem enfrentou censura e monitoramento devido ao seu teor revolucionário, especialmente após sua repercussão em cidades como São Paulo e Rio de Janeiro.

\page

«Tudo aquilo que José Celso Martinez Corrêa tentou e --- na minha opinião
--- não conseguiu inteiramente em {\it Roda viva} está realizado, com
perfeita coerência, na parte final de {\it Os fuzis}. O espetáculo
estoura os limites do palco com a sua violência, alastra-se pela
plateia, {\bf AGRIDE O ESPECTADOR COM O SEU ÓDIO, CONQUISTA-O COM O SEU AMOR}.
Tudo isto sem qualquer apelo à gratuidade, sem qualquer concessão à
facilidade.»



{\vfill\scale[factor=5]{Yan Michalski, «Os fuzis de D. Tereza Carrar». {\bf Jornal do}}\setupinterlinespace[line=1.5ex]\scale[factor=5]{{\bf Brasil}, Rio de Janeiro, 6 jul. 1968, p.\,10.}}


\page

\MyCover{CARVALHO_TUSP_THUMB}

\page %----------------------------------------------------------|

\Hedra

\starttikzpicture[remember picture,overlay]
\node at (1.5,-2) {\externalfigure[logo_proac.png][width=3cm]};
\node at (4.5,-1.92) {\externalfigure[logo_cult.png][width=1.2cm]};
\node at (7.5,-2.25) {\externalfigure[logo_gov.png][width=3cm]};

\stoptikzpicture


\stoptext %---------------------------------------------------------|




