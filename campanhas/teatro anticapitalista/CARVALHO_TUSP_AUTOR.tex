\% AUTOR_LIVRO_AUTOR.tex
% Preencher com o nome das cor ou composição RGB (ex: [r=0.862, g=0.118, b=0.118]) 
\usecolors[crayola] 			   % Paleta de cores pré-definida: wiki.contextgarden.net/Color#Pre-defined_colors

% Cores definidas pelo designer:
% MyGreen		r=0.251, g=0.678, b=0.290 % 40ad4a
% MyCyan		r=0.188, g=0.749, b=0.741 % 30bfbd
% MyRed			r=0.820, g=0.141, b=0.161 % d12429
% MyPink		r=0.980, g=0.780, b=0.761 % fac7c2
% MyGray		r=0.812, g=0.788, b=0.780 % cfc9c7
% MyOrange		r=0.980, g=0.671, b=0.290 % faab4a

% Configuração de cores
\definecolor[MyColor][x=e7cd16]	  [r=0.862, g=0.118, b=0.118]      % ou ex: [r=0.862, g=0.118, b=0.118] % corresponde a RGB(220, 30, 30)
\definecolor[MyColorText][black] [r=0.655, g=0.663, b=0.675]      % ou ex: [r=0.862, g=0.118, b=0.118] % corresponde a RGB(167, 169, 172)

% Classe para diagramação dos posts
\environment{marketing.env}		   

% Cabeço e rodapé: Informações (caso queira trocar alguma coisa)
 		\def\MensagemSaibaMais  {SAIBA MAIS:}
 		\def\MensagemSite		{HEDRA.COM.BR}
 		\def\MensagemLink       {LINK NA BIO}

\starttext %--------------------------------------------------------|

\hyphenation{ANTICAPITALISTA}

\Mensagem{SOBRE OS AUTORES}

\vskip-30pt
\startMyCampaign

\hyphenpenalty=10000
\exhyphenpenalty=10000

OS AUTORES DE

{\bf TEATRO

 ANTICAPITALISTA:
O CASO TUSP (1966--1969)}
\stopMyCampaign
 \scale[factor=60]{\MyPortrait{maria}\quad\MyPortrait{sergio}}

\vfill\scale[factor=6]{\Seta\,Maria Lívia Nobre Goes\quad\scale[factor=7.3]{\Seta\,Sérgio de Carvalho}}

\page %----------------------------------------------------------|

\hyphenpenalty=10000
\exhyphenpenalty=10000

 {\bf MARIA LÍVIA NOBRE GOES} é dramaturga e pesquisadora teatral. É associada ao grupo Companhia do Latão, no qual contribui com projetos que revisitam a história brasileira criticamente, utilizando o teatro como ferramenta de reflexão social. 
\page

{\bf SÉRGIO DE CARVALHO} é dramaturgo, diretor teatral e professor da {\cap ECA-USP}. Ficou conhecido publicamente por sua atuação no grupo Companhia do Latão, que é referência no teatro político brasileiro contemporâneo. Ao longo de sua trajetória, Sérgio conserva duas importantes influências, a saber: o dramaturgo Bertolt Brecht e o pensamento crítico marxista.

\page %----------------------------------------------------------|


 Em {\bf TEATRO ANTICAPITALISTA: O CASO TUSP (1966--1969)}, os autores buscam compreender as experiências teatrais e políticas dos grupos de teatro universitário da cidade de São Paulo no início da segunda metade do século {\cap XX}, em especial a experiência do {\cap TUSP} e de seu teatro politizado.

\page

\MyCover{CARVALHO_TUSP_THUMB}

\page %----------------------------------------------------------|

\Hedra

\starttikzpicture[remember picture,overlay]
\node at (1.5,-2) {\externalfigure[logo_proac.png][width=3cm]};
\node at (4.5,-1.92) {\externalfigure[logo_cult.png][width=1.2cm]};
\node at (7.5,-2.25) {\externalfigure[logo_gov.png][width=3cm]};

\stoptikzpicture

\stoptext %---------------------------------------------------------|
