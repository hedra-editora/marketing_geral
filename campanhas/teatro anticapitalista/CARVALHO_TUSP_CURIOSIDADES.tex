% AUTOR_LIVRO_CURIOSIDADES.tex
% Preencher com o nome das cor ou composição RGB (ex: [r=0.862, g=0.118, b=0.118]) 
\usecolors[crayola] 			   % Paleta de cores pré-definida: wiki.contextgarden.net/Color#Pre-defined_colors

% Cores definidas pelo designer:
% MyGreen		r=0.251, g=0.678, b=0.290 % 40ad4a
% MyCyan		r=0.188, g=0.749, b=0.741 % 30bfbd
% MyRed			r=0.820, g=0.141, b=0.161 % d12429
% MyPink		r=0.980, g=0.780, b=0.761 % fac7c2
% MyGray		r=0.812, g=0.788, b=0.780 % cfc9c7
% MyOrange		r=0.980, g=0.671, b=0.290 % faab4a

% Configuração de cores
\definecolor[MyColor][x=e7cd16]      % ou ex: [r=0.862, g=0.118, b=0.118] % corresponde a RGB(220, 30, 30)
\definecolor[MyColorText][black]     % ou ex: [r=0.862, g=0.118, b=0.118] % corresponde a RGB(167, 169, 172)

% Classe para diagramação dos posts
\environment{marketing.env}		   

\starttext %---------------------------------------------------------|

\hyphenpenalty=10000
\exhyphenpenalty=10000

\Mensagem{EM CONTEXTO} %Sempre usar esse header

\startMyCampaign

\hyphenpenalty=10000
\exhyphenpenalty=10000

VOCÊ CONHECE A HISTÓRIA DO {\bf TUSP} 
E SEU {\bf TEATRO POLÍTICO}?


\stopMyCampaign

\page %---------------------------------------------------------| 


\MyPhoto{IMAGEM8}

\scale[factor=5]{Atores do {\cap TUSP} antes da apresentação no Sindicato dos Bancários.}

\page

\hyphenpenalty=10000
\exhyphenpenalty=10000



O {\cap TUSP}, Teatro dos Universitários de São Paulo, foi um grupo teatral formado por estudantes militantes, de 1966 a 1969. Surgiu como uma iniciativa independente, sem ligação oficial com instituições, e envolveu alunos de diversas faculdades. O grupo teve uma curta, mas marcante trajetória, destacando-se como um {\bf MODELO DE TEATRO POLÍTICO}.

\page %---------------------------------------------------------|

Criado em meio à ditadura militar no Brasil, o grupo contou com a colaboração de artistas renomados, como Paulo José e Flávio Império, e produziu espetáculos de alto nível, como «A exceção e a regra» e «Os fuzis de Dona Tereza Carrar», ambas baseadas em textos de {\bf BERTOLT BRECHT}.

\page


\MyPhoto{IMAGEM9}

\scale[factor=5]{Projeção e sombras em {\em A exceção e a regra}.}


\page

Além das peças, o grupo organizava debates, publicava a revista aParte e promovia ações culturais e educativas. Embora não se declarasse explicitamente como um grupo de «teatro político», suas atividades dialogavam diretamente com a {\bf RESISTÊNCIA À DITADURA E AS LUTAS PELA DEMOCRACIA}.


\page

A repressão, no entanto, levou ao fim do coletivo em 1969, quando muitos de seus membros foram presos, exilados ou forçados à clandestinidade.

\page

O {\cap TUSP} é lembrado por sua capacidade de unir {\bf CRIATIVIDADE ARTÍSTICA E COMPROMISSO POLÍTICO}, mostrando que o teatro pode ser uma ferramenta poderosa de transformação social. 

\page %---------------------------------------------------------|

\MyCover{CARVALHO_TUSP_THUMB}

\page

\Hedra


\starttikzpicture[remember picture,overlay]
\node at (1.5,-2) {\externalfigure[logo_proac.png][width=3cm]};
\node at (4.5,-1.92) {\externalfigure[logo_cult.png][width=1.2cm]};
\node at (7.5,-2.25) {\externalfigure[logo_gov.png][width=3cm]};

\stoptikzpicture





\stoptext %---------------------------------------------------------|