% Preencher com o nome das cor ou composição RGB (ex: [r=0.862, g=0.118, b=0.118]) 
\usecolors[crayola] 			   % Paleta de cores pré-definida: wiki.contextgarden.net/Color#Pre-defined_colors

% Cores definidas pelo designer:
% MyGreen		r=0.251, g=0.678, b=0.290 % 40ad4a
% MyCyan		r=0.188, g=0.749, b=0.741 % 30bfbd
% MyRed			r=0.820, g=0.141, b=0.161 % d12429
% MyPink		r=0.980, g=0.780, b=0.761 % fac7c2
% MyGray		r=0.812, g=0.788, b=0.780 % cfc9c7
% MyOrange		r=0.980, g=0.671, b=0.290 % faab4a

% Configuração de cores
\definecolor[MyColor][x=ad0406]      % ou ex: [r=0.862, g=0.118, b=0.118] % corresponde a RGB(220, 30, 30)
\definecolor[MyColorText][white]     % ou ex: [r=0.862, g=0.118, b=0.118] % corresponde a RGB(167, 169, 172)

% Classe para diagramação dos posts
\environment{marketing.env}		   

\starttext %---------------------------------------------------------|

\Mensagem{DIA DOS NAMORADOS} %Sempre usar esse header

\startMyCampaign
\hyphenpenalty=10000
\exhyphenpenalty=10000

«Casamento. Quanta tristeza, miséria, humilhação; quantas lágrimas e maldições; que
agonia e sofrimento essa palavra tem trazido à humanidade.»
\stopMyCampaign


{\vfill\scale[factor=fit]{\Seta\,Emma Goldman em \bf Sobre anarquismo, sexo e casamento}}

\page %---------------------------------------------------------| 

\hyphenpenalty=10000
\exhyphenpenalty=10000

\MyPicture{./goldman2.jpg}

{\vfill\scale[factor=6]{\Seta\,Emma Goldman (1869-1940)}}


\page %---------------------------------------------------------|

\MyCover{./GOLDMAN_CASAMENTO_THUMB.pdf}

\page %---------------------------------------------------------|

\Hedra

\stoptext %---------------------------------------------------------|