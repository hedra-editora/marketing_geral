% AUTOR_LIVRO_EFEMERIDE.tex
% Preencher com o nome das cor ou composição RGB (ex: [r=0.862, g=0.118, b=0.118]) 
\usecolors[crayola] 			   % Paleta de cores pré-definida: wiki.contextgarden.net/Color#Pre-defined_colors

% Cores definidas pelo designer:
% MyGreen		r=0.251, g=0.678, b=0.290 % 40ad4a
% MyCyan		r=0.188, g=0.749, b=0.741 % 30bfbd
% MyRed			r=0.820, g=0.141, b=0.161 % d12429
% MyPink		r=0.980, g=0.780, b=0.761 % fac7c2
% MyGray		r=0.812, g=0.788, b=0.780 % cfc9c7
% MyOrange		r=0.980, g=0.671, b=0.290 % faab4a

% Configuração de cores
\definecolor[MyColor][MaximumRed]      % ou ex: [r=0.862, g=0.118, b=0.118] % corresponde a RGB(220, 30, 30)
\definecolor[MyColorText][black]     % ou ex: [r=0.862, g=0.118, b=0.118] % corresponde a RGB(167, 169, 172)

% Classe para diagramação dos posts
\environment{marketing.env}	

% Cabeço e rodapé: Informações (caso queira trocar alguma coisa)
        \def\MensagemSaibaMais  {SAIBA MAIS:}
        \def\MensagemSite       {HEDRA.COM.BR}
        \def\MensagemLink       {LINK NA BIO}

%\environment{extra.env}	   

\starttext %---------------------------------------------------------|

\hyphenpenalty=10000
\exhyphenpenalty=10000

\def\MyBackgroundMessage{206 ANOS DE MARX}

\MyBackground{MARX2.jpg}

\vfill
\startMyCampaign
«É muito mais\\ seguro ser temido que amado, quando se deve ser\\ desprovido de um dos dois.»
\stopMyCampaign

\page %---------------------------------------------------------| 

% \hyphenpenalty=10000
% \exhyphenpenalty=10000

% \startnarrower[3*right]
% «Comparo a fortuna a um desses rios ruinosos que, quando se iram, alagam as planícies, arruínam as árvores e os edifícios, levam terra desta parte, põem-na em outro lugar: qualquer um foge em sua presença, todos cedem ao seu ímpeto sem poder impedi-lo de modo algum.»
% \stopnarrower

% \starttikzpicture[remember picture,overlay]
% \node at (7.8,3.84)
% {\externalfigure
%               [./fortuna1.jpg]
%               [width=.3\textwidth]};
% \stoptikzpicture

% \vfill\scale[factor=fit]{\Seta\,Maquiavel, em {\it O príncipe}, trad.\,José Martins}

% \page %---------------------------------------------------------| 

% \hyphenpenalty=10000
% \exhyphenpenalty=10000

% \startnarrower[4*left]
% «Ainda que sejam assim, aos homens nada impede que, quando os tempos estão calmos, tomem providências, com proteções e com diques: de modo que, ao se avolumarem depois, ou iriam por um canal ou o seu ímpeto não seria nem tão violento nem tão danoso.»
% \stopnarrower


% \starttikzpicture[remember picture,overlay]
% \node at (1.1,4.5)
% {\externalfigure
%               [./fortuna2.jpg]
%               [width=.3\textwidth]};

% \node at (4.5,-.2){\scale[factor=fit]{\Seta\,{\it A alegoria da Fortuna} (1658-1659), do pintor italiano Salvator Rosa}};

% \stoptikzpicture


\page %---------------------------------------------------------|

\MyCover{MARX_MANIFESTO_THUMB.png}

\page %---------------------------------------------------------|

\Hedra

\stoptext %---------------------------------------------------------|