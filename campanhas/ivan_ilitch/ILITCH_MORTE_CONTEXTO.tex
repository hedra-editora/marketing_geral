% AUTOR_LIVRO_CURIOSIDADE.tex
% Vamos falar sobre isso "curiosidades"
% > "EM CONTEXTO"

% Preencher com o nome das cor ou composição RGB (ex: [r=0.862, g=0.118, b=0.118]) 
\usecolors[crayola] 			   % Paleta de cores pré-definida: wiki.contextgarden.net/Color#Pre-defined_colors

% Cores definidas pelo designer:
% MyGreen		r=0.251, g=0.678, b=0.290 % 40ad4a
% MyCyan		r=0.188, g=0.749, b=0.741 % 30bfbd
% MyRed			r=0.820, g=0.141, b=0.161 % d12429
% MyPink		r=0.980, g=0.780, b=0.761 % fac7c2
% MyGray		r=0.812, g=0.788, b=0.780 % cfc9c7
% MyOrange		r=0.980, g=0.671, b=0.290 % faab4a

% Configuração de cores
\definecolor[MyColor][x=e95558]      % ou ex: [r=0.862, g=0.118, b=0.118] % corresponde a RGB(220, 30, 30)
\definecolor[MyColorText][black]     % ou ex: [r=0.862, g=0.118, b=0.118] % corresponde a RGB(167, 169, 172)

% Classe para diagramação dos posts
\environment{marketing.env}		   

\starttext %---------------------------------------------------------|

\hyphenpenalty=10000
\exhyphenpenalty=10000

\Mensagem{TOLSTOÍSMO} %Sempre usar esse header

\startMyCampaign

\hyphenpenalty=10000
\exhyphenpenalty=10000

VEGETARIANISMO, CASTIDADE E CRISTIANISMO NÃO ORTODOXO: CONHEÇA A {\bf SEITA DE TOLSTÓI}

\stopMyCampaign

\page

\MyPicture{gandhi}

\vfill\Seta{Mahatma Gandhi e outros residentes da fazenda de Tolstói, 1910}

\page %---------------------------------------------------------| 

\hyphenpenalty=10000
\exhyphenpenalty=10000

Tolstói bon vivant, gostava de beber e curtir a noite com mulheres e um carteado.
Mas a década de 1880 aprofundou uma série de crises existenciais pelas quais o
escritor havia passado e o levou a uma fase que ele próprio definiu como
sua “redenção moral”.

\page

Já praticante do vegetarianismo, ele abriu mão
dos direitos autorais de algumas obras em prol dos camponeses e
sistematizou uma série de preceitos filosóficos e religiosos que,
reunidos, passaram a ser conhecidos como {\bf TOLSTOÍSMO}, doutrina
baseada no cristianismo, mas acrescida de outras concepções, que
repercutiu no mundo todo e fez com que Tolstói fosse excomungado da
Igreja Ortodoxa.

\page %---------------------------------------------------------|

\MyCover{./TOLSTOI_MORTE_THUMB.jpeg}

\page %---------------------------------------------------------|

\Hedra

\stoptext %---------------------------------------------------------|