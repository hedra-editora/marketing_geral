% Preencher com o nome das cor ou composição RGB (ex: [r=0.862, g=0.118, b=0.118]) 
\usecolors[crayola] 			   % Paleta de cores pré-definida: wiki.contextgarden.net/Color#Pre-defined_colors

% Cores definidas pelo designer:
% MyGreen		r=0.251, g=0.678, b=0.290 % 40ad4a
% MyCyan		r=0.188, g=0.749, b=0.741 % 30bfbd
% MyRed			r=0.820, g=0.141, b=0.161 % d12429
% MyPink		r=0.980, g=0.780, b=0.761 % fac7c2
% MyGray		r=0.812, g=0.788, b=0.780 % cfc9c7
% MyOrange		r=0.980, g=0.671, b=0.290 % faab4a

% Configuração de cores
\definecolor[MyColor][x=361606]      % ou ex: [r=0.862, g=0.118, b=0.118] % corresponde a RGB(220, 30, 30)
\definecolor[MyColorText][white]     % ou ex: [r=0.862, g=0.118, b=0.118] % corresponde a RGB(167, 169, 172)

% Classe para diagramação dos posts
\environment{marketing.env}		   

\starttext %---------------------------------------------------------|

\Mensagem{DESTAQUE}

\MyCover{./TOLSTOI_MORTE_THUMB.jpeg}


\page %---------------------------------------------------------| 

\startMyCampaign

\hyphenpenalty=10000
\exhyphenpenalty=10000
\setupinterlinespace[line=1.5ex]
{\tfxx «Sim, havia vida, está partindo, partindo, e não tenho como a deter. Sim. Para que me enganar? Afinal, é evidente para todos, menos para mim, que vou morrer, tratando-se de uma questão apenas do número de semanas ou dias — pode até ser agora. Pois havia a luz, e agora são trevas. Pois eu estava aqui, e agora vou para lá! Para onde?»}

\stopMyCampaign

\scale[factor=7]{\Seta\,{\bf A morte de Ivan Ilitch}, Lev Tolstói}

\page %---------------------------------------------------------|

\MyPhoto{ILITCH_MORTE_IMAGE-2.jpg}

{\vfill\scale[factor=6]{\Seta\,{\bf LEV TOLSTÓI} (1828--1910)}}

\page %---------------------------------------------------------|

\Hedra

\stoptext %---------------------------------------------------------|