% Preencher com o nome das cor ou composição RGB (ex: [r=0.862, g=0.118, b=0.118]) 
\usecolors[crayola] 			   % Paleta de cores pré-definida: wiki.contextgarden.net/Color#Pre-defined_colors

% Cores definidas pelo designer:
% MyGreen		r=0.251, g=0.678, b=0.290 % 40ad4a
% MyCyan		r=0.188, g=0.749, b=0.741 % 30bfbd
% MyRed			r=0.820, g=0.141, b=0.161 % d12429
% MyPink		r=0.980, g=0.780, b=0.761 % fac7c2
% MyGray		r=0.812, g=0.788, b=0.780 % cfc9c7
% MyOrange		r=0.980, g=0.671, b=0.290 % faab4a

% Configuração de cores
\definecolor[MyColor][x=361606]      % ou ex: [r=0.862, g=0.118, b=0.118] % corresponde a RGB(220, 30, 30)
\definecolor[MyColorText][white]     % ou ex: [r=0.862, g=0.118, b=0.118] % corresponde a RGB(167, 169, 172)

% Classe para diagramação dos posts
\environment{marketing.env}		   

\starttext %---------------------------------------------------------|

\Mensagem{TOLSTÓI NO SÉCULO 21}

\startMyCampaign

\hyphenpenalty=10000
\exhyphenpenalty=10000

NOVA TRADUÇÃO
DE CLÁSSICO {\bf RENOVA
O REALISMO DE
TOLSTÓI} PARA
O NOSSO TEMPO

\stopMyCampaign

%\vfill\scale[lines=1.5]{\MyStar[MyColorText][none]}

\page %---------------------------------------------------------| 

\MyCover{./TOLSTOI_MORTE_THUMB.jpeg}

\page %---------------------------------------------------------| 

\hyphenpenalty=10000
\exhyphenpenalty=10000

Tradutor de {\bf CLÁSSICOS DA LITERATURA RUSSA}, como {\it O mestre e Margarida}, de Mikhail Bulgákov, {\it Memórias do subsolo}, de Fiódor Dostoiévski, e {\it Anna Kariênina}, de Tolstói, {\bf IRINEU FRANCO PERPETUO} acaba de verter para o português um clássico insuperável do realismo russo: {\it A morte de Ivan Ilitch}.

\page %---------------------------------------------------------|

\MyPhoto{irineu.png}

\vfill\scale[factor=6]{\Seta\,O tradutor Irineu Franco Perpetuo}

\page %---------------------------------------------------------|

\hyphenpenalty=10000
\exhyphenpenalty=10000

Ganhador do Prêmio Jabuti de Tradução em 2014 pela sua versão de {\it Vida e Destino}, de Vassili Grossman, Irineu Franco Perpetuo renova a leitura de {\bf A MORTE DE IVAN ILITCH} com uma tradução que, sem deixar de lado o rigor e o respeito ao original, evidencia em língua portuguesa a impressionante força da literatura de Tolstói nos nossos tempos.

\page %---------------------------------------------------------|

\Hedra

\stoptext %---------------------------------------------------------|