% Preencher com o nome das cor ou composição RGB (ex: [r=0.862, g=0.118, b=0.118]) 
\usecolors[crayola] 			   % Paleta de cores pré-definida: wiki.contextgarden.net/Color#Pre-defined_colors

% Cores definidas pelo designer:
% MyGreen		r=0.251, g=0.678, b=0.290 % 40ad4a
% MyCyan		r=0.188, g=0.749, b=0.741 % 30bfbd
% MyRed			r=0.820, g=0.141, b=0.161 % d12429
% MyPink		r=0.980, g=0.780, b=0.761 % fac7c2
% MyGray		r=0.812, g=0.788, b=0.780 % cfc9c7
% MyOrange		r=0.980, g=0.671, b=0.290 % faab4a

% Configuração de cores
\definecolor[MyColor][x=e95558]      % ou ex: [r=0.862, g=0.118, b=0.118] % corresponde a RGB(220, 30, 30)
\definecolor[MyColorText][Black]  % ou ex: [r=0.862, g=0.118, b=0.118] % corresponde a RGB(167, 169, 172)

% Classe para diagramação dos posts
\environment{marketing.env}		   

% Comandos & Instruções %%%%%%%%%%%%%%%%%%%%%%%%%%%%%%%%%%%%%%%%%%%%%%%%%%%%%%%%%%%%%%%|

% Cabeço e rodapé: Informações (caso queira trocar alguma coisa)
 		\def\MensagemSaibaMais 	{SAIBA MAIS:}
 		\def\MensagemSite		{HEDRA.COM.BR}
 		\def\MensagemLink		{LINK NA BIO}

\starttext %---------------------------------------------------------|

\Mensagem{PRÉ-VENDA}

\MyCover{./TOLSTOI_MORTE_THUMB.jpeg}

\vfill\scale[factor=6]{\Seta\,{\bf 30\% DE DESCONTO}} % Data entra somente após aprovação da Mayara

\page %---------------------------------------------------------|

\hyphenpenalty=10000
\exhyphenpenalty=10000

Publicada em 1886, a narrativa leve, mas nem por isso menos profunda, centra-se na história da vida e da morte do senhor Ivan Ilitch Golovin, promovendo para o leitor uma reflexão existencial, enquanto narra os caminhos da vida da personagem principal, de sua atividade junto ao judiciário, por meio da qual chega num ponto alto de sucesso, para então adoecer e sucumbir.
{\bf TRADUÇÃO INÉDITA DE IRINEU FRANCO PERPETUO}.

\page %---------------------------------------------------------|

\hyphenpenalty=10000
\exhyphenpenalty=10000

\MyPicture{ILITCH_MORTE_IMAGE-2}

\vfill
\scale[factor=5]{\Seta\,De R\$\,XXXXX {\bf por R\$\,XXXXXX}, 100 páginas, 1ª edição.}

\page %---------------------------------------------------------|

\Hedra

\stoptext %---------------------------------------------------------|