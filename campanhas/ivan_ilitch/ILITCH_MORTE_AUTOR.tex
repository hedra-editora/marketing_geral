\usecolors[crayola]

% Configuração de cores TickleMePink
\definecolor[MyColor][TickleMePink]      % ou ex: [r=0.862, g=0.118, b=0.118] % corresponde a RGB(220, 30, 30)
\definecolor[MyColorText][black]  % ou ex: [r=0.862, g=0.118, b=0.118] % corresponde a RGB(167, 169, 172)

% Classe para diagramação dos posts
\environment{marketing.env}        

% Cabeço e rodapé: Informações (caso queira trocar alguma coisa)
        \def\MensagemSaibaMais  {SAIBA MAIS:}
        \def\MensagemSite       {HEDRA.COM.BR}
        \def\MensagemLink       {LINK NA BIO}
      
\environment{extra.env}

\starttext  %---------------------------------------------------------|

\def\MyBackgroundMessage{TOLSTOÍSMO}
\MyBackground{./ILITCH_MORTE_IMAGE-1}

\startMyCampaign
\hyphenpenalty=10000
\exhyphenpenalty=10000

\position(0,7.8){\scale[factor=3]{\Seta\,{\bf LEV TOLSTÓI} (1828--1910)}}
\stopMyCampaign

\page %----------------------------------------------------------|

\Mensagem{TOLSTOÍSMO}

\setupbackgrounds[page][background=color,backgroundcolor=MyColor]

Pulvinar ante, a ultricies magna {\bf TRECHO EM DESTAQUE, MAS PODE HAVER MAIS DE UM}, 
mas sempre em negrito e caixa alta. Lorem ipsum dolor sit amet, consectetur adipiscing elit. Praesent sit amet pulvinar ante, a ultricies
magna. Etiam placerat quis tellus sed ultrices. Duis aliquet sed quam non
tincidunt. Donec sit amet tempor urna. Quisque auctor justo enim. Curabitur
vel est consectetur, sodales orci a, eleifend lacus. 

A década de 1880 aprofundou uma série de crises existenciais pelas quais o
escritor havia passado e o levou a uma fase que ele próprio definiu como
sua ``redenção moral''. Já praticante do vegetarianismo, ele abriu mão
dos direitos autorais de algumas obras em prol dos camponeses e
sistematizou uma série de preceitos filosóficos e religiosos que,
reunidos, passaram a ser conhecidos como \emph{tolstoísmo}, doutrina
baseada no cristianismo, mas acrescida de outras concepções, que
repercutiu no mundo todo e fez com que Tolstói fosse excomungado da
Igreja Ortodoxa.

\page %----------------------------------------------------------|

\MyCover{./Imagens/THUMB_LIVRO.pdf}

\page %----------------------------------------------------------|

\Hedra

\stoptext %---------------------------------------------------------|