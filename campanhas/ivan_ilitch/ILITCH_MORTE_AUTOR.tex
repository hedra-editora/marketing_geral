\usecolors[crayola]

% Configuração de cores TickleMePink
\definecolor[MyColor][x=361606]      % ou ex: [r=0.862, g=0.118, b=0.118] % corresponde a RGB(220, 30, 30)
\definecolor[MyColorText][white]  % ou ex: [r=0.862, g=0.118, b=0.118] % corresponde a RGB(167, 169, 172)

% Classe para diagramação dos posts
\environment{marketing.env}        

% Cabeço e rodapé: Informações (caso queira trocar alguma coisa)
        \def\MensagemSaibaMais  {SAIBA MAIS:}
        \def\MensagemSite       {HEDRA.COM.BR}
        \def\MensagemLink       {LINK NA BIO}
      
\environment{extra.env}

\starttext  %---------------------------------------------------------|

\def\MyBackgroundMessage{QUEM FOI}
\MyBackground{./ILITCH_MORTE_IMAGE-6}

\startMyCampaign
\hyphenpenalty=10000
\exhyphenpenalty=10000

\position(0,7.8){\scale[factor=3]{\Seta\,{\bf LEV TOLSTÓI} (1828--1910)}}
\stopMyCampaign

\page %----------------------------------------------------------|

\Mensagem{QUEM FOI}

\setupbackgrounds[page][background=color,backgroundcolor=MyColor]

Um dos mais importantes e influen-tes escritores do seu tempo, \mbox{\bf LEV TOLSTÓI} foi o principal representante do realismo russo.

\MyPicture{ILITCH_MORTE_IMAGE-7}

\page %----------------------------------------------------------|

Seu primeiro texto, {\it Infância}, saiu em 1852 na revista {\it O contemporâneo}. Após seu casamento com Sófia Andréievna em 1862, deu-se início a fase de seus longos romances, de {\it Guerra e paz} (1863--1869) até {\it Anna Kariênina} (1873--1878).


\page %----------------------------------------------------------|

 Seu último romance foi {\it Ressurreição} (1889). Embora seja considerado o mestre insuperado do gênero que se costumou chamar {\bf ROMANCE PSICOLÓGICO} do século 19, também aventurou-se por outros formatos, como contos breves, diários e escritos teóricos sobre pedagogia, arte e religião.

\page %----------------------------------------------------------|

\MyCover{./TOLSTOI_MORTE_THUMB.jpeg}

\page %----------------------------------------------------------|

\Hedra

\stoptext %---------------------------------------------------------|