% Preencher com o nome das cor ou composição RGB (ex: [r=0.862, g=0.118, b=0.118]) 
\usecolors[crayola] 			   % Paleta de cores pré-definida: wiki.contextgarden.net/Color#Pre-defined_colors

% Cores definidas pelo designer:
% MyGreen		r=0.251, g=0.678, b=0.290 % 40ad4a
% MyCyan		r=0.188, g=0.749, b=0.741 % 30bfbd
% MyRed			r=0.820, g=0.141, b=0.161 % d12429
% MyPink		r=0.980, g=0.780, b=0.761 % fac7c2
% MyGray		r=0.812, g=0.788, b=0.780 % cfc9c7
% MyOrange		r=0.980, g=0.671, b=0.290 % faab4a

% Configuração de cores
\definecolor[MyColor][x=361606]      % ou ex: [r=0.862, g=0.118, b=0.118] % corresponde a RGB(220, 30, 30)
\definecolor[MyColorText][white]     % ou ex: [r=0.862, g=0.118, b=0.118] % corresponde a RGB(167, 169, 172)

% Classe para diagramação dos posts
\environment{marketing.env}		   

\starttext %---------------------------------------------------------|

\Mensagem{FOFOCA LITERÁRIA}

\startMyCampaign

\hyphenpenalty=10000
\exhyphenpenalty=10000

{\bf QUANDO LOU SALOMÉ E RAINER MARIA RILKE
CONHECERAM LEV TOLSTÓI}

\stopMyCampaign

%\vfill\scale[lines=1.5]{\MyStar[MyColorText][none]}

\page %---------------------------------------------------------| 

\MyCover{./TOLSTOI_MORTE_THUMB.jpeg}

\page %---------------------------------------------------------| 

\hyphenpenalty=10000
\exhyphenpenalty=10000

Em 31 de maio de 1900, em uma viagem à Moscou, os escritores Rainer Maria Rilke e Lou Salomé encontram, por acaso, o pintor Leonid Pasternak, sua mulher e o filho, Boris, que depois escreveria {\it Doutor Jivago}. Por meio de Pasternak, o casal descobre que Tolstói estava em sua casa de campo, Iasnaia Poliana.

\page %---------------------------------------------------------|

\hyphenpenalty=10000
\exhyphenpenalty=10000

Rilke e Salomé, então, vão ao encontro do grande escritor, que os faz esperar por bastante tempo. Salomé assim o descreveu em seus escritos autobiográficos: «os olhos vivos brilhavam no meio de um rosto infeliz e como que atormentado, sua vivacidade parecendo para além de todas as coisas, afastada de todas as coisas. Ele parecia um pequeno camponês encantado, um ser mágico\ldots»

\page %---------------------------------------------------------|

Se para Lou Salomé o encontro foi festivo, como recorda o biógrafo Dorian Astor, não o foi para Rilke, que se sentiu atormentado e inseguro com sua identidade de poeta diante do «monstro sagrado da literatura russa».

\page %---------------------------------------------------------|

\Hedra

\stoptext %---------------------------------------------------------|