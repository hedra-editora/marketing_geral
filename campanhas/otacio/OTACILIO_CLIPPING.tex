% AUTOR_LIVRO_CLIPPING.tex
% Preencher com o nome das cor ou composição RGB (ex: [r=0.862, g=0.118, b=0.118]) 
\usecolors[crayola] 			   % Paleta de cores pré-definida: wiki.contextgarden.net/Color#Pre-defined_colors

% Cores definidas pelo designer:
% MyGreen		r=0.251, g=0.678, b=0.290 % 40ad4a
% MyCyan		r=0.188, g=0.749, b=0.741 % 30bfbd
% MyRed			r=0.820, g=0.141, b=0.161 % d12429
% MyPink		r=0.980, g=0.780, b=0.761 % fac7c2
% MyGray		r=0.812, g=0.788, b=0.780 % cfc9c7
% MyOrange		r=0.980, g=0.671, b=0.290 % faab4a

% Configuração de cores
\definecolor[MyColor][x=4cb7c4]      % ou ex: [r=0.862, g=0.118, b=0.118] % corresponde a RGB(220, 30, 30)
\definecolor[MyColorText][black]  % ou ex: [r=0.862, g=0.118, b=0.118] % corresponde a RGB(167, 169, 172)

% Classe para diagramação dos posts
\environment{marketing.env}		   

% Comandos & Instruções %%%%%%%%%%%%%%%%%%%%%%%%%%%%%%%%%%%%%%%%%%%%%%%%%%%%%%%%%%%%%%%|

% Cabeço e rodapé: Informações (caso queira trocar alguma coisa)
 		\def\MensagemSaibaMais 	{SAIBA MAIS:}
 		\def\MensagemSite		{HEDRA.COM.BR}
 		\def\MensagemLink		{LINK NA BIO}

\starttext %---------------------------------------------------------|

\Mensagem{NA IMPRENSA}

\MyPhoto{OTACILIO_CARTA} %Usar este tamanho de imagem


\page %---------------------------------------------------------|

\hyphenpenalty=10000
\exhyphenpenalty=10000

«Sagaz e carismático, {\bf OTACÍLIO} se destacava nas rimas dos versos poéticos, nas diferentes métricas, apresentadas em pelejas – os duelos entre repentistas. 
(\ldots)
Cada passagem no livro é marcada por contextualização social e política do País e do Nordeste, porque não só influenciou a vida dos cantadores, mas também a sua temática musical.»

{\vfill\scale[factor=5]
{\Seta\,	
	Trecho da resenha de {\bf Augusto Diniz},}
\setupinterlinespace[line=1.5ex]\scale[factor=5]{
   da Revista Carta Capital, em 30 de março.}
}

\page %---------------------------------------------------------|

\MyCover{OTACILIO_CAPA.pdf}

\page %---------------------------------------------------------|

\Hedra

\stoptext %---------------------------------------------------------|