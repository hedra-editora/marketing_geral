% AUTOR_LIVRO_CURIOSIDADES.tex
% Preencher com o nome das cor ou composição RGB (ex: [r=0.862, g=0.118, b=0.118]) 
\usecolors[crayola] 			   % Paleta de cores pré-definida: wiki.contextgarden.net/Color#Pre-defined_colors

% Cores definidas pelo designer:
% MyGreen		r=0.251, g=0.678, b=0.290 % 40ad4a
% MyCyan		r=0.188, g=0.749, b=0.741 % 30bfbd
% MyRed			r=0.820, g=0.141, b=0.161 % d12429
% MyPink		r=0.980, g=0.780, b=0.761 % fac7c2
% MyGray		r=0.812, g=0.788, b=0.780 % cfc9c7
% MyOrange		r=0.980, g=0.671, b=0.290 % faab4a

% Configuração de cores
\definecolor[MyColor][x=cfc9c7]      % ou ex: [r=0.862, g=0.118, b=0.118] % corresponde a RGB(220, 30, 30)
\definecolor[MyColorText][black]     % ou ex: [r=0.862, g=0.118, b=0.118] % corresponde a RGB(167, 169, 172)

% Classe para diagramação dos posts
\environment{marketing.env}		   

\starttext %---------------------------------------------------------|

\hyphenpenalty=10000
\exhyphenpenalty=10000

\Mensagem{EM CONTEXTO} %Sempre usar esse header

\startMyCampaign

\hyphenpenalty=10000
\exhyphenpenalty=10000

%\MyPicture{F1}

SETE FATOS QUE ANTECEDERAM O {\bf GOLPE MILITAR
DE 1964}

\stopMyCampaign



\page %---------------------------------------------------------| 
\MyPicture{F3}

{\bf 25/08/1961} \hfill
1/7


Jânio Quadros renuncia à Presidência da República.

\page %---------------------------------------------------------|
\MyPicture{F2}

\mbox{}\vfill
{\bf 07/09/1961 }\hfill
2/7\blank[1ex]


João Goulart assume a Presidência da República.


\page %---------------------------------------------------------|
{\bf 01/11/1961}\hfill 
3/7\blank[1ex]

\hyphenpenalty=10000
\exhyphenpenalty=10000
Na abertura do 1º Congresso de Lavradores e Trabalhadores
Agrícolas, Goulart defende “reformas de base”. As principais
medidas propostas eram: desapropriação de terras improdutivas, concessão das terras devolutas aos trabalhadores rurais, incentivo à produção cooperativa dentre outras medidas ligadas à
proposta de reforma agrária.

\page %---------------------------------------------------------|

\MyPicture{F5}

{\bf 16/02/1962}\hfill 
4/7\blank[1ex]

\hyphenpenalty=10000
\exhyphenpenalty=10000
O Governador do Rio Grande do Sul, Leonel Brizola, decreta a
desapropriação da subsidiária norte-americana International 
Telegraph and Telephone (ITT).


\page %---------------------------------------------------------|
\MyPicture{F6}


{\bf 26/06/1962}\hfill 
5/7\blank[1ex]

\hyphenpenalty=10000
\exhyphenpenalty=10000
O Primeiro-Ministro, Tancredo Neves, renuncia ao cargo. Todo
o gabinete ministerial se demite em virtude da pressão política
provocada por declarações de Goulart em favor de uma reforma
agrária.


\page %---------------------------------------------------------|
\MyPicture{F8}

{\bf 05/07/1962}\hfill 
6/7\blank[1ex]

\hyphenpenalty=10000
\exhyphenpenalty=10000
João Goulart sanciona a Lei do 13º salário. Deflagra-se a greve
nacional contra o aumento dos preços de bens de consumo.
A greve gera conflitos nas grandes capitais, resultando em 700
feridos e 42 mortos no Rio de Janeiro.

\page %---------------------------------------------------------|
\MyPicture{F7.png}
{\bf 12/09/1963}\hfill 
7/7\blank[1ex]

\hyphenpenalty=10000
\exhyphenpenalty=10000
Revolta dos Sargentos, no Rio de Janeiro e Brasília. A rebelião foi
motivada pela decisão do Supremo Tribunal Federal de reafirmar
a inelegibilidade de militares de baixa patente para cargos em
órgãos do Poder Legislativo.	

\page %---------------------------------------------------------|



\MyCover{1964_CAPA.jpg}

\page %---------------------------------------------------------|

\Hedra

\stoptext %---------------------------------------------------------|