% AUTOR_LIVRO_PREVENDA.tex
% Pré-venda
% "PRÉ-VENDA"

% Preencher com o nome das cor ou composição RGB (ex: [r=0.862, g=0.118, b=0.118]) 
\usecolors[crayola] 			   % Paleta de cores pré-definida: wiki.contextgarden.net/Color#Pre-defined_colors

% Cores definidas pelo designer:
% MyGreen		r=0.251, g=0.678, b=0.290 % 40ad4a
% MyCyan		r=0.188, g=0.749, b=0.741 % 30bfbd
% MyRed			r=0.820, g=0.141, b=0.161 % d12429
% MyPink		r=0.980, g=0.780, b=0.761 % fac7c2
% MyGray		r=0.812, g=0.788, b=0.780 % cfc9c7
% MyOrange		r=0.980, g=0.671, b=0.290 % faab4a

% Configuração de cores
\definecolor[MyColor][MaximumGreen]      % ou ex: [r=0.862, g=0.118, b=0.118] % corresponde a RGB(220, 30, 30)
\definecolor[MyColorText][white]  % ou ex: [r=0.862, g=0.118, b=0.118] % corresponde a RGB(167, 169, 172)

% Classe para diagramação dos posts
\environment{marketing.env}		   

% Comandos & Instruções %%%%%%%%%%%%%%%%%%%%%%%%%%%%%%%%%%%%%%%%%%%%%%%%%%%%%%%%%%%%%%%|

% Cabeço e rodapé: Informações (caso queira trocar alguma coisa)
 		\def\MensagemSaibaMais 	{SAIBA MAIS:}
 		\def\MensagemSite		{HEDRA.COM.BR}
 		\def\MensagemLink		{LINK NA BIO}

\starttext %---------------------------------------------------------|

\Mensagem{PRÉ-VENDA}

\startMyCampaign

{\bf A REBILIÃO DOS MANÉS}\\ 
\small de Pedro Arantes, \\Fernando Frias e \\Maria Luiza Meneses 

\vfill

\Seta 07.03.2024 - 31.03.2024

\stopMyCampaign

\page %---------------------------------------------------------|

\MyCover{ARANTES_MANES_THUMB.jpeg}

\vfill

Pŕe-venda com desconto: R\$ 39,90

\page %---------------------------------------------------------|

\hyphenpenalty=10000
\exhyphenpenalty=10000

{\bf 8/1: A rebelião dos manés} é uma análise política da história do tempo presente que problematiza uma inversão decisiva nas lutas sociais do Brasil: por que a direita se tornou ativista e audaz enquanto a esquerda novamente está refém do realismo político e da gestão comportada do sistema? 

\page

Ao investigarem os traços distintivos do ataque bolsonarista a Brasília em 8 de janeiro de 2023, os autores analisam como a extrema-direita incorporou os impulsos políticos, rebeldes e estéticos da esquerda, reconfigurando-os ao seu modo e em sentido golpista.  

\page %---------------------------------------------------------|

O livro apresenta uma análise ímpar da visualidade e do imaginário dos novos rebeldes, de suas tramas e desfaçatez política, bem como aponta caminhos ainda possíveis para uma esquerda que precisa retomar a imaginação coletiva, a crítica radical e a rebeldia insurgente a fim de alterar o curso da história em favor dos despossuídos.

\page

%\MyPicture{ARANTES_MANES_AUTOR1.png}

\tfx {\bf Pedro Arantes} é arquiteto e urbanista (FAU-USP),
professor de História da Arte na Unifesp, Campus Guarulhos. É autor de
livros e artigos sobre movimentos sociais, arte e política, guerras
culturais, direito à cidade, habitação popular e educação. 

\page %---------------------------------------------------------|

\tfx {\bf Fernando Frias} é graduado em História pela
Universidade de São Paulo (FFLCH-USP) e licenciado pela Faculdade de
Educação da USP. É mestrando em História da Arte pela EFLCH-Unifesp.
Integra o grupo de pesquisa MAAR (Mídias, Artes, Afetos e Resistência),
coordenado pela profa. Yanet Aguilera e o grupo de pesquisa e estudos em
guerras culturais, coordenado pelo prof. Pedro Arantes (ambos na
EFLCH-Unifesp).

\page

\tfx {\bf Maria Luiza Meneses} é graduanda em História da Arte (Unifesp).
É responsável pelos projetos Pinacoteca Digital Mauá (2019--) e Falando em
Arte (2020--). Foi curadora da exposição {\it Travessias do Moderno em
Mauá} (2022) e assistente pessoal da curadora Diane Lima, com ênfase em
pesquisa, produção e curadoria durante a 35ª Bienal de São Paulo
(2022--2023). Atua nos coletivos Rede Latino Americana de Estudantes de
História da Arte (RedLEHA), Nacional TROVOA e Rede Graffiteiras Negras
do Brasil.

\page

\Hedra

\stoptext %---------------------------------------------------------|
	