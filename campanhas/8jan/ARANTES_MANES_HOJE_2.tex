% Preencher com o nome das cor ou composição RGB (ex: [r=0.862, g=0.118, b=0.118]) 
\usecolors[crayola] 			   % Paleta de cores pré-definida: wiki.contextgarden.net/Color#Pre-defined_colors

% Cores definidas pelo designer:
% MyGreen		r=0.251, g=0.678, b=0.290 % 40ad4a
% MyCyan		r=0.188, g=0.749, b=0.741 % 30bfbd
% MyRed			r=0.820, g=0.141, b=0.161 % d12429
% MyPink		r=0.980, g=0.780, b=0.761 % fac7c2
% MyGray		r=0.812, g=0.788, b=0.780 % cfc9c7
% MyOrange		r=0.980, g=0.671, b=0.290 % faab4a

% Configuração de cores
\definecolor[MyColor][MaximumGreen]      % ou ex: [r=0.862, g=0.118, b=0.118] % corresponde a RGB(220, 30, 30)
\definecolor[MyColorText][white]  % ou ex: [r=0.862, g=0.118, b=0.118] % corresponde a RGB(167, 169, 172)

% Classe para diagramação dos posts
\environment{marketing.env}		   

% Comandos & Instruções %%%%%%%%%%%%%%%%%%%%%%%%%%%%%%%%%%%%%%%%%%%%%%%%%%%%%%%%%%%%%%%|

% Cabeço e rodapé: Informações (caso queira trocar alguma coisa)
 		\def\MensagemSaibaMais 	{SAIBA MAIS:}
 		\def\MensagemSite		{HEDRA.COM.BR}
 		\def\MensagemLink		{LINK NA BIO}

\starttext %---------------------------------------------------------|

\Mensagem{LANÇAMENTO NA USP}

\startMyCampaign

{É HOJE}\setupinterlinespace[line=1ex]\\
{\bf 8/1: A REBELIÃO DOS MANÉS}
\stopMyCampaign

\vfill

% Colocar sempre imagens quadradas
\scale[factor=fit]{\MyPortrait{ARANTES_MANES_AUTOR-PEDROQUADRADO}\quad\MyPortrait{ARANTES_MANES_AUTOR-FERNANDOQUADRADO}\quad\MyPortrait{ARANTES_MANES_AUTOR-MARIAQUADRADO}\quad\MyPortrait{ARANTES_MANES_CONVIDADO_SAFATLE}}


{\vfill

{\scale[factor=6]{\Seta\,{\bf CONVIDADOS} Pedro Arantes, Fernando}
\setupinterlinespace[line=1.7ex]\scale[factor=6]{Frias, Maria L. Meneses e Vladimir Safatle}}
\setupinterlinespace[line=2ex]\scale[factor=6]{\Seta\,{\bf 12 DE ABRIL 18H} na {\cap FFLCH-USP}}




\vfill

% Colocar sempre imagens quadradas

\page %---------------------------------------------------------|

\hyphenpenalty=10000
\exhyphenpenalty=10000

Evento de lançamento do livro {\bf 8/1: A REBELIÃO DOS MANÉS} terá a participação especial do filósofo {\bf VLADIMIR SAFATLE}, que conversa com os autores sobre os temas do livro.

{\vfill\scale[factor=5]{\Seta\,O debate acontece na {\bf sala 14} do prédio da}\setupinterlinespace[line=1.7ex]\scale[factor=5]{Filosofia/Ciências Sociais, na {\bf {\cap FFLCH-USP}}, que }
\setupinterlinespace[line=1.7ex]\scale[factor=5]{fica na Av. Prof. Luciano Gualberto, 315, SP.}}

\page %---------------------------------------------------------|


\MyCover{ARANTES_MANES_THUMB}
\vfill

\Seta\,{\bf 50\%} de desconto

\page %---------------------------------------------------------|

\setcharacterkerning[reset]\setupbackgrounds[header][background=MyGuidesCleaned]
\Times
\switchtobodyfont[50pt]
\mbox{}\blank[10mm]
\hfill hedra \hfill \mbox{}  

\vfill
\FormularMImedium\bold
\switchtobodyfont[12pt]
\setcharacterkerning[packed]

\position(-1,9.25mm){\blackrule[width=1.2\paperwidth, height=.6pt, color=MyColorText]}

\starttikzpicture[remember picture,overlay]
\node at (3,-.8)
{
\startcombination[4*1]
{\externalfigure[fflch-pb.png][width=.1\textwidth]}{}
{\externalfigure[ppgfil-pb.png][width=.1\textwidth]}{}
{\externalfigure[capes.png][width=.15\textwidth]}{}
{\externalfigure[usp-pb.png][width=.15\textwidth]}{}
\stopcombination
};
\stoptikzpicture


\stoptext %---------------------------------------------------------|