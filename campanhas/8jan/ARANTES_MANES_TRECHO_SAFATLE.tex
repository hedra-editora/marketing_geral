% AUTOR_LIVRO_TRECHO.tex
% Preencher com o nome das cor ou composição RGB (ex: [r=0.862, g=0.118, b=0.118]) 
\usecolors[crayola] 			   % Paleta de cores pré-definida: wiki.contextgarden.net/Color#Pre-defined_colors

% Cores definidas pelo designer:
% MyGreen		r=0.251, g=0.678, b=0.290 % 40ad4a
% MyCyan		r=0.188, g=0.749, b=0.741 % 30bfbd
% MyRed			r=0.820, g=0.141, b=0.161 % d12429
% MyPink		r=0.980, g=0.780, b=0.761 % fac7c2
% MyGray		r=0.812, g=0.788, b=0.780 % cfc9c7
% MyOrange		r=0.980, g=0.671, b=0.290 % faab4a

% Configuração de cores
\definecolor[MyColor][MaximumGreen]      % ou ex: [r=0.862, g=0.118, b=0.118] % corresponde a RGB(220, 30, 30)
\definecolor[MyColorText][white]     % ou ex: [r=0.862, g=0.118, b=0.118] % corresponde a RGB(167, 169, 172)

% Classe para diagramação dos posts
\environment{marketing.env}		   

\starttext %---------------------------------------------------------|

\Mensagem{DESTAQUE}

\startMyCampaign

\hyphenpenalty=10000
\exhyphenpenalty=10000

%\startcombination[2*1]
%{\externalfigure[ARANTES_MANES_IMPRENSA_FOLHA.png][width=.45\textwidth]}{}
%{\externalfigure[ARANTES_MANES_IMPRENSA_FOLHA4.png][width=.5\textwidth]}{}
%\stopcombination
%\MyPhoto{ARANTES_MANES_THUMB.jpeg}
\placefigure[top,right]{}{\scale[factor=30]{\MyPortrait{ARANTES_MANES_CONVIDADO_SAFATLE}}}

%\startplacefigure[location={top,left},title={}]
%\externalfigure[ARANTES_MANES_THUMB.jpeg][width=5mm]
%\stopplacefigure

%\startplacefigure[location={bottom,right},title={}]
%\externalfigure[ARANTES_MANES_CONVIDADO_SAFATLE][width=5mm]
%\stopplacefigure

\setupinterlinespace[line=1.3ex]\tfxx «Esse livro é muito bem vindo, não só para decifrar a constituição do imaginário da extrema direita, mas ele aponta também para um problema estrutural para podermos pensar o presente: a decomposição profunda da esquerda como força insurgente, como força insurrecional, e a transformação e a passagem dessa força à extrema direita.»

%SAFATLE: Esse livro é muito bem vindo, com sua pesquisa minuciosa, preciosa, não só para  decifrar a constituição do imaginário da extrema direita, seus transformismos, por uma certa apropriação de imagens, de discursos, de práticas que eram próprias da esquerda, mas ele aponta também para um problema que se consolidou entre nós e que se coloca como um problema estrutural para podemos pensar o presente: a decomposição profunda da esquerda como força insurgente, como força insurrecional, e a transformação e a passagem dessa força à extrema direita. É uma chave importante de análise do processo, o livro trás exemplos muito significativos, muito fortes, não só do ponto de vista da comunicação, de uma mutação na comunicação, mas trata-se efetivamente de uma mutação na posição da direita brasileira, que foi fagocitada pelo seu extremo.

%SAFATLE: Este livro fornece subsídios preciosos ao fazer uma análise ao mesmo tempo política e estética. A integração da análise estética dentro do quadro é um ganho muito interessante. Há materiais aqui que que eu desconhecia. E são materiais que mostram muito claramente não só essa tentativa de fagocitar um imaginário insurgente, como se fosse simplesmente uma espécie de estratégia de comunicação, mas de se reconhecer nesse imaginário! De se ver e de se autocompreender como fazendo parte de uma revolução. Seja ela conservadora, mas uma revolução.

\stopMyCampaign

{\vfill\scale[factor=6]{\Seta\,Trecho da fala de {\bf Vladimir Safatle}, no evento}\setupinterlinespace[line=1.5ex]\scale[factor=6]{de lançamento realizado em 12 de abril, na}\setupinterlinespace[line=1.5ex]\scale[factor=6]{FFLCH-USP.}}

\page %---------------------------------------------------------| 

\MyCover{ARANTES_MANES_THUMB.jpeg}

\page %---------------------------------------------------------|

\Hedra

\stoptext %---------------------------------------------------------|