% ARANTES_MANES_EDICAO.tex
% Preencher com o nome das cor ou composição RGB (ex: [r=0.862, g=0.118, b=0.118]) 
\usecolors[crayola] 			   % Paleta de cores pré-definida: wiki.contextgarden.net/Color#Pre-defined_colors

% Cores definidas pelo designer:
% MyGreen		r=0.251, g=0.678, b=0.290 % 40ad4a
% MyCyan		r=0.188, g=0.749, b=0.741 % 30bfbd
% MyRed			r=0.820, g=0.141, b=0.161 % d12429
% MyPink		r=0.980, g=0.780, b=0.761 % fac7c2
% MyGray		r=0.812, g=0.788, b=0.780 % cfc9c7
% MyOrange		r=0.980, g=0.671, b=0.290 % faab4a

% Configuração de cores
\definecolor[MyColor][MaximumGreen]      % ou ex: [r=0.862, g=0.118, b=0.118] % corresponde a RGB(220, 30, 30)
\definecolor[MyColorText][white]  % ou ex: [r=0.862, g=0.118, b=0.118] % corresponde a RGB(167, 169, 172)

% Classe para diagramação dos posts
\environment{marketing.env}		   
\starttext %---------------------------------------------------------|

\Mensagem{LANÇAMENTO: «8/1»}

\startMyCampaign

\hyphenpenalty=10000
\exhyphenpenalty=10000
Conheça
uma análise ímpar 
do imaginário dos 
{\bf 
NOVOS 
REBELDES}
de direita

\stopMyCampaign

%\vfill\scale[lines=1.5]{\MyStar[MyColorText][none]}

\page %---------------------------------------------------------| 

\MyCover{ARANTES_MANES_THUMB}

\page %---------------------------------------------------------| 

\hyphenpenalty=10000
\exhyphenpenalty=10000


{\bf 8/1: A REBELIÃO DOS MANÉS} problematiza uma {\bf INVERSÃO} decisiva nas lutas sociais do Brasil: por que a {\bf DIREITA} se tornou ativista e audaz enquanto a {\bf ESQUERDA} novamente está refém do realismo político e da gestão comportada do sistema? 

\page

Além de apresentar uma análise ímpar da visualidade e do imaginário dos {\bf NOVOS REBELDES}, o livro aponta caminhos possíveis para uma {\bf ESQUERDA} que precisa retomar a imaginação coletiva, a crítica radical e a rebeldia insurgente.

\page 
\MyPicture{001.jpeg}

\vfill
\scale[factor=fit]{STF durante a invasão com “Perdel Mané” grafado nas vidraças, Foto: Renato Guariba}
\page

\hyphenpenalty=10000
\exhyphenpenalty=10000

«Entre gritos de cólera, frenesi destrutivo, {\it selfies} 
autoincriminatórios, pedidos de {\bf INTERVENÇÃO MILITAR}, facadas e
 pauladas em obras de arte, defecação simbólica, focos de incêndio etc.,
 destacava-se a inscrição nas vidraças do Supremo Tribunal Federal ({\cap STF}) e na escultura da Justiça: “{\bf PERDEU MANÉ}”.»
\vfill

\setupinterlinespace[line=1.5ex]\scale[factor=5]{\bf 8/1: A REBELIÃO DOS MANÉS}

\page

\Hedra

\stoptext %---------------------------------------------------------|




