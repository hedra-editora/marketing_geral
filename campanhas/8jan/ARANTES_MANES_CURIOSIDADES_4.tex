% ARANTES_MANES_CURIOSIDADES_4.tex
% Preencher com o nome das cor ou composição RGB (ex: [r=0.862, g=0.118, b=0.118]) 
\usecolors[crayola] 			   % Paleta de cores pré-definida: wiki.contextgarden.net/Color#Pre-defined_colors

% Cores definidas pelo designer:
% MyGreen		r=0.251, g=0.678, b=0.290 % 40ad4a
% MyCyan		r=0.188, g=0.749, b=0.741 % 30bfbd
% MyRed			r=0.820, g=0.141, b=0.161 % d12429
% MyPink		r=0.980, g=0.780, b=0.761 % fac7c2
% MyGray		r=0.812, g=0.788, b=0.780 % cfc9c7
% MyOrange		r=0.980, g=0.671, b=0.290 % faab4a

% Configuração de cores
\definecolor[MyColor][MaximumGreen]      % ou ex: [r=0.862, g=0.118, b=0.118] % corresponde a RGB(220, 30, 30)
\definecolor[MyColorText][white]  % ou ex: [r=0.862, g=0.118, b=0.118] % corresponde a RGB(167, 169, 172)

% Classe para diagramação dos posts
\environment{marketing.env}		   

\starttext %---------------------------------------------------------|

\hyphenpenalty=10000
\exhyphenpenalty=10000

\Mensagem{EM CONTEXTO} %Sempre usar esse header

\startMyCampaign

\hyphenpenalty=10000
\exhyphenpenalty=10000

A tentativa de {\bf REESCRITA}
do 8/1: manés {\bf INVADEM} a Wikipédia
 

\stopMyCampaign

\page %---------------------------------------------------------| 

\hyphenpenalty=10000
\exhyphenpenalty=10000

Apoiadores de Bolsonaro tentaram {\bf REESCREVER} a história sobre a empreitada golpista do 8 de janeiro na página da {\bf WIKIPÉDIA}. 

\page

Usuários não registrados fizeram mudanças no texto relativizando os ataques às sedes dos {\bf TRÊS PODERES}. Eles acusavam que {\bf “ESQUERDISTAS INFILTRADOS”} teriam incitado os protestos e que tudo o que ocorreu naquela data seria uma {\bf “MANIFESTAÇÃO DA VONTADE POPULAR LEGÍTIMA”}.

% \page

% O {\bf VERBETE} foi criado enquanto os atos começavam a se desenvolver em Brasília e foi {\bf EDITADO} mais de 500 vezes apenas em um dia.

% \page

% Cerca de 20 minutos depois da sua criação, um usuário trocou o termo {\bf “TERRORISTAS BOLSONARISTAS”} por “manifestantes”, o que foi {\bf REVERTIDO} poucos minutos depois.

% \page 

% Quase um ano depois, esse mesmo {\bf USUÁRIO} substituiu “manifestações golpistas no Brasil após as eleições de 2022” por “manifestações da vontade popular legítima fruto de uma democracia instituída pela própria {\bf CONSTITUIÇÃO FEDERAL DO BRASIL} após as eleições de 2022”.

\page

O volume de edições na página era tão grande que essas foram {\bf RESTRINGIDAS}.
A Wikipédia tornou-se um verdadeiro {\bf CABO DE GUERRA}, em que estava em jogo a verdade sobre os acontecimentos do fatídio 8 de janeiro.


\page %---------------------------------------------------------|

\MyCover{ARANTES_MANES_THUMB}

\page %---------------------------------------------------------|

\Hedra

\stoptext %---------------------------------------------------------|