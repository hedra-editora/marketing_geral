% ARANTES_MANES_CURIOSIDADES_3.tex
% Preencher com o nome das cor ou composição RGB (ex: [r=0.862, g=0.118, b=0.118]) 
\usecolors[crayola] 			   % Paleta de cores pré-definida: wiki.contextgarden.net/Color#Pre-defined_colors

% Cores definidas pelo designer:
% MyGreen		r=0.251, g=0.678, b=0.290 % 40ad4a
% MyCyan		r=0.188, g=0.749, b=0.741 % 30bfbd
% MyRed			r=0.820, g=0.141, b=0.161 % d12429
% MyPink		r=0.980, g=0.780, b=0.761 % fac7c2
% MyGray		r=0.812, g=0.788, b=0.780 % cfc9c7
% MyOrange		r=0.980, g=0.671, b=0.290 % faab4a

% Configuração de cores
\definecolor[MyColor][MaximumGreen]      % ou ex: [r=0.862, g=0.118, b=0.118] % corresponde a RGB(220, 30, 30)
\definecolor[MyColorText][white]  % ou ex: [r=0.862, g=0.118, b=0.118] % corresponde a RGB(167, 169, 172)

% Classe para diagramação dos posts
\environment{marketing.env}		   

\starttext %---------------------------------------------------------|

\hyphenpenalty=10000
\exhyphenpenalty=10000

\Mensagem{EM CONTEXTO} %Sempre usar esse header

\startMyCampaign

\hyphenpenalty=10000
\exhyphenpenalty=10000

a {\bf DIREITA BRASILEIRA}: da 
{\bf DITADURA MILITAR} à invasão
bolsonarista 
do 8/1

\stopMyCampaign

\page %---------------------------------------------------------| 

\hyphenpenalty=10000
\exhyphenpenalty=10000

% A tomada e a destruição dos palácios de Brasília seriam grandes {\bf FEITOS REVOLUCIONÁRIOS}, mas nunca integraram o imaginário das esquerdas brasileiras. Mesmo nos 21 anos de ditadura entrincheirada em Brasília, não houve nenhum atentado relevante, muito menos ameaça de tomada de palácios.

\page

Com a tomada dos palácios de Brasília pelos {\bf “PATRIOTAS”} em fúria, no 8 de janeiro de 2023, percebemos que algo está {\bf INVERTIDO} no espelho político e ativista da esquerda e da direita. 

\page

Afinal, a tomada de palácios por massas de civis em {\bf REBELIÃO} não era uma forma histórica de {\bf INSURREIÇÃO DE ESQUERDAS}, comunistas, operários e camponeses derrubando monarquias e ditaduras?

\page


A {\bf DITADURA MILITAR} (1964--1985) considerava terroristas a insurgência e a guerrilha. 

\page

Na nova quadra da história, a {\bf ESQUERDA} institucional abandona o  horizonte revolucionário e passa ao posto de principal {\bf GUARDIÃ DA ORDEM}, do Estado de Direito e da democracia liberal burguesa, adotando os aparatos de repressão e tecnologias de identificação facial em massa para enquadrar rebeldes.


\page

 O {\bf GOVERNO LULA} também apresentou nova legislação para aumento de penas e novas tipificações nos “crimes contra o Estado democrático de Direito”, com penas de até 40 anos de prisão ({\cap PL} 3611/2023).

\page

Se a esquerda não retomar a imaginação coletiva, a {\bf CRÍTICA RADICAL} e a rebeldia insurgente, a vanguarda reacionária seguirá ocupando as ruas, comandando os negócios e a vida política real, do Congresso às periferias — e o que será de nós?



% Percebemos essa inversão com um paradoxo, que surge na transição do {\bf REGIME MILITAR} para a {\bf NOVA REPÚBLICA}: como a {\bf ESQUERDA}, mesmo derrotada militarmente pela ditadura, saiu {\bf VITORIOSA} do ponto de vista político, moral e simbólico?

% \page 

% % {\bf OLAVO DE CARVALHO} estava convencido de que havia uma ação organizada e deliberada da esquerda em torno do {\bf GRAMSCISMO} (ou marxismo cultural). Acreditava num grau tal de centralismo e orquestração que a sua hipótese tornava-se uma teoria da conspiração, mas teve faro político e percebeu um movimento real na {\bf SOCIEDADE BRASILEIRA}.

% \page

% Na narrativa da direita, o {\bf AVANÇO PROGRESSISTA} se comprovaria com seis eleições presidenciais sucessivas --- 22 anos de {\bf PRESIDÊNCIAS DE ESQUERDA}, ou seja, mais tempo que os 21 anos da ditadura.

% \page


% {\bf OLAVO DE CARVALHO} defende que {\bf NOVA DIREITA} reacionária precisaria dar um basta a esse longo {\bf CICLO DE HEGEMONIA} política e cultural progressista e liberal. Realizar ascensão similar, mas às avessas, e promover uma {\bf REVOLUÇÃO CULTURAL} a partir de seus valores. 

% \page %---------------------------------------------------------|

% % «Em 1964, a Marcha com Deus pela Família mobilizou grandes contingentes
% % de conservadores, reacionários e fascistas pelas ruas brasileiras,
% % desempenhando um relevante papel simbólico e midiático na época. No
% % entanto, a efetiva ação que desencadeou o golpe e a subsequente
% % instauração de um regime ditatorial no país foi conduzida
% % primordialmente por militares e certos setores do Parlamento. Não houve
% % um envolvimento direto e decisivo de ativistas ou grupos paramilitares
% % civis de direita na tomada de poder, que instaurou 21 anos de ditadura.»

 \page
 \MyCover{ARANTES_MANES_THUMB}

 \page %---------------------------------------------------------|

 \Hedra

 \stoptext %------------------------------------------------------