% ARANTES_MANÉS_CURIOSIDADES.tex
% Preencher com o nome das cor ou composição RGB (ex: [r=0.862, g=0.118, b=0.118]) 
\usecolors[crayola] 			   % Paleta de cores pré-definida: wiki.contextgarden.net/Color#Pre-defined_colors

% Cores definidas pelo designer:
% MyGreen		r=0.251, g=0.678, b=0.290 % 40ad4a
% MyCyan		r=0.188, g=0.749, b=0.741 % 30bfbd
% MyRed			r=0.820, g=0.141, b=0.161 % d12429
% MyPink		r=0.980, g=0.780, b=0.761 % fac7c2
% MyGray		r=0.812, g=0.788, b=0.780 % cfc9c7
% MyOrange		r=0.980, g=0.671, b=0.290 % faab4a

% Configuração de cores
\definecolor[MyColor][MaximumGreen]      % ou ex: [r=0.862, g=0.118, b=0.118] % corresponde a RGB(220, 30, 30)
\definecolor[MyColorText][white]  % ou ex: [r=0.862, g=0.118, b=0.118] % corresponde a RGB(167, 169, 172)

% Classe para diagramação dos posts
\environment{marketing.env}		   


\starttext %---------------------------------------------------------|

\hyphenpenalty=10000
\exhyphenpenalty=10000

\Mensagem{EM CONTEXTO} %Sempre usar esse header

\startMyCampaign

\hyphenpenalty=10000
\exhyphenpenalty=10000


Relembre {\bf PERSONAGENS ICÔNICOS}

do 8/1 %Aqui a manchete pode ser mais longa

\stopMyCampaign

\page %---------------------------------------------------------| 

\hyphenpenalty=10000
\exhyphenpenalty=10000



\MyPicture{ARANTES_MANES_CURIOSIDADES_1}
\vfill
William da Silva Lima, o “{\bf GOLPISTA TOGADO}” que roubou toga do {\cap STF} durante a invasão do 8/1

\page

\MyPicture{ARANTES_MANES_CURIOSIDADES_3}

Fábio Alexandre Oliveira que sentou na cadeira do ministro {\bf ALEXANDRE DE MORAES}


\page

\MyPicture{ARANTES_MANES_CURIOSIDADES_2}

Antônio Claudio Alves Ferreira, que apareceu nas filmagens do Palácio do Planalto destruindo um relógio que pertenceu a {\bf DOM JOÃO VI}

\page

\MyPicture{ARANTES_MANES_CURIOSIDADES_4}

Débora Rodrigues dos Santos, que foi flagrada escrevendo a fase “{\bf PERDEU, MANÉ}“ na estátua da Justiça em frente ao prédio do {\cap STF}.

\page %---------------------------------------------------------|

\MyCover{ARANTES_MANES_THUMB}

\page %---------------------------------------------------------|

\Hedra

\stoptext %---------------------------------------------------------|