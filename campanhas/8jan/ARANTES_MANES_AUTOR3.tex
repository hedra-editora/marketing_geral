% AUTOR_LIVRO_AUTOR.tex
% Preencher com o nome das cor ou composição RGB (ex: [r=0.862, g=0.118, b=0.118]) 
\usecolors[crayola]                            % Paleta de cores pré-definida: wiki.contextgarden.net/Color#Pre-defined_colors

% Cores definidas pelo designer:
% MyGreen                r=0.251, g=0.678, b=0.290 % 40ad4a
% MyCyan                r=0.188, g=0.749, b=0.741 % 30bfbd
% MyRed                        r=0.820, g=0.141, b=0.161 % d12429
% MyPink                r=0.980, g=0.780, b=0.761 % fac7c2
% MyGray                r=0.812, g=0.788, b=0.780 % cfc9c7
% MyOrange                r=0.980, g=0.671, b=0.290 % faab4a

% Configuração de cores
\definecolor[MyColor][x=c0e016]          [r=0.862, g=0.118, b=0.118]      % ou ex: [r=0.862, g=0.118, b=0.118] % corresponde a RGB(220, 30, 30)
\definecolor[MyColorText][black] [r=0.655, g=0.663, b=0.675]      % ou ex: [r=0.862, g=0.118, b=0.118] % corresponde a RGB(167, 169, 172)

% Classe para diagramação dos posts
\environment{marketing.env}                   

% Cabeço e rodapé: Informações (caso queira trocar alguma coisa)
                 \def\MensagemSaibaMais  {SAIBA MAIS:}
                 \def\MensagemSite                {HEDRA.COM.BR}
                 \def\MensagemLink       {LINK NA BIO}

\starttext %--------------------------------------------------------|

\Mensagem{SOBRE O AUTOR}

\hyphenpenalty=10000
\exhyphenpenalty=10000

%\startMyCampaign

\MyPicture{ARANTES_MANES_AUTOR6}

%\stopMyCampaign

\vfill\scale[factor=6]{\Seta\,MARIA LUIZA MENEZES}

\page %----------------------------------------------------------|

\hyphenpenalty=10000
\exhyphenpenalty=10000

Graduanda em História da Arte (Unifesp), é responsável pelos projetos Pinacoteca Digital Mauá (2019--) e Falando em Arte (2020--). Foi curadora da exposição {\it Travessias do Moderno em Mauá} (2022) e assistente pessoal da curadora Diane Lima, com ênfase em
pesquisa, produção e curadoria durante a 35ª Bienal de São Paulo (2022--2023). Atua nos coletivos Rede Latino Americana de Estudantes de História da Arte (Red{\cap LEHA}), Nacional {\cap TROVOA} e Rede Graffiteiras Negras do Brasil.

\page %----------------------------------------------------------|

\MyCover{ARANTES_MANES_THUMB}

\page %----------------------------------------------------------|

\Hedra

\stoptext %---------------------------------------------------------|