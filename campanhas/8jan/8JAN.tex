% Preencher com o nome das cor ou composição RGB (ex: [r=0.862, g=0.118, b=0.118]) 
\usecolors[crayola] 			   % Paleta de cores pré-definida: wiki.contextgarden.net/Color#Pre-defined_colors

% Cores definidas pelo designer:
% MyGreen		r=0.251, g=0.678, b=0.290 % 40ad4a
% MyCyan		r=0.188, g=0.749, b=0.741 % 30bfbd
% MyRed			r=0.820, g=0.141, b=0.161 % d12429
% MyPink		r=0.980, g=0.780, b=0.761 % fac7c2
% MyGray		r=0.812, g=0.788, b=0.780 % cfc9c7
% MyOrange		r=0.980, g=0.671, b=0.290 % faab4a

% Configuração de cores
\definecolor[MyColor][MaximumGreen]      % ou ex: [r=0.862, g=0.118, b=0.118] % corresponde a RGB(220, 30, 30)
\definecolor[MyColorText][white]  % ou ex: [r=0.862, g=0.118, b=0.118] % corresponde a RGB(167, 169, 172)

% Classe para diagramação dos posts
\environment{marketing.env}		   


% Comandos & Instruções %%%%%%%%%%%%%%%%%%%%%%%%%%%%%%%%%%%%%%%%%%%%%%%%%%%%%%%%%%%%%%%|

% Pesos para os títulos:
%		\startMyCampaign...		 \stopMyCampaign
%		\stopMyCampaignSection...   \stopMyCampaignSection
% Cabeço e rodabé: Opções
% 		\Mensagem{AGORA É QUE SÃO ELAS}
% 		\Hashtag{campanha de natal}
% 		\Mensagem{campanha de natal}
% Aplicação de imagens: 
% 		\MyCover{capa.pdf}  	% Aplicação de capa de livro com sombra
%		\MyPicture{Imagem.png}  % Imagem com aplicação de filtro segundo cor MyColorText
%		\MyPhoto{}			    % Aplicação simples de imagem com tamamho \textwidth

% Logos e selos: 				
% \Hedra
% \HedraAyllon	% Não está pronto
% \HedraAcorde	% Não está pronto
% \Ayllon		% Não está pronto
% \Acorde		% Não está pronto

% Atalhos: 						
% 		\Seta  % Seta para baixo

% Aplicação de imagem com legenda:		
% 		\placefigure{Legenda}{\externalfigure[drop2-1.png][width=\textwidth]}

% Cabeço e rodabé: Informações (caso queira trocar alguma coisa)
% 		\def\MensagemSaibaMais{SAIBA MAIS:}
% 		\def\MensagemSite{HEDRA.COM.BR}
% 		\def\MensagemLink{LINK NA BIO}

% Alteração de várias cores de background:
% \setupbackgrounds[page][background=color,backgroundcolor=MyGray]

% Estrela: 
% \vfill\scale[lines=2]{\MyStar[MyColorText][none]} 					% Estrela pequena  
% \startpositioning 											% Estrela grande
%  \position(-1,-.3){\scale[scale=980]{\MyStar[white][none]}}
% \stoppositioning

%%%%%%%%%%%%%%%%%%%%%%%%%%%%%%%%%%%%%%%%%%%%%%%%%%%%%%%%%%%%%%%%%%%%%%%%%%%%%%%%%%%%%%%|

\starttext
%\showframe  %Para mostrar somente as linhas.

\hyphenpenalty=10000
\exhyphenpenalty=10000

 \startMyCampaign
 

 {\bf A REBELIÃO 

 DOS  MANÉS}

 \Seta EM BREVE

 \stopMyCampaign
 
 \vfill\scale[lines=2]{\MyStar[MyColorText][none]} 

 \page %---------------------------------------------------------|

\Mensagem{lançamento}

\MyCover{capa}

\page

\MyPicture{001}

\page

\MyPicture{002}


\page

O livro {\bf A REBELIÃO DOS MANÉS}, a ser lançado em breve, analisa como a extrema-direita brasileira incorporou com muita habilidade
os impulsos políticos, rebeldes e estéticos da esquerda, reconfigurando-os ao
seu modo e em sentido golpista.

\vfill

\scale[factor=fit]{\tfxx {\bf Pedro Arantes, Fernando Frias e Maria L.\ Meneses} (org.)}

\page

%\Mensagem{A revolução dos manés}

«O 8 de janeiro é um caleidoscópio de forças e vertigens, em imagens e atos,
apresentando de forma fugaz e impactante dilemas e fraturas do Brasil atual.
Daquele espetáculo, entre o grotesco e o surpreendente, há importantes
aprendizados sobre o que motiva e o que impulsiona a indignação e a capacidade
de assumir riscos em defesa de causas e crenças para transformar o rumo da
história.»

\page

Como pode a “direita ordeira” ter feito este {\bf QUEBRA-QUEBRA}? “Patriotas”, 
cidadãos de bem, “manés” seriam capazes de
atos tão violentos? Os {\bf DEFENSORES DA ORDEM}, da tradição, da
família e da propriedade não seriam os agentes de tanta violência. A resposta
imedita: «Esquerdistas infiltrados.»

\page 

«As cenas filmadas em tempo real, e depois coletadas a partir das redes e
celulares dos extremistas, inundaram a mídia no mesmo dia e nas semanas
seguintes, produzindo não apenas a inversão no espelho da rebeldia insurgente
esquerda-direita, mas o seu estilhaçamento para uma parcela da direita.»

\page

\Hedra

\stoptext

% POST 8/1:
% 
% Thumb em png (jpg)
% 
% Chamada: "8/1: a rebelião dos manés" analisa como a extrema-direita incorporou
% os impulsos políticos, rebeldes e estéticos da esquerda, reconfigurando-os ao
% seu modo e em sentido golpista.
% 
% Destaques: "O 8 de janeiro é um caleidoscópio de forças e vertigens, em
% imagens e atos, apresentando de forma fugaz e impactante dilemas e fraturas do
% Brasil atual. Daquele espetáculo, entre o grotesco e o surpreendente, há
% importantes aprendizados sobre o que motiva e o que impulsiona a indignação e
% a capacidade de assumir riscos em defesa de causas e crenças para transformar
% o rumo da história (32)."
% 
% "As cenas filmadas em tempo real, e depois coletadas a partir das redes e
% celulares dos extremistas, inundaram a mídia no mesmo dia e nas semanas
% seguintes, produzindo não apenas a inversão no espelho da rebeldia insurgente
% esquerda-direita, mas o seu estilhaçamento para uma parcela da direita.""
% 
% Como fica, diante do quebra-quebra em Brasília, a “direita ordeira”? Viu-se em
% um reflexo distorcido: “patriotas”, cidadãos de bem, “manés” seriam capazes de
% tais atos de depredação e vandalismo? Defensores da ordem, da tradição, da
% família e da propriedade seriam agentes de tanta violência? A resposta
% escapista foi: havia esquerdistas infiltrados.
% 
% Imagens: 6 e 7
% 
% Minibios: