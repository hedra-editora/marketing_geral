%AUTOR_LIVRO_CLIPPING.tex
%Preencher com o nome das cor ou composição RGB (ex: [r=0.862, g=0.118, b=0.118]) 
\usecolors[crayola] 			   % Paleta de cores pré-definida: wiki.contextgarden.net/Color#Pre-defined_colors

% Cores definidas pelo designer:
% MyGreen		r=0.251, g=0.678, b=0.290 % 40ad4a
% MyCyan		r=0.188, g=0.749, b=0.741 % 30bfbd
% MyRed			r=0.820, g=0.141, b=0.161 % d12429
% MyPink		r=0.980, g=0.780, b=0.761 % fac7c2
% MyGray		r=0.812, g=0.788, b=0.780 % cfc9c7
% MyOrange		r=0.980, g=0.671, b=0.290 % faab4a

% Configuração de cores
\definecolor[MyColor][MaximumGreen]      % ou ex: [r=0.862, g=0.118, b=0.118] % corresponde a RGB(220, 30, 30)
\definecolor[MyColorText][white]  % ou ex: [r=0.862, g=0.118, b=0.118] % corresponde a RGB(167, 169, 172)

% Classe para diagramação dos posts
\environment{marketing.env}		   

% Comandos & Instruções %%%%%%%%%%%%%%%%%%%%%%%%%%%%%%%%%%%%%%%%%%%%%%%%%%%%%%%%%%%%%%%|

% Cabeço e rodapé: Informações (caso queira trocar alguma coisa)
 		\def\MensagemSaibaMais 	{SAIBA MAIS:}
 		\def\MensagemSite		{HEDRA.COM.BR}
 		\def\MensagemLink		{LINK NA BIO}

\starttext %---------------------------------------------------------|

\Mensagem{NA IMPRENSA}

\MyPhoto{ARANTES_MANES_IMPRENSA_OPERAMUNDI}\par

\hyphenpenalty=10000
\exhyphenpenalty=10000%

\page

\hyphenpenalty=10000
\exhyphenpenalty=10000

\MyPhoto{ARANTES_MANES_IMPRENSA_OPERAMUNDI2.png}

\godown[.5cm]

\setupinterlinespace[line=1.5ex]{\tfxx «Nós, da esquerda, estamos agora associados a defensores da ordem, da democracia, do Estado de Direito e das instituições. Mas que democracia é essa? Quais instituições são essas? Que sociedade essas instituições mantém?»}
{\vfill\scale[factor=3.5]{\Seta\,Trecho da entrevista no {\bf 20 minutos - Opera Mundi}, realizada}\setupinterlinespace[line=.5ex]\scale[factor=3.5]{em 26 de abril, no YouTube: https://www.hedra.com.br/r/manesoperamundi}%\setupinterlinespace[line=1.5ex]\scale[factor=5]{as linhas nos códigos.}}

\page

\MyPhoto{ARANTES_MANES_IMPRENSA_FOLHA1.png} %Usar este tamanho de imagem
\MyPhoto{ARANTES_MANES_IMPRENSA_FOLHA_MANCHETE.png}
%\externalfigure{ARANTES_MANES_IMPRENSA_FOLHA4.png}[0.5\textwidth]
%\startcombination[2*1]
%{\externalfigure[ARANTES_MANES_IMPRENSA_FOLHA.png][width=.45\textwidth]}{}
%{\externalfigure[ARANTES_MANES_IMPRENSA_FOLHA4.png][width=.5\textwidth]}{}
%\stopcombination

\page %---------------------------------------------------------|%


%\MyPhoto{ARANTES_MANES_IMPRENSA_FOLHA3.png}
\MyPhoto{ARANTES_MANES_IMPRENSA_FOLHA5.png}
%\godown[.5]

%\startcombination[2*1]
%{\externalfigure[ARANTES_MANES_IMPRENSA_FOLHA4.png][width=.45\textwidth]}{}
%{\externalfigure[ARANTES_MANES_IMPRENSA_FOLHA5.png][width=.45\textwidth]}{}
%\stopcombination

\page %---------------------------------------------------------|

\MyPhoto{ARANTES_MANES_IMPRENSA_LAYMERT.png} %Usar este tamanho de imagem

\page %---------------------------------------------------------|%

\hyphenpenalty=10000
\exhyphenpenalty=10000%

«Muito já se escreveu, e se leu, sobre a escandalosa ocupação da Praça dos Três Poderes, uma semana depois da posse do Presidente Lula. (...) Tenho, porém, a impressão de que os autores foram certeiros ao investigarem a participação dos populares bolsonaristas no episódio golpista do 8 de Janeiro.»

{\vfill\scale[factor=5]{\Seta\,Trecho da resenha de {\bf Laymert Garcia dos Santos},}\setupinterlinespace[line=1.5ex]\scale[factor=5]{em 6 de maio, para A Terra é Redonda:}\setupinterlinespace[line=1.5ex]\scale[factor=5]{https://www.hedra.com.br/r/manesresenhalaymert}}

\page

\MyCover{ARANTES_MANES_THUMB.jpeg}

\page %---------------------------------------------------------|

\setcharacterkerning[reset]\setupbackgrounds[header][background=MyGuidesCleaned]
\Times
\switchtobodyfont[50pt] 

\mbox{}\blank[30mm]
\hfill \switchtobodyfont[50pt] hedra \hfill \mbox{} 


\stoptext %---------------------------------------------------------|