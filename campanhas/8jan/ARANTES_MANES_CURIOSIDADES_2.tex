% ARANTES_MANES_CURIOSIDADES.tex
% Preencher com o nome das cor ou composição RGB (ex: [r=0.862, g=0.118, b=0.118]) 
\usecolors[crayola] 			   % Paleta de cores pré-definida: wiki.contextgarden.net/Color#Pre-defined_colors

% Cores definidas pelo designer:
% MyGreen		r=0.251, g=0.678, b=0.290 % 40ad4a
% MyCyan		r=0.188, g=0.749, b=0.741 % 30bfbd
% MyRed			r=0.820, g=0.141, b=0.161 % d12429
% MyPink		r=0.980, g=0.780, b=0.761 % fac7c2
% MyGray		r=0.812, g=0.788, b=0.780 % cfc9c7
% MyOrange		r=0.980, g=0.671, b=0.290 % faab4a

% Configuração de cores
\definecolor[MyColor][MaximumGreen]      % ou ex: [r=0.862, g=0.118, b=0.118] % corresponde a RGB(220, 30, 30)
\definecolor[MyColorText][white]  % ou ex: [r=0.862, g=0.118, b=0.118] % corresponde a RGB(167, 169, 172)

% Classe para diagramação dos posts
\environment{marketing.env}		   

\setupbodyfont [FormularMI,14pt] 


\starttext %---------------------------------------------------------|

\hyphenpenalty=10000
\exhyphenpenalty=10000

\Mensagem{EM CONTEXTO} %Sempre usar esse header

\startMyCampaign

\hyphenpenalty=10000
\exhyphenpenalty=10000

Relembre {\bf MOMENTOS ICÔNICOS} que
antecederam a invasão bolsonarista
no 8/1

\stopMyCampaign

\page %---------------------------------------------------------| 

\hyphenpenalty=10000
\exhyphenpenalty=10000

\scale[factor=fit]{\MyPhoto{ARANTES_MANES_CURIOSIDADES_2FOTO2}\quad\MyPhoto{ARANTES_MANES_CURIOSIDADES_2FOTO1}}

Em maio de 2022, apoiador de Daniel Silveira foi para um ato vestido como o sósia brasileiro do {\bf“VIKING DO CAPITÓLIO”}. Como o homem que se destacou em meio a multidão que invadiu o congresso americano, ele usava chifres e peles, mas dessa vez o look contava com um cocar indígena e o rosto pintado de {\bf VERDE E AMARELO}.

\page

\MyPhoto{ARANTES_MANES_CURIOSIDADES_2FOTO3}

Em Irati, no interior do Paraná, durante um dos protestos contra o resultado das urnas das eleições, bolsonaristas foram flagrados fechando a rodovia, formando uma roda e cantando o {\bf HINO NACIONAL} para um pneu.

\page %---------------------------------------------------------|


\MyPhoto{ARANTES_MANES_CURIOSIDADES_2FOTO5}

Em novembro de 2022, bolsonaristas soltam fogos em comemoração a {\bf PRISÃO} do ministro Alexandre de Moraes. As comemorações pararam quando eles se deram conta de que se tratava de uma {\bf FAKE NEWS}.

\page

\MyPhoto{ARANTES_MANES_CURIOSIDADES_2FOTO4}


O {\bf PATRIOTA DO CAMINHÃO} foi um verdadeiro fenômeno da internet. As filmagens mostram um bolsonarista que, na tentativa de impedir a passagem de um caminhão durante o bloqueio na {\cap BR}-232, acabou pendurado em um caminhão em movimento.



\page
\MyCover{ARANTES_MANES_THUMB}

\page %---------------------------------------------------------|

\Hedra

\stoptext %---------------------------------------------------------|

