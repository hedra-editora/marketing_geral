% AUTOR_LIVRO_AUTOR.tex
% Preencher com o nome das cor ou composição RGB (ex: [r=0.862, g=0.118, b=0.118]) 
\usecolors[crayola]                % Paleta de cores pré-definida: wiki.contextgarden.net/Color#Pre-defined_colors

% Cores definidas pelo designer:
% MyGreen       r=0.251, g=0.678, b=0.290 % 40ad4a
% MyCyan        r=0.188, g=0.749, b=0.741 % 30bfbd
% MyRed         r=0.820, g=0.141, b=0.161 % d12429
% MyPink        r=0.980, g=0.780, b=0.761 % fac7c2
% MyGray        r=0.812, g=0.788, b=0.780 % cfc9c7
% MyOrange      r=0.980, g=0.671, b=0.290 % faab4a

% Configuração de cores
\definecolor[MyColor][TickleMePink]      % ou ex: [r=0.862, g=0.118, b=0.118] % corresponde a RGB(220, 30, 30)
\definecolor[MyColorText][black]  % ou ex: [r=0.862, g=0.118, b=0.118] % corresponde a RGB(167, 169, 172)

% Classe para diagramação dos posts
\environment{marketing.env}        

% Cabeço e rodapé: Informações (caso queira trocar alguma coisa)
        \def\MensagemSaibaMais  {SAIBA MAIS:}
        \def\MensagemSite           {HEDRA.COM.BR}
        \def\MensagemLink           {LINK NA BIO}

\starttext  %---------------------------------------------------------|
\Mensagem{MANCHETE CATIVANTE}

% Foto para background
\MyBackground{CHOMSKY_ANARQUISMO_3}

\startstandardmakeup[background=backgroundimage]
\startMyCampaign
\vfill\scale[factor=4]{\Seta\,NOME DO AUTOR (1900--2000)}
\stopMyCampaign
\stopstandardmakeup

\page 
\Mensagem{MANCHETE CATIVANTE}


\hyphenpenalty=10000
\exhyphenpenalty=10000

Pulvinar ante, a ultricies magna {\bf TRECHO EM DESTAQUE, MAS PODE HAVER MAIS DE UM}, 
mas sempre em negrito e caixa alta. Lorem ipsum dolor sit amet, consectetur adipiscing elit. Praesent sit amet pulvinar ante, a ultricies
magna. Etiam placerat quis tellus sed ultrices. Duis aliquet sed quam non
tincidunt. Donec sit amet tempor urna. Quisque auctor justo enim. Curabitur
vel est consectetur, sodales orci a, eleifend lacus. 

\page %----------------------------------------------------------|

\MyCover{THUMB_LIVRO.pdf}

\page %----------------------------------------------------------|

\Hedra

\stoptext %---------------------------------------------------------|