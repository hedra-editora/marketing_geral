% AUTOR_LIVRO_JORNAL_CLIPPING.tex
% Pré-venda
% "PRÉ-VENDA"

% Preencher com o nome das cor ou composição RGB (ex: [r=0.862, g=0.118, b=0.118]) 
\usecolors[crayola] 			   % Paleta de cores pré-definida: wiki.contextgarden.net/Color#Pre-defined_colors

% Cores definidas pelo designer:
% MyGreen		r=0.251, g=0.678, b=0.290 % 40ad4a
% MyCyan		r=0.188, g=0.749, b=0.741 % 30bfbd
% MyRed			r=0.820, g=0.141, b=0.161 % d12429
% MyPink		r=0.980, g=0.780, b=0.761 % fac7c2
% MyGray		r=0.812, g=0.788, b=0.780 % cfc9c7
% MyOrange		r=0.980, g=0.671, b=0.290 % faab4a

% Configuração de cores
\definecolor[MyColor][Cornflower]      % ou ex: [r=0.862, g=0.118, b=0.118] % corresponde a RGB(220, 30, 30)
\definecolor[MyColorText][Black]  % ou ex: [r=0.862, g=0.118, b=0.118] % corresponde a RGB(167, 169, 172)

% Classe para diagramação dos posts
\environment{marketing.env}		   

% Comandos & Instruções %%%%%%%%%%%%%%%%%%%%%%%%%%%%%%%%%%%%%%%%%%%%%%%%%%%%%%%%%%%%%%%|

% Cabeço e rodapé: Informações (caso queira trocar alguma coisa)
 		\def\MensagemSaibaMais 	{SAIBA MAIS:}
 		\def\MensagemSite		{HEDRA.COM.BR}
 		\def\MensagemLink		{LINK NA BIO}

\starttext %---------------------------------------------------------|

\Mensagem{NA IMPRENSA}

\MyPhoto{THUMB_IMPRENSA.jpeg} %Usar este tamanho de imagem

\page %---------------------------------------------------------|

\hyphenpenalty=10000
\exhyphenpenalty=10000

Pulvinar ante, a ultricies magna {\bf TRECHO EM DESTAQUE, MAS PODE HAVER MAIS DE UM}, sempre em negrito e caixa alta. Lorem ipsum dolor sit amet,
consectetur ou {\bf ALGUM OUTRO DESTAQUE}. «O que diz a matéria? É possível citar um trecho curto dela, sempre entre aspas francesas?» Caso contrário, conte com suas palavras a importância da matéria. Na próxima página haverá espaço para uma citação maior, caso precise.

\page %---------------------------------------------------------|

\hyphenpenalty=10000
\exhyphenpenalty=10000

«Isso não é obrigatório mas, caso queira, é possível continuar a citação nesta página. Se fizer isso, não esqueça dos créditos. Tipo data, jornal, de quem se trata (ainda que seja repetitivo).»

{\vfill\scale[factor=5]{\Seta\,Trecho da entrevista com {\bf João}, do Fantástico, em}\setupinterlinespace[line=1.5ex]\scale[factor=5]{1º de abril, para a Folha de S.\,Paulo. Lembre de quebrar}\setupinterlinespace[line=1.5ex]\scale[factor=5]{as linhas nos códigos.}}

\page %---------------------------------------------------------|

\MyCover{THUMB_LIVRO.pdf}

\page %---------------------------------------------------------|

\Hedra

\stoptext %---------------------------------------------------------|