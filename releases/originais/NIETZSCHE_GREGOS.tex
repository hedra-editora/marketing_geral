O solo do qual a filosofia nasceu não é unívoco. E, mais, implica na pergunta pela “pessoa” por detrás de cada empreendimento filosófico: "Eu conto a história de tais filósofos de um modo simplificado: espero destacar apenas o ponto de cada sistema que é um pedaço de personalidade." Para Friederich Nietzsche, autor de A filosofia na era trágica dos gregos (1873), livro póstumo e inacabado, o problema da origem da filosofia revela uma procedência que nos escapa, “uma atmosfera pessoal, uma coloração de que se pode lançar mão a fim obter a imagem do filósofo”.

Neste livro não lemos as palavras do menino prodígio da filologia clássica alemã a fim 􏰃􏰀de traduzir e interpretar; e tampouco do romântico pensador iconoclasta. Aqui, Nietzsche defende a tese de que os pensadores anteriores a Platão foram os únicos que ousaram compreender a dimensão trágica que rege a vida dos homens. Se ao homem antigo era “incrivelmente árduo apreender o conceito como conceito”, ao moderno cumpre, inversamente, sublimar o que há de mais pessoal nas significações abstratas. Reféns de um conhecimento pulverizado, somos levados a aceitar uma insuportável inacessibilidade: "A tarefa a ser levada a cabo por um filósofo no interior de uma efetiva cultura, formada segundo um estilo unitário, não se deixa adivinhar com perfeita clareza a partir de nossas condições e vivências, porque 􏰂􏰁simplesmente não dispomos de tal cultura."

Por meio de uma curiosa intuição, Nietzsche traz à tona “aquilo que nenhum conhecimento posterior poderá nos roubar: o grande homem”. Acreditando que espíritos afins já de longe se reconhecem, o autor de A filosofia na era trágica dos gregos permite-se reproduzir tais homens com a ponta de sua pena. E cita, dentre eles, os primeiros filósofos, que se prestam aqui a modelos-vivos: Tales de Mileto, Anaximandro de Mileto, Heráclito de Éfeso, Parmênides de Eleia, Zenão de Eleia, Anaxágoras de Clazômenas, etc.

Ao enfatizar não só as antigas hipóteses de interpretação do homem e do universo (mas também as vidas singulares), o filósofo alemão não pretende cultuar personalidades ou erigir ídolos. E tampouco poderia ser diferente. Afinal de contas: ``Outros povos possuem santos, enquanto que os gregos, por sua vez, têm sábios''.