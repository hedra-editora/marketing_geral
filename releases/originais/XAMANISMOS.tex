%ORELHA

Os xamanismos têm sido objeto de fascínio há mais de 500 anos. Mesmo assim,\textit{Xamanismos amerídios} é uma das primeiras publicações em português dedicadas ao assunto depois de mais de duas décadas. O livro conta com artigos de pesquisadores indígenas e não indígenas sobre os modos de pensar e agir das culturas ameríndias, mas sem a obsessão por definir ou identificar o fenômeno xamânico como religião arcaica. A investigação passa, nesta edição, pela multiplicidade dos xamanismos de 19 povos: de uma ponta a outra da América da Argentina e do Brasil, passando por Venezuela, Colômbia, Peru, México e alcançando o Canadá.

O livro também passa pela poética e política dos xamanismos no mundo contemporâneo: a agência \textit{est-ética} e a centralidade do sensível, com a potência de criar e transformar mundos, que se alteram ao longo do tempo e adaptam-se à realidade na qual estão inseridos, e sua consequente abertura ao \textit{outro}.

\textit{Xamanismos amerídios} explora diversas faces do tema por meio de teorias transversais e etnografias que ressoam umas nas outras, em uma encruzilhada de de saberes e práticas em diferentes contextos e histórias, por onde circulam corpos, substâncias, objetos, povos, linguagens e performances. Trata-se de uma antropologia em transformação, campo engajado na tentativa de compreender outros regimes de conhecimento e de ir além das categorias ocidentais da realidade.
%PRETAS

\textbf{Xamanismos ameríndios} reúne artigos de autores indígenas e não indígenas a respeito dos xamanismos de 20 povos das Américas, originários de diversas regiões da Amazônia, Aridoamérica, Brasil Central, Canadá, Chaco, Llanos de Venezuela e Mesoamérica. Foi organizado a partir de dois grupos de trabalho, um no \textsc{vii} Congresso da Associação Portuguesa de Antropologia e outro na 6ª Reunião Equatorial de Antropologia, ambos em 2019. Contém 18 textos, nos quais são explorados modos xamânicos particulares de conceber o mundo e agir nele: da permeação estética ---  a agência no mundo por meio da beleza e da poesia --- associada à centralidade do sensível através de sons, cheiros, imagens, texturas e movimentos à sua \textit{potência transformativa} de criar e transformar mundos. Explora também as alterações das práticas xamânicas ao longo do tempo e sua consequente abertura ao outro, em resposta ao mundo no qual estão inseridas.

