\usemodule[tikz]
\setuppapersize[A4]
\usecolors[crayola]
%\setupbackgrounds[paper][background=color,backgroundcolor=Almond]
	
	\definefontfeature
		[default]
		[default]
		[expansion=quality,protrusion=quality,onum=yes]
	\setupalign[fullhz,hanging]
	\definefontfamily [mainface] [sf] [Formular]
	\setupbodyfont[mainface,11pt]

% Indenting [4.4 cont-enp.p.65]
			\setupindenting[yes, 3ex]  % none small medium big next first dimension
			\indenting[next]           % never not no yes always first next
			
			% [cont-ent.p.76]
			\setupspacing[broad]  %broad packed
			% O tamanho do espaço entre o ponto final e o começo de uma sentença. 


\startsetups[grid][mypenalties]
    \setdefaultpenalties
    \setpenalties\widowpenalties{2}{10000}
    \setpenalties\clubpenalties {2}{10000}
\stopsetups

\setuppagenumbering
  [location={}]            % Estilo dos números de páginat

\setuphead[subject]
[style=bfb]		

\setuplayout[
          location=middle,
          %
          leftedge=0mm,
          leftedgedistance=0mm,
          leftmargin=20mm,
          leftmargindistance=0mm,
          width=100mm,
          rightmargindistance=0mm,
          rightmargin=20mm,
          rightedgedistance=0mm,
          rightedge=0mm,
          backspace=20mm,
          %
          top=21mm,
          topdistance=0mm,
          header=0mm,
          headerdistance=0mm,
          height=250mm,
          footerdistance=0mm,
          footer=0mm,
          bottomdistance=0mm,
          bottom=21mm,
          topspace=21mm,
        setups=mypenalties,
]

\setupalign[right]

\starttext
{\bfb Eram os deuses queer?}

\blank[big]

\noindent{\it Da cosmogonia babilônica à mesoamericana, escritor chinês questiona dualismo de gênero de mitologias primordiais em um impulso de descolonizar os saberes}

\blank[1cm]

\inoutermargin[width=60mm,hoffset=1cm,style=tfx,,voffset=7.5cm]{
\externalfigure[CAMINHOS_QUEER_THUMB.pdf][width=60mm]
}


\inoutermargin[width=70mm,hoffset=1cm,voffset=8cm,style=tfx]{
\noindent{\bf Título} {\em Antigos caminhos queer:\\ uma exploração decolonial}\\
{\bf Autor} Zairong Xiang\\
{\bf Tradução} Paula Faro, colaboração de\\ Gil Vicente Lourenção\\
{\bf Editora} n-1 edições\\
{\bf ISBN} 978-65-6119-011-4\\
{\bf Pág.} 270\\
{\bf Lançamento} 25/04\\
{\bf Preço} R\$\,80,00
}

\noindent{}A pesquisa do professor, curador e escritor Zairong Xiang tem sido
pautada pelo que ele mesmo costuma chamar de práticas de ensino e
curadoria com cosmologia e cosmopolitanismo. Partindo de uma complexa
diversidade cultural, especificidades históricas e escrituras em inglês,
espanhol, francês, chinês e nahuatl, a obra de Zairong apresenta um viés
muitíssimo original. Se, à primeira vista, o seu ecletismo ex-cêntrico
pode soar um pouco enigmático, lendo os seus textos e ouvindo suas
palestras, tudo parece fazer sentido, especialmente em um país como o
Brasil, onde a promiscuidade cultural tem sido sempre um ponto de partida
inevitável.

O principal argumento de Zairong é que o colonialismo tem afetado as
traduções de culturas não ocidentais antigas, na tentativa de fortalecer
os seus próprios paradigmas.

\blank[big]

\subject{Trecho do livro}

{\it Antigos caminhos queer}, por sua vez, defende um profundo
desaprendizado das categorias coloniais/modernas que funcionaram, desde
o alvorecer do colonialismo europeu no século {\sc xvi} até o
presente, para manter na obscuridade as formas e teorias de
corporeidades e {\it queerness} das fontes mais antigas. Isto é feito
simultaneamente por meio de um exercício decolonial de
aprender-a-aprender com cosmologias não ocidentais e não modernas, o que
nos ajuda a abordar um rico imaginário {\it ueer} que foi quase
perdido para o pensamento moderno.

\vfill

\externalfigure[logo_n1.pdf][width=.2\textwidth]

\starttikzpicture[remember picture,overlay]
\node at (15.8,25)
{\externalfigure
              [dots.png]
              [width=.15\textwidth]};
\stoptikzpicture

\stoptext