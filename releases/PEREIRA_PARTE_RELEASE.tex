\setuppapersize[A4]
\usecolors[crayola]
\setupbackgrounds[paper][background=color,backgroundcolor=Almond]
	
	\definefontfeature
		[default]
		[default]
		[expansion=quality,protrusion=quality,onum=yes]
	\setupalign[fullhz,hanging]
	\definefontfamily [mainface] [sf] [Formular]
	\setupbodyfont[mainface,11pt]

% Indenting [4.4 cont-enp.p.65]
			\setupindenting[yes, 3ex]  % none small medium big next first dimension
			\indenting[next]           % never not no yes always first next
			
			% [cont-ent.p.76]
			\setupspacing[broad]  %broad packed
			% O tamanho do espaço entre o ponto final e o começo de uma sentença. 


\startsetups[grid][mypenalties]
    \setdefaultpenalties
    \setpenalties\widowpenalties{2}{10000}
    \setpenalties\clubpenalties {2}{10000}
\stopsetups

\setuppagenumbering
  [location={}]            % Estilo dos números de páginat

\setuphead[subject]
[style=bfb]		

\setuplayout[
          location=middle,
          %
          leftedge=0mm,
          leftedgedistance=0mm,
          leftmargin=20mm,
          leftmargindistance=0mm,
          width=100mm,
          rightmargindistance=0mm,
          rightmargin=20mm,
          rightedgedistance=0mm,
          rightedge=0mm,
          backspace=20mm,
          %
          top=21mm,
          topdistance=0mm,
          header=0mm,
          headerdistance=0mm,
          height=250mm,
          footerdistance=0mm,
          footer=0mm,
          bottomdistance=0mm,
          bottom=21mm,
          topspace=21mm,
        setups=mypenalties,
]

\setupalign[right]

\starttext
{\bfb O gesto inútil, e necessário,\\ da poesia}

\blank[big]

\noindent{\it Em seu novo livro de poemas, Iuri Pereira explora os caminhos e possibilidades para a poesia no século 21}

\blank[1cm]

\inoutermargin[width=60mm,hoffset=1cm,style=tfx,,voffset=6.5cm]{
\externalfigure[PEREIRA_PARTE_THUMB.pdf][width=60mm]
}


\inoutermargin[width=70mm,hoffset=1cm,voffset=7.5cm,style=tfx]{
\noindent{\bf Título} {\em Parte de tudo}\\
{\bf Autor} Iuri Pereira\\
{\bf Editora} Hedra\\
{\bf ISBN} 978-65-89705-33-8\\
{\bf Pág.} 76\\
{\bf Preço} R\$\,54,09
}

\noindent{}A poesia é um acidente, gesto minoritário e inútil. Corrida em raia inválida levando a um mato sem cachorro. Mas também é picada e desvio, intervalo e campo de visão, gramática biográfica e trincheira, palco de uma primeira pessoa que nunca está lá quando a procuramos fora.

Alguns poemas são partes de fala, partes de um diálogo, conversa com e contra a ausência. Alguns são memórias de passagens que articulam feixes de traços heterogêneos de uma experiência. Outros são devoluções de atropelamentos que vêm de fora, da história, indigestão, esconjuro e defesa. Sempre são passagens fixadas do tempo, que só foge e nos leva com ele. Retornos do assombro de certos movimentos totalizados num retrato.

Em todos os casos, ninharias, fiapos de um pequeno cotidiano. Não traz consolo, não confere promoções, não dá fama, não vende, não é televisionável, mal pode dizer que tem leitores para encher as cadeiras de uma pequena sala, a poesia. 

Mesmo sendo, então, talvez, um capricho, tem sua humilde dignidade de ação emancipada do ato, de produção — poesia — que ultrapassa o produto, para ser menos que marca, menos que registro, menos que livro, só resquício, indício, melodia e decalque de tudo.

%\page

\subject{Sobre o autor}

{\bf Iuri Pereira} (São Paulo, 1973), graduado em Letras na USP, mestre e doutor em Teoria Literária na Unicamp, é autor de {\it Dez poemas da vizinhança vazia} (Hedra, 2012), {\it No parque} (SM, 2016), com a artista argentina Rebeca Luciani, e {\it Sinto muito} (Peirópolis, 2023), com o artista Marcelo Cipis. Organizou os livros {\it Farsa de Inês Pereira} (Hedra, 2012), de Gil Vicente, {\it Desenganos da vida humana} (Hedra, 2013), de Gregório de Matos, e {\it Escritos sobre literatura} (Hedra, 2014), de Sigmund Freud.

% \page

% \subject{Trechos do livro}

%   \startitemize
%     \item
%     {\bf A miséria dos homens frente aos animais}

%     \startitemize
%     \item
%       Observe o rebanho a pastar: ele nada sabe do que é o ontem e o hoje;
%       saltita aqui e acolá, come, descansa, digere, novamente saltita, noite e
%       dia, dia após dia. Em resumo, preso ao seu prazer e desprazer, estancado
%       no instante, não se entristece nem se enfastia. Ver isso é difícil para
%       o homem, que se vangloria de sua humanidade perante o animal, mas
%       contempla enciumado a sorte deste --- pois o homem apenas quer, como o
%       animal, viver sem fastio e sem dor; mas o quer em vão, por não querer
%       como aquele. O homem pergunta ao animal: “por que nada me diz de sua
%       sorte e apenas me fita?” O animal quer responder e dizer: “acontece que
%       eu sempre esqueço o que quero dizer” --- mas já esquece essa resposta e
%       silencia, e o homem se espanta.
%     \stopitemize
%   \item
%     {\bf A ciência domina a vida humana}

%     \startitemize
%     \item
%       Aliás, hoje é vangloriado o fato de que “a ciência começa a dominar a
%       vida”: é possível que se chegue a isso, mas a vida assim dominada não
%       tem muito valor, pois é menos vida e garante menos vida para o futuro do
%       que outrora, quando se dominava a vida não pelo saber, mas por instintos
%       e fortes alucinações. Mas esta não deve ser, como dissemos, uma época de
%       personalidades harmoniosas, perfeitas e maduras, mas a do trabalho mais
%       ordinário e mais útil possível. Isso significa que os homens devem
%       direcionar-se aos propósitos da época para trabalhar o mais cedo
%       possível. Eles devem trabalhar na fábrica das utilidades universais
%       antes de se tornar maduros --- porque seria um luxo dispensar do
%       “mercado de trabalho” uma grande quantidade de força. Cegam-se alguns
%       pássaros para que eles cantem melhor; não acredito que os homens de hoje
%       cantem melhor do que os de outrora, mas sei que se cegam na atualidade.
%       Mas o instrumento, o terrível instrumento que utilizam para cegar é uma
%       luz por demais rútila, súbita e cambiante.
%     \stopitemize
%   \stopitemize
 
 \stoptext