\setuppapersize[A4]
\usecolors[crayola]
\setupbackgrounds[paper][background=color,backgroundcolor=Almond]
	
	\definefontfeature
		[default]
		[default]
		[expansion=quality,protrusion=quality,onum=yes]
	\setupalign[fullhz,hanging]
	\definefontfamily [mainface] [sf] [Formular]
	\setupbodyfont[mainface,11pt]

% Indenting [4.4 cont-enp.p.65]
			\setupindenting[yes, 3ex]  % none small medium big next first dimension
			\indenting[next]           % never not no yes always first next
			
			% [cont-ent.p.76]
			\setupspacing[broad]  %broad packed
			% O tamanho do espaço entre o ponto final e o começo de uma sentença. 


\startsetups[grid][mypenalties]
    \setdefaultpenalties
    \setpenalties\widowpenalties{2}{10000}
    \setpenalties\clubpenalties {2}{10000}
\stopsetups

\setuppagenumbering
  [location={}]            % Estilo dos números de páginat

\setuphead[subject]
[style=bfb]		

\setuplayout[
          location=middle,
          %
          leftedge=0mm,
          leftedgedistance=0mm,
          leftmargin=20mm,
          leftmargindistance=0mm,
          width=100mm,
          rightmargindistance=0mm,
          rightmargin=20mm,
          rightedgedistance=0mm,
          rightedge=0mm,
          backspace=20mm,
          %
          top=21mm,
          topdistance=0mm,
          header=0mm,
          headerdistance=0mm,
          height=250mm,
          footerdistance=0mm,
          footer=0mm,
          bottomdistance=0mm,
          bottom=21mm,
          topspace=21mm,
        setups=mypenalties,
]

\setupalign[right]

\starttext
{\bfb O perigoso, corrosivo e\\ envenenador da vida}

\blank[big]

\noindent{\it O consagrado filósofo Friedrich Nietzsche discute a doença do século: o sentido histórico}

\blank[1cm]

\inoutermargin[width=60mm,hoffset=1cm,style=tfx,,voffset=6.5cm]{
\externalfigure[NIETZSCHE_UTILIDADE_THUMB.pdf][width=60mm]
}


\inoutermargin[width=70mm,hoffset=1cm,voffset=7.5cm,style=tfx]{
\noindent{\bf Título} {\em Sobre a utilidade e a desvantagem da história para a vida}\\
{\bf Autor} Friedrich Nietzsche\\
{\bf Tradutor} André Itaparica\\
{\bf Editora} Hedra\\
{\bf ISBN} 978-85-7715-768-6\\
{\bf Pág.} 122\\
{\bf Pré-venda} 07/05\\
{\bf Lançamento} 07/06\\
{\bf Preço} R\$\,59,00
}

\noindent{}Obra fundamental para compreender a filosofia da história e a filosofia da
cultura em Nietzsche, {\it Sobre a utilidade e a desvantagem da história para a
vida}, publicada em 1874, é a segunda das quatro considerações extemporâneas do
autor, série de livros caracterizada pelo desejo de “intervir
extemporaneamente --- isto é, contra a época, sobre a época e a favor de uma
época futura”. Nesta consideração, são discutidos os princípios, limites e
objetivos do saber histórico.

Contudo, as invectivas de Nietzsche não se dirigem apenas à cultura histórica
do século 19, mas também às próprias concepções de ciência e de
conhecimento que permeiam essa pesquisa e têm consequências na cultura como um
todo. Para o filósofo, tratar a história com a pretensão da suposta
objetividade é mera erudição sem relação com a vida e com a renovação da cultura ---
é apenas uma forma de conhecimento que não conduz à ação. A história como
ciência objetiva não é apenas erro e ilusão: é desserviço à vida.

Não será à toa, portanto, que Nietzsche exortará, ao final desta consideração,
a juventude a libertar-se da educação histórica que lhe é impingida e a
praticar a história a serviço da vida, por meio dos ponto de vista {\it a-histórico}
---  “a arte e a força de poder esquecer”, isto é, a capacidade de abandonar a
memória coletiva --- e {\it supra-histórico} --- a percepção do que “dá à
existência o caráter da eternidade e identidade, a arte e a religião”.
 
Apesar da tímida recepção na época de sua publicação, {\it Sobre a utilidade e a
desvantagem da história para a vida} tornou-se, com o passar do tempo, um texto
incontornável na obra de Nietzsche pelas provocações, complexidades e
ambiguidades que contém.

\page

\subject{Sobre o autor}

Friedrich Nietzsche (Röcken, 1844--Weimar, 1900), filósofo e filólogo
alemão, foi crítico mordaz da cultura ocidental e um dos pensadores mais
influentes da modernidade. Descendente de pastores protestantes, optou,
no entanto, pela carreira acadêmica. Aos 25 anos, tornou-se professor de
letras clássicas na Universidade da Basileia, onde se aproximou do
compositor Richard Wagner. Serviu como enfermeiro voluntário na guerra
franco-prussiana, mas contraiu difteria, que lhe comprometeu a saúde
definitivamente. Retornou à Basileia e passou a frequentar mais a casa
de Wagner. Em 1879, devido a constantes recaídas, deixou a universidade
e passou a receber uma renda anual. A partir daí assumiu uma vida
errante, dedicando-se exclusivamente à reflexão e à redação de suas
obras, dentre as quais se destacam: {\em O nascimento da tragédia}
(1872), {\em Considerações Extemporâneas} (1873--1874), {\em Assim
falava Zaratustra} (1883--1885), {\em Para além do bem e mal} (1886),
{\em A genealogia da moral} (1887) e {\em O anticristo} (1895). Em 1889,
apresentou os primeiros sintomas de problemas mentais, provavelmente
decorrentes de sífilis. Faleceu em 1900.

\subject{Sobre o tradutor}

André Luis Mota Itaparica é doutor em filosofia pela Universidade de São
Paulo (USP) e professor da Universidade Federal do Recôncavo da Bahia
(UFRB). É autor de {\em Nietzsche: Estilo e moral} (Discurso/Unijuí,
2001), {\em Verdade e linguagem em Nietzsche} (Edufba, 2014), numerosos
artigos e contribuições a obras sobre Nietzsche, Crítica da Moral,
Idealismo, Realismo, Natureza, Cultura etc.

\page

\subject{Trechos do livro}

  \startitemize
    \item
    {\bf A miséria dos homens frente aos animais}

    \startitemize
    \item
      Observe o rebanho a pastar: ele nada sabe do que é o ontem e o hoje;
      saltita aqui e acolá, come, descansa, digere, novamente saltita, noite e
      dia, dia após dia. Em resumo, preso ao seu prazer e desprazer, estancado
      no instante, não se entristece nem se enfastia. Ver isso é difícil para
      o homem, que se vangloria de sua humanidade perante o animal, mas
      contempla enciumado a sorte deste --- pois o homem apenas quer, como o
      animal, viver sem fastio e sem dor; mas o quer em vão, por não querer
      como aquele. O homem pergunta ao animal: “por que nada me diz de sua
      sorte e apenas me fita?” O animal quer responder e dizer: “acontece que
      eu sempre esqueço o que quero dizer” --- mas já esquece essa resposta e
      silencia, e o homem se espanta.
    \stopitemize
  \item
    {\bf A ciência domina a vida humana}

    \startitemize
    \item
      Aliás, hoje é vangloriado o fato de que “a ciência começa a dominar a
      vida”: é possível que se chegue a isso, mas a vida assim dominada não
      tem muito valor, pois é menos vida e garante menos vida para o futuro do
      que outrora, quando se dominava a vida não pelo saber, mas por instintos
      e fortes alucinações. Mas esta não deve ser, como dissemos, uma época de
      personalidades harmoniosas, perfeitas e maduras, mas a do trabalho mais
      ordinário e mais útil possível. Isso significa que os homens devem
      direcionar-se aos propósitos da época para trabalhar o mais cedo
      possível. Eles devem trabalhar na fábrica das utilidades universais
      antes de se tornar maduros --- porque seria um luxo dispensar do
      “mercado de trabalho” uma grande quantidade de força. Cegam-se alguns
      pássaros para que eles cantem melhor; não acredito que os homens de hoje
      cantem melhor do que os de outrora, mas sei que se cegam na atualidade.
      Mas o instrumento, o terrível instrumento que utilizam para cegar é uma
      luz por demais rútila, súbita e cambiante.
    \stopitemize
  \stopitemize
 
  % \item
  %   {\bf A educação distante da experiência}

  %   \startitemize
  %   \item
  %     Sobretudo destruindo uma superstição, a crença na necessidade daquela
  %     forma de educação. Porém, pensa-se que não haveria outra possibilidade
  %     senão a da nossa tão lamentável realidade. Basta alguém examinar a
  %     literatura das últimas décadas produzida por nossas escolas e
  %     estabelecimentos de ensino superiores: ele verificará, para seu espanto
  %     e desgosto, como o objetivo geral do ensino é pensado uniformemente, em
  %     toda mudança de sugestões, em toda sofreguidão de contradições; como
  %     temerosamente se admite o resultado atual, o “homem culto”, como hoje é
  %     entendido, como o fundamento necessário e racional de um ensino
  %     ulterior. Mas aquele cânone monótono soaria assim: o jovem deve começar
  %     com um saber acerca da cultura, não com um saber acerca da vida e muito
  %     menos com um saber acerca da própria vida e vivência. Ainda mais, esse
  %     saber acerca da cultura, como saber histórico, é misturado e
  %     administrado ao jovem; isto é, sua cabeça é entupida com um número
  %     descomunal de conceitos extraídos, no máximo, do conhecimento indireto
  %     de épocas e povos pretéritos, não da observação direta da vida. Seu
  %     anseio é entorpecido e igualmente inebriado pelo grande teatro de que
  %     seria possível sumarizar em si as mais altas e mais marcantes
  %     experiências das épocas antigas, justamente as maiores épocas. É o mesmo
  %     método absurdo que conduz nossos jovens artistas plásticos a museus e
  %     galerias, e não ao ateliê de um mestre e, sobre tudo, ao ateliê da
  %     mestra única, a natureza. Como se se pudesse prever, como um passeante
  %     fugidio, na história das coisas passadas, seus pendores e artes, seu
  %     produto vital! Como se a própria vida não fosse um ofício, que se
  %     aprende profunda e firmemente, e que se exerce com labor, quando não
  %     impede que incompeten- tes e falastrões saiam do ovo!
  %   \stopitemize

\stoptext