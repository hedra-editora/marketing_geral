\setuppapersize[A4]
\usecolors[crayola]
\setupbackgrounds[paper][background=color,backgroundcolor=Almond]
\mainlanguage[pt]
	
	\definefontfeature
		[default]
		[default]
		[expansion=quality,protrusion=quality,onum=yes]
	\setupalign[fullhz,hanging]
	\definefontfamily [mainface] [sf] [Formular]
	\setupbodyfont[mainface,11pt]

% Indenting [4.4 cont-enp.p.65]
			\setupindenting[yes, 3ex]  % none small medium big next first dimension
			\indenting[next]           % never not no yes always first next
			
			% [cont-ent.p.76]
			\setupspacing[broad]  %broad packed
			% O tamanho do espaço entre o ponto final e o começo de uma sentença. 


\startsetups[grid][mypenalties]
    \setdefaultpenalties
    \setpenalties\widowpenalties{2}{10000}
    \setpenalties\clubpenalties {2}{10000}
\stopsetups

\setuppagenumbering
  [location={}]            % Estilo dos números de páginat

\setuphead[subject]
[style=bfb]		

\setuplayout[
          location=middle,
          %
          leftedge=0mm,
          leftedgedistance=0mm,
          leftmargin=20mm,
          leftmargindistance=0mm,
          width=100mm,
          rightmargindistance=0mm,
          rightmargin=20mm,
          rightedgedistance=0mm,
          rightedge=0mm,
          backspace=20mm,
          %
          top=21mm,
          topdistance=0mm,
          header=0mm,
          headerdistance=0mm,
          height=250mm,
          footerdistance=0mm,
          footer=0mm,
          bottomdistance=0mm,
          bottom=21mm,
          topspace=21mm,
        setups=mypenalties,
]

\setupalign[right]

\starttext
{\bfb No sentido do girassol}

\blank[big]

\noindent{\it A trajetória circular, do concreto ao transcendente, na poesia de Orides Fontela}

\blank[1cm]

\inoutermargin[width=60mm,hoffset=1cm,style=tfx,,voffset=6.5cm]{
\externalfigure[ORIDES_HELIANTO_THUMB.png][width=60mm]
}

\inoutermargin[width=70mm,hoffset=1cm,voffset=7.5cm,style=tfx]{
\noindent{\bf Título} {\em Helianto}\\
{\bf Autor} Orides Fontela\\
{\bf Editora} Hedra\\
{\bf ISBN} 978-85-7715-751-8\\
{\bf Pág.} 80\\
{\bf Pré-venda} 29/05\\
{\bf Lançamento} XX/XX\\
{\bf Preço} R\$\,XXXXX
}

\noindent{}O título {\it Helianto}, do segundo livro de Orides
Fontela, publicado em 1973, é sinônimo de girassol. Essa flor que
acompanha o sentido do astro maior orienta também o olhar do leitor. Os
poemas de {\it Helianto} são, afinal, guiados pela
imagem da circularidade. A partir desse movimento fundamental da
natureza --- explícito em poemas como “Oscila” e “Paisagem em círculo”
--- a poeta aprofunda a experiência ao mesmo tempo telúrica e elevada
que já explorara em {\it Transposição}, seu primeiro
livro.

Celebrada, na época da publicação, por José Paulo Paes e Antonio
Candido, a poeta investiga nos poemas de {\it
Helianto} a apreensão e fixação do que é efêmero. Da mesma maneira, o
percurso que vai das imagens do firmamento ao solo reafirma a tensão da
obra de estreia, entre o transcendente e o concreto, além do
aprofundamento da especulação teológica e das experiências místicas, nas
quais a justaposição de {\em criação humana} e {\em natureza} compõe uma
forma de ascensão ao sagrado.     

\subject{Sobre a autora}

{\bf Orides Fontela} (1940--1998) nasceu em São João
da Boa Vista, onde concluiu o curso normal e tornou-se professora. Foi
surpreendida, depois de publicar alguns poemas no jornal da cidade, pelo
entusiasmo do antigo colega de escola, o jovem crítico Davi Arrigucci
Jr., que fez questão de levar apresentar a produção literária da amiga
aos professores e ao público da antiga {\sc FFCL},
atual {\sc FFLCH}, de modo que o primeiro livro de
Orides, {\it Transposição} (1969), já nasceu
consagrado. Já morando em São Paulo e cursando filosofia, a poeta
combinou leituras acadêmicas ao misticismo cristão e à meditação
oriental --- arranjo que deixou marcas em sua obra. Depois de
{\it Helianto}, publicado em 1973, seu terceiro
livro, {\it Alba} (1983), conquistou o prêmio Jabuti
de Poesia. {\it Teia} (1996) foi contemplado com o
prêmio da Associação Paulista de Críticos de Arte
({\sc apca}). Seus poemas foram elogiados, em
diversos momentos, por críticos do porte de Antonio Candido, Décio de
Almeida Prado, Alcides Villaça, Augusto Massi e José Miguel Wisnik. Esse
reconhecimento contribuiu para que a autora, em momentos pontuais,
alcançasse mais leitores, mas só recentemente sua obra vêm conquistando
a atenção que merece.


\subject[title={Obras completas de Orides
Fontela},reference={obras-completas-de-orides-fontela}]

Depois de publicar, em volume único, as obras completas de Orides
Fontela, em 2015, a Hedra agora apresenta os livros da autora
separadamente, com breve apresentação dos editores.
{\it Transposição}, primeiro livro da autor, já está
disponível no site da editora. {\it Helianto} sai em
pré-venda no fim do mês de maio de 2024.

\subject{Dois poemas de {\em Helianto}}

\startcolumns[n=2]\adaptlayout[width=2\textwidth]
{\bf Helianto}

\startblockquote
\noindent{}Cânon\\
da flor completa\\
metro\,/\,valência\,/\,rito\\
da flor\\
verbo\\\blank[big]

\noindent{}círculo\\
exemplar de helianto\\
flor e\\
mito\\\blank[big]

\noindent{}ciclo\\
do complexo espelho\\
flor e\\
ritmo\\\blank[big]

\noindent{}cânon\\
da luz perfeita\\
capturada fixa\\
na flor\\
verbo.
\stopblockquote

\column
{\bf Rosácea}

\startblockquote
\noindent{}Rosa primária quíntupla\\
abstrato vitral\\
das figuras do ser.\\\blank[big]

\noindent{}Ritmo em círculo, cinco\\
tempos de um mesmo ponto\\
interno, que se acende\\
no infinito. Rosa\\
não rosa: arquitetura\\
corforma do possível.\\\blank[big]

\noindent{}Abstrato vitral\\
das figuras do ser.
\stopblockquote
\stopcolumns


  
\stoptext