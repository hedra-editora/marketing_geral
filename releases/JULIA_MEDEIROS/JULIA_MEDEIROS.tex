\setuppapersize[A4]
\usecolors[crayola]
\setupbackgrounds[paper][background=color,backgroundcolor=Almond]
	
	\definefontfeature
		[default]
		[default]
		[expansion=quality,protrusion=quality,onum=yes]
	\setupalign[fullhz,hanging]
	\definefontfamily [mainface] [sf] [Formular]
	\setupbodyfont[mainface,11pt]

% Indenting [4.4 cont-enp.p.65]
			\setupindenting[yes, 3ex]  % none small medium big next first dimension
			\indenting[next]           % never not no yes always first next
			
			% [cont-ent.p.76]
			\setupspacing[broad]  %broad packed
			% O tamanho do espaço entre o ponto final e o começo de uma sentença. 


\startsetups[grid][mypenalties]
    \setdefaultpenalties
    \setpenalties\widowpenalties{2}{10000}
    \setpenalties\clubpenalties {2}{10000}
\stopsetups

\setuppagenumbering
  [location={}]            % Estilo dos números de páginat

\setuphead[subject]
[style=bfb]		

\setuplayout[
          location=middle,
          %
          leftedge=0mm,
          leftedgedistance=0mm,
          leftmargin=20mm,
          leftmargindistance=0mm,
          width=100mm,
          rightmargindistance=0mm,
          rightmargin=20mm,
          rightedgedistance=0mm,
          rightedge=0mm,
          backspace=20mm,
          %
          top=21mm,
          topdistance=0mm,
          header=0mm,
          headerdistance=0mm,
          height=250mm,
          footerdistance=0mm,
          footer=0mm,
          bottomdistance=0mm,
          bottom=21mm,
          topspace=21mm,
        setups=mypenalties,
]

\setupalign[right]

\starttext
{\bfb Abolição e emancipação \\ das mulheres}

\blank[big]

\noindent{\it O primeiro romance de Júlia Lopes de Almeida e as questões que o Brasil ainda não resolveu}

\blank[1cm]

\inoutermargin[width=60mm,hoffset=1cm,style=tfx,,voffset=6.5cm]{
\externalfigure[JULIA_MEDEIROS_THUMB.pdf][width=60mm]
}


\inoutermargin[width=70mm,hoffset=1cm,voffset=7.5cm,style=tfx]{
\noindent{\bf Título} {\em A família Medeiros}\\
{\bf Autor} Júlia Lopes de Almeida\\
{\bf Organizadores} Anna Faedrich e Rafael Balseiro Zin\\
{\bf Editora} Hedra\\
{\bf ISBN} 978-85-7715-721-1\\
{\bf Pág.} 280\\
{\bf Pré-venda} 30/05\\
{\bf Lançamento} XX/XX\\
{\bf Preço} R\$\,XXXXX
}

\noindent{} Com a {\it A família Medeiros}, a Editora Hedra inicia a 
publicação das Obras Completas de Júlia Lopes de Almeida, que serão 
publicadas em 18 volumes. 

O romance veio a público, pela primeira vez, em 1891, 
em folhetim, na {\it Gazeta de Notícias}, e em livro no ano seguinte. 
A edição de referência para este volume é a última, de 1919, ano em 
que a obra foi reeditada e a autora teve a chance de revisá-la, 
conferindo-lhe forma definitiva, depois de quase trinta anos da 
primeira edição. 

Ambientado na região de Campinas, no estado de São Paulo, o livro retrata os
costumes e conflitos de duas gerações da família do Comendador
Medeiros, um cafeicultor brutal que resiste à iminente libertação dos
escravizados. Por sua vez, Eva, sua sobrinha, e Otávio, seu filho, 
defendem abertamente os ideais abolicionistas e republicanos. No seio 
da Família Medeiros encontram-se, portanto, duas questões centrais do 
Brasil que ainda estão por resolver, apesar dos avanços recentes: o racismo 
e a emancipação das mulheres.   

Cada uma das duas gerações administra uma propriedade rural: a Fazenda Genoveva,
conduzida pela mão forte do Comendador e seus asseclas, insiste na 
brutalidade da exploração da mão de obra escravizada, que, por sua
vez, resiste articulando uma revolta, um dos pontos altos do enredo. 
Trata-se do oposto do que ocorre na fazenda Mangueiral, sob a responsabilidade 
de Eva, cujas atividades são conduzidas com respeito à dignidade humana 
por meio da partilha dos lucros. 

O registro desse ambiente social e político conturbado, no estado de 
São Paulo dos últimos anos do século {\sc xix}, faz de {\it A
família Medeiros} uma obra fundamental para a compreensão do Brasil 
contemporâneo. Além do retrato de um momento crucial da nossa história 
--- os acordes finais da crise do Segundo Reinado, a abolição da escravidão 
e a Proclamação da República ---, o livro surpreende pela atualidade 
de passagens em que o ambiente familiar, cindido pelo debate político,
se radicaliza, refletindo as chagas abertas da sociedade brasileira.    

\page

\subject{Sobre a autora}

{\bf Júlia Lopes de Almeida} (1862--1934) nasceu no Rio de Janeiro. 
Considerada um verdadeiro fenômeno literário, escreveu romances, contos,
novelas, peças teatrais, crônicas, ensaios, livros didáticos e infantis.
Estreou como escritora em 1881, incentivada pelo pai, com uma crônica 
publicada na {\it Gazeta de Campinas}. Entusiasta da modernidade e 
das mentalidades  daquele período de efervescência cultural e intenso 
otimismo, compôs em seus textos um amplo painel da {\it Belle Époque}
carioca. Seu primeiro romance, {\it Memórias de Marta}, foi publicado em
folhetim, na {\it Tribuna Liberal}, do Rio de Janeiro, de 1888 a 1889.
Em seu casarão no bairro de Santa Teresa, oferecia celebrados saraus nos 
jardins, então conhecidos como {\it Salão Verde} --- onde ocorreram 
algumas das reuniões de criação da Academia Brasileira de Letras, de que
Júlia Lopes teria participado, se não tivesse sido afastada da cadeira 
que ocuparia sob o argumento de que nossa academia deveria seguir o modelo
da francesa, frequentada apenas por homens. Apesar dessa deslealdade de 
parceiros, a autora não seguiu atuante e incansável no meio literário, 
jornalístico e intelectual brasileiro e na luta pela emancipação feminina, 
aconselhando mulheres a trabalharem e a terem sua própria fonte de renda
para não dependerem dos homens, criticando filósofos misóginos, contestando
a falta de educação para as mulheres, mas, sobretudo, o tipo de educação 
que recebiam em casa, destinada apenas ao casamento e à futilidade. Desde
sua morte, em 1934, foi gradativa e injustamente alijada da memória e história 
literárias, processo que esta coleção de Obras Completas pretende reverter.


\subject{Sobre os organizadores}

{\bf Anna Faedrich} é doutora em Letras, com especialização em Teoria da
Literatura ({\sc pucrs}), professora de literatura brasileira na
Universidade Federal Fluminense ({\sc uff}) e coordenadora do projeto de
pesquisa {\it Literatura de autoria feminina na belle époque brasileira:
memória, esquecimento e repertórios de exclusão}. É autora de {\it Teorias
da autoficção} ({\sc e}d{\sc uerj}, 2022) e {\it Escritoras
silenciadas} (Macabéa/\,Fundação Biblioteca Nacional, 2022).

{\bf Rafael Balseiro Zin} é sociólogo e doutor em Ciências Sociais, pela
{\sc puc--sp}, onde atua como pesquisador no Núcleo de Estudos em Arte,
Mídia e Política (Neamp/\,{\sc cnp}q). Nos últimos anos, entre outros
temas, tem se dedicado a investigar a trajetória intelectual das escritoras
abolicionistas no Brasil, com especial atenção ao legado de Maria Firmina dos
Reis e Júlia Lopes de Almeida.


\page

\subject{Trechos do livro}

  \startitemize
    \item
    {\bf A miséria dos homens frente aos animais}

    \startitemize
    \item
      Observe o rebanho a pastar: ele nada sabe do que é o ontem e o hoje;
      saltita aqui e acolá, come, descansa, digere, novamente saltita, noite e
      dia, dia após dia. Em resumo, preso ao seu prazer e desprazer, estancado
      no instante, não se entristece nem se enfastia. Ver isso é difícil para
      o homem, que se vangloria de sua humanidade perante o animal, mas
      contempla enciumado a sorte deste --- pois o homem apenas quer, como o
      animal, viver sem fastio e sem dor; mas o quer em vão, por não querer
      como aquele. O homem pergunta ao animal: “por que nada me diz de sua
      sorte e apenas me fita?” O animal quer responder e dizer: “acontece que
      eu sempre esqueço o que quero dizer” --- mas já esquece essa resposta e
      silencia, e o homem se espanta.
    \stopitemize
  \item
    {\bf A ciência domina a vida humana}

    \startitemize
    \item
      Aliás, hoje é vangloriado o fato de que “a ciência começa a dominar a
      vida”: é possível que se chegue a isso, mas a vida assim dominada não
      tem muito valor, pois é menos vida e garante menos vida para o futuro do
      que outrora, quando se dominava a vida não pelo saber, mas por instintos
      e fortes alucinações. Mas esta não deve ser, como dissemos, uma época de
      personalidades harmoniosas, perfeitas e maduras, mas a do trabalho mais
      ordinário e mais útil possível. Isso significa que os homens devem
      direcionar-se aos propósitos da época para trabalhar o mais cedo
      possível. Eles devem trabalhar na fábrica das utilidades universais
      antes de se tornar maduros --- porque seria um luxo dispensar do
      “mercado de trabalho” uma grande quantidade de força. Cegam-se alguns
      pássaros para que eles cantem melhor; não acredito que os homens de hoje
      cantem melhor do que os de outrora, mas sei que se cegam na atualidade.
      Mas o instrumento, o terrível instrumento que utilizam para cegar é uma
      luz por demais rútila, súbita e cambiante.
    \stopitemize
  \stopitemize
 
  % \item
  %   {\bf A educação distante da experiência}

  %   \startitemize
  %   \item
  %     Sobretudo destruindo uma superstição, a crença na necessidade daquela
  %     forma de educação. Porém, pensa-se que não haveria outra possibilidade
  %     senão a da nossa tão lamentável realidade. Basta alguém examinar a
  %     literatura das últimas décadas produzida por nossas escolas e
  %     estabelecimentos de ensino superiores: ele verificará, para seu espanto
  %     e desgosto, como o objetivo geral do ensino é pensado uniformemente, em
  %     toda mudança de sugestões, em toda sofreguidão de contradições; como
  %     temerosamente se admite o resultado atual, o “homem culto”, como hoje é
  %     entendido, como o fundamento necessário e racional de um ensino
  %     ulterior. Mas aquele cânone monótono soaria assim: o jovem deve começar
  %     com um saber acerca da cultura, não com um saber acerca da vida e muito
  %     menos com um saber acerca da própria vida e vivência. Ainda mais, esse
  %     saber acerca da cultura, como saber histórico, é misturado e
  %     administrado ao jovem; isto é, sua cabeça é entupida com um número
  %     descomunal de conceitos extraídos, no máximo, do conhecimento indireto
  %     de épocas e povos pretéritos, não da observação direta da vida. Seu
  %     anseio é entorpecido e igualmente inebriado pelo grande teatro de que
  %     seria possível sumarizar em si as mais altas e mais marcantes
  %     experiências das épocas antigas, justamente as maiores épocas. É o mesmo
  %     método absurdo que conduz nossos jovens artistas plásticos a museus e
  %     galerias, e não ao ateliê de um mestre e, sobre tudo, ao ateliê da
  %     mestra única, a natureza. Como se se pudesse prever, como um passeante
  %     fugidio, na história das coisas passadas, seus pendores e artes, seu
  %     produto vital! Como se a própria vida não fosse um ofício, que se
  %     aprende profunda e firmemente, e que se exerce com labor, quando não
  %     impede que incompeten- tes e falastrões saiam do ovo!
  %   \stopitemize

\stoptext