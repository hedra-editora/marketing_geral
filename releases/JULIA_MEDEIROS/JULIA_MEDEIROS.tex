\setuppapersize[A4]
\usecolors[crayola]
\setupbackgrounds[paper][background=color,backgroundcolor=Almond]
\mainlanguage[pt]
	
	\definefontfeature
		[default]
		[default]
		[expansion=quality,protrusion=quality,onum=yes]
	\setupalign[fullhz,hanging]
	\definefontfamily [mainface] [sf] [Formular]
	\setupbodyfont[mainface,11pt]

% Indenting [4.4 cont-enp.p.65]
			\setupindenting[yes, 3ex]  % none small medium big next first dimension
			\indenting[next]           % never not no yes always first next
			
			% [cont-ent.p.76]
			\setupspacing[broad]  %broad packed
			% O tamanho do espaço entre o ponto final e o começo de uma sentença. 


\startsetups[grid][mypenalties]
    \setdefaultpenalties
    \setpenalties\widowpenalties{2}{10000}
    \setpenalties\clubpenalties {2}{10000}
\stopsetups

\setuppagenumbering
  [location={}]            % Estilo dos números de páginat

\setuphead[subject]
[style=bfb]		

\setuplayout[
          location=middle,
          %
          leftedge=0mm,
          leftedgedistance=0mm,
          leftmargin=20mm,
          leftmargindistance=0mm,
          width=100mm,
          rightmargindistance=0mm,
          rightmargin=20mm,
          rightedgedistance=0mm,
          rightedge=0mm,
          backspace=20mm,
          %
          top=21mm,
          topdistance=0mm,
          header=0mm,
          headerdistance=0mm,
          height=250mm,
          footerdistance=0mm,
          footer=0mm,
          bottomdistance=0mm,
          bottom=21mm,
          topspace=21mm,
        setups=mypenalties,
]

\setupalign[right]

\starttext
{\bfb Emancipação das mulheres \\ e abolição da escravidão}

\blank[big]

\noindent{\it O primeiro romance de Júlia Lopes de Almeida e as questões que o Brasil ainda não resolveu}

\blank[1cm]

\inoutermargin[width=60mm,hoffset=1cm,style=tfx,,voffset=6.5cm]{
\externalfigure[JULIA_MEDEIROS_THUMB.pdf][width=60mm]
}

\inoutermargin[width=70mm,hoffset=1cm,voffset=7.5cm,style=tfx]{
\noindent{\bf Título} {\em A família Medeiros}\\
{\bf Autor} Júlia Lopes de Almeida\\
{\bf Organizadores} Anna Faedrich e Rafael Balseiro Zin\\
{\bf Editora} Hedra\\
{\bf ISBN} 978-85-7715-721-1\\
{\bf Pág.} 280\\
{\bf Pré-venda} 30/05\\
{\bf Lançamento} XX/XX\\
{\bf Preço} R\$\,XXXXX
}

\noindent{}Com a {\it A família Medeiros}, a Editora Hedra inicia a 
publicação das Obras Completas de Júlia Lopes de Almeida, em 18 volumes. 

O romance veio a público, pela primeira vez, em 1891, 
em folhetim, na {\it Gazeta de Notícias}, e em livro no ano seguinte. 
A edição de referência para este volume é a última, de 1919, ano em 
que a obra foi reeditada e a autora teve a chance de revisá-la, 
conferindo-lhe forma definitiva, depois de quase trinta anos da 
primeira edição. 

Ambientado na região de Campinas, no estado de São Paulo, o livro retrata os
costumes e conflitos de duas gerações da família do Comendador
Medeiros, um cafeicultor brutal que resiste à iminente libertação dos
escravizados. Sua sobrinha Eva e seu filho Otávio, por sua vez, 
defendem abertamente os ideais abolicionistas e republicanos. 

Cada uma das duas gerações administra uma propriedade rural: a 
Fazenda Genoveva, conduzida pela mão forte do Comendador e seus 
asseclas, insiste na brutalidade da exploração da mão de obra 
escravizada, que, por sua vez, resiste articulando uma revolta, 
um dos pontos altos do enredo. Trata-se do oposto do que ocorre 
na fazenda Mangueiral, sob a responsabilidade de Eva, cujas atividades 
são conduzidas com respeito à dignidade humana por meio da partilha 
dos lucros. 

O registro desse ambiente social e político conturbado, no estado de 
São Paulo dos últimos anos do século 19, faz de {\it A
família Medeiros} uma obra fundamental para a compreensão do Brasil 
contemporâneo. Além do retrato de um momento crucial da nossa história 
--- os momentos finais da crise do Segundo Reinado, a abolição da escravidão 
e a Proclamação da República ---, o livro surpreende pela atualidade 
de passagens em que o ambiente familiar, cindido pelo debate político,
se radicaliza, refletindo duas chagas abertas da sociedade brasileira 
que ainda estão por resolver, apesar dos avanços recentes: o racismo 
e a emancipação das mulheres.        

\subject{Sobre a autora}

{\bf Júlia Lopes de Almeida} (1862--1934) nasceu no Rio de Janeiro. 
Considerada um verdadeiro fenômeno literário, escreveu romances, contos,
novelas, peças teatrais, crônicas, ensaios, livros didáticos e infantis.
Estreou como escritora em 1881, incentivada pelo pai, com uma crônica 
publicada na {\it Gazeta de Campinas}. Entusiasta da modernidade e 
das mentalidades daquele período de efervescência cultural e intenso 
otimismo, compôs em seus textos um amplo painel da {\it Belle Époque}
carioca. Autora atuante e incansável 
no meio literário, jornalístico e intelectual brasileiro e na luta pela 
emancipação feminina, Júlia Lopes aconselhou mulheres a trabalharem e 
a terem sua própria fonte de renda para não dependerem dos homens, 
criticando filósofos misóginos, contestando a falta de educação para 
as mulheres e, sobretudo, o tipo de educação 
que recebiam em casa, destinada apenas ao casamento e à futilidade. Desde
sua morte, em 1934, foi gradativa e injustamente alijada da memória e história 
literárias, processo que esta coleção de Obras Completas pretende reverter.


\subject{Sobre os organizadores}

{\bf Anna Faedrich} é doutora em Letras, com especialização em Teoria da
Literatura ({\sc pucrs}), professora de literatura brasileira na
Universidade Federal Fluminense ({\sc uff}) e coordenadora do projeto de
pesquisa {\it Literatura de autoria feminina na belle époque brasileira:
memória, esquecimento e repertórios de exclusão}. É autora de {\it Teorias
da autoficção} ({\sc e}d{\sc uerj}, 2022) e {\it Escritoras
silenciadas} (Macabéa/\,Fundação Biblioteca Nacional, 2022).

{\bf Rafael Balseiro Zin} é sociólogo e doutor em Ciências Sociais, pela
{\sc puc--sp}, onde atua como pesquisador no Núcleo de Estudos em Arte,
Mídia e Política (Neamp/\,{\sc cnp}q). Nos últimos anos, entre outros
temas, tem se dedicado a investigar a trajetória intelectual das escritoras
abolicionistas no Brasil, com especial atenção ao legado de Maria Firmina dos
Reis e Júlia Lopes de Almeida.

\subject{Trecho do livro}

  \startitemize
    \item
    {\bf Um debate sobre a abolição em uma família escravista}

    \startitemize
    \item
      --- A vida agora no Brasil é um inferno. Em São Paulo, um tal Luiz Gama e outro que tal Antônio Bento especulam com os pobres dos lavradores, tirando-lhes os escravos. Os jornalistas do Rio são a mesma corja. Eles acoitam os pretos fugidos para os alugarem por sua conta e irem fazer conferências públicas, nos teatros, pregando a emancipação! É por isso que a gente séria, os chama de “pescadores de águas turvas”. José do  Patrocínio é o chefe dessa bandalheira, que, se o país tivesse governo, já teria acabado. É por isso mesmo que muitos liberais e muitos conservadores estão se passando para o partido republicano… 

Otávio estremeceu, mas absteve-se de falar. Deixaria passar a onda amarga em silêncio. Reservava-se para depois.

Supunha poder demolir pouco a pouco o brônzeo egoísmo do pai e vê-lo  enfim cooperar na grande obra de humanidade e patriotismo. Precisava procurar com cuidado as ocasiões propícias para o completo desenvolvimento da sua ideia. Naquele momento tudo seria inútil; o comendador, muito exaltado, não o escutaria, e ele era incapaz nesse dia de sustentar com o velho, para cujos braços voltava cheio de alegria, uma questão qualquer. Susteve-se, enquanto o pai continuava amaldiçoando o tempo dos abusos e dos ataques à propriedade alheia!

--- Se eles se lembrarem de vir a Santa Genoveva --- exclamava --- , os bandidos dos abolicionistas, eu sei como os hei de receber: a tiro! Defendo a minha propriedade, estou no meu direito. A culpa é também das autoridades, que não amoldaçam esses cachorros dos jornais que latem, latem para os outros morderem!
Nesse ponto, bateram de manso à porta, e uma voz de mulher perguntou de fora:

---  Dá licença, meu tio?

---  Mau, lá vem a lambisgoia!\ldots{} Entre!
Otávio levantou-se, e recuando um pouco, encostou-se ao piano; a porta, impelida docemente, deu passagem à mesma pessoa que ele vira de costas, dando milho às aves.

---  Você chegou em bem má ocasião… ---  disse o comendador secamente.

---  Demoro-me pouco…
Otávio não fora notado e observava com atenção a recém-chegada.
Era uma mulher nova, esbelta, morena, de fartos cabelos negros, rosto oval, olhos franjados por longas pestanas, feições regulares sem serem belas, andar firme, cabeça erguida sem afetação. Tinha a voz grave, a atitude serena. Vestia com simplicidade o seu vestido de percale, escrupulosamente ajustado.

---  Que temos? ---  indagou o tio.

---  Venho pedir-lhe que perdoe ao Manoel Sabino; ele promete ser obediente daqui por diante. Mande tirar-lhe os ferros, sim?

---  Asneira! Deixe-se disso, que não é da competência das moças. Se não quiser ver o negro com os ferros, não olhe para ele. Era o que faltava!

---  Não olho, mas nem assim deixo de saber que os traz.
O comendador deu uma gargalhada. Pelos olhos de Eva passou um relâmpago de indignação, mas conteve-se e um sorriso de desdém arqueou-lhe os lábios.

---  Já não sei quantas vezes tenho, a seu pedido, perdoado faltas dos escravos! Olhe, é melhor que se vá preparar para o jantar; aqui está meu filho, que chegou hoje, e espero amigos nesta meia-hora…

Eva voltou os olhos para Otávio, a quem cumprimentou friamente, sem avançar um passo; depois, num tom de quem se desculpa, disse:

---  Eu não sabia da sua chegada; venho neste momento…

---  De alguma senzala ---  interrompeu com ironia o tio.

---  É verdade ---  confirmou ela --- , de uma senzala. Fui ver a Josefa, que está doente. À saída encontrei o Manoel, que me pediu que o apadrinhasse; prometi vir em seu socorro e atravessei logo para aqui\ldots{}

---  Não deve prometer o que não pode cumprir.
Eva olhou para o primo, como a pedir-lhe auxílio; Otávio, aproximando-se do fazendeiro, disse, comovido:

---  A minha chegada justificará a clemência que tiver para com ele; em nome da grande alegria de nos tornarmos a ver, peço-lhe, meu pai, que atenda aos rogos da prima Eva.
O comendador fingiu refletir um momento, e, voltando-se para a sobrinha, disse:

---  Está bom! Por hoje perdoo, mas não torne a fazer semelhantes pedidos; não torne a fazer!

---  Obrigada ---  e Eva saiu da sala sem precipitação.

Otávio sentiu avivar-se-lhe a curiosidade a respeito da história daquela prima, que não conhecera nunca, e que vinha encontrar debaixo do teto paterno, tratado por uns como um anjo, e por outros como um demônio. Avaliou um momento a triste posição de Eva, recebendo por caridade a sombra de um telhado e o pão de um velho e encarniçado inimigo de seu pai. Absteve-se, contudo, de qualquer pergunta naquela ocasião em que via o comendador excitado contra ela; pensou sensatamente que qualquer informação seria apaixonada, e reservou-se para mais tarde, quando o visse de ânimo tranquilo. E no fundo do seu espírito havia já a convicção de que a opinião de Noêmia era a justa: “Eva é um anjo!”, dissera ela, e ele compreendia-a depois de ter presenciado aquela cena.

Só os anjos arrostam com a má vontade dos poderosos a favor dos fracos e dos oprimidos; só os anjos suportam injúrias com humildade quando a causa que advogam é a dos desgraçados.

\stoptext