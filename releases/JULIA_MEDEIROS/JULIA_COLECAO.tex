\setuppapersize[A4]
\usecolors[crayola]
\setupbackgrounds[paper][background=color,backgroundcolor=Almond]
\mainlanguage[pt]
	
	\definefontfeature
		[default]
		[default]
		[expansion=quality,protrusion=quality,onum=yes]
	\setupalign[fullhz,hanging]
	\definefontfamily [mainface] [sf] [Formular]
	\setupbodyfont[mainface,11pt]

% Indenting [4.4 cont-enp.p.65]
			\setupindenting[yes, 3ex]  % none small medium big next first dimension
			\indenting[next]           % never not no yes always first next
			
			% [cont-ent.p.76]
			\setupspacing[broad]  %broad packed
			% O tamanho do espaço entre o ponto final e o começo de uma sentença. 


\startsetups[grid][mypenalties]
    \setdefaultpenalties
    \setpenalties\widowpenalties{2}{10000}
    \setpenalties\clubpenalties {2}{10000}
\stopsetups

\setuppagenumbering
  [location={}]            % Estilo dos números de páginat

\setuphead[subject]
[style=bfb]		

\setuplayout[
          location=middle,
          %
          leftedge=0mm,
          leftedgedistance=0mm,
          leftmargin=20mm,
          leftmargindistance=0mm,
          width=100mm,
          rightmargindistance=0mm,
          rightmargin=20mm,
          rightedgedistance=0mm,
          rightedge=0mm,
          backspace=20mm,
          %
          top=21mm,
          topdistance=0mm,
          header=0mm,
          headerdistance=0mm,
          height=250mm,
          footerdistance=0mm,
          footer=0mm,
          bottomdistance=0mm,
          bottom=21mm,
          topspace=21mm,
        setups=mypenalties,
]

\setupalign[right]

\starttext
{\bfb Obras completas de \\ Júlia Lopes de Almeida}

\blank[big]

\noindent{\it Editora Hedra publica as obras completas da autora em 18 volumes}

\blank[1cm]

\inoutermargin[width=60mm,hoffset=1cm,style=tfx,,voffset=6.5cm]{
\startcombination[2*2]
{\externalfigure[JULIALOPES_AVIUVA.png][width=\marginwidth]}{}
{\externalfigure[JULIALOPES_FALENCIA.png][width=\marginwidth]}{}
{\externalfigure[JULIALOPES_FAMILIA.png][width=\marginwidth]}{}
{\externalfigure[JULIALOPES_MEMORIAS.png][width=\marginwidth]}{}
\stopcombination
}


\inoutermargin[width=70mm,hoffset=1cm,voffset=7.5cm,style=tfx]{
\noindent{\bf Título} {\em Obras completas}\\
{\bf Autor} Júlia Lopes de Almeida\\
{\bf Volumes} 18\\
{\bf Organizadores} Anna Faedrich e Rafael Balseiro Zin\\
{\bf Editora} Hedra\\
}

\noindent{}Nos 90 anos de morte de Júlia Lopes de Almeida, a Editora Hedra
inicia a publicação das Obras Completas da autora. Sob a organização
de Anna Faedrich e Rafael Balseiro Zin, pesquisadores responsáveis
pelo resgate da memória de Júlia Lopes, os romances, contos, novelas,
crônicas, peças de teatro e textos de não ficção da autora serão 
apresentados ao público em edições prefaciadas por acadêmicas e 
acadêmicos de todo o Brasil. 

O projeto inicial prevê a publicação de 18 volumes, apresentados a
seguir: \\ \\ 

\startitemize[n]
\setupinterlinespace[line=1.8ex] % Ajuste o valor conforme necessário

\item A família Medeiros (romance)
\item A viúva Simões (romance)
\item Memórias de Marta (romance)
\item A falência (romance)
\item A intrusa (romance)
\item Cruel Amor (romance)
\item A Silveirinha (romance)
\item A casa verde (romance)
\item Pássaro tonto (romance)
\item O funil do diabo (romance)
\item Tríptico sobre a vida e a cultura nos campos
\item Contos e novelas  
\item Contos para crianças e adolescentes 
\item Teatro 
\item Livros de conselhos 
\item Crônicas  
\item Crônicas de viagem
\item Ensaios e conferências

\stopitemize

\page

Os destaques da coleção são os seguintes:

\startitemize[n]
\setupinterlinespace[line=1.8ex] % Ajuste o valor conforme necessário

  \item Todos os volumes contêm prefácio de especialistas na obra de Júlia Lopes de Almeida;

  \item O {\it Tríptico sobre a vida e a cultura nos campos} contém o romance {\it Correio da roça} (1913), os poemas e a prosa de {\it A árvore} (1906) e o texto sobre “a cultura de flores” de {\it Jardim florido}, respeitando assim o projeto da autora, no qual esses três livros compunham uma unidade;
  
  \item Reunião de todos os contos e novelas da autora em dois volumes: um voltado para o público adulto, outro para os jovens leitores;
  
  \item Reunião inédita da dramaturgia completa da autora: desde as peças anteriormente publicadas em 1917, em um só volume --- {\it Quem ama não perdoa}, {\it Doidos de amor} e {\it Nos jardins de Saul} --- passando pelas peças manuscritas --- {\it O caminho do céu}, {\it O dinheiro dos outros}, {\it Vai raiar o sol}, {\it A senhora marquesa}, {\it A última entrevista} e {Laura} --- até {\it A herança}, publicada em 1909;
  
  \item Reunião dos livros de conselhos --- {\it Livro das noivas} (1896) e {\it Livro das donas e donzelas} (1906) --- em um só volume, de importância história para a pesquisa a respeito de costumes e hábitos das mulheres brasileiras da virada do século XIX para o século XX;

  \item Reunião das crônicas da autora publicadas na imprensa periódica nas colunas {\it Eles e elas}, {\it Dois dedos de prosa} e {\it A violeta};

  \item Reunião em volume único das crônicas de viagem da autora: {\it Cenas e paisagens do Espírito Santo} (1912) e {\it Jornadas no meu país} (1920);
  
  \item Reunião das obras de não ficção da autora em volume único: {\it Maternidade} (1925) e as conferências “A mulher e a arte” (sem data), “Padre José Maurício” (1917), “Brasil” (1922) e “Oração a Santa Doroteia” (1923).

\stopitemize

\startcombination[2*2]
{\externalfigure[JULIALOPES_AVIUVA.png][width=.5\textwidth]}{}
{\externalfigure[JULIALOPES_FALENCIA.png][width=.5\textwidth]}{}
{\externalfigure[JULIALOPES_FAMILIA.png][width=.5\textwidth]}{}
{\externalfigure[JULIALOPES_MEMORIAS.png][width=.5\textwidth]}{}
\stopcombination


\page

\subject{Sobre a autora}

{\bf Júlia Lopes de Almeida} (1862--1934) nasceu no Rio de Janeiro. 
Considerada um verdadeiro fenômeno literário, escreveu romances, contos,
novelas, peças teatrais, crônicas, ensaios, livros didáticos e infantis.
Estreou como escritora em 1881, incentivada pelo pai, com uma crônica 
publicada na {\it Gazeta de Campinas}. Entusiasta da modernidade e 
das mentalidades daquele período de efervescência cultural e intenso 
otimismo, compôs em seus textos um amplo painel da {\it Belle Époque}
carioca. Autora atuante e incansável 
no meio literário, jornalístico e intelectual brasileiro e na luta pela 
emancipação feminina, Júlia Lopes aconselhou mulheres a trabalharem e 
a terem sua própria fonte de renda para não dependerem dos homens, 
criticando filósofos misóginos, contestando a falta de educação para 
as mulheres e, sobretudo, o tipo de educação 
que recebiam em casa, destinada apenas ao casamento e à futilidade. Desde
sua morte, em 1934, foi gradativa e injustamente alijada da memória e história 
literárias, processo que esta coleção de Obras Completas pretende reverter.


\subject{Sobre os organizadores}

{\bf Anna Faedrich} é doutora em Letras, com especialização em Teoria da
Literatura (PUCRS), professora de literatura brasileira na
Universidade Federal Fluminense (UFF) e coordenadora do projeto de
pesquisa {\it Literatura de autoria feminina na belle époque brasileira:
memória, esquecimento e repertórios de exclusão}. É autora de {\it Teorias
da autoficção} (EdUERJ, 2022) e {\it Escritoras
silenciadas} (Macabéa/\,Fundação Biblioteca Nacional, 2022).

{\bf Rafael Balseiro Zin} é sociólogo e doutor em Ciências Sociais, pela
PUC-SP, onde atua como pesquisador no Núcleo de Estudos em Arte,
Mídia e Política (Neamp/\,CNPq). Nos últimos anos, entre outros
temas, tem se dedicado a investigar a trajetória intelectual das escritoras
abolicionistas no Brasil, com especial atenção ao legado de Maria Firmina dos
Reis e Júlia Lopes de Almeida.

\stoptext