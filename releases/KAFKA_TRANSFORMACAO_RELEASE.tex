\setuppapersize[A4]
\usecolors[crayola]
\setupbackgrounds[paper][background=color,backgroundcolor=Almond]
\mainlanguage[pt]
	
	\definefontfeature
		[default]
		[default]
		[expansion=quality,protrusion=quality,onum=yes]
	\setupalign[fullhz,hanging]
	\definefontfamily [mainface] [sf] [Formular]
	\setupbodyfont[mainface,11pt]

% Indenting [4.4 cont-enp.p.65]
			\setupindenting[yes, 3ex]  % none small medium big next first dimension
			\indenting[next]           % never not no yes always first next
			
			% [cont-ent.p.76]
			\setupspacing[broad]  %broad packed
			% O tamanho do espaço entre o ponto final e o começo de uma sentença. 


\startsetups[grid][mypenalties]
    \setdefaultpenalties
    \setpenalties\widowpenalties{2}{10000}
    \setpenalties\clubpenalties {2}{10000}
\stopsetups

\setuppagenumbering
  [location={}]            % Estilo dos números de páginat

\setuphead[subject]
[style=bfb]		

\setuplayout[
          location=middle,
          %
          leftedge=0mm,
          leftedgedistance=0mm,
          leftmargin=20mm,
          leftmargindistance=0mm,
          width=100mm,
          rightmargindistance=0mm,
          rightmargin=20mm,
          rightedgedistance=0mm,
          rightedge=0mm,
          backspace=20mm,
          %
          top=21mm,
          topdistance=0mm,
          header=0mm,
          headerdistance=0mm,
          height=250mm,
          footerdistance=0mm,
          footer=0mm,
          bottomdistance=0mm,
          bottom=21mm,
          topspace=21mm,
        setups=mypenalties,
]

\setupalign[right]

\starttext
{\bf A transformação de Kafka nos cem anos de sua morte}

\blank[big]

\noindent{\it Nova tradução do clássico “A metamorfose” evidencia o humor do
original e abre suas possibilidades de sentido}

\blank[1cm]

\inoutermargin[width=60mm,hoffset=1cm,style=tfx,,voffset=6.5cm]{
\externalfigure[KAFKA_TRANSFORMACAO_THUMB.pdf][width=60mm]
}


\inoutermargin[width=70mm,hoffset=1cm,voffset=7.5cm,style=tfx]{
\noindent{\bf Título} {\em A transformação}\\
{\bf Autor} Franz Kafka\\
{\bf Tradutor} Celso Donizete Cruz\\
{\bf Editora} Hedra\\
{\bf ISBN} XXXXXXXXXXXXXX\\
{\bf Pág.} 122\\
{\bf Pré-venda} 12/06\\
{\bf Lançamento} 12/07\\
{\bf Preço} R\$\,XXXXXXX
}

\noindent{}Publicada em 1915, a novela {\em Die Verwandlung} é traduzida comumente
para o português como {\em A metamorfose}. Muitos leitores conhecem a
famosa narrativa sobre Gregor Samsa, que “certa manhã, ao despertar de
um sonho inquieto, descobriu-se em sua cama transformado num
insuportável inseto”. No entanto, poucos sabem que o título original
também pode ser traduzido como {\em A transformação}, termo mais amplo e
versátil, adotado pelo tradutor Celso Donizete Cruz, mestre em língua e
literatura alemãs pela USP.

{\em A transformação} pode referir-se a mudanças físicas, psicológicas
ou situacionais, sem necessariamente carregar a mesma conotação natural
e completa de “metamorfose”, que tem uma conotação mais biológica e
específica, frequentemente associada a mudanças naturais e inevitáveis,
como a metamorfose de uma lagarta em borboleta.

Para alguns, a palavra “metamorfose” induz a pensar em um personagem
trágico e mais triste, evocando uma mudança biológica que pode ser
inevitável e dolorosa. Por outro lado, “transformação” sugere uma
conotação mais alegre e psicológica, indicando um crescimento ou
evolução pessoal que é intencional e positivo. Com isso, abre-se o
sentido da narrativa para o lúdico, o tom jocoso e brincalhão que muitos
contemporâneos perceberam em Kafka. É impossível esquecer, nesse
sentido, a conhecida anedota de que o autor, quando fazia leituras
públicas de seus manuscritos em cervejarias de Praga, durante as quais
algumas mulheres desmaiavam ao ouvir suas imagens terríveis, o fazia
entre gargalhadas. Ao falar do humor de seu texto, evidenciado em opções
de tradução, conseguimos, assim, retomar um outro lado de sua literatura
que às vezes passa despercebido.

Com a tradução de {\em A transformação}, também conseguimos recuperar a
repetição sonora do substantivo alemão do título original,
{\em Verwandlung}, que ecoa na forma verbal {\em verwandelt}
(transformado), no fim da primeira frase da narrativa. E não foram
poucos intelectuais que optaram pelo rigor com a língua original. Como o
escritor argentino Jorge Luis Borges, que também criticava o título
consagrado nas traduções, argumentando que a língua alemã possui a
palavra {\em Metamorphose}, e Kafka a adotaria se sua intenção fosse de
fato privilegiar a mutação biológica, o que não é o caso. E também o
crítico Otto Maria Carpeaux, em um texto de 1941, marco inaugural da
recepção crítica de Kafka no Brasil, referiu-se à história de Gregor
Samsa como\ldots{} {\em A transformação}!

Além da novidade da tradução, a edição da Hedra também traz uma pequena
fortuna crítica, que inclui esse texto de Carpeaux, responsável por
introduzir o público brasileiro na literatura do tcheco, e o ensaio
“Kafka. A propósito do décimo aniversário de sua morte”, em que Walter
Benjamin inova a interpretação de Kafka à época ao deslocá-la de uma questão propriamente judaica para o contexto da modernidade como um todo.

\subject{Sobre o autor}

Franz Kafka (Praga, 1883--Klosterneuburg, 1924) é um dos autores mais lidos 
e influentes do século 20. Oriundo de uma abastada família judaica de
comerciantes, sua infância é marcada pela relação conflituosa com o pai.
Frequenta na juventude uma escola alemã de Praga, cursa química e
posteriormente direito na Universidade Karl-Ferdinand, seguindo depois 
uma bem-sucedida carreira como funcionário público na área de segurança do trabalho. 
Além de {\it A transformação}, Kafka publicou em vida {\it O~foguista}, {\it A~sentença} e {\it O artista da fome}. {\it O~processo}, {\it O~castelo} e {\it América} 
(este último, inacabado) foram publicados postumamente, graças à intervenção de 
seu amigo Max Brod, que se recusou a seguir o testamento de Kafka, no qual 
determinava a destruição de todos os seus escritos inéditos. A sua obra inclui
ainda contos, diários e uma significativa correspondência com sua noiva Felice
Bauer, que ele jamais desposaria. Falece ao 39 anos, vítima de tuberculose.

\subject{Sobre o tradutor}

Celso Donizete Cruz, mestre em língua e literatura alemãs pela Universidade de São Paulo, foi professor da Universidade Federal de Sergipe. Além de obras traduzidas do alemão, inglês e italiano, é de sua autoria {\it As metamorfoses de Kafka} (Annablume, 2008), um estudo comparativo das mais de doze traduções de {\it A transformação} publicadas no Brasil.

\subject{Trechos do livro}

  \startitemize
    \item
    Certa manhã, ao despertar de um sonho inquieto, Gregor Samsa descobriu-se
    em sua cama transformado num insuportável inseto. Deitado de costas, duras
    como um casco, ele viu, ao erguer um pouco a cabeça, sua barriga arredondada,
    pardacenta, repartida por pregas arqueadas, do alto da qual a coberta, já
    quase toda caída, escorregava. Diante de seus olhos moviam-se
    desesperadas suas várias pernas, ridiculamente finas em comparação com
    suas proporções de antes.
  \item
    Em {\it A transformação}, um jovem é subitamente transformado num horrível inseto que os seus próprios parentes querem matar. O homem, submergido
    pela vida banal de todos os dias, não é mais a imagem de Deus; não se
    pode deter essa queda onde se desejaria, em alguma etapa propícia; e a queda torna-se radical até se perder o direito de existir. (Do ensaio “Franz Kafka e o mundo invisível”, de Otto Maria Carpeaux.)
    \item
    É mais fácil extrair conclusões especulativas das notas póstumas de
    Kafka que investigar um único dos temas que aparecem em seus contos e
    romances. No entanto somente esses temas podem lançar alguma luz sobre
    as forças arcaicas que atravessam a obra de Kafka --- forças,
    entretanto, que com igual justificação poderíamos identificar no mundo
    contemporâneo. (Do ensaio “Kafka. A propósito do décimo aniversário de sua morte”, de Walter Benjamin.)
  \stopitemize
 
\stoptext