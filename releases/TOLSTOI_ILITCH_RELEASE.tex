\setuppapersize[A4]
\usecolors[crayola]
\setupbackgrounds[paper][background=color,backgroundcolor=Almond]
	
	\definefontfeature
		[default]
		[default]
		[expansion=quality,protrusion=quality,onum=yes]
	\setupalign[fullhz,hanging]
	\definefontfamily [mainface] [sf] [Formular]
	\setupbodyfont[mainface,11pt]

% Indenting [4.4 cont-enp.p.65]
			\setupindenting[yes, 3ex]  % none small medium big next first dimension
			\indenting[next]           % never not no yes always first next
			
			% [cont-ent.p.76]
			\setupspacing[broad]  %broad packed
			% O tamanho do espaço entre o ponto final e o começo de uma sentença. 


\startsetups[grid][mypenalties]
    \setdefaultpenalties
    \setpenalties\widowpenalties{2}{10000}
    \setpenalties\clubpenalties {2}{10000}
\stopsetups

\setuppagenumbering
  [location={}]            % Estilo dos números de páginat

\setuphead[subject]
[style=bfb]		

\setuplayout[
          location=middle,
          %
          leftedge=0mm,
          leftedgedistance=0mm,
          leftmargin=20mm,
          leftmargindistance=0mm,
          width=100mm,
          rightmargindistance=0mm,
          rightmargin=20mm,
          rightedgedistance=0mm,
          rightedge=0mm,
          backspace=20mm,
          %
          top=21mm,
          topdistance=0mm,
          header=0mm,
          headerdistance=0mm,
          height=250mm,
          footerdistance=0mm,
          footer=0mm,
          bottomdistance=0mm,
          bottom=21mm,
          topspace=21mm,
        setups=mypenalties,
]

\setupalign[right]

\hyphenation{inte-res-ses}
\hyphenation{ama-cia-da}
\hyphenation{ca-mi-nho}

\starttext
{\bfb Um mergulho íntimo na psicologia russa do século 19}

\blank[big]

\noindent{\it A partir da crítica sutil e irônica do universo de classe média na Rússia Imperial, Tolstói vai ao extremo da mente de seu protagonista e de sua crise existencial}



\blank[1cm]

\inoutermargin[width=60mm,hoffset=1cm,style=tfx,,voffset=4.5cm]{
\externalfigure[TOLSTOI_MORTE_THUMB.jpeg][width=60mm]
}


\inoutermargin[width=70mm,hoffset=1cm,voffset=5.5cm,style=tfx]{
\noindent{\bf Título} {\em A morte de Ivan Ilitch}\\
{\bf Autor} Lev Tolstói\\
{\bf Tradutor} Irineu Franco Perpetuo\\
{\bf Editora} Hedra\\
{\bf ISBN} 978-85-7715-894-2\\
{\bf Pág.} 100\\
{\bf Pré-venda} 22/05\\
{\bf Lançamento} 22/05\\
{\bf Preço} R\$\,49,00
}

\noindent{}Após se consagrar com a publicação de longos
  romances, entre os quais {\it Guerra e paz} (1863--1869) e {\it Anna Kariênina}
  (1873--1878), Tolstói publica, em 1886, {\it A morte de Ivan Ilitch}, que
  integra uma série de textos breves que o autor produziu por volta de seus 50 anos.

  Nessa concisa narrativa, o escritor russo detém-se sobre a “corriqueira”, e por isso
  mesmo “terrível”, vida de um funcionário público médio, preenchida por
  ambições mesquinhas e relações regidas por interesses e conveniências.
  É apenas a partir do distanciamento, que a iminência da morte lhe impõe,
  que Ivan Ilitch torna-se capaz de reavaliar a vida que levou até
  então. Diante do seu inevitável fim, o protagonista não enxerga mais
  sentido nos seus esforços infindos para seguir à risca os protocolos
  sociais e adequar-se ao mundo das aparências.
  Percebe como suas escolhas priorizaram aquilo que se esperava de um homem que
  ocupava sua posição social, mas que muitas delas não foram mais do que
  fontes de aborrecimento, em nada contribuindo para sua verdadeira
  felicidade.

  Através da existência medíocre de Ivan Ilitch, o romance
  tece uma crítica sutil, salpicada com amaciada ironia, que expõe as
  futilidades e pequenezas da classe média na Rússia
  Imperial. Mas, no longo caminho da mediocridade cotidiana à morte solitária, a narrativa se esvai da preocupação comezinha e atinge algo sublime nas reflexões da personagem. Assim, saímos sutilmente do universo cotidiano, mesquinho, ganancioso e pretensioso
  das repartições, para irmos, paulatinamente, ao
  mais íntimo psicológico de Ivan Ilitch.

\page

\subject{Sobre o autor}

Lev Tolstói (1828--1910), tido como um dos mais
  importantes e influentes escritores do seu tempo, foi o principal
  representante do realismo russo. Ainda menino, perdeu ambos os pais,
  sendo educado por tutores e depois por uma tia. Em 1845, ingressou na
  Universidade de Kazan, mas não chegou a concluí-la, sendo, no fim das
  contas, um autodidata --- era conhecedor de muitas línguas e
  filosofias. Seu primeiro texto, {\it Infância}, saiu em 1852 na
  revista {\it O contemporâneo}. Após seu casamento com Sófia Andréievna
  em 1862, deu-se início a fase de seus longos romances, de {\it Guerra
  e paz} (1863--1869) até {\it Anna Kariênina} (1873--1878). Seu último
  romance foi {\it Ressurreição} (1889). Embora seja considerado “o
  mestre insuperado do gênero que se costumou chamar romance psicológico
  do século 19”, também aventurou-se por outros formatos, como contos
  breves, diários e escritos teóricos sobre pedagogia, arte e religião.
  Na década de 1880, Tolstói viveu uma fase que ele próprio definiu como
  sua {\it redenção moral}. Foi nessa época que sistematizou uma série
  de preceitos filosóficos e religiosos que, reunidos, passaram a ser
  conhecidos como tolstoísmo, doutrina baseada no cristianismo, mas
  acrescida de outras concepções, que repercutiu no mundo todo e fez com
  que Tolstói fosse excomungado da Igreja Ortodoxa.

\subject{Sobre o tradutor}

Irineu Franco Perpetuo é jornalista, tradutor e crítico de música. Autor, entre outros, de {\it História concisa da música clássica brasileira} (Alameda editorial, 2018). Entre suas traduções, consta {\it Vida e Destino}, de Vassili Grossman (Ed.\,Alfaguara, 2014, Prêmio Jabuti de Tradução).

\page

\subject{Trechos do livro}

  \startitemize
    \item
    Exigia da vida familiar apenas o jantar doméstico, a dona de casa, o leito, confortos que ela podia lhe proporcionar e, principalmente, a decência na aparência externa, que determinava a opinião da sociedade. (p.\,36)
    \item
    Na verdade, havia ali o que acontece com todas pessoas que não são ricas de verdade, mas querem parecer ricas e, dessa forma, só se parecem umas com as outras: damasco, ébano, flores, tapetes e bronzes, escuros e brilhantes — tudo que todas as pessoas de um determinado tipo fazem para se parecer com todas as pessoas de um determinado tipo. No caso dele, tudo era tão imitativo que nem chamava a atenção; para Ivan Ilitch, porém, parecia muito especial. (p.\,42)
    \item
    Mas o que é isso? Por quê? Não pode ser. Não pode ser que a vida tenha sido tão sem sentido e abjeta. E se ela foi apenas tão abjeta e sem sentido, então por que morrer, e morrer no sofrimento? Alguma coisa não bate. “Talvez eu não tenha vivido como deveria” --- passou-lhe de repente pela cabeça. (p.\,81)
    \item
    Veio-lhe à mente que aquilo que antes parecia uma completa impossibilidade, que não tivesse passado a vida como devia, pudesse ser verdade. Veio-lhe à mente que suas intenções quase imperceptíveis de lutar contra aquilo que as pessoas de posição elevada consideravam bom, que ele imediatamente afastava de si, essas intenções é que podiam ser a realidade, e todo o resto era impróprio. Seu trabalho, a organização de sua vida, sua família, os interesses da sociedade e do serviço, tudo isso podia ser impróprio. Tentou defender tudo aquilo perante si mesmo. E, de repente, sentiu toda a debilidade do que defendia. Não havia nada a defender. (p.\,88)
  \stopitemize
 
\stoptext