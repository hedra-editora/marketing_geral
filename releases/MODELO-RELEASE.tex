\setuppapersize[A4]
\usecolors[crayola]
\setupbackgrounds[paper][background=color,backgroundcolor=Almond]
\mainlanguage[pt]	

	\definefontfeature
		[default]
		[default]
		[expansion=quality,protrusion=quality,onum=yes]
	\setupalign[fullhz,hanging]
	\definefontfamily [mainface] [sf] [Formular]
	\setupbodyfont[mainface,11pt]

% Indenting [4.4 cont-enp.p.65]
			\setupindenting[yes, 3ex]  % none small medium big next first dimension
			\indenting[next]           % never not no yes always first next
			
			% [cont-ent.p.76]
			\setupspacing[broad]  %broad packed
			% O tamanho do espaço entre o ponto final e o começo de uma sentença. 


\startsetups[grid][mypenalties]
    \setdefaultpenalties
    \setpenalties\widowpenalties{2}{10000}
    \setpenalties\clubpenalties {2}{10000}
\stopsetups

\setuppagenumbering
  [location={}]            % Estilo dos números de páginat

\setuphead[subject]
[style=bfb]		

\setuplayout[
          location=middle,
          %
          leftedge=0mm,
          leftedgedistance=0mm,
          leftmargin=20mm,
          leftmargindistance=0mm,
          width=100mm,
          rightmargindistance=0mm,
          rightmargin=20mm,
          rightedgedistance=0mm,
          rightedge=0mm,
          backspace=20mm,
          %
          top=21mm,
          topdistance=0mm,
          header=0mm,
          headerdistance=0mm,
          height=250mm,
          footerdistance=0mm,
          footer=0mm,
          bottomdistance=0mm,
          bottom=21mm,
          topspace=21mm,
        setups=mypenalties,
]

\setupalign[right]

\starttext
{\bfc Título} \\
{\tfb Autor}

\blank[big]


\noindent {\it Aqui um pequeno parágrafo sobre o livro... este livro é demais o autor faz isso e aquilo e tudo mais hehe}

\blank[1.5cm]

\inoutermargin[width=60mm,hoffset=1cm,style=tfx,,voffset=2.35cm]{
\externalfigure[MINULESCU_NOITE_THUMB][width=50mm]}

\inoutermargin[width=70mm,hoffset=1.1cm,voffset=3cm,style=tfx]
{\noindent{\bf Título} {\it Livro legal}\\
{\bf Autor} Fulaninho da Silva\\
{\bf Tradução} Beltrano Souza\\
{\bf Apresentação} Sicrano Andrade\\
{\bf Editora} Hedra\\
{\bf ISBN} XXXx\\
{\bf Pág.} 110\\
{\bf Preço} 43,00
}

\inoutermargin[width=70mm,hoffset=1.1cm,voffset=6.5cm,style=tfx]
{{\bf Sobre o autor} Lorem ipsum dolor sit amet, consectetur adipisicing elit, sed do eiusmod
tempor incididunt ut labore et dolore magna aliqua. Ut enim ad minim veniam,
quis nostrud exercitation ullamco laboris nisi ut aliquip ex ea commodo}

\inoutermargin[width=70mm,hoffset=1.1cm,voffset=10cm,style=tfx]
{{\bf Sobre o tradutor} Lorem ipsum dolor sit amet, consectetur adipisicing elit, sed do eiusmod
tempor incididunt ut labore et dolore magna aliqua. Ut enim ad minim veniam,
quis nostrud exercitation ullamco laboris nisi ut aliquip ex ea commodo
consequat. Duis aute irure dolor in reprehenderit in voluptate velit esse
cillum dolore eu fugiat nulla pariatur. Excepteur sint occaecat cupidatat non
proident, sunt in culpa qui officia deserunt mollit anim id est laborum. }

% \inoutermargin[width=70mm,hoffset=1.1cm,voffset=16.6cm,style=tfx]
% {{\bf Sobre o apresentador} Leonardo Francisco Soares é professor do Instituto de Letras e Linguística da Universidade Federal de Uberlândia ({\cap ILEEL/UFU}) e do programa de pós-graduação em Estudos Literários do {\cap ILEEL/UFU}. Publicou, dentre outros, um texto na coletânea {\it Guerra e literatura: ensaios em emergência} (Alameda, 2022)}


\inoutermargin[width=70mm,hoffset=-10cm,voffset=18.5cm,style=tfx]
{\definefontfamily [Times] [rm] [Times New Roman]
                   [tf=file:TimesLTStd-Roman.otf]

\setcharacterkerning[reset] \switchtobodyfont[Times,50pt] hedra \hfill \mbox{}
}

\noindent{\tfb \it Linha fina, impactante, que sirva como título para os parágrafos a seguir}

\blank[.5cm]

Conhecido sobretudo pela sua contribuição ao movimento simbolista, foi com suas incursões pelo fantástico que Ion Minulescu se firmou como um dos maiores escritores da Romênia. Publicado originalmente em 1930, {\it Para serem lidas à noite} marca o ponto de virada do autor rumo ao sobrenatural. 
Desde sua provocante nota de abertura – “leia-as de noite, ou então, não as leia nunca” –, consolida-se um universo enigmático e encantador. Composto por quatro contos, o livro é um convite à exploração das nuances da imaginação, combinando mistério, humor e ironia, e ecoando grandes mestres simbolistas, como Villiers de L'Isle-Adam, Oscar Wilde, Henri de Régnier e Edgar Allan Poe.


Os contos reunidos, de prosa envolvente e estrutura narrativa inovadora, se articulam em um jogo de espelhos. Cada enigma narrativo é apresentado dentro de outro, como uma caixa de Pandora que se abre para revelar novos segredos a cada página virada, promovendo uma leitura imersiva e cheia de surpresas.



% funcionam como labirintos nos quais os limites entre o real e o irreal desvanecem. Seja recorrendo a célebres tópicos como o pacto com o diabo, seja concebendo intrigantes relatos como o de uma gravata comprada na cidade de Braîla, a obra propõe uma viagem pelos meandros do espírito. 


% O autor utiliza esse recurso para mostrar como as histórias pessoais se entrelaçam com o tecido da sociedade, abordando questões universais que ressoam com a experiência humana.

% Outro traço distintivo do estilo de Minulescu é a convivência do cômico e do sombrio.
% Através do humor sutil com que os personagens enfrentam as adversidades e da pungente ironia do narrador, o escritor propõe uma reflexão crítica sobre a condição humana e transparece engenhosamente a fragilidade da fronteira que separa o ridículo e o trágico. 

% Além disso, a obra é marcada por uma fina ironia que permeia o discurso. O humor sutil com que os personagens enfrentam suas adversidades e dilemas existenciais não apenas atenua a tensão, como também provoca uma reflexão crítica sobre a condição humana. Esse equilíbrio entre o cômico e o sombrio é uma característica distintiva do estilo do autor, que habilmente nos lembra do caráter tênue da linha que separa o ridículo e o trágico.

Mas, acima de tudo, destaca-se a ambientação noturna, que, sugerida já no título e reforçada pela nota inicial, domina as narrativas e proporciona um cenário ideal para os mistérios, para os “jogos de mostras e máscaras” articulados por Minulescu e a serem vislumbrados pelo leitor notívago. 

{\it Para serem lidas à noite} é uma obra-prima da literatura fantástica que merece um lugar de destaque em qualquer estante, tanto pela autenticidade estilística quanto pelo talento com que Minulescu desafia nossas percepções do real.

% \subject{Sobre o autor}

% Ion Minulescu, nascido em 1881 em Bucareste, destaca-se como uma das figuras mais fascinantes da literatura romena do século {\cap XX}. Escritor, poeta, crítico literário e dramaturgo, sua trajetória é marcada por uma riqueza criativa que transita entre o simbolismo e o fantástico, influenciando gerações de leitores e escritores.

% Sua formação em direito na vibrante Paris proporcionou a Minulescu um contato profundo com as correntes literárias da época, em especial o simbolismo francês. Ao longo de sua carreira, ele conseguiu traduzir essa influência em uma linguagem poética e sofisticada, que revela uma sensibilidade única na representação dos sentimentos e da realidade.

% Minulescu assina em 1908 o manifesto literário, {\it Acendam as tochas}, veementemente antitradicional, no qual se pedia ao jovem que “acendesse as tochas”, promovesse “a liberdade e a individualidade na arte” e abandonasse as formas ultrapassadas, herdadas dos antecessores. Não muito mais tarde, na inconformista revista literária {\it Insula} (1912), publicará uma série de artigos ousados e combativos, dirigidos contra  certas tradições estabelecidas na literatura e nas artes.

% Embora seja amplamente reconhecido por sua contribuição ao movimento simbolista, Minulescu destaca-se também na esfera da literatura fantástica, como verifica-se em seu célebre {\it Para serem lidas à noite}. Suas incursões nesse gênero revelam um talento excepcional para criar atmosferas oníricas e enredos que desafiam a lógica e mesclam o cotidiano com o extraordinário, conferindo às suas narrativas uma aura quase mágica.

% O legado de Minulescu não se limita apenas à sua escrita ficcional; sua atuação como jornalista e editor também foi fundamental para o desenvolvimento cultural na Romênia. O autor não hesitou em se posicionar critica e combativamente em relação às questões sociais e políticas de seu tempo, defendendo que a literatura deve dialogar com a realidade. 

% Falecido em 1944, Minulescu permanece uma figura indispensável da literatura romena, com uma obra que transcende fronteiras e continua a ressoar profundamente nos leitores contemporâneos. 


% \blank[big]
\page
\subject{Sobre o tradutor}

Fernando Klabin nasceu em São Paulo e formou-se em Ciência Política pela Universidade de Bucareste, onde foi agraciado com a Ordem do Mérito Cultural da Romênia no grau de Oficial, em 2016. Além de tradutor exerce atividades ocasionais como fotógrafo, escritor, ator e artista plástico.

\subject{Sobre o apresentador}

Leonardo Francisco Soares é professor associado do Instituto de Letras e Linguística da Universidade Federal de Uberlândia ({\cap ILEEL/UFU}) e professor permanente do programa de pós-graduação em Estudos Literários do {\cap ILEEL/UFU}. Publicou, dentre outros, um texto na coletânea {\it Guerra e literatura: ensaios em emergência} (Alameda, 2022)

\subject{Trecho do livro}

  \startitemize
    \item
    {\bf Capítulo {\it Bate-papo com o coisa-ruim}}

    % \startitemize
    % \item
    %  A imaginação dos poetas, na maior parte das vezes, ultrapassa a realidade e estrangula o verossímil. Ainda bem que a maioria das pessoas que frequenta a Igreja não lê poesia, e aqueles que lêem e acreditam na conversa fiada dos poetas não vão à Igreja.

    % \stopitemize

    \startitemize
    \item
      Jamais esquecerei aquele momento de terror, acentuado pela vergonha de não poder manifestá-lo diante da pessoa que o produzira em nós dois.
Amarelo como a cera, de olhos arregalados atrás de nós, Oreste não conseguiu segurar a emoção diante daquela constatação fantástica. Com a voz embargada pela síncope suprema em que sua alma parecia deixar o corpo, ele sussurrou tão baixo que mal se fez ouvir:
  
     --- Onde está sua sombra, Seu Damian? Você não faz sombra sobre a terra?

    \stopitemize


%   \item
%     {\bf Capítulo {\it O homem do coração de ouro}}
% \startitemize
% \item --- Onde está o anel?\unknown Por que você arrancou a pedra?\unknown\\
% --- Não fui eu quem arrancou.\\
% --- Então quem foi?\\
% --- Ele!\unknown\\
% --- Ele quem?\unknown\\
% --- O homem do coração de ouro!\\
% --- Admirável título para uma novela fantástica!, exclamei.\\

%     \stopitemize

% \startitemize
% \item
% --- Você teria a bondade de me dizer quantos anos tem?\\
% --- Trezentos e onze anos, e cento e noventa e oito dias,
% considerando, claro, os trinta dias dos anos bissextos.\\
% --- E por que é que você está há tanto tempo por aqui?\\
% --- Não posso morrer até estar completo, como todos os
% mortais.\\
% --- Falta-lhe algo?\\
% --- Sim\unknown\\
% --- Algum órgão importante?\\
% --- O mais importante de todos\unknown O coração! [\unknown]

  \stopitemize



\stoptext