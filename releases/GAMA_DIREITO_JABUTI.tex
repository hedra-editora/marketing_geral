\setuppapersize[A4]
\usecolors[crayola]
\setupbackgrounds[paper][background=color,backgroundcolor=Almond]
\mainlanguage[pt]
	
	\definefontfeature
		[default]
		[default]
		[expansion=quality,protrusion=quality,onum=yes]
	\setupalign[fullhz,hanging]
	\definefontfamily [mainface] [sf] [Formular]
	\setupbodyfont[mainface,11pt]

% Indenting [4.4 cont-enp.p.65]
			\setupindenting[yes, 3ex]  % none small medium big next first dimension
			\indenting[next]           % never not no yes always first next
			
			% [cont-ent.p.76]
			\setupspacing[broad]  %broad packed
			% O tamanho do espaço entre o ponto final e o começo de uma sentença. 


\startsetups[grid][mypenalties]
    \setdefaultpenalties
    \setpenalties\widowpenalties{2}{10000}
    \setpenalties\clubpenalties {2}{10000}
\stopsetups

\setuppagenumbering
  [location={}]            % Estilo dos números de páginat

\setuphead[subject]
[style=bfb]		

\setuplayout[
          location=middle,
          %
          leftedge=0mm,
          leftedgedistance=0mm,
          leftmargin=20mm,
          leftmargindistance=0mm,
          width=100mm,
          rightmargindistance=0mm,
          rightmargin=20mm,
          rightedgedistance=0mm,
          rightedge=0mm,
          backspace=20mm,
          %
          top=21mm,
          topdistance=0mm,
          header=0mm,
          headerdistance=0mm,
          height=250mm,
          footerdistance=0mm,
          footer=0mm,
          bottomdistance=0mm,
          bottom=21mm,
          topspace=21mm,
        setups=mypenalties,
]

\setupalign[right]

\starttext
{\bfb Obra com 70 textos de Luiz Gama vence o Prêmio Jabuti}

\blank[big]

\noindent{\it {\bf Direito (1870-1875)}, com escritos da época em que o autor iniciou a carreira de advogado, marca presença na primeira edição do Jabuti Acadêmico}

\blank[1cm]

\inoutermargin[width=60mm,hoffset=1cm,style=tfx,,voffset=6.5cm]{
\externalfigure[GAMA_DIREITO_THUMB.png][width=60mm]
}


\inoutermargin[width=70mm,hoffset=1cm,voffset=7.5cm,style=tfx]{
\noindent{\bf Título} {\em Direito (1870-1875)}\\
{\bf Autor} Luiz Gama\\
{\bf Organizador} Bruno Rodrigues de Lima\\
{\bf Editora} Hedra\\
{\bf ISBN} 978-85-7715-734-1\\
{\bf Pág.} 488\\
{\bf Preço} R\$\,149,90
}

\noindent{}Na última terça-feira (6/8), o livro {\em Direito (1870-1875)}, de Luiz Gama, venceu o Prêmio Jabuti na categoria {\bf Direito}. Com organização de Bruno Rodrigues de Lima, doutor em História do Direito pela Universidade de Frankfurt e coordenador das Obras completas do abolicionista baiano, {\em Direito} reúne 70 textos que, em grande parte, nunca tinham sido publicados em livro.

A obra é fruto de uma década de trabalho de Bruno Rodrigues, que pesquisou pelos inéditos em arquivos da imprensa e do judiciário paulista, fluminense, gaúcho, paranaense, mineiro e baiano. O resultado é uma obra ímpar, que vai desde as maiores causas abolicionistas pelas quais Gama advogou, até missivas íntimas e textos líricos.

Aqui, por exemplo, o leitor encontra uma das raríssimas menções públicas de Gama à “Questão Netto”, a ação coletiva pela qual ele representou a demanda de liberdade de 217 pessoas. Simplesmente, a maior ação de liberdade das Américas. Também lemos a polêmica pública em que o autor se envolveu com José de Alencar, então Ministro da Justiça. O autor de {\it Iracema} saiu à imprensa para apoiar a demissão de um cargo público de Gama, cuja resposta não poderia ser mais veemente: “se dobrasse-me subserviente perante um juiz prevaricador, que, aconselhado, proferia despachos manifestamente contrários à lei; se pactuasse com os ladrões devassos e não requeresse a manumissão de indivíduos postos ilegalmente em cativeiro; se, numa palavra, guardasse profundo silêncio perante os salteadores do poder e da liberdade, seria mantido no emprego de amanuense de Polícia, e acatado pela administração!”.

Ao lado das grandes questões públicas, {\em Direito} traz textos delicados, como a carta que escreveu para o amigo José Carlos Rodrigues em novembro de 1870. Ao recordar momentos ao lado do companheiro que mudara-se para Nova York, Gama revela pequenos detalhes de seu cotidiano, dos utensílios de sua casa aos ambientes e acontecimentos que o marcaram. Na missiva, lemos uma rica recordação de sua infância, em um raro quadro da Bahia de 1837.

Também lemos a carta de Gama ao seu filho, então com onze anos. Com lirismo, Gama orienta o jovem sobre o que dizer, o que evitar, o que fazer, o que combater, o que ser, no que crer, o que ler. Diante de tantas causas que defendeu em nome da liberdade e contra as elites do Império, Gama pensava que sofreria uma atentado fatal a qualquer momento, o que o motivou a deixar suas últimas palavras para o filho.

\subject{Sobre o autor}

{\bf Luiz Gonzaga Pinto da Gama} nasceu livre em Salvador da Bahia no dia 21
de junho de 1830 e morreu na cidade de São Paulo, como herói da
liberdade, em 24 de agosto de 1882. Filho de Luiza Mahin, africana
livre, e de um fidalgo baiano cujo nome nunca revelou, Gama foi
escravizado pelo próprio pai, na ausência da mãe, e vendido para o sul
do país no dia 10 de novembro de 1840. Dos dez aos dezoito anos de
idade, Gama viveu escravizado em São Paulo e, após conseguir provas de
sua liberdade, fugiu do cativeiro e assentou praça como soldado (1848).
Depois de seis anos de serviço militar (1854), Gama tornou-se escrivão
de polícia e, em 1859, publicou suas {\it Primeiras trovas burlescas},
livro de poesias escrito sob o pseudônimo Getulino, que marcaria o seu
ingresso na história da literatura brasileira. Desde o período em que
era funcionário público, Gama redigiu, fundou e contribuiu com veículos
de imprensa, tornando-se um dos principais jornalistas de seu tempo. Mas
foi como advogado, posição que conquistou em dezembro de 1869, que
escreveu a sua obra magna, a luta contra a escravidão por dentro do
direito, que resultou no feito assombroso --- sem precedentes no
abolicionismo mundial --- de conferir a liberdade para aproximadamente
750 pessoas através das lutas nos tribunais.

\subject{Sobre o organizador}

{\bf Bruno Rodrigues de Lima} é advogado e historiador do direito, graduado em Direito pela
Universidade do Estado da Bahia (UNEB-Cabula), mestre em Direito, Estado
e Constituição pela Universidade de Brasília (UnB) e doutor em
História do Direito pela Universidade de Frankfurt, Alemanha, com tese
sobre a obra jurídica de Luiz Gama. Em 2022, ganhou o Prêmio Walter Kolb de melhor tese de doutorado da Universidade de Frankfurt. Atualmente, é pesquisador de pós-doutorado no Instituto Max Planck de História do Direito e Teoria do Direito. Pela EDUFBA,
publicou o livro {\it Lama \& Sangue -- Bahia 1926} (2018).

\subject{Trechos do livro}

  \startitemize
    \item
    A religião, a moral, o direito e a liberdade são gêneros deteriorados que não têm cotação nos mercados do Império.

    Aí compra-se e vende-se o homem; açoitam-se-lhe as carnes, monetariza-se-lhe o suor e o sangue, o homem é o escravo, o escravo é o dinheiro e a questão é essencialmente econômica!
  \item
    Tu evitas a amizade e as relações dos grandes homens; eles são como o
    oceano que aproxima-se das costas para corroer os penedos.

    Sê republicano, como o foi o Homem-Cristo. Faze-te artista; crê, porém,
    que o estudo é o melhor entretenimento, e o livro o melhor amigo.
  \stopitemize
 
\stoptext