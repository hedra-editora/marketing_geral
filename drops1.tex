\usemodule[tikz]
\usetikzlibrary{shadows.blur}
\environment{star.env}


\definepapersize    [post]        [width=9.31cm, height=9.31cm]
\definepapersize    [14x21]       [width=140mm, height=210mm]
\setuppapersize     [post]        [14x21]

    
\setuplayout[location=middle, % Centralizar página dentro da 14x21
             marking=empty]   % Colocar marcas de corte

% Define cores RGB
\definecolor[minhacor][r=0.5, g=0.7, b=0.9]

% Cores pré-definidas
% https://wiki.contextgarden.net/Color#Pre-defined_colors
\usecolors[crayola]

% Define cor de fundo (background)
\setupbackgrounds[page][background=color,backgroundcolor=Salmon]

% Retira numeração de página
\setuppagenumbering[location=none]
    

\definefontfamily [FormularRegular] [sans] [Formular]
                   [tf=file:Formular-Regular.otf,
                    it=file:Formular-Italic.otf,
                    bf=file:Formular-Bold.otf,
                    bi=file:Formular-BoldItalic.otf]

\definefontfamily [FormularMI] [sans] [Formular]
                   [tf=file:Formular-regular-mi.otf,
                    bf=file::Formular-black-mi.otf]                   

\definefontfamily [FormularMImedium] [sans] [Formular]
                   [tf=file:Formular-medium-mi.otf,
                    bf=file:Formular-bold-mi.otf]              
\definefontfamily [FormularLight] [sans] [Formular]
                   [tf=file:Formular-Light.otf,
                    it=file:Formular-LightItalic.otf]
\definefontfamily [Times] [rm] [Times New Roman]
                   [tf=file:TimesLTStd-Roman.otf]

        
\setupbodyfont [FormularMI,15pt] 
            
\definecharacterkerning[packed] 
\setupcharacterkerning[packed][factor=-.05] 


\setuplayout[
          location=middle,
          width=89mm,
          height=93mm,
          leftmargin=2mm,
          rightmargin=2mm,
          backspace=2mm,
          header=2mm,
          headerdistance=0mm,
          footer=1mm,
          footerdistance=0mm,
          topspace=0mm,
          leftedge=0mm,
          leftedgedistance=0mm,
          leftmargindistance=0mm,
          rightmargindistance=0mm,
          rightedgedistance=0mm,
          rightedge=0mm,
          topdistance=0mm,
          bottomdistance=0mm,
    %      top=21mm,
    %      bottom=21mm,
]


\definelayout[MyLayout][
          height=93mm,
%          backspace=5mm,
%          header=6.6mm,
          footer=9mm,
%          topspace=0mm,
%          topdistance=0mm,
%          headerdistance=3mm,
          footerdistance=1mm,
%          bottomdistance=0mm,
]

\definelayer[MyGuides]
\setuplayer[MyGuides][width=\paperwidth, height=.8pt, color=black]
\setlayer[MyGuides]{
    \position(\leftmarginwidth,2.5mm){\FormularMImedium\bold  PRÉ-VENDA A PARTIR DE 22.08.20}
    \position(0,8mm){\blackrule[width=\paperwidth, height=.6pt, color=black]}
    \position(0,\dimexpr\paperheight-\footerheight-.7mm){\blackrule[width=\paperwidth, height=.6pt, color=black]}
    \position(\leftmarginwidth,\dimexpr\paperheight-\footerheight+1.5mm){\switchtobodyfont[7pt]\FormularMImedium\bold SAIBA MAIS:} 
    \position(5cm,\dimexpr\paperheight-\footerheight+1.5mm){\switchtobodyfont[7pt]\FormularMImedium\bold ↙HEDRA.COM.BR}
    \position(5cm,\dimexpr\paperheight-\footerheight+4.5mm){\switchtobodyfont[7pt]\FormularMImedium\bold ↙LINK NA BIO}
    }        


\definelayer[MyGuidesHashtag]
\setuplayer[MyGuidesHashtag][width=\paperwidth, height=0pt, color=black]
\setlayer[MyGuidesHashtag]{
    %\position(\leftmarginwidth,2.5mm){\FormularMImedium\bold  PRÉ-VENDA A PARTIR DE 22.08.20}
    %\position(0,8mm){\blackrule[width=\paperwidth, height=.6pt, color=black]}
    \position(0,\dimexpr\paperheight-9mm){\blackrule[width=\paperwidth, height=.6pt, color=black]}
    \position(2.5mm,\dimexpr\paperheight-7mm){\FormularMI \#hedraedicoes} 
    %\position(5cm,\dimexpr\paperheight-\footerheight+1.5mm){\switchtobodyfont[7pt]\FormularMImedium\bold ↙HEDRA.COM.BR}
    %\position(5cm,\dimexpr\paperheight-\footerheight+4.5mm){\switchtobodyfont[7pt]\FormularMImedium\bold ↙LINK NA BIO}
    }               


\definelayer[MyGuidesCleaned] 
\setlayer[MyGuidesCleaned]{}

\setupalign[right]


\setuplocalinterlinespace[small]
\setupbackgrounds[header][background=MyGuidesCleaned]
\setupinterlinespace[line=2.5ex]


%%%%%%%%%%%%%%%%%%%%%%%%%%%%%%%%%%%%%%%%%%%%%%%%%%%%%%%%%%%%%%%%%|

\starttext
%\showframe  %Para mostrar somente as linhas.

\bgroup
\FormularMI\bold
\switchtobodyfont[28pt]\bold
\setupinterlinespace[line=1.9ex]
\setcharacterkerning[packed]
ESCRAVO X

ESCRAVISADO

\egroup

\vfill
\scale[lines=2]{\MyStar[black][none]}

\page %---------------------------------------------------------|

\setcharacterkerning[packed]

No aparato crítico das {\bf NARRATIVAS
DA ESCRAVIDÃO}, empregou-se a
palavra “escravizado” no lugar de
“escravo”.

\page %---------------------------------------------------------|

\setuplayout[MyLayout]
\kerncharacters[-.05]{Saiba mais e conheça nossa loja em}
Saiba mais e conheça nossa loja em
Mussum Ipsum, cacilds vidis litro abertis.  Suco de cevadiss, é um leite
 divinis, qui tem lupuliz, matis, aguis e fermentis. Cevadis im ampola
 pa arma uma pindureta. Mauris nec dolor in eros commodo tempor. Aenean
 aliquam molestie leo, vitae iaculis nisl. Paisis, filhis, espiritis
 santis. Nullam volutpat risus nec leo commodo, ut interdum diam
 laoreet. Sed non consequat odio. Quem num gosta di mim que vai caçá
 sua turmis! Interagi no mé, cursus quis, vehicula ac nisi. Não sou
 faixa preta cumpadi, sou preto inteiris, inteiris. Si u mundo tá muito
 paradis? Toma um mé que o mundo vai girarzis! Manduma pindureta quium
 dia nois paga. Copo furadis é disculpa de bebadis, arcu quam euismod
 magna. Sapien in monti palavris qui num significa nadis i pareci latim.

\setupbackgrounds[header][background=MyGuidesHashtag]


\page %---------------------------------------------------------|



\setcharacterkerning[reset]\setupbackgrounds[header][background=MyGuidesCleaned]
\Times
\switchtobodyfont[50pt] 

\mbox{}\blank[1.6cm]
\hfill \switchtobodyfont[50pt] hedra \hfill \mbox{}  



\stoptext